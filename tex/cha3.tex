% LaTeX source for textbook ``How to think like a computer scientist''
% Copyright (C) 1999  Allen B. Downey

% This LaTeX source is free software; you can redistribute it and/or
% modify it under the terms of the GNU General Public License as
% published by the Free Software Foundation (version 2).

% This LaTeX source is distributed in the hope that it will be useful,
% but WITHOUT ANY WARRANTY; without even the implied warranty of
% MERCHANTABILITY or FITNESS FOR A PARTICULAR PURPOSE.  See the GNU
% General Public License for more details.

% Compiling this LaTeX source has the effect of generating
% a device-independent representation of a textbook, which
% can be converted to other formats and printed.  All intermediate
% representations (including DVI and Postscript), and all printed
% copies of the textbook are also covered by the GNU General
% Public License.

% This distribution includes a file named COPYING that contains the text
% of the GNU General Public License.  If it is missing, you can obtain
% it from www.gnu.org or by writing to the Free Software Foundation,
% Inc., 59 Temple Place - Suite 330, Boston, MA 02111-1307, USA.

% This is an Icelandic translation/adaptation by Hrafn Loftsson of the orginal book by Allen B. Downey.

\chapter{Föll}

\section{Kommutölur}
\index{kommutala}
\index{tag!double}
%\index{double (floating-point)}

Í síðasta kafla áttum við í nokkrum vandræðum með tölur sem eru ekki heiltölur.
Við leystum vandamálið að hluta til með því að reikna út prósentur í stað brots.
Almennari lausn er hins vegar að nota kommutölur (e. floating-point) sem geta staðið fyrir bæði brot og heiltölur.
Tvenns konar kommutölur eru í C++, {\tt float} og {\tt double}.
Í þessari bók munum við eingöngu nota {\tt double} tölur.

Þú getur búið til kommutölur og gefið þeim gildi með sömu málskipan og notuð er fyrir önnur tög.
Dæmi:

\begin{verbatim}
  double pi;
  pi = 3.14159;
\end{verbatim}
%
Það er einnig mögulegt að lýsa yfir breytu og gefa henni gildi á sama tíma:

\begin{verbatim}
  int x = 1;
  String empty = "";
  double pi = 3.14159;
\end{verbatim}
%
Reyndar vill svo til að þessi leið er mjög algeng.
Samsett yfirlýsing (e. declaration) og gildisveiting (e. asssignment) er kölluð {\bf frumstilling} (e. initialization).

\index{frumstilling}

Þó svo að kommutölur séu gagnlegar þá valda þær stundum ruglingi vegna þess að svo virðist sem heiltölur og kommutölur skarist. 
T.d., er gildið {\tt 1} heiltala, kommutala eða hvoru tveggja?

C++ gerir greinarmun á heiltölugildinu {\tt 1} og kommutölugildinu {\tt 1.0} þrátt fyrir að þau virðast standa fyrir sömu töluna.
Ástæðan er sú að gildin tvö tilheyra mismunandi tögum og almennt gildir að ekki er leyfilegt að framkvæma gildisveitingu ``á milli'' taga.
Þetta er t.d. ekki leyfilegt í C++

\begin{verbatim}
    int x = 1.1;
\end{verbatim}
%
vegna þess að breytan á vinstri hlið gildisveitingarinnar er af taginu {\tt int} og gildið á hægri hlið er {\tt double}.
Það er aftur á móti auðvelt að gleyma þessari reglu, sérstaklega vegna þess að í sumum tilvikum breytir C++ þýðandi einu tagi í annað á sjálfvirkan hátt.
T.d. ætti 

\begin{verbatim}
    double y = 1;
\end{verbatim}
%
tæknilega séð ekki að vera leyfilegt en C++ þýðandi leyfir þetta með því að breyta sjálfur
{\tt int} (gildinu 1) í {\tt double}.
Þessi ``linkind'' af hálfu þýðandans getur verið þægileg fyrir forritarann en getur jafnframt leitt til vandræða.
Dæmi:

\begin{verbatim}
    double y = 1 / 3;
\end{verbatim}
%
Hér gætir þú búist við því að breytan {\tt y} fái gildið {\tt 0,333333}, sem er leyfilegt kommutölugildi, en í raun fær hún gildið {\tt 0.0}.
Ástæðan er sú að segðin á hægri hlið gildisveitingarinnar er hlutfall tveggja heiltalna og því framkvæmir C++ {\em heiltöludeilingu}, hvers niðurstaða er heiltölugildið {\tt 0}.
Þegar því gildi er breytt (af þýðandanum) í kommutölu þá er niðurstaðan {\tt 0,0}.

Ein leið til að leysa þetta vandamál er að gera gildi segðarinnar á hægri hlið að kommutölu:

\begin{verbatim}
    double y = 1.0 / 3.0;
\end{verbatim}
%
Þetta veldur því að {\tt y} fær gildið {\tt 0,333333} eins og við var að búast.

\index{útreikningur!kommutala}

Allar þær reikniaðgerðir sem við höfum séð hingað til -- samlagning, frádráttur, margföldun og deiling -- virka á kommutölum sem og á heiltölum.
Það er hins vegar athyglisvert að undirliggjandi vélarmálsútreikningur er mismunandi eftir því hvort um kommutölur eða heiltölur er að ræða.
Flestir örgjörvar hafa einmitt sérstakan búnað til að framkvæma aðgerðir á kommutölum.

\section{Breyting á {\tt double} í {\tt int}}
\label{rounding}
\index{afrúnun}
\index{tagmótun}

Eins og ég nefndi áður þá breytir C++ þýðandinn {\tt int}
í {\tt double} á sjálfvirkan hátt ef þörf er á vegna þess að engar upplýsingar tapast í breytingunni.
Á hinn bóginn þá krefst breyting á {\tt double} í {\tt int} afrúnunar (e. rounding off).
C++ framkvæmir þá aðgerð ekki sjálfvirkt og því þarft þú, sem forritari, að vera meðvitaður um brotið sjálft (þ.e. sá hluti sem kemur á eftir kommunni) tapast.

Einfaldasta leiðin til að breyta kommutölu í heiltölu er að nota {\bf tagmótun} (e. typecast).
Tagmótun dregur nafn sitt af því að það gefur þér kost á því að ``móta'' gildi sem tilheyrir einu tagi yfir í annað tag.

Málskipan fyrir tagmótun er eins og málskipan fyrir fallakall.
Dæmi:

\begin{verbatim}
  double pi = 3.14159;
  int x = int (pi);
\end{verbatim}
%
{\tt int} fallið skilar heiltölu þannig að {\tt x} fær gildið 3.
Að móta kommutölu í heiltölu hefur í för með sér að talan er rúnuð niður (e. rounded down) jafnvel þó svo að brotið sé 0.99999999.

Fyrir sérhvert tag í C++ er til samsvarandi fall sem tagmótar sitt viðfang í viðeigandi tag.

\section{Stærðfræðiföll}
\index{Math function}
\index{fall!stærðfræði}
\index{segð}
\index{viðfang}

Í stærðfræði hefur þú væntanlega séð föll eins og $\sin$ og $\log$ og hefur lært að reikna út gildi segða eins og $\sin(\pi/2)$ og $\log(1/x)$.
Fyrst reiknar þú út gildi segðar innan sviga en það er kallað {\bf viðfang} (e. argument) fallsins.
T.d. er $\pi/2$ u.þ.b. 1,571 og $1/x$ er 0,1 (ef $x$ hefur gildið 10).

Eftir þetta getur þú ákvarðar gildi fallsins sjálfs, annað hvort með því að fletta upp í töflu eða með því að framkvæma ýmsa útreikninga.
$\sin$ af 1,571 er 1 og $\log$ af 0,1 er -1 (ef við gerum ráð fyrir því að $\log$ standi fyrir 
lógariþma með grunn 10).

Þetta ferli er hægt að endurtaka til að ákvarða gildi flóknari segða eins og $\log(1/\sin(\pi/2))$.
Fyrst ákvörðum við gildi viðfangs innsta fallsins (þ.e. ($\pi/2$)), síðan reiknum við út gildi fallsins (þ.e. $\sin$), o.s.frv.

C++ býður upp á mengi af innbyggðum (e. built-in) föllum sem inniheldur flestar þær stærðfræðilegu aðgerðir sem þú getur ímyndað þér.
Kallað er á þessi stærðfræðiföll með því að nota málskipan sem er sambærileg við stærðfræðilega táknun: 

\begin{verbatim}
     double log = log (17.0);
     double angle = 1.5;
     double height = sin (angle);
\end{verbatim}
%
\index{stærðfræðifall!log}
\index{stærðfræðifall!sin}

Fyrsta setningin að ofan gefur breytunni {\tt log} gildið lógariþmi af 17 (grunnur $e$).
Það er einnig til fall {\tt log10} sem reiknar út lógariþma miðað við grunn 10.

Þriðja setningin reiknar út sínus af gildinu sem geymt er í breytunni {\tt angle}.
C++ gerir ráð fyrir því að gildin sem notuð eru með sínus fallinu, og öðrum hornaföllum ({\tt cos}, {\tt tan}), séu í {\em radian}.
Til að breyta gráðum í radian getur þú deilt með 360 og margfaldað með $2 \pi$.  

Ef þú þekkir ekki gildið á $\pi$ (með 15 aukastöfum!) þá getur þú reiknað það út með því að nota {\tt acos} fallið.
Arccosínus (eða andhverfa cosínus) af -1 er $\pi$ vegna þess að cosínus af $\pi$ er -1.

\begin{verbatim}
  double pi = acos(-1.0);
  double degrees = 90;
  double angle = degrees * 2 * pi / 360.0;
\end{verbatim}
\index{pi}
%
Áður en þú getur notað eitthvað af stærðfræðiföllunum þarftu að taka inn (e. include) sérstaka {\bf hausaskrá} (e. header file) í forritið þitt.
Hausaskrá inniheldur upplýsingar um föll, sem eru skilgreind annars staðar en í þínu eigin forriti, og sem þýðandinn þarf á að halda.
Í ``Hello, world!'' forritinu tókum við t.d. inn haus nefndan {\tt iostream} með því að nota {\bf include} setningu:

\begin{verbatim}
#include <iostream>
using namespace std;
\end{verbatim}
%
{\tt iostream} inniheldur upplýsingar um inntaks- og úttaksstrauma (e. I/O streams), þ.m.t. um úttaksstrauminn {\tt cout}.
C++ notar öflugan eiginleika sem kallaður er {\bf nafnasvið} (e. namespaces) sem gerir þér t.d. kleift að útfæra {\tt cout} á þinn eigin máta.
Í flestum tilvikum notum við reyndar hina stöðluðu útfærslu sem skilgreind er í nafnasviðinu std.
Við látum þýðandann vita af þessu með línunni

\begin{verbatim}
using namespace std;
\end{verbatim}

Þumalputtareglan er sú að þú átt að skrifa {\tt using namespace std;} í hvert sinn sem þú ætlar að nota {\tt iostream}.

\index{header skrá}
\index{cmath}
\index{iostream}

Á sambærilegan hátt inniheldur {\tt cmath} hausaskráin upplýsingar um stærðfræðiföll.
Þú getur tekið þá skrá inn, ásamt {\tt iostream}, í upphafi forritsins þíns: 

\begin{verbatim}
#include <cmath>
\end{verbatim}

Hausaskrár sem byrja á `c' gefa til kynna að þær hafi upphaflega verið búnar til fyrir forritunarmálið {\bf C}.

\section {Samsetning}
\label{composition}
\index{samsetning}
\index{segð}

Föll í C++ geta verið samsett á sama hátt og stærðfræðiföll.
Þetta merkir að þú getur notað eina segð sem hluta af annarri.
Þú getur t.d. notað hvaða segð sem er sem viðfang í fall:

\begin{verbatim}
    double x = cos (angle + pi/2);
\end{verbatim}
\index{stærðfræðifall!cos}
%
Í þessari setningu er deilt í gildið {\tt pi} með tveimur og gildinu á breytunni {\tt angle} bætt við niðurstöðuna.
Summan er síðan send sem viðfang í fallið {\tt cos}.

Þú getur einnig sent niðurstöðuna úr einu falli sem viðfang í annað fall:

\begin{verbatim}
    double x = exp (log (10.0));
\end{verbatim}
\index{stærðfræðifall!exp}
%
Hér er tekinn lógariþmi (með grunn $e$) af 10 og niðurstaðan (nefnum hana $t$) síðan send inn í exp fallið sem reiknar $e$ í veldinu $t$.
Breytan {\tt x} fær að lokum gildið úr heildarniðurstöðunni sem ég vona að þú vitir hver er!

\section{Nýjum föllum bætt við}
\index{fall!skilgreining}
\index{main}
\index{fall!main}

Hingað til höfum við eingöngu notað föll sem eru innbyggð í C++ en við getum einnig bætt við nýjum föllum.
Reyndar vill svo til að við höfum þegar bætt við einu nýju falli: {\tt main}.
Fallið {\tt main} er sérstakt að því leyti til að það gefur til kynna hvar keyrsla forritsins á að byrja en málskipan fyrir {\tt main} er sú sama og fyrir hvaða aðra fallaskilgreiningu sem er:

\begin{verbatim}
  void NAFN ( LISTI AF VIÐFÖNGUM ) {
    SETNINGAR
  }
\end{verbatim}
%
Þú getur gefið fallinu þínu hvaða nafn sem er með þeirri undantekningu að þú getur hvorki kallað það 
{\tt main} né notað annað C++ lykilorð.
Listinn af viðföngum skilgreinir hvaða upplýsingar (ef nokkrar) þarf að gefa fallinu þegar það er notað (þegar {\bf kallað} (e. call)) er á það.

{\tt main} tekur engin viðföng eins og sjá má með tómum svigum {\tt ()} í skilgreiningunni á fallinu.
Fyrstu tvö föllin sem við munum skrifa taka heldur engin viðföng -- málskipanin lítur þá svona út:

\begin{verbatim}
  void newLine () {
    cout << endl;
  }
\end{verbatim}
%
Þessu falli hefur verið gefið nafnið {\tt newLine}.
Það inniheldur aðeins eina setningu, þ.e. skrifar út stafinn {\tt endl} sem stendur fyrir nýja línu.

Úr {\tt main} getum við kallað á þetta nýja fall með því að nota málskipan sem er svipuð þeirri sem við notum þegar við köllum á innbyggð föll:

\begin{verbatim}
int main ()
{
  cout << "First Line." << endl;
  newLine ();
  cout << "Second Line." << endl;
  return 0;
}
\end{verbatim}
%
Úttakið úr forritinu er:

\begin{verbatim}
First line.

Second line.
\end{verbatim}
%
Taktu eftir auka (tómu) línunni á milli línanna tveggja.
En hvað ef við þyrftum á meira bili að halda á milli línanna?
Við gætum kallað á þetta sama fall nokkrum sinnum:

\begin{verbatim}
int main ()
{
  cout << "First Line." << endl;
  newLine ();
  newLine ();
  newLine ();
  cout << "Second Line." << endl;
  return 0;
}
\end{verbatim}
%
Annar möguleiki væri sá að skrifa nýtt fall, t.d. með nafninu {\tt threeLine}, sem skrifar út þrjár nýjar línur: 

\begin{verbatim}
void threeLine ()
{
  newLine ();  newLine ();  newLine ();
}

int main ()
{
  cout << "First Line." << endl;
  threeLine ();
  cout << "Second Line." << endl;
  return 0;
}
\end{verbatim}
%
Hér eru nokkur atriði sem vert er að gefa gaum:

\begin{itemize}

\item Hægt er að kalla á sama fallið aftur og aftur. Reyndar vill svo til að það er einmitt algengt og gagnlegt.

\item Hægt er að láta eitt fall kalla á annað.
Í forritinu að ofan kallar {\tt main} á {\tt threeLine} og {\tt threeLine} kallar á {\tt newLine}.
Þetta er einnig algengt og gagnlegt. 

\item Í {\tt threeLine} skrifaði ég þrjár setningar í einni og sömu línunni sem er setningafræðilega rétt (mundu að bil og tómar línur breyta ekki merkingu forrits í C++).
Á hinn bóginn bendi ég á að það er yfirleitt betra að hafa eina setningu í hverri línu því þannig verður forritið læsilegra.
Í þessari bók brýt ég stundum þessa reglu til að spara pláss.

\end{itemize}

Á þessum tímapunkti er kannski ekki ljóst hvað ávinnst með því að búa til öll þessi föll.
Fyrir því eru margar ástæður en forritið að ofan sýnir fram á tvær þeirra:

\begin{enumerate}

\item Með því að búa til nýtt fall þá getur þú gefið safni setninga nafn.
Föll geta einfaldað forrit með því að hylja flókna reikninga/aðgerðir ``á bak við'' eina skipun og með því að nota orð sem eru okkur töm í stað ``skrýtinna'' tákna.
Hvort er læsilegra, {\tt newLine} eða {\tt cout << endl}?

\item Það að búa til fall getur eytt endurteknum kóða og þar með gert forrit styttra.
Einföld leið til að prenta t.d. níu nýjar línur í röð væri að kalla á {\tt threeLine} þrisvar sinnum.
Hvernig myndir þú prenta út 27 nýjar línur?

\end{enumerate}

\section {Skilgreiningar og notkun}

Ef við tökum saman alla kóðabútana úr kaflanum hér á undan þá lítur forritið í heild sinni svona út:

\begin{verbatim}
#include <iostream>
using namespace std;

void newLine ()
{
  cout << endl;
}

void threeLine ()
{
  newLine ();  newLine ();  newLine ();
}

int main ()
{
  cout << "First Line." << endl;
  threeLine ();
  cout << "Second Line." << endl;
  return 0;
}
\end{verbatim}

Þetta forrit inniheldur þrjár fallaskilgreiningar: {\tt newLine}, {\tt threeLine} og {\tt main}.

Skilgreiningin á {\tt main} inniheldur setningu sem kallar á {\tt threeLine}.
Á sama hátt kallar {\tt threeLine} þrisvar sinnum á {\tt newLine}.

Taktu eftir að skilgreiningin á sérhverju falli kemur á undan þeim stað þar sem fallið er notað (þar sem kallað er á það).
C++ krefst þess að skilgreiningin á falli komi á undan fyrstu noktun þess.
Þú ættir að prófa að þýða þetta forrit með föllunum í annarri röð til að sjá hvers konar villumeldingar þú færð.

\section {Forrit með mörgum föllum}

Þegar þú skoðar forrit sem inniheldur nokkur föll þá er ruglandi að lesa forritið frá ``toppi til táar''
því það endurspeglar ekki {\bf keyrsluröð} (e. order of execution) forritsins.

Keyrslan hefst alltaf í fyrstu setningunni í {\tt main}, burtséð frá því hvar {\tt main} er staðsett í forritinu (það er reyndar oft neðst í kóðanum).
Setningar eru keyrðar, ein í einu, í röð þangað til komið er að fallakalli.
Fallaköll eru eins og krókur í flæði keyrslunnar.
Í stað þess að fara í næstu setningu þá er stokkið í fyrstu línu fallsins sem kallað er á, allar setningar fallsins keyrðar og síðan stokkið til baka og þráðurinn tekinn upp þar sem frá var horfið.

Þetta hljómar svo sem einfalt en mundu að eitt fall getur kallað á annað.
Þannig að þegar við eru í miðri keyrslu á {\tt main} gætum við þurft að stökkva burt og keyra setningarnar í {\tt threeLine}.
En meðan setningarnar í {\tt threeLine} eru keyrðar þá eru við ``trufluð'' (e. interrupted) þrisvar sinnum til að keyra {\tt newLine}.

Sem betur fer þurfum við sem forritarar ekki að hafa áhyggjur af þessum stökkvum í föll því C++ þýðandinn býr til kóða sem sér um þetta fyrir okkur.
Þegar {\tt newLine} hættir heldur forritið áfram á réttum stað í {\tt threeLine} og kemst að lokum til baka í {\tt main} þar sem keyrslunni lýkur.

Boðskapurinn er sem sagt sá að þegar þú lest forritið þá skaltu ekki lesa það frá toppi til táar heldur fylgja keyrsluflæðinu (e. flow of execution).

\section {Leppar og viðföng}
\index{leppur}
\index{viðfang}

Sum af þeim innbyggðu föllum sem við höfum skoðað hafa {\bf leppa} (e. formal parameters)
en þeir geyma þau gildi sem við látum fall hafa til að það geti gert það sem það á að gera.
Ef þú þarft t.d. að finna sínus af einhverri tölu þá þarftu að gefa til kynna hver talan er.
Þ.e. {\tt sin} tekur {\tt double} gildi sem viðfang.

Sum föll hafa fleiri en einn lepp.
Dæmi um það er fallið {\tt pow} sem tekur tvö {\tt double} gildi, grunninn og veldisvísinn.

Í sérhverju þessara tilfella þarf bæði að taka fram hversu margir lepparnir eru og af hvaða tagi þeir eru.
Það ætti því ekki að koma á óvart að þegar þú skrifar fallaskilgreiningu þá inniheldur listinn yfir leppana einnig tag þeirra:
Dæmi:

\begin{verbatim}
  void printTwice (char phil) {
    cout << phil << phil << endl;
  }
\end{verbatim}
%
Þetta fall er með lepp með nafninu {\tt phil} sem er af taginu {\tt char}.
Gildið sem kemur inn, hvert svo sem það er (á þessum tímapunkti vitum við það ekki), er prentað tvisvar og ný lína á eftir.
Ég notaði hér nafnið {\tt phil} til að gefa til kynna að þú getur notað hvaða nafn sem er á leppum en almennt séð þá ættir þú að velja eitthvað meira lýsandi nafn en {\tt phil}.

Til að kalla á þetta fall verðum við að gefa því {\tt char} gildi.
Við gætum t.d. skrifað {\tt main} fallið svona:

\begin{verbatim}
  int main () {
    printTwice ('a');
    return 0;
  }
\end{verbatim}
%
{\tt char} gildið í kallinu á {\tt printTwice} er kallað {\bf viðfang} (e. actual parameter/argument).
Talað er um að senda viðfangið (e. to pass the argument) til fallsins.
Í þessu tilviki {\tt 'a'} sent sem viðfang í {\tt printTwice} sem prentar gildið út tvisvar.

Ef við hefðum breytu af taginu {\tt char} þá gætum við notað hana sem viðfang í staðinn:

\begin{verbatim}
  int main () {
    char argument = 'b';
    printTwice (argument);
    return 0;
  }
\end{verbatim}
%
Hér er eitt mikilvægt atriðið: Nafnið á breytunni sem við sendum sem viðfang hefur ekkert að gera með nafnið á leppnum ({\tt phil}).
Ég endurtek:
\begin{quote}

{\bf Nafnið á breytunni sem við sendum sem viðfang hefur ekkert að gera með nafnið á leppnum.}

\end{quote}

Nöfnin geta verið þau sömu en þau geta líka verið mismunandi.
Það er mikilvægt að gera sér grein fyrir því að leppurinn og viðfangið eru ekki sami hluturinn
þó svo að þau hafi sama gildið (í þessu tilviki stafinn {\tt 'b'}).

Gildið sem sent er sem viðfang verður að hafa sama tagið og leppurinn í fallinu sem kallað er á.
Þessi regla er mikilvæg en getur verið ruglandi því C++ breytir stundum einu tagi í annað á sjálfvirkan hátt.
Á þessum tímapunkti skaltu muna þessa almennu reglu en við munum ræða undantekningar frá henni síðar.

\section {Leppar og breytur eru staðværar}

Gildissvið (e. scope) leppa og breytna er aðeins innan í eigin föllum.
Innan marka {\tt main} er ekki neinn hlutur með nafninu {\tt phil} aðgengilegur.
Þýðandinn mun kvarta ef þú reynir að nota það nafn innan í {\tt main}.
Á sama hátt er enginn hlutur með nafninu {\tt argument} aðgengilegur inni í {\tt printTwice}.

Breytur eins og þessar eru sagðar vera {\bf staðværar} (e. local).
Það getur verið gott að teikna {\bf staflarit} (e. stack diagram) til að gera sér grein fyrir leppum og staðværum breytum.
Staflarit sýna (eins og stöðurit) gildi á sérhverri breytu en breyturnar eru innan í stærri boxum sem gefa til kynna hvaða föllum þær tilheyra.

Staflarit fyrir {\tt printTwice} lítur svona út:

\vspace{0.1in}
\centerline{\epsfig{figure=stack.eps}}
\vspace{0.1in}
%
Í hvert skipti sem kallað er á fall til nýtt {\bf tilvik} af upplýsingum um fallið (kallað {\bf kvaðningafærsla} (e. activation record)).
Sérhvert tilvik af fallinu inniheldur leppana og staðværu breyturnar.
Í myndinni er tilvik af fallinu sýnt sem box með nafni fallsins að utanverðu en að innanverðu eru breytur og leppar.

Í dæminu hefur {\tt main} eina staðværa breytu, {\tt argument}, og enga leppa.
{\tt printTwice} hefur engar staðværar breytur en einn lepp með nafninu {\tt phil}.

\section {Föll með marga leppa}
\index{leppur!margir}
\index{fall!margir leppar}
%\index{class!Time}

Málskipanin sem notuð er til að lýsa yfir og kalla á föll með mörgum leppum veldur of villum við þýðingu.
Í fyrsta lagi þarf að muna að það þarf að lýsa yfir tagi á sérhverjum lepp.
Dæmi:
\begin{verbatim}
  void printTime (int hour, int minute) {
    cout << hour;
    cout << ":";
    cout << minute;
  }
\end{verbatim}
%
Það er freistandi að skrifa {\tt (int hour, minute)} en sá ritháttur er aðeins löglegur fyrir yfirlýsingar á breytum en ekki á leppum!

Annað sem getur valdið ruglingi er að þú þarft ekki að lýsa yfir tagi á viðföngunum.
Eftirfarandi er rangt:

\begin{verbatim}
    int hour = 11;
    int minute = 59;
    printTime (int hour, int minute);   // WRONG!
\end{verbatim}
%
Ástæðan er sú að þýðandinn getur séð hvert tagið á viðföngunum, breytunum {\tt hour} og {\tt minute}, er með því að fletta upp í skilgreiningunni á þeim.
Það er sem sagt óþarfi og óleyfilegt að taka fram tagið á viðföngunum.
Rétta málskipanin er {\tt printTime (hour, minute)}.

\section {Föll sem skila gildi}
\index{frjósöm föll}
\index{föll!frjósöm}

Þú ættir að hafa tekið eftir því að sum af þeim föllum sem við höfum notað, t.d. stærðfræðiföll, skila af sér gildi.
Önnur föll, eins og {\tt newLine}, framkvæma aðgerð en skila ekki gildi.
Þetta vekur upp nokkrar spurningar:

\begin{itemize}

\item Hvað gerist ef þú kallar á fall og þú gerir ekkert við niðurstöðuna (þ.e. þú setur niðurstöðuna hvorki inn í breytu né notar hana sem hluta af stærri segð)?

\item Hvað gerist ef þú nota fall sem ekki skilar gildi sem hluta af segð, eins og {\tt newLine() + 7}?

\item Getum við skrifað föll sem skila gildum eða sitjum við uppi með að skrifa föll eins og {\tt newLine} og {\tt printTwice}?

\end{itemize}

Svarið við þriðju spurningunni er ``já'', þ.e. við getum skrifað föll sem skila gildum og við munum einmitt gera það í næstu köflum.
Ég mun láta þér eftir að svara fyrstu tveimur spurningunum með prófunum.
Það er góð leið að spyrja þýðandann í sérhvert sinn sem þig vantar svar varðandi það hvort tiltekið atriði er leyfilegt eður ei í C++.

\section{Orðalisti}

\begin{description}

\item[kommutala (e. floating-point):] Tag breytu (eða gildi) sem getur geymt brot sem og heiltölur.
Það eru nokkur kommutölutög í C++ en við munum nota {\tt double} í þessari bók.

\item[frumstilling (e. initialization):]  Setning sem lýsir yfir breytu og gefur henni gildi á sama tíma.

\item[fall (e. function):]  Röð setninga sem bera tiltekið nafn og framkvæma tiltekna(r) aðgerð(ir).
Föll taka 0 eða fleiri viðföng og geta skilað af sér gildi.

\item[leppur (e. parameter):]  Geymir upplýsingar sem gefnar eru upp þegar kallað er á fall.
Leppar eru eins og breytur í þeim skilningi að þeir innihelda bæði gildi og eru af tilteknu tagi.

\item[viðfang (e. argument):]  Gildi sem gefið er upp þegar kallað er á fall.
Gildið verður að vera af sama tagi og viðkomandi leppur.

\item[fallakall (e. function call):]  Veldur því að fall er keyrt. 

\index{kommutala}
\index{fall}
\index{leppur}
\index{viðfang}
\index{fallakall}
\index{frumstilling}

\end{description}