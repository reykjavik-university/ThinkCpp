% LaTeX source for textbook ``How to think like a computer scientist''
% Copyright (C) 1999  Allen B. Downey

% This LaTeX source is free software; you can redistribute it and/or
% modify it under the terms of the GNU General Public License as
% published by the Free Software Foundation (version 2).

% This LaTeX source is distributed in the hope that it will be useful,
% but WITHOUT ANY WARRANTY; without even the implied warranty of
% MERCHANTABILITY or FITNESS FOR A PARTICULAR PURPOSE.  See the GNU
% General Public License for more details.

% Compiling this LaTeX source has the effect of generating
% a device-independent representation of a textbook, which
% can be converted to other formats and printed.  All intermediate
% representations (including DVI and Postscript), and all printed
% copies of the textbook are also covered by the GNU General
% Public License.

% This distribution includes a file named COPYING that contains the text
% of the GNU General Public License.  If it is missing, you can obtain
% it from www.gnu.org or by writing to the Free Software Foundation,
% Inc., 59 Temple Place - Suite 330, Boston, MA 02111-1307, USA.

% This is an Icelandic translation/adaptation of the orginal book by Allen B. Downey

\chapter{Ítrun}

\section{Fjölgildisveiting}
\index{gildisveiting}
\index{setning!gildisveiting}
\index{fjölgildisveiting}

Ég hef ekki nefnt það áður að í C++ er leyfilegt að gefa breytu gildi oftar en einu sinni.
Tilgangur gildisveitingar nr. 2 er að skipta út gömlu gildi tiltekinnar breytu fyrir nýtt gildi.

\begin{verbatim}
  int fred = 5;
  cout << fred;
  fred = 7;
  cout << fred;
\end{verbatim}
%
Úttakið úr þessu forriti er {\tt 57} vegna þess að í fyrra skiptið sem við skrifum út {\tt fred}
þá er gildi breytunnar 5 en í síðari skiptið er gildið 7.

Þessi tegund af {\bf fjölgildisveitingu} (e. multiple assignment) er ástæðan fyrir því að
ég lýsti breytum sem einskonar {\em gámi} (e. container) fyrir gildi.
Þegar þú gefur breytu gildi þá breytir þú innihaldi gámsins eins og sést á eftirfarandi mynd:

\vspace{0.1in}
\centerline{\epsfig{figure=assign2.eps}}
\vspace{0.1in}

Þegar um fjölgildisveitingu er að ræða þá er sérstaklega mikilvægt að gera greinarmun á gildisveitingu og samanburði.
C++ notar {\tt =} táknið fyrir gildisveitingu en það er freistandi að túlka setningu eins og {\tt a = b} sem samanburð (jöfnuð).
Svo er hins vegar ekki!

Í fyrsta lagi þá er samanburður víxlin (e. commutative) en gildisveiting er það ekki.
Í stærðfræði ef t.d. $a = 7$ þá er $7 = a$.
En í C++ þá er setningin {\tt a = 7;} lögleg en {\tt 7 = a;} aftur á móti ekki. Af hverju ekki?

Jafnframt gildir í stærðfræði að setning um jöfnuð (e. statement of equality) er sönn án tillits til tímasetningar.
Ef $a = b$ núna, þá verður $a$ alltaf jafnt $b$.
Í C++ getur gildisveiting gert tvær breytur jafnar en þær þurfa hins vegar ekki alltaf að vera jafnar eftir það!

\begin{verbatim}
  int a = 5;
  int b = a;     // a and b are now equal
  a = 3;         // a and b are no longer equal
\end{verbatim}
%
Þriðja línan breytir gildinu á {\tt a} en breytir hins vegar ekki gildinu á {\tt b} og því eru breyturnar tvær ekki lengur jafnar.
Í mörgum forritunarmálum er annað tákn notað fyrir gildisveitingu, t.d. {\tt <-} eða {\tt :=}, til að koma í veg fyrir rugling.

Þrátt fyrir að fjölgildisveiting sé oft gagnleg þá skaltu nota hana með varúð.
Ef gildi breytna eru stöðugt að breytast á mismunandi stöðum í forritunu þínu þá getur það orðið ólæsilegra og erfiðara að kemba.

\section{Ítrun}
\index{ítrun}

Eitt af því sem tölvur eru oft notaðar í er að sjálfvirknivæða einhæf verkefni.
Tölvur er, ólíkt fólki, góðar í því að endurtaka sama eða svipað verkefni án þess að gera villur.

Við höfum hingað til séð forrit sem beitir endurkvæmni til að framkvæma endurtekningar, t.d. {\tt nLines} og {\tt countdown}.
Þessi tegund af endurtekningum er kölluð {\bf ítrun} (e. iteration) og C++ hefur ýmsa innbyggða eiginleika sem gera okkur auðveldara að skrifa forrit sem notar ítranir.

%Við ætlum nú að skoða tvo af þessum eiginleikum:
%{\tt while} setningu og {\tt for} setningu.

Við ætlum nú að skoða einn af þessum eiginleikum: {\tt while} setningu.

\section{{\tt While} setning}
\index{setning!while}
\index{while setning}

Með því að nota {\tt while} setningu getum við endurskrifað {\tt countdown} fallið:

\begin{verbatim}
int countdown (int n) {
  while (n > 0) {
    cout << n << endl;
    n = n-1;
  }
  cout << "Blastoff!" << endl;
  return 0;
}
\end{verbatim}
%
Þú getur nánast lesið {\tt while} setningu eins og um enskan texta væri að ræða.
Merkingin hér er: ``While {\tt n} is greater than
zero, continue displaying the value of {\tt n} and then reducing
the value of {\tt n} by 1.  When you get to zero, output the
word `Blastoff!'''

Við getum lýst keyrsluflæðinu í {\tt while} setningu á formlegri hátt:

\begin{enumerate}

\item Gildið á skilyrðinu innan í svigunum er ákvarðað. Útkoman er annaðhvort {\tt true} eða {\tt false}.

\item Ef skilyrðið er ósatt (false) þá er strax hætt í {\tt while} setningunni og haldið áfram keyrslu í næstu setningu á eftir.

\item Ef skilyrðið er satt (true) þá er sérhver setning í blokkinni sem afmarkast af slaufusvigunum keyrð og síðan farið til baka í skref 1.

\end{enumerate}

Þessi tegund af flæði er kölluð {\bf lykkja} (e. loop) vegna þess að þriðja skrefið stekkur til baka í fyrsta skrefið.
Athugaðu að ef skilyrðið er ósatt (false) strax í upphafi þá eru setningarnar innan lykkjunnar aldrei keyrðar.
Setningarnar innan í lykkju eru kallaðar {\bf meginmál} (e. body) lykkjunnar.

\index{lykkja}
\index{lykkja!meginmál}
\index{lykkja!óendanleg}
\index{meginmál!lykkja}
\index{óendanleg lykkja}

Það er mikilvægt að meginmál lykkjunnar breyti gildi einnar eða fleiri breytna þannig að skilyrðið verði false á einhverjum tímapunkti og lykkjan hætti þá keyrslu.
Að öðrum kosti mun lykkjan halda áfram sínum ítrunum (að eilífu!) og í því tilviki er um {\bf óendanlega lykkju} (e. infinite loop) að ræða.
%An endless source of amusement for computer scientists is the observation
%that the directions on shampoo, ``Lather, rinse, repeat,'' are an infinite loop.

Í tilviki {\tt countdown} getum við sannað að lykkjan muni klárast.
Við vitum að gildið á {\tt n} minnkar um 1 í hverri {\bf ítrun} lykkjunnar þannig að {\tt n} fær gildið 0 að lokum.
Í öðrum tilfellum er kannski ekki svo auðvelt að segja til um þetta:

\begin{verbatim}
  void sequence (int n) {
    while (n != 1) {
      cout << n << endl;
      if (n%2 == 0) {           // n is even
        n = n / 2;
      } else {                  // n is odd
        n = n*3 + 1;
      }
    }
  }
\end{verbatim}
%
Skilyrði fyrir áframhaldi þessarar lykkju er {\tt n != 1} og hún heldur því áfram þangað til {\tt n} er 1 (sem gerir skilyrðin false).
Í sérhverri ítrun skrifar fallið út gildið á {\tt n} og athugar síðan hvort það er slétt tala eða oddatala.
Ef {\tt n} er slétt þá er deilt í gildið með tveimur.
Ef það er oddatala þá er gildinu skipt út fyrir $3n+1$.
Ef upphafsgildið (þ.e. gildi viðfangsins sem sent er í {\tt sequence}) á {\tt n} er t.d. 3 þá skrifast út eftirfarandi röð:
3, 10, 5, 16, 8, 4, 2, 1.

Þar sem {\tt n} hækkar stundum eða lækkar þá er ekki auðvelt að sanna að {\tt n} fái að lokum gildið 1 og þar með að lykkjan klárist að lokum.
Fyrir tiltekin gildi á {\tt n} getum við sannað að lykkjan klárist.
Ef upphafsgildið á {\tt n} er t.d. veldi af tveimur þá mun gildið á {\tt n} vera slétt tala í sérhverri ítrun lykkjunnar þangað til það verður að lokum 1.
Dæmið að ofan endar með þannig röð, þ.e. röðinni sem byrjar á 16.

Það er hins vegar athyglisvert vandamál að reyna að sanna að lykkjan hætti keyrslu fyrir {\em öll} gildi á {\tt n}.
Hingað til hefur engum tekist að sanna eða afsanna það!

\section{Töflur}
\index{tafla}
\index{lógariþmi}

Eitt af því sem hentugt er að nota lykkur í er að búa til töflugögn.
Fyrir tíma tölvunnar þurfti fólk að reikna lógariþma, sínus og cósínus, og önnur stærðfræðiföll, í höndunum.
Til að gera þetta auðveldara þá voru til bækur sem innhéldu langar töflur þar sem hægt var að fletta upp gildum fyrir ýmis föll.
Það var tímafrekt og leiðinlegt að búa þessar töflur til og sumar þeirra innihéldu fullt af villum.

Ein fyrstu viðbrögðin við því þegar tölvur komu á sjónarsviðið voru:
``Þetta er frábært!  Við getum notað tölvurnar til að búa til töflur sem innihalda engar villur.''
Það reyndist vera (næstum því) rétt en skammsýnt.
Brátt urðu nefnilega tölvur og reiknivélar svo útbreiddar að það var engin þörf á þessum töflum lengur.

Jæja, næstum því.
Það vill reyndar svo til að fyrir tilteknar aðgerðir nota tölvur töflur til að nálga svörin og nota síðan útreikninga til að bæta nálganirnar.
Í sumum tilvikum hafa verið villur í undirliggjandi töflum -- sú þekktasta í töflunni sem fyrsta útgáfa af Intel Pentium örgjörvanum notaði til að framkvæma kommutöludeilingu.

\index{deiling!kommutala}

Þrátt fyrir að ``log tafla'' sé ekki eins gagnleg og áður fyrr þá má nota hana sem gott dæmi um ítrun.
Eftirfarandi forrit skrifar út röð gilda í vinstri dálki og samsvarandi lógariþma í hægri dálki:

\begin{verbatim}
  double x = 1.0;
  while (x < 10.0) {
    cout << x << "\t" << log(x) << "\n";
    x = x + 1.0;
  }
\end{verbatim}
%
Stafarunan \verb+\t+ stendur fyrir {\bf dálkastafinn} (e. tab character).
Runan \verb+\n+ stendur fyrir {\bf newline} (ný lína) stafinn.
Báðar þessar runur er hægt að setja hvar sem er inn í streng en í þessu dæmi mynda runurnar heilan streng án nokkurra annara stafa.

Dálkastafurinn veldur því að bendillinn ``stekkur'' til hægri þangað til hann lendir á einum af {\bf dálkastoppum} (e. tab stops) sem erum yfirleitt eftir hverja átta stafi.

Eins og við sjáum hér rétt á eftir þá er dálkastafurinn hentugur til að skrifa út texta sem passar í dálka.

Stafurinn {\bf newline} virkar á nákvæmlega sama hátt og {\tt endl}, þ.e. veldur því að bendillinn hoppar í næstu línu fyrir neðan.
Ég nota yfirleitt {\tt endl} ef newline stafurinn stendur einn og sér en annars nota ég \verb+\n+ ef hann er hluti strengs.

Úttak þessa forrits er:

\begin{verbatim}
1      0
2      0.693147
3      1.09861
4      1.38629
5      1.60944
6      1.79176
7      1.94591
8      2.07944
9      2.19722
\end{verbatim}
%
Ef þér finnst þessi gildi vera undarleg, mundu þá að {\tt log} fallið notar grunn $e$.
Vegna þess hversu veldi af tveimur eru mikilvæg í tölvunarfræði þá viljum við oft reikna út lógariþma með grunn 2.
Það getum við gert með því að nota eftirfarandi formúlu:

\[ \log_2 x = \frac {log_e x}{log_e 2} \]
%
Með því að breyta úttakssetningunni í 

\begin{verbatim}
      cout << x << "\t" << log(x) / log(2.0) << endl;
\end{verbatim}
%
þá fáum við:

\begin{verbatim}
1      0
2      1
3      1.58496
4      2
5      2.32193
6      2.58496
7      2.80735
8      3
9      3.16993
\end{verbatim}
%
Hér sjáum við að 1, 2, 4 og 8 eru veldi af tveimur vegna þess að lógariþminn með grunn 2 er heil tala.
Ef við vildum finna lógariþma af öðrum tölum af veldinu tveimur þá gætum við breytt forritinu á þennan hátt:

\begin{verbatim}
  double x = 1.0;
  while (x < 100.0) {
    cout << x << "\t" << log(x) / log(2.0) << endl;
    x = x * 2.0;
  }
\end{verbatim}
%
Í stað þess að bæta einhverju við {\tt x} í sérhverri ítrun lykkjunnar, og fá þannig {\bf jafnmunarunu} (e. arithmetic sequence),
þá margföldum við {\tt x} með einhverju og fáum út {\bf kvótarunu} (e. geometric sequence).
Niðurstaðan er:

\begin{verbatim}
1      0
2      1
4      2
8      3
16     4
32     5
64     6
\end{verbatim}
%
Við notum dálkastafinn á milli dálka og því er staðsetning seinni dálksins ekki háð fjölda tölustafa í fyrri dálkinum.

Það má vera að log töflur séu ekki gagnlegar nú til dags en hins vegar er mikilvægt fyrir tölvunarfræðinga að þekkja veldi af tveimur!
Prófaðu að breyta forritinu þannig að það skrifi út veldi af tveimur upp í 65536 (þ.e. $2^{16}$).
Prentaðu það út og mundu það!


\section{Tvívíðar töflur}
\index{tafla!tvívíð}

Í tvívíðri töflu (e. two-dimensional table) getur þú valið röð og dálk og lesið gildið þar sem röðin og dálkurinn mætast.
Gott dæmi um tvívíða töflu er margföldunartafla.
Segjum að þig langi til að prenta út margföldunartöflu fyrir gildin 1 til 6.

Þú gætir byrjað með því að skrifa einfalda lykkju sem prentar út margfeldi af 2, öll í sömu línu:

\begin{verbatim}
  int i = 1;
  while (i <= 6) {
    cout << 2*i << "   ";
    i = i + 1;
  }
  cout << endl;
\end{verbatim}
%
Fyrsta línan frumstillir breytuna {\tt i} en hún er hér notuð sem teljari eða {\bf lykkjubreyta} (e. loop variable).
Þegar lykkjan keyrir þá hækkar gildið á {\tt i} úr 1 í 6 og lykkjan hættir keyrslu þegar {\tt i} er 7.
Í sérhverri ítrun lykkjunnar prentum við út gildið {\tt 2*i} ásamt þremur bilum.
Við skrifum allt úttakið í einni og sömu línunni því við sleppum {\tt endl} úr fyrstu úttakssetningunni.

\index{lykkjubreyta}
\index{breyta!lykkja}

Úttak þessa forrits er:

\begin{verbatim}
2   4   6   8   10   12
\end{verbatim}
%
Næsta skref er að {\bf hjúpa} (e. encapsulate) og {\bf alhæfa} (e. generalize).

\section {Hjúpun og alhæfing}

Hjúpun (e. encapsulation) merkir yfirleitt að taka hluta af kóða og ``pakka'' (e. wrap) honum inn í fall og þannig nýta kostina sem almennt fylgja föllum.
Við höfum séð tvö dæmi um hjúpun, þ.e. þegar við skrifuðum {\tt printParity} í kafla~\ref{alternative} og {\tt
isSingleDigit} í kafla~\ref{bool}.

Alhæfing merkir að taka eitthvað einstakt tilfelli, eins og að prenta út margfeldi af tveimur, og gera það almennara, eins og að prenta út margfeldi af hvaða heiltölu sem er.

\index{hjúpun}
\index{alhæfing}

Hér er fall sem hjúpar lykkjuna að ofan og gerir hana almennari þannig að hún prenti út margeldi af {\tt n}.

\begin{verbatim}
void printMultiples (int n)
{
  int i = 1;
  while (i <= 6) {
    cout << n*i << "   ";
    i = i + 1;
  }
  cout << endl;
}
\end{verbatim}
%
Eina sem ég þurfti að gera til að hjúpa var að bæta fyrstu línunni við, þ.e. að tilgreina nafn fallsins, lepp og skilagildi.
Til að alhæfa þurfti ég aðeins að skipta út gildinu 2 fyrir leppinn {\tt n}.

Ef við köllum á þetta fall með viðfanginu 2 þá fáum við sama úttak og áður.
Ef við köllum með viðfanginu 3 fæst úttakið: 

\begin{verbatim}
3   6   9   12   15   18
\end{verbatim}
%
og með viðfanginu 4 fæst úttakið:

\begin{verbatim}
4   8   12   16   20   24 
\end{verbatim}
%
Á þessum tímapunkti ættir þú að sjá hvernig við förum að því að prenta út margföldunartöflu:
Við munum kalla endurtekið á {\tt printMultiples} með mismunandi viðfögnum.
Reyndar vill svo til að við munum nota aðra lykkjum til að ítra yfir raðirnar:

\begin{verbatim}
  int i = 1;
  while (i <= 6) {
    printMultiples (i);
    i = i + 1;
  }    
\end{verbatim}
%
Taktu eftir hversu svipuð þessi lykkja er þeirri inni í fallinu {\tt printMultiples}.
Eina sem ég gerði var að skipta út úttakssetningunni fyrir fallakall.

Úttakið úr þessu forriti er

\begin{verbatim}
1   2   3   4   5   6   
2   4   6   8   10   12   
3   6   9   12   15   18   
4   8   12   16   20   24   
5   10   15   20   25   30   
6   12   18   24   30   36   
\end{verbatim}
%
sem er (örlítið bjöguð) margföldunartafla.
Ef þessi bjögun angrar þig þá ættir þú að prófa að skipta út bilum fyrir dálkastafinn og athuga hvernig taflan lítur út eftir þá breytingu.

\section{Föll}
\index{fall}

Í síðasta kafla nefndi ég ``og þannig nýta kostina sem almennt fylgja föllum''.
Á þessum tímapunkti veltir þú kannski fyrir þér hvaða kostir þetta eru. 
Hér er nokkrar ástæður fyrir gagnsemi falla:

\begin{itemize}

\item Með því að gefa röð setninga tiltekið nafn þá gerir þú forritið þitt bæði læsilegra og auðveldara að kemba.

\item Með því að brjóta langt forrit upp í föll þá skiptir þú því í mismunandi hluta, getur unnið í einstökum hlutum þess án tillits til annarra hluta,
og síðan sett þessa hluta saman til að mynda eina heild.

\item Föll styðja bæði við endurkvæmni og ítrun.

\item Vel hönnuð föll eru oft gagnleg fyrir mörg forrit. Þegar þú hefur skrifað fall þá getur þú oft endurnýtt það í öðrum forritum.

\end{itemize}

\section{Meira um hjúpun}
\index{hjúpun}
\index{forritun!hjúpun}

Í þeim tilgangi að sýna hjúpun aftur þá ætla ég núna að taka kóðann úr síðasta kafla og pakka honum inn í fallið {\tt printMultTable}:

\begin{verbatim}
void printMultTable () {
  int i = 1;
  while (i <= 6) {
    printMultiples (i);
    i = i + 1;
  }
}
\end{verbatim}
%
Þetta ferli sem ég hef sýnt þér er algengt þróunarferli í forritun.
Þú þróar forrit smám saman með því að bæta setningu við {\tt main} eða á einhverjum öðrum stað, 
og þegar forritið virkar þá dregur þú kóðann út og pakkar honum inn í fall.

Ástæðan fyrir því að þetta er gagnlegt er að stundum veistu ekki í upphafi hvernig best er að skipta forritinu þínu upp í einstök föll.
Þessi aðferð gerir þér kleift að hanna um leið og þú skrifar forritið.

\section{Staðværar breytur}

Á þessum tímapunkti gætir þú velt því fyrir þér hvernig standi á því að hægt sé að nota sömu breytuna {\tt i}
í bæði {\tt printMultiples} og {\tt printMultTable}.
Töluðum við ekki um að eingöngu er hægt að lýsa yfir breytu einu sinni?
Og veldur það ekki vandræðum ef eitt fall breytir gildinu á breytunni?

Svarið við báðum þessum spurningum er ``nei''.
Ástæðan er sú að {\tt i} í {\tt printMultiples} og {\tt i} í {\tt printMultTable} eru {\em ekki sama breytan}.
Þær bera vissulega sama nafnið en minnissvæðið sem þær standa fyrir er ekki það sama og þess vegna hefur það ekki áhrif á aðra breytuna að breyta gildi hinnar.

\index{staðvær breyta}
\index{breyta!staðvær}

Mundu að breyta sem er lýst yfir í falli er staðvær (e. local).
Það er ekki hægt að nálgast gildi staðværrar breytu utan frá (þ.e. utan fallsins sem skilgreinir breytuna)
og því getur þú haft margar breytur með sama nafni svo framarlega sem þær eru ekki skilgreindar í sama fallinu.

Staflaritið fyrir þetta forrit sýnir glögglega að breyturnar tvær með nafnið {\tt i} standa ekki fyrir sama minnissvæði.
Gildi þeirra getur verið mismunandi og breyting á annarri breytunni hefur engin áhrif á hina breytuna.

\vspace{0.1in}
\centerline{\epsfig{figure=stack4.eps}}
\vspace{0.1in}
%
Taktu eftir að gildið á leppnum {\tt n} í {\tt printMultiples} er það sama og gildið á {\tt i} í {\tt printMultTable}.
Á hinn bóginn hleypur gildið á {\tt i} í {\tt printMultiple} úr 1 upp í {\tt n}.
Á myndinni þá vill svo til að þetta gildi er 3 en í næstu ítrun lykkjunnar verður það 4.

Til að minnka líkur á ruglingi þá er yfirleitt góð forritunarvenja að nota mismunandi nöfn á breytum í mismunandi föllum.
Aftur á móti eru líka ástæður fyrir því að nota sömu breytunöfn.
Það er t.d. algengt að nota nöfnin {\tt i}, {\tt j} og {\tt k} fyrir lykkjubreytur.
Ef þú forðast að nota þau í einu falli vegna þess að þú notar þau í öðru falli þá verður forritið þitt líklega bara ólæsilegra.

\index{lykkjubreyta}
\index{breyta!lykkja}

\section{Meira um alhæfingu}
\index{alhæfing}

Tökum annað dæmi um alhæfingu.
Gerum ráð fyrir að þú viljir skrifa forrit sem prentar út margföldunartöflu af hvaða stærð sem er, þ.e. ekki bara 6x6 töflu.
Þú gætir bætt leppi við {\tt printMultTable}:

\begin{verbatim}
void printMultTable (int high) {
  int i = 1;
  while (i <= high) {
    printMultiples (i);
    i = i + 1;
  }
}
\end{verbatim}
%
Hér hef ég skipt út gildinu 6 með leppnum {\tt high}.
Með því að kalla á {\tt printMultTable} með viðfanginu 7 þá fæst:

\begin{verbatim}
1   2   3   4   5   6   
2   4   6   8   10   12   
3   6   9   12   15   18   
4   8   12   16   20   24   
5   10   15   20   25   30   
6   12   18   24   30   36   
7   14   21   28   35   42   
\end{verbatim}
%
Þetta er ágætt en gott væri ef taflan gæti sýnt jafnmargar raðir og dálka.
Ég þarf því að bæta öðrum lepp við {\tt printMultiples} sem tilgreinir þann fjölda dálka sem ég vil sjá.

Svona til að pirra þig þá ætla ég líka að kalla þennan lepp {\tt high} og þar með sýna að mismunandi föll geta haft leppa (eins og staðværar breytur) með sama nafni:

\begin{verbatim}
void printMultiples (int n, int high) {
  int i = 1;
  while (i <= high) {
    cout << n*i << "   ";
    i = i + 1;
  }    
  cout << endl;
}

void printMultTable (int high) {
  int i = 1;
  while (i <= high) {
    printMultiples (i, high);
    i = i + 1;
  }
}

int main() {
  printMultTable(7);
}

\end{verbatim}
%
Taktu eftir því að þegar ég bætti nýjum leppi við þá þurfti ég að breyta fyrstu línunni í fallinu og einnig þeim stað þar sem kallað er á fallið í {\tt printMultTable}.
Eins og gera mátti ráð fyrir þá skrifar þetta forrit út 7x7 margföldunartöflu:

\begin{verbatim}
1   2   3   4   5   6   7   
2   4   6   8   10   12   14   
3   6   9   12   15   18   21   
4   8   12   16   20   24   28   
5   10   15   20   25   30   35   
6   12   18   24   30   36   42   
7   14   21   28   35   42   49
\end{verbatim}
%
Þegar maður gerir fall almennara þá sér maður stundum að niðurstaðan hefur eiginleika sem ekki voru fyrirséðir.
Þú tekur kannski eftir því núna að margföldunartaflan er samhverf (e. symmetric) því $ab = ba$ og því birtast öll gildi í töflunni tvisvar.
Við gætum sparað okkur blekið (að því gefnu að úttakið væri sent á prentara) með því að skrifa bara út helming töflunnar.
Við þurfum aðeins að breyta einni línu í {\tt printMultTable} til að gera það.
Breyttu

\begin{verbatim}
      printMultiples (i, high);
\end{verbatim}
%
í

\begin{verbatim}
      printMultiples (i, i);
\end{verbatim}
%
og þá færðu

\begin{verbatim}
1   
2   4   
3   6   9   
4   8   12   16   
5   10   15   20   25   
6   12   18   24   30   36   
7   14   21   28   35   42   49  
\end{verbatim}
%
Ég læt þig um að finna út hvernig þetta virkar!

\section{Orðalisti}

\begin{description}

\item[lykkja (e. loop):]  Setning sem er endurtekin á meðan að tiltekið skilyrði er satt.

\item[óendanleg lykkja (e. infinite loop):]  Lykkja, hvers skilyrði er ávallt satt.

\item[meginmál (e. body):]  Setningarnar innan í lykkju.

\item[ítrun (e. iteration):]  Ein umferð í gegnum meginmál lykkju, þ.m.t. ákvörðun á gildi skilyrðisins.

\item[dálkalykill (e. tab):] Sérstakur stafur, skrifaður sem \verb+\t+ í C++, sem gerir það að verkum að 
bendillinn færist í næsta dálkastopp í núverandi línu.

\item[hjúpa (e. encapsulate):]  Að setja tiltekna virkni inn í sérstaka einingu, t.d. fall, og einangra virknina (t.d. með því að nota staðværar breytur) frá öðrum hluta forritsins.

\item[staðvær breyta (e. local variable):]  Breyta sem er skilgreind inni í falli og hvers líftimi er aðeins innan í fallinu.
Ekki er hægt að nálgast staðværar breytur utan frá og þær hafa engin áhrif á virkni annarra falla.

\item[alhæfa (e. generalize):]  Að breyta einhverju sem er óþarflega sérstakt (eins og fast gildi) yfir í eitthvað sem er almennara (eins og breytu eða lepp)
Alhæfing gerir kóðann fjölhæfara og líklegri til að verða endurnýtanlegur.

\item[þróunarferli (e. development process):]  Ferli til að þróa forrit.
Í þessum kafla hef ég sýnt sérstaka aðferð sem byggir á því að skrifa kóða sem framkvæmir einfalda, sérstaka hluti og hef síðan beitt hjúpun og alhæfingu.

\index{lykkja}
\index{óendanleg lykkja}
\index{meginmál}
\index{dálkalykill}
\index{lykkja!óendanleg}
\index{ítrun}
\index{hjúpun}
\index{alhæfing}
\index{staðvær breyta}
\index{breyta!staðvær}
\index{þróunarferli}

\end{description}

