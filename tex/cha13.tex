% LaTeX source for textbook ``How to think like a computer scientist''
% Copyright (C) 1999  Allen B. Downey

% This LaTeX source is free software; you can redistribute it and/or
% modify it under the terms of the GNU General Public License as
% published by the Free Software Foundation (version 2).

% This LaTeX source is distributed in the hope that it will be useful,
% but WITHOUT ANY WARRANTY; without even the implied warranty of
% MERCHANTABILITY or FITNESS FOR A PARTICULAR PURPOSE.  See the GNU
% General Public License for more details.

% Compiling this LaTeX source has the effect of generating
% a device-independent representation of a textbook, which
% can be converted to other formats and printed.  All intermediate
% representations (including DVI and Postscript), and all printed
% copies of the textbook are also covered by the GNU General
% Public License.

% This distribution includes a file named COPYING that contains the text
% of the GNU General Public License.  If it is missing, you can obtain
% it from www.gnu.org or by writing to the Free Software Foundation,
% Inc., 59 Temple Place - Suite 330, Boston, MA 02111-1307, USA.

% This is an Icelandic translation/adaptation by Hrafn Loftsson of the orginal book by Allen B. Downey.

%\chapter{Objects of Vectors}
\chapter{Hlutir með vektorum}

%\section{Enumerated types}
\section{Upptalningartög}
\index{tag!upptalning}
\index{upptalningartag}
\index{vörpun}

Í kaflanum hér á undan ræddi ég um vörpun á milli raunverulegra gilda, eins og gilda eða lita tiltekinna spila, 
og innri framsetningar, eins og heiltalna og strengja.
Þó svo að við höfum annars vegar búið til vörpun á milli gilda og heiltalna og hins vegar á milli lita og heiltalna
þá benti ég á að vörpunin sjálf kemur ekki beint fram í forritskóðanum.

Reyndar vill svo til að C++ hefur eiginleika sem kallast {\bf upptalningartag} (e. enumerated type)
sem gerir kleift (1) að gera vörpun sem hluta af forritskóða og (2) að skilgreina mengi af gildum sem vörpunin samanstendur af.
Hér er t.d. skilgreiningin á upptalningartagi fyrir {\tt Suit} og {\tt Rank}:

\begin{verbatim}
enum Suit { CLUBS, DIAMONDS, HEARTS, SPADES };

enum Rank { ACE=1, TWO, THREE, FOUR, FIVE, SIX, SEVEN, EIGHT, NINE,
TEN, JACK, QUEEN, KING };
\end{verbatim}
%
Sjálfgefin vörpun er sú að fyrsta gildið í upptalningartagi varpast í heiltöluna 0, annað gildið í 1, o.s.frv.
Í {\tt Suit} taginu stendur því gildið {\tt CLUBS} fyrir heiltöluna 0, {\tt DIAMONDS} fyrir 1, o.s.frv.

Skilgreiningin á {\tt Rank} yfirskrifar sjálfgefnu vörpunina og tilgreinir að {\tt ACE} standi fyrir heiltöluna 1.
Þar með varpast {\tt TWO} í 2,  {\tt THREE} í 3, o.s.frv.

Um leið og við höfum skilgreint þessi tög getum við notað þau hvar sem er.
Meðlimabreytunum {\tt rank} og {\tt suit} getur t.d. verið lýst yfir með tögunum {\tt Rank} og {\tt Suit}:

\begin{verbatim}
struct Card
{
  Rank rank;
  Suit suit;

  Card (Suit s, Rank r);
};
\end{verbatim}
%
Taktu eftir að tag leppanna í smiðnum hefur líka breyst.
Til að ``smíða'' card hlut getum við þá notað gildi úr upptalningartögunum sem viðföng:

\begin{verbatim}
  Card card (DIAMONDS, JACK);
\end{verbatim}
%
Það er hefð fyrir því að gildi úr upptalningartögum séu nöfn með stórum stöfum.
Þessi yfirlýsing á card hlut er miklu læsilegri en sú sem við notuðum áður með heiltölum:

\begin{verbatim}
  Card card (1, 11);
\end{verbatim}
%
Við getum notað gildi upptalningartaga sem vísi inn í vektora vegna þess að við vitum 
að þessi gildi standa fyrir heiltölur.
Af þessum sökum mun gamla {\tt print} fallið okkar virka á nokkurra breytinga.
Við þurfum hins vegar að gera breytingar á {\tt buildDeck} fallinu:

\begin{verbatim}
  int index = 0;
  for (Suit suit = CLUBS; suit <= SPADES; suit = Suit(suit+1)) {
    for (Rank rank = ACE; rank <= KING; rank = Rank(rank+1)) {
      deck[index].suit = suit;
      deck[index].rank = rank;
      index++;
    }
  }
\end{verbatim}
%
Með því að nota upptalningartögin er kóðinn læsilegri en það er ein flækja.
Strangt til tekið getum við ekki beitt útreikningum á upptalningartög þannig að {\tt suit++} er ekki löglegt.
Hins vegar breytir C++ upptalningargildinu suit í segðinni {\tt suit+1} í heiltölu.
Þar með getum við tekið niðurstöðuna og notað tagmótun (e. typecast) til að ``kasta'' gildinu til baka í upptalningartag: 

\begin{verbatim}
  suit = Suit(suit+1);
  rank = Rank(rank+1);
\end{verbatim}
%
Það er reyndar til betri leið til að gera þetta með því að yfirskrifa {\tt ++} virkjann fyrir
upptalningartögin en við fjöllum ekki um það í þessari bók. 

\section{{\tt switch} setning}
\index{switch setning}
\index{setning!switch}

Það er erfitt að ræða upptalningartög án þess að nefna {\tt switch} setningar því oft haldast þær í hendur.
{\tt switch} setning er önnur leið til að setja fram keðjur af if-skilyrðissetningum og oft reyndar bæði læsilegri og skilvirkari. 
Hún lítur svona út:

\begin{verbatim}
  switch (symbol) {
  case '+':
    perform_addition ();
    break;
  case '*':
    perform_multiplication ();
    break;
  default:
    cout << "I only know how to perform addition and multiplication" << endl;
    break;
  }
\end{verbatim}
%
Þessi {\tt switch} setning er jafngild eftirfarandi keðju af if-setningum:

\begin{verbatim}
  if (symbol == '+') {
    perform_addition ();
  } else if (symbol == '*') {
    perform_multiplication ();
  } else {
    cout << "I only know how to perform addition and multiplication" << endl;
  }
\end{verbatim}
%
{\tt break} setningarnar eru nauðsynlegar í sérhverri ``grein'' í 
{\tt switch} setningu vegna þess að án {\tt break} setningar þá ``fellur'' keyrsluflæðið í næsta tilvik (e. case).
Án {\tt break} setninga myndi táknið {\tt +} verða þess valdandi að forritið myndi framkvæma
samlagningu (e. addition), síðan margföldun (e. multiplication) og að lokum prenta út villuskilaboðin.
Af og til er þessi eiginleiki reyndar gagnlegur en í flestum tilvikum koma upp villur í keyrslu þegar 
{\tt break} setningar gleymast.

\index{break setning}
\index{setning!break}

{\tt switch} setningar virka fyrir heiltölur, stafi og upptalningatög.
Til að breyta {\tt Suit} í samsvarandi streng þá getum við t.d. gert eftirfarandi: 

\begin{verbatim}
  switch (suit) {
  case CLUBS:     return "Clubs";
  case DIAMONDS:  return "Diamonds";
  case HEARTS:    return "Hearts";
  case SPADES:    return "Spades";
  default:        return "Not a valid suit";
  }
\end{verbatim}
%
Í þessu dæmi þurfum við ekki á {\tt break} setningu að halda vegna þess að 
{\tt return} setningar valda því að keyrsluflæðið fer til baka til þess sem kallaði í stað þess
að ``falla'' í næsta tilvik. 

\index{default}

Almennt séð er góður forritunarstíll að hafa sjálfgefið ({\tt default}) tilvik í sérhverri     
{\tt switch} setningu til að meðhöndla villur eða óvænt gildi. 

\section{Stokkar}
\label{deck}
\index{stokkur}
\index{vektor!af spilum}

Í köflunum hér á undan unnum við með vektor af hlutum 
en ég nefndi það líka að það væri hægt að búa til hluti 
sem innihalda vektor sem meðlimabreytu.
Í þessum kafla ætla ég að búa til nýjan hlut, {\tt Deck},
sem inniheldur vektor af {\tt Card}.

\index{meðlimabreyta}
\index{breyta!meðlimur}

Strúktúrskilgreiningin lítur svona út:

\begin{verbatim}
struct Deck {
  vector<Card> cards;

  Deck (int n);
};

Deck::Deck (int size)
{
  vector<Card> temp (size);
  cards = temp;
}
\end{verbatim}
%
Nafnið á meðlimabreytunni er {\tt cards} sem hjálpar okkur að gera
greinarmun á {\tt Deck} (stokknum) sjálfum og vektor af {\tt Card} sem stokkurinn inniheldur.


\index{smiður}

Sem stendur er aðeins um einn smið að ræða.
Sá nýtir sér staðværa breytu með nafninu {\tt temp} sem er upphafsstillt
með því að vekja upp smiðinn í {\tt vector} klasanum (stærðin, size, er send sem viðfang).
Síðan afritar smiðurinn {\tt temp} yfir í meðlimabreytuna {\tt cards}.

Nú getum við þá búið til stokk af spilum á eftirfarandi hátt: 

\begin{verbatim}
  Deck deck (52);
\end{verbatim}
%
Stöðurit fyrir {\tt Deck} hlut lítur svona út:

\index{stöðurit}
\index{smiður}

\vspace {0.1in}
\centerline{\epsfig{figure=deckobject.eps}}
\vspace {0.1in}

Hluturinn með nafnið {\tt deck} á sér eina meðlimabreytu með nafnið {\tt cards}
sem er vektor af {\tt Card} hlutum.
Til að nálgast spil í stokknum þurfum við að setja saman málskipanina fyrir að nálgast meðlimabreytu
og málskipanina fyrir að velja stak úr vektor.
Segðin {\tt deck.cards[i]} er t.d. i-ta spilið í stokknum og
{\tt deck.cards[i].suit} er litur þess.

Eftirfarandi lykkja

\begin{verbatim}
  for (int i = 0; i<52; i++) {
    deck.cards[i].print();
  }
\end{verbatim}
%
sýnir hvernig er hægt að ferðast um í stokknum og prenta út sérhvert spil.

\section {Annar smiður}
\index{smiður}

Það getur verið gagnlegt að upphafsstilla spilin í {\tt Deck} hlutnum.
Við gætum notað fallið {\tt buildDeck} úr fyrri kafla (með smávægilegum breytingum) 
en það er líklega eðlilegra að skrifa annan smið í {\tt Deck}.

\index{lykkja!hreiðruð}

\begin{verbatim}
Deck::Deck ()
{
  vector<Card> temp (52);
  cards = temp;

  int i = 0;
  for (Suit suit = CLUBS; suit <= SPADES; suit = Suit(suit+1)) {
    for (Rank rank = ACE; rank <= KING; rank = Rank(rank+1)) {
      cards[i].suit = suit;
      cards[i].rank = rank;
      i++;
    }
  }
}
\end{verbatim}
%
Taktu eftir því hversu líkt þetta fall er {\tt buildDeck} -- við þurftum aðeins að breyta málskipaninni til að gera þetta að smið.
Nú getum við búið til hefðbundinn 52-spila stokk með einfaldri yfirlýsingu: {\tt Deck deck;}

\section {Meðlimaföll í {\tt Deck}}
\index{meðlimaföll}
%\index{function!member}

Nú þegar við höfum sérstakan {\tt Deck} hlut þá er eðlilegt að gera öll föll, sem hafa með stokk að gera, að
meðlimaföllum í {\tt Deck}.
Einn augljós kandídat er fallið {\tt printDeck} (kafli~\ref{printdeck}).
Svona lítur það út eftir að því hefur verið breytt í meðlimafall í {\tt Deck}:

\index{printDeck}

\begin{verbatim}
void Deck::print () const {
  for (int i = 0; i < cards.length(); i++) {
    cards[i].print ();
  }
}
\end{verbatim}
%
Að venju getum við vísað í meðlimabreytu núverandi hlutar (t.d. {\tt cards}) án þess að nota punktatáknun (e. dot notation).

Það er hins vegar ekki augljóst fyrir sum önnur föll hvort þau ættu að vera
meðlimaföll í {\tt Card}, í {\tt Deck} eða föll sem standa eitt og sér (e. nonmember function)
og taka {\tt Card} og {\tt Deck} hluti sem viðföng.
Útgáfan af {\tt find} úr fyrri kafla tekur t.d. {\tt Card} og {\tt Deck} hluti sem viðföng
en við gætum gert fallið að meðlimafalli annars hvors hlutarins. 
Til að æfa þig skaltu núna endurskrifa {\tt find} sem meðlimafall í {\tt Deck} sem tekur
{\tt Card} hlut sem viðfang. 

Það að skrifa {\tt find} sem meðlimafall í {\tt Card} er aftur á móti dálítið snúið.
Hér er mín útgáfa: 

\begin{verbatim}
int Card::find (const Deck& deck) const {
  for (int i = 0; i < deck.cards.length(); i++) {
    if (equals (deck.cards[i], *this)) return i;
  }
  return -1;
}
\end{verbatim}
%
Fyrsta ``vandamálið'' er að við verðum að nota lykilorðið {\tt this}
til að vísa í það spil sem fallið er vakið upp með.

\index{strúktúrskilgreining}

Annað vandamálið er það að C++ gerir ekki auðvelt að skrifa strúktúrskilgreiningar
sem vísa hvor á aðra.
Vandamálið er að þegar þýðandinn les fyrri skilgreininguna þá hefur það ekki séð þá seinni.

Ein lausn er að lýsa {\tt Deck} yfir (e. declare) á undan {\tt Card} og síðan skilgreina (e. define) {\tt Deck} eftir það:

\begin{verbatim}
// declare that Deck is a structure, without defining it
struct Deck;

// that way we can refer to it in the definition of Card
struct Card
{
  int suit, rank;

  Card ();
  Card (int s, int r);

  void print () const;
  bool isGreater (const Card& c2) const;
  int find (const Deck& deck) const;
};

// and then later we provide the definition of Deck
struct Deck {
  vector<Card> cards;

  Deck ();
  Deck (int n);
  void print () const;
  int find (const Card& card) const;
};
\end{verbatim}


\section{Að stokka}
\label{shuffle}
\index{stokka}

Í flestum spilum þarf að vera hægt að stokka spilin, þ.e. setja þau í handahófskennda röð í stokknum.
Við sáum í kafla~\ref{random} hvernig hægt er að búa til slembitölur en það er ekki augljóst hvernig hægt er 
að nota þær til að stokka spil.

Ein leið er sú að líkja eftir því hvernig við sjálf stokkum, t.d. að skipta stokknum í tvennt
og setja hann svo saman á ný með því að velja eitt spil í einu frá hvorum helmingnum.
Þetta er kallað fullkomin stokkun (e. perfect shuffle).
Það getur tekið okkur um 7 ítranir þangað til spilin eru vel stokkuð með þessari aðferð.
Forrit hefur hins vegar þann pirrandi eiginleika að stokka fullkomlega í sérhvert sinn sem er í raun ekki mjög handahófskennt!
Það vill reyndar svo til að eftir 8 fullkomnar stokkanir þá er stokkurinn í sömu röð og þegar byrjað var.
Þú getur séð umfjöllun um þessa fullyrðingu á {\tt http://www.wiskit.com/marilyn/craig.html} eða leitað á vefnum
með leitarorðunum ``perfect shuffle.''

Betra stokkunaralgrím er að ferðast um í stokknum, spil fyrir spil, og skipta á núverandi spili og öðru spili (sem valið er af handahófi) í sérhverri ítrun. 

\index{sauðakóði}

Hér má sjá drög af því hvernig þetta algrím virkar.
Til að setja fram forritið nota ég sambland af C++ setningum og ensku -- þetta er stundum kallað 
called {\bf sauðakóði} (e. pseudocode):

\begin{verbatim}
  for (int i=0; i<cards.length(); i++) {
    // choose a random number between i and cards.length()
    // swap the ith card and the randomly-chosen card
  }
\end{verbatim}
%
Kosturinn við að nota sauðakóða er oft sá að hann gefur skýrt til kynna hvaða föll eru nauðsynleg.
Í þessu tilviki þurfum við fall eins og {\tt randomInt} sem velur heiltölu, á handahófskenndan hátt, á milli
viðfanganna {\tt low} og {\tt high}, og fall eins og {\tt swapCards} sem tekur tvo vísa og skiptir á spilunum í viðkomandi sætum.

\index{slembitölur}

Þú ættir að geta fundið út hvernig skrifa á {\tt randomInt} með því að kíkja til baka í kafla~\ref{random}.
Þú þarft þó að passa þig á því að búa ekki til vísa sem eru út fyrir bilið (e. out of range). 

%\index{swapCards}
%\index{reference}

Þú ættir líka að geta skrifað {\tt swapCards} sjálf(ur). 

%I will leave the remaining implementation of these functions as an exercise to the reader.

\section{Röðun}
\label{sorting}
\index{röðun}

Nú þegar við getum stokkað spilin þurfum við líka að geta raðað þeim í rétta röð.
Það vill reyndar svo til að algrímið fyrir röðun er mjög svipað stokkunaralgríminu.

Við munum aftur ferðast um í spilastokknum og skipta á núverandi spili og öðru spili í sérhverri ítrun.
Eini munurinn er sá að í stað þess að velja annað spil af handahófi þá munum við finna lægsta spilið sem eftir er í stokknum.
Það sem ég á við með ``eftir er í stokknum'' eru spil sem eru jöfn eða hærri heldur er núverandi vísir {\tt i}.

\begin{verbatim}
  for (int i=0; i<cards.length(); i++) {
    // find the lowest card at or to the right of i
    // swap the ith card and the lowest card
  }
\end{verbatim}
%
Sauðakóðinn hjálpar hér aftur við hönnunina á {\bf hjálparföllum}.
Í þessu tilviki getum við aftur notað {\tt swapCards} fallið og við þurfum því einungis eitt nýtt fall, {\tt findLowestCard},
sem tekur vektor af spilum og vísi sem gefur til kynna hvar byrja á að leita í vektornum.

Þetta ferli að nota sauðakóða til að finna út hvaða hjálparföll eru nauðsynleg eru stundum kallað 
{\bf ofansækin hönnun} (e. top-down design), sem er andstaðan við {\bf neðansækna hönnun} (e. bottom-up design) sem ég ræddi um í kafla~\ref{counting}.

\index{ofansækin hönnun}
\index{hönnun!ofansækin}
\index{neðansækin hönnun}
\index{hönnun!bottom-up}
\index{hjálparfall}
%\index{fall!helper}

Þú ættir núna að spreyta þig á útfærslunni á þessum sauðakóða.

\section {Hlutstokkur}
\index{hlutstokkur}

Hvernig ættum við að tákna hönd eða annað hlutmengi af spilastokki (hlutstokk)?
Ein einföld leið er að búa til {\tt Deck} hlut sem hefur færri en 52 spil. 

Við gætum skrifað fall, {\tt subdeck}, sem tekur vektor af spilum og vísa (sem tákna bil)
og skilar nýjum vektor af spilum sem innheldur viðkomandi hlutmengi úr stokknum:

\begin{verbatim}
Deck Deck::subdeck (int low, int high) const {
  Deck sub (high-low+1);
	
  for (int i = 0; i<sub.cards.length(); i++) {
    sub.cards[i] = cards[low+i];
  }
  return sub;
}
\end{verbatim}
%
Til að búa til staðværu breytuna {\tt subdeck} notum við hér smiðinn í {\tt Deck} sem tekur stærðina á nýja stokknum (hlutstokknum)
sem viðfang og upphafsstillir ekki spilin í stokknum.
Spilin eru í raun upphafsstillt þegar þau eru afrituð úr upphaflega stokknum.

Stærðin á hlutstokknum er {\tt high-low+1} vegna þess að spilin í bæði {\tt low} og {\tt high} eru innifalin.
Svona útreikningur getur verið ruglandi og hefur oft í för með sér ``off-by-one'' villur.
Það er oft gott að teikna mynd til að koma í veg fyrir þessar villur.

\index{smiður}
%\index{overloading}

%As an exercise, write a version of {\tt findBisect} that takes a
%subdeck as an argument, rather than a deck and an index range.  Which
%version is more error-prone?  Which version do you think is more
%efficient?

\section{Að stokka og gefa}
\index{stokka}
\index{gefa}

Í kafla~\ref{shuffle} skrifaði ég sauðakóða fyrir stokkunaralgrímið.
Ef við gerum nú ráð fyrir því að eiga fallið {\tt shuffle}, sem stokkar viðfang sitt, þá getum við búið til og stokkað spilastokk: 

\begin{verbatim}
  Deck deck;               // create a standard 52-card deck
  deck.shuffle ();         // shuffle it
\end{verbatim}
%
Síðan getum við gefið spil með því að nota {\tt subdeck}:

\begin{verbatim}
  Deck hand1 = deck.subdeck (0, 4);
  Deck hand2 = deck.subdeck (5, 9);
  Deck pack = deck.subdeck (10, 51);
\end{verbatim}
%
Þessi kóði gefur fyrstu 5 spilin til einnar handar, næstu 5 spil til annarrar handar og afgangurinn fer í bunkann.

Hvarflaði að þér að við ættum að gefa eitt spil í einu til sérhverrar handar eins og algengt er í raunverulegum spilum?
Ég hugsaði um það en áttaði mig svo á því að það væri ekki nauðsynlegt í forriti.
Það að gefa eitt spil í einu er yfirleitt gert til að leiðrétta ófullkomna stokkun og gera það að verkum að erfiðara sé fyrir þann sem gefur að svindla.
Í tilviki tölvu þarf ekki að hugsa um þessi atriði. 

%This example is a useful reminder of one of the dangers of engineering
%metaphors: sometimes we impose restrictions on computers that are
%unnecessary, or expect capabilities that are lacking, because we
%unthinkingly extend a metaphor past its breaking point.  Beware of
%misleading analogies.


\section {Mergesort}
\index{skilvirkni}
\index{röðun}
\index{mergesort}

Í kafla~\ref{sorting} sáum við einfalt röðunaralgrím sem vill svo til að er ekki mjög skilvirkt.
Til að raða $n$ hlutum (spilum) þarf algrímið að ferðast um í vektornum $n$ sinnum (for-lykkjan)
og sérhver umferð (finna lægsta spil) tekur tíma sem er í réttu hlutfalli við $n$.
Heildartíminn sem algrímið tekur er því í réttu hlutfalli við $n^2$.

Í þessum kafla mun ég setja fram mun skilvirkara röðunaralgrím sem kallast {\bf mergesort}.
Tíminn sem mergesort tekur að raða $n$ hlutum er í réttu hlutfalli við $n \log n$.
Þetta virðist ekki vera mikil bæting en eftir því sem $n$ stækkar því meiri munur verður á $n^2$ og $n \log n$.
Prófaðu nokkur gildi fyrir $n$ og sjáðu muninn. 

Grundvallarhugmyndin á bak við mergesort er eftirfarandi:
Ef þú hefur tvo hlutstokka, sem hvor um sig er þegar raðaður, þá er einfalt (og skilvirkt) að setja þá saman (e. merge) í einn raðaðan stokk.
Prófaðu þetta með raunverulegum spilastokki:

\begin{enumerate}

\item Búðu til tvo hlutstokka með um 10 spilum í hvorum fyrir sig og raðaðu þeim þannig að þegar spilin snúa upp þá er lægsta spilið efst.
Settu báða stokkana fyrir framan þig (þannig að spilin snúi upp). 

\item Berðu saman efstu spilin í hvorum stokki og veldu það lægra. 
Snúðu því við og bættu því við nýja samsetta stokkinn (sem upphaflega er tómur). 

\item Endurtaktu skref tvö þangað til annar stokkurinn er tómur. 
Bættu þá restinni af spilunum við samsetta stokkinn. 

\end{enumerate}

Niðurstaðan ætti að vera einn raðaður stokkur.
Svona lítur þetta út í sauðakóða:

\begin{verbatim}
  Deck merge (const Deck& d1, const Deck& d2) {
    // create a new deck big enough for all the cards
    Deck result (d1.cards.length() + d2.cards.length());

    // use the index i to keep track of where we are in
    // the first deck, and the index j for the second deck
    int i = 0;
    int j = 0;
		
    // the index k traverses the result deck
    for (int k = 0; k<result.cards.length(); k++) {
			
      // if d1 is empty, d2 wins; if d2 is empty, d1 wins;
      // otherwise, compare the two cards
			
      // add the winner to the new deck
    }
    return result;
  }
\end{verbatim}
%
Ég ákvað að gera {\tt merge} að falli sem stendur eitt og sér (e. nonmember function) vegna þess að viðföngin tvö eru samhverf (e. symmetric). 

Besta leiðin til að prófa {\tt merge} er að búa til stokk, nota subdeck til að búa til tvær (litlar) hendur
og nota síðan röðunaralgrímið úr síðasta kafla til að raða höndunum tveimur.
Eftir það getur þú sent hendurnar tvær (hlutstokkana) inn í {\tt merge} til að sjá hvort það virkar. 

\index{prófanir}

Ef þú færð þetta til að virka ættir þú að prófa eftirfarandi útfærslu af {\tt mergeSort}:

\begin{verbatim}
Deck Deck::mergeSort () const {
  // find the midpoint of the deck
  // divide the deck into two subdecks
  // sort the subdecks using sort
  // merge the two halves and return the result
}
\end{verbatim}
%
Taktu eftir því að núverandi hlutur er skilgreindur sem {\tt const} vegna þess að 
{\tt mergeSort} breytir honum ekki. Í staðinn býr fallið til nýjan {\tt Deck} hlut og skilar honum. 

Fjörið byrjar fyrst núna ef þetta gengur hjá þér! 
Það sem er töfrandi við mergesort er að fallið er endurkvæmt.
Af hverju ættir þú að kalla á gömlu, óskilvirku útgáfunum af {\tt sort} þegar þú ætlar að raða hlutstokkunum?
Af hverju ekki að kalla á {\tt mergeSort} sem þú ert einmitt að skrifa? 

\index{endurkvæmni}

Það er ekki einungis góð hugmynd heldur er það {\em nauðsynlegt} til að algrímið í heild sinni verði skilvirkara eins og ég lofaði.
Til að fá það til að virka verðum við þó að bæta við grunnþrepi þannig að ekki verði um óendanlega endurkvæmni að ræða.
Einfalt grunnþrep er hlutstokkur með 0 eða einu spili.
Ef {\tt mergesort} fær svona lítinn hlutstokk þá getur fallið einfaldlega skilað stokknum óbreyttum til baka því hann er þá sannarlega þegar raðaður.

Endurkvæma útgáfan af {\tt mergesort} lítur einhvern veginn svona út: 

\begin{verbatim}
Deck Deck::mergeSort (Deck deck) const {
  // if the deck is 0 or 1 cards, return it

  // find the midpoint of the deck
  // divide the deck into two subdecks
  // sort the subdecks using mergesort
  // merge the two halves and return the result
}
\end{verbatim}
%
Að venju eru tvær leiðir til að hugsa um endurkvæm föll:
Hægt er fylgja keyrsluflæði þess í heild sinni eða ``taka það trúanlegt'' (e. make the leap of faith).
Ég setti þetta dæmi fram vísvitandi til að hvetja þig til að ``taka það trúanlegt''.

\index{taka trúanlegt}

Þegar þú notaðir {\tt sort} til að raða hlutstokkum þá fannst þér ekki ástæða til að fylgja keyrsluflæði þess í heild sinni, ekki satt?
Þú gerðir einfaldlega ráð fyrir því að {\tt sort} fallið virkaði vegna þess að þú hafðir áður aflúsað það. 
Eina sem þú þurftir að breyta til að gera {\tt mergeSort} endurkvæmt var að skipta út einu röðunaralgrími fyrir annað. 
Það er engin ástæða til að líta á nýja forritið á annan hátt. 

Reyndar þurftir þú að hugsa um að ná grunnþrepinu réttu og að það myndi á einhverjum tímapunkti reyna á grunnþrepið en að öðru leyti ætti ekki að vera neitt sérstakt vandamál að skrifa endurkvæmu lausnina.  Gangi þér vel!

\section{Orðalisti}

\begin{description}

\item[sauðakóði (e. pseudocode):]  Leið til að hanna forrit með því að skrifa gróf drög í blöndu af náttúrulegu tungumáli og C++.

\item[hjálparfall (e. helper function):]  Oftast lítið fall, sem eitt og sér gerir ekki neitt sérlega gagnlegt, en hjálpar öðru gagnlegra falli.

%\item[bottom-up design:]  A method of program development that
%uses pseudocode to sketch solutions to large problems and
%design the interfaces of helper functions.

\item[mergesort:]  Algrím til að raða safni af gildum.
Mergesort er skilvirkara en hið einfalda röðunaralgrím í fyrri kafla, sérstaklega fyrir stór söfn. 

\index{sauðakóði}
\index{hjálparfall}
%\index{bottom-up design}
%\index{program development!bottom-up}
%\index{function!helper}
\index{mergesort}


\end{description}

