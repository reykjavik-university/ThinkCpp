% LaTeX source for textbook ``How to think like a computer scientist''
% Copyright (C) 1999  Allen B. Downey

% This LaTeX source is free software; you can redistribute it and/or
% modify it under the terms of the GNU General Public License as
% published by the Free Software Foundation (version 2).

% This LaTeX source is distributed in the hope that it will be useful,
% but WITHOUT ANY WARRANTY; without even the implied warranty of
% MERCHANTABILITY or FITNESS FOR A PARTICULAR PURPOSE.  See the GNU
% General Public License for more details.

% Compiling this LaTeX source has the effect of generating
% a device-independent representation of a textbook, which
% can be converted to other formats and printed.  All intermediate
% representations (including DVI and Postscript), and all printed
% copies of the textbook are also covered by the GNU General
% Public License.

% This distribution includes a file named COPYING that contains the text
% of the GNU General Public License.  If it is missing, you can obtain
% it from www.gnu.org or by writing to the Free Software Foundation,
% Inc., 59 Temple Place - Suite 330, Boston, MA 02111-1307, USA.

% This is an Icelandic translation/adaptation by Hrafn Loftsson of the orginal book by Allen B. Downey.

\chapter{Strúktúrar}
\label{structs}
\index{struct}

%\section{Compound values}
\section{Samsett gildi}

Flest af þeim gagnatögum sem við höfum unnið með hingað til standa fyrir eitt tiltekið
gildi -- heiltölu, kommutölu og boole gildi.
Strengir eru öðruvísi að því leyti til að þeir eru settir saman úr smærri gildum, þ.e. stöfum.
Strengir eru því dæmi um {\bf samsett} gildi (e. compound type).

Við gætum þurft að meðhöndla samsett gildi sem einn tiltekinn hlut og við gætum þurft að komast í einstaka hluta samsetta gildisins.
%This ambiguity is useful.

Það er einnig gagnlegt fyrir þig að geta búið til þín eigin samsettu gildi.
C++ býður upp á tvær leiðir til að gera það: {\bf strúktúra} (e. structures) og {\bf klasa} (e. classes).
Við munum byrja á því að fjalla um strúktúra og förum síðan í klasa í kafla~\ref{class} (það er ekki mikill munur á milli þeirra).
%Í því sem á eftir kemur mun ég oft nota hugtakið \emph{hlutur} í stað hugtaksins strúktúr. 

\section{{\tt Point} hlutir}
\index{Point}
\index{struct!Point}

Við skulum skoða stærðfræðilegan punkt (eins og í hnitakerfi) sem dæmi um samsett gildi.
Punktur er í raun tvær tölur (hnit) sem við meðhöndlum sem einn tiltekinn hlut.
Í stærðfræði eru punktar oft skrifaðir innan sviga með kommu á milli hnitanna.
T.d. gefur $(0, 0)$ til kynna upphafspunkt (hnitamiðju) og $(x, y)$ stendur fyrir punkt sem er $x$ einingar til hægri og $y$ einingar upp miðað við upphafspunktinn.

Eðlileg leið til að tákna punkt í C++ er að nota tvær kommutölur, {\tt double}.
Spurningin er hins vegar hvernig hægt er að setja þessi tvö gildi saman í samsettan hlut eða strúktúr.
Það er hægt með því að nota {\tt struct} skilgreiningu:

\begin{verbatim}
struct Point {
  double x, y;
};  
\end{verbatim}
%
{\tt struct} skilgreining er yfirleitt sett fram utan fallaskilgreininga, í upphafi forrits (á eftir {\tt include} setningum).

Þessi skilgreining gefur til kynna að í strúktúrnum eru tvö stök (gildi), nefnd {\tt x} og {\tt y}.
Þessi stök eru kölluð {\bf tilvikabreytur} (e. instance variables), en ég mun skýra síðar hver ástæðan er fyrir því.

Það er algeng villa að gleyma semíkommunni í enda strúktúrskilgreiningar.
Það virðist skrýtið að setja semíkommu á eftir slaufusviga en þú venst því fljótt.

Þegar þú hefur skilgreint nýjan strúktúr þá getur þú búið til breytur af því tagi:

\begin{verbatim}
  Point blank;
  blank.x = 3.0;
  blank.y = 4.0;   
\end{verbatim}
%
Fyrsta línan er hefðbundin yfirlýsing á breytu: {\tt blank} er af taginu {\tt Point}.
Næstu tvær línur frumstilla tilvikabreytur strúktúrsins.
Hér er punktatáknun notuð á svipaðan hátt og þegar kallað er á fall sem tilheyrir tilteknum hlut, eins og í {\tt fruit.length()}.
Munurinn er auðvitað sá að fallakalli fylgir viðfangalisti, jafnvel þó hann sé tómur.

\index{yfirlýsing}
\index{setning!yfirlýsing}
\index{tilvísun}
\index{stöðurit}
\index{staða}

Niðurstaðan af þessum gildisveitingum sést í eftirfarandi stöðuriti:

\vspace{0.1in}
\centerline{\epsfig{figure=point.eps}}
\vspace{0.1in}

Að venju birtist nafn breytunnar {\tt blank} utan kassans en gildi hennar innan hans.
Í þessu tilviki er gildið samsettur hlutur með tveimur tilvikabreytum.

\section{Aðgangur að tilvikabreytum}
\index{struct!tilvikabreytur}

Þú getur lesið gildi tilvikabreytu með því að nota sömu málskipan og við notuðum til að gefa henni gildi:

\begin{verbatim}
    int x = blank.x;
\end{verbatim}
%
Segðin {\tt blank.x} merkir ``farðu í hlutinn með nafninu {\tt blank} og náðu í gildið á {\tt x}.''
Í þessu tilviki gefum við staðværri breytu með nafninu {\tt x} það gildi.
Taktu eftir því að það er enginn ``árekstur'' á milli staðværu breytunnar {\tt x} og tilvikabreytunnar {\tt x}.
Tilgangur punktatáknunar er einmitt sá að gefa til kynna hvaða breytu verið er að vísa í án þess að einhver margræðni sé til staðar.

Hægt er að nota punktatáknun sem hluta af hvaða C++ segð sem er, þannig að eftirfarandi er t.d. löglegt:

\begin{verbatim}
  cout << blank.x << ", " << blank.y << endl;
  double distance = blank.x * blank.x + blank.y * blank.y;
\end{verbatim}
%
Fyrri línan skrifar út {\tt 3, 4} og seinni línan reiknar út gildið 25.

\section{Aðgerðir á strúktúrum}
\index{struct!aðgerðir}

Flestum af þeim aðgerðum sem við höfum notað á önnur tög, eins og stærðfræðivirkjarnir (e. mathematical operators) ( {\tt +}, {\tt \%}, o.s.frv.)
og samanburðarvirkjarnir (e. comparison operators) ({\tt ==}, {\tt >}, o.s.frv.) er ekki hægt að beita á strúktúra.
Reyndar er hægt að breyta merkingu þessara virkja fyrir ný tög en við munum ekki fjalla um það í þessari bók.

Gildisveitingarvirkjanum (e. assignment operator) er aftur á móti hægt að beita á strúktúra.
Hægt er að nota hann á tvo vegu: til að frumstilla tilvikabreytu strúktúrs eða til að afrita gildi tilvikabreytu úr einum strúktúr í annan.
Frumstilling lítur svona út:

\begin{verbatim}
  Point blank = { 3.0, 4.0 };
\end{verbatim}
%
Tilvikabreytur breytunnar {\tt blank} fá hér gildin úr slaufusviganum, í þeirri röð sem þau eru sett fram.
Þannig að hér fær {\tt blank.x} fyrsta gildið (3.0) og {\tt blank.y} annað gildið (4.0). 

Því miður er eingöngu hægt að nota þessa málskipan í frumstillingu en ekki í gildisveitingarsetningu.
Eftirfarandi er því ekki löglegt:

\begin{verbatim}
  Point blank;
  blank = { 3.0, 4.0 };       // WRONG !!
\end{verbatim}
%
Það er eðlilegt að velta því fyrir sér hver ástæðan er fyrir því að jafn eðlileg setning sé óleyfileg.
Ég er ekki alveg viss en vandamálið gæti verið það að þýðandinn veit ekki hvert tagið á hægri hliðinni er.
Ef þú bætir við tagmótun (e. typecast) þá er allt í fína lagi:

\begin{verbatim}
  Point blank;
  blank = (Point){ 3.0, 4.0 };
\end{verbatim}
%

Það er jafnframt leyfilegt að gefa einum strúktúr gildi annars strúkturs. Dæmi: 

\begin{verbatim}
  Point p1 = { 3.0, 4.0 };
  Point p2 = p1;
  cout << p2.x << ", " <<  p2.y << endl;
\end{verbatim}
%
Úttakið úr þessu forriti er {\tt 3, 4}.

\section{Strúktúrar sem viðföng}
\index{viðfang}
\index{struct!viðfang}

Þú getur haft strúktúr sem lepp í falli, t.d.:

\begin{verbatim}
void printPoint (Point p) {
  cout << "(" << p.x << ", " << p.y << ")" << endl;
}
\end{verbatim}
%
{\tt printPoint} tekur punkt sem viðfang og skrifar gildi hans út á staðlaðan hátt.
Ef þú kallar á fallið með {\tt printPoint (blank)} þá skrifast út {\tt (3, 4)}.

Tökum annað dæmi.  Við getum endurskrifað {\tt distance} fallið úr kafla~\ref{distance} þannig að það taki tvo punkta sem viðföng í stað fjögurrra kommutalna:

\begin{verbatim}
double distance (Point p1, Point p2) {
  double dx = p2.x - p1.x;
  double dy = p2.y - p1.y;
  return sqrt (dx*dx + dy*dy);
}
\end{verbatim}

\section{Kallað með gildi}
\index{stikun færibreytna}
\index{kall!með gildi}

Það er mikilvægt að gera sér grein fyrir því að þegar strúktúr er sendur sem viðfang í fall þá er viðfangið (e. argument/actual parameter) og leppurinn (e. formal parameter) ekki sama breytan.
Um er að ræða tvær breytur (önnur í þeim sem kallar (e. caller) og hin í þeim sem kallað er á (e. callee)) sem hafa sama gildið, a.m.k. í upphafi.
Þegar við t.d. köllum á {\tt printPoint} þá lítur stöðuritið svona út: 

\vspace{0.1in}
\centerline{\epsfig{figure=point2.eps}}
\vspace{0.1in}
%
Ef {\tt printPoint} breytir annarri (eða báðum) tilvikabreytum {\tt p} þá mun það ekki hafa nein áhrif á {\tt blank} (auðvitað er engin ástæða fyrir {\tt printPoint} að breyta leppnum sínum).

Þessi tegund af {\bf stikun færibreytna} (e. parameter passing) er kölluð ``kall með gildi'' (e. ``pass by value'')
vegna þess að gildi (e. value) strúktúrsins (eða hvaða tags sem er) er sent til fallsins og afritað yfir í leppinn.

\section{Kallað með tilvísun}
\index{stikun færibreytna}
\index{kall!með tilvísun}
\index{tilvísun}

Önnur aðferð við stikun færibreytna í C++ er ``kall með tilvísun'' (e. ``pass by reference'').
Þessi aðferð gerir þér kleift að senda strúktúr í fall og breyta gildum hans!

Þú gætir t.d. speglað punkti um 45-gráðu línuna með því að skipta á gildum hnitanna tveggja.
Augljósasta leiðin (en ekki sú rétta) er að útfæra {\tt reflect} fallið á þennan hátt:

\begin{verbatim}
void reflect (Point p)      // WRONG !!
{
  double temp = p.x;
  p.x = p.y;
  p.y = temp;
}
\end{verbatim}
%
Þetta virkar ekki vegna þess að það að gera breytingar á leppnum í {\tt reflect} hefur engin áhrif á þann sem kallaði.

Í staðinn verðum við að tilgreina að við ætlum að senda viðfangið með tilvísun (e. by reference).
Það gerum við með því að bæta tákninu {\tt \&} við í skilgreiningu á leppunum: 

\begin{verbatim}
void reflect (Point& p)
{
  double temp = p.x;
  p.x = p.y;
  p.y = temp;
}
\end{verbatim}
%
Núna getum við kallað á fallið á hefðbundin hátt:

\begin{verbatim}
  printPoint (blank);
  reflect (blank);
  printPoint (blank);
\end{verbatim}
%
Úttak forritsins er eins og við gerum ráð fyrir: 

\begin{verbatim}
(3, 4)
(4, 3)
\end{verbatim}
%
Svona lítur stöðurit út fyrir þetta forrit:

\vspace{0.1in}
\centerline{\epsfig{figure=point3.eps}}
\vspace{0.1in}
%
Leppurinn {\tt p} er tilvísun í strúktúr með nafnið {\tt blank}.
Hefðbundin táknun fyrir tilvísun er punktur með ör sem bendir á það sem tilvísunin vísar á.

Það mikilvæga sem hægt er að lesa úr þessu stöðuriti er að allar breytingar sem 
{\tt reflect} gerir á {\tt p} munu einnig hafa áhrif á {\tt blank}.

Það að stika strúktúr með tilvísun (e. by reference) er sveigjanlegra heldur en stika hann með gildi (e. by value)
vegna þess að sá sem kallað er á getur breytt strúktúrnum.
Það er jafnframt hraðvirkara vegna þess að kerfið þarf ekki að afrita heilan strúktúr.
Á hinn bóginn má segja að það sé ekki eins öruggt vegna þess að það er erfiðara að gera sér grein fyrir hvaða breytingar eru gerðar hvar.
Samt sem áður eru strúktúrar yfirleitt stikaðir með tilvísun í C++ forritum og ég mun fylgja þeirri venju í þessari bók.

\section{Rétthyrningar}
\index{Rétthyrningur}
\index{struct!rétthyrningur}

Gerum nú ráð fyrir að við viljum búa til strúktúr sem stendur fyrir rétthyrning.
Hvaða upplýsingar þurfum við að veita til þess að skilgreina rétthyrning?
Til að einfalda málið skulum við gera ráð fyrir því að rétthyrningnum sé stillt upp lárétt eða lóðrétt en ekki miðað við tiltekið horn (gráður).

Það eru nokkrir möguleikar.
Ég gæti tilgreint miðju rétthyrnings (tvö hnit) og stærð hans (breidd og hæð), eða tilgreint eitt af hornum hans ásamt stærðinni, eða tilgreint hnit tveggja andstæðra horna.

Algeng leið er að tilgreina efra vinstra horn rétthyrningsins ásamt stærðinni.
Til að gera það í C++ þá skilgreinum við strúktúr sem inniheldur {\tt Point} og tvær kommutölur.

\begin{verbatim}
struct Rectangle {
  Point corner;
  double width, height;
};  
\end{verbatim}
%
Taktu eftir að einn strúktúr getur innifalið annan og það er reyndar mjög algengt.
Það þýðir auðvitað að til að búa til {\tt Rectangle} þá þurfum við fyrst að búa til {\tt Point}:

\begin{verbatim}
  Point corner = { 0.0, 0.0 };
  Rectangle box = { corner, 100.0, 200.0 };
\end{verbatim}
%
Þessi kóði býr til nýjan {\tt Rectangle} strúktúr og frumstillir tilvikabreytur hans.
Myndin sýnir áhrif þess.

\vspace{0.1in}
\centerline{\epsfig{figure=rectangle.eps}}
\vspace{0.1in}
%
Við getum nálgast {\tt width} og {\tt height} á venjulegan máta:

\begin{verbatim}
  box.width += 50.0;
  cout << box.height << endl;
\end{verbatim}
%
Til að nálgast tilvikabreytur {\tt corner} getum við notað tímabundna breytu:

\begin{verbatim}
  Point temp = box.corner;
  double x = temp.x;
\end{verbatim}
%
En við gætum líka sett þessar tvær setningar saman í eina:

\index{samsetning}

\begin{verbatim}
  double x = box.corner.x;
\end{verbatim}
%
Það er eðlilegast að lesa þessa setningu frá hægri til vinstri: 
``Sæktu {\tt x} úr {\tt corner} úr {\tt box} og láttu staðværu breytuna {\tt x} fá það gildi.''

Fyrst við erum að tala um samsetningu setninga þá bendi ég jafnframt á að þú getur reyndar búið til {\tt Point} og {\tt Rectangle} á sama tíma:

\begin{verbatim}
  Rectangle box = { { 0.0, 0.0 }, 100.0, 200.0 };
\end{verbatim}
%
Innri slaufusviginn tilgreinir hnit hornsins sem stendur þá fyrir fyrsta gildið af þremur sem þarf fyrir {\tt Rectangle}.
Þessi setning er dæmi um {\bf hreiðraðan strúktúr} (e. nested structure).

\index{hreiðraður strúktúr}

\section{Strúktúrar sem skilagildi}
\index{struct!skilagildi}
\index{skilagildi}
\index{setning!skilagildi}

Þú getur skrifað föll sem skila strúktúrum.
Eftirfarandi fall {\tt findCenter} tekur {\tt Rectangle} sem viðfang og skilar {\tt Point} sem inniheldur hnit miðjunnar á {\tt Rectangle}:

\begin{verbatim}
Point findCenter (Rectangle& box)
{
  double x = box.corner.x + box.width/2;
  double y = box.corner.y + box.height/2;
  Point result = {x, y};
  return result;
}
\end{verbatim}
%
Til að kalla á þetta fall verðum við að senda box sem viðfang (taktu eftir að það er sent með tilvísun) og setjum skilagildið inn í {\tt Point} breytu:

\begin{verbatim}
  Rectangle box = { {0.0, 0.0}, 100, 200 };
  Point center = findCenter (box);
  printPoint (center);
\end{verbatim}
%
Úttakið úr þessu forriti er {\tt (50, 100)}.

\section {Að senda önnur tög með tilvísun}
\index{stikun færibreytna}
\index{kall með tilvísun}
\index{tilvísun}

Það eru ekki eingöngu strúktúrar sem hægt er að senda með tilvísun í föll.
Það sama gildir um öll önnur tög sem við höfum séð.
Til að skipta á gildum tveggja heiltölubreytna getum við t.d. gert eftirfarandi: 

\begin{verbatim}
void swap (int& x, int& y)
{
  int temp = x;
  x = y;
  y = temp;
}
\end{verbatim}
%
Við myndum kalla á þetta fall á hefðbundin hátt:

\begin{verbatim}
  int i = 7;
  int j = 9;
  swap (i, j);
  cout << i << j << endl;
\end{verbatim}
%
Úttakið úr þessu forriti er {\tt 97}.
Teiknaðu stöðurit fyrir þetta forrit til að fullvissa þig um að þetta sé rétt.
Ef lepparnir {\tt x} og {\tt y} væru skilgreindir sem venjulegir leppar (þ.e. án {\tt \&}), þá myndi {\tt swap} ekki virka rétt.
Í því tilviki myndi það breyta {\tt x} og {\tt y} án þess að hafa nokkur áhrif á {\tt i} og {\tt j}.

Þegar fólk byrjar að senda hluti eins og heiltölur með tilvísun þá reynir það stundum að nota segðir sem tilvísunarviðföng.
Dæmi:

\begin{verbatim}
  int i = 7;
  int j = 9;
  swap (i, j+1);         // WRONG!!
\end{verbatim}
%
Þetta er ekki leyfilegt vegna þess að segðin {\tt j+1} er ekki breyta -- hún stendur ekki fyrir minnishólf sem hægt er að vísa til.
Það getur verið svolítið erfitt að átta sig á því hvers konar segðir er hægt að senda með tilvísun en það er ágætis þumalputtaregla að segja að tilvísunarviðföng þurfa að vera breytur.

\section{Að sækja gögn af lyklaborði}
\label{input}
\index{inntak!lyklaborð}

Forritin sem við höfum skrifað hingað til eru nokkuð fyrirsjáanleg, þ.e. þau gera nákvæmlega það sama í hvert skipti sem þau eru keyrð.
Í flestum tilvikum viljum við, aftur á móti, forrit sem lesa inntak frá notanda og bregðast við því.

Það eru margar leiðir til að sækja gögn, þ.m.t. lyklaborðsinntak, músarhreyfingar og músarsmellir, svo og framandi aðferðir eins og talgreining og sjónhimnuskönnun.
Í þessari bók munum við aðeins sækja gögn af lyklaborði.

\index{straumur}
\index{cin}
\index{cout}

Í hausaskránni {\tt iostream} er hluturinn {\tt cin} skilgreindur en hann meðhöndlar inntak á svipaðan hátt og {\tt cout} meðhöndlar úttak.
Við getum gert eftirfarandi til að sækja heiltölugildi frá notanda:

\begin{verbatim}
  int x;
  cin >> x;
\end{verbatim}
%
{\tt >>} virkinn veldur því að forritið stoppar keyrslu og bíður eftir að notandinn slái eitthvað inn af lyklaborði.
Ef notandi slær inn tölustaf þá breytir forritið honum í heiltölugildi og geymir það í {\tt x}.

\index{virki!{\tt >>}}

Ef notandinn slær inn eitthvað annað en tölustaf þá mun keyrsluumhverfið ekki koma með neina villu.
Í staðinn mun eitthvað merkingarlaust gildi verða sett inn í {\tt x} og keyrslan heldur áfram.

Sem betur fer höfum við leið til að athuga hvort inntakssetning gekk upp.
Við getum kallað á {\tt good} fallið sem tileyrir {\tt cin} til að athuga það sem kallað er {\bf stream state}.
{\tt good} skilar {\tt bool}: ef true, þá gekk síðasta inntakssetning upp.
Ef ekki þá vitum við að einhver fyrri aðgerð virkaði ekki og jafnframt að næsta aðgerð mun heldur ekki virka.

Að sækja inntak frá notanda gæti því litið svona út:

\begin{verbatim}
#include <iostream>

using namespace std;

int main ()
{
  int x;

  // prompt the user for input
  cout << "Enter an integer: ";

  // get input
  cin >> x;

  // check and see if the input statement succeeded
  if (cin.good() == false) {
    cout << "That was not an integer." << endl;
    return -1;
  }

  // print the value we got from the user
  cout << x << endl;
  return 0;
}
\end{verbatim}
%
Það er einnig hægt að nota {\tt cin} til að lesa {\tt string}:

\begin{verbatim}
  string name;

  cout << "What is your name? ";
  cin >> name;
  cout << name << endl;
\end{verbatim}
%
Því miður les þessi setning aðeins fyrsta orðið í inntakinu en næstu orð verða lesin í næstu inntakssetningu.
Þannig að ef þú keyrir þetta forrit og skrifar inn fullt nafn (með bili á milli fyrra og seinna nafns) þá mun það eingöngu skrifa út fyrra nafnið.

Vegna þessara vandamála (vanhæfni til að meðhöndla villur og skrýtin virkni) þá forðast ég að nota {\tt >>} virkjann
nema að ég sé að lesa gögn sem ég veit að innihalda ekki neinar villur.

Í staðinn nota ég fall sem heitir {\tt getline} og er skilgreint í hausaskránni {\tt string}.

\begin{verbatim}
  string name;

  cout << "What is your name? ";
  getline (cin, name);
  cout << name << endl;
\end{verbatim}
%
Fyrra viðfangið í {\tt getline} er {\tt cin} sem tilgreinir hvaðan inntakið er að koma.
Seinna viðfangið er nafnið á strengnum sem á að geyma það sem lesið er.

{\tt getline} les heila línu allt að Return (Enter) tákninu.
Þetta fall er sem sagt t.d. gagnlegt að nota fyrir strengi sem innihalda bil.

Reyndar er {\tt getline} gagnlegt til að lesa inntak af hvaða tagi sem er.
Ef þú vildir t.d. að notandinn slæi inn heiltölu þá gætir þú lesið inntakið sem streng og síðan athugað hvort strengurinn stæði fyrir löglega heiltölu.
Ef inntakið væri löglegt þá myndir þú breyta því í heiltölugildi.
Ef ekki þá gætir þú prentað út viðeigandi villuskilaboð og beðið notandann að reyna aftur.

Þú getur notað fallið {\tt atoi} (sem skilgreint er í hausaskránni {\tt cstdlib}) til að breyta streng í heiltölu. 
Við mun einmitt gera það í kafla~\ref{parsing}.

\section{Orðalisti}

\begin{description}

\item[strúktúr (e. structure):]  Safn gagna sem eru sett saman til að mynda einn tiltekinn hlut.

\item[tilvikabreyta (e. instance variable):]  Ein af þeim breytum sem tilheyra strúktúr.

\item[tilvísun (e. reference):]  Gildi sem vísar í breytu eða strúktúr. Í stöðuriti er tilvísun táknuð með ör.

\item[kall með gildi (e. call/pass by value):]  Aðferð við stikun færibreytna þar sem gildi viðfangsins (e. argument) er afritað inn í leppinn (e. formal parameter) en viðfangið og leppurinn standa fyrir mismunandi minnishólf.

\item[kall með tilvísun (e. call/pass by reference):]  Aðferð við stikun færibreytna þar sem leppurinn er tilvísun í viðfangið.
Breytingar á leppnum hafa því áhrif á viðfangið.

\index{strúktúr}
\index{tilvikabreyta}
\index{tilvísun}
\index{kall með gildi}
\index{kall með tilvísun}

\end{description}

