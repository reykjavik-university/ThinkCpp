% LaTeX source for textbook ``How to think like a computer scientist''
% Copyright (C) 1999  Allen B. Downey

% This LaTeX source is free software; you can redistribute it and/or
% modify it under the terms of the GNU General Public License as
% published by the Free Software Foundation (version 2).

% This LaTeX source is distributed in the hope that it will be useful,
% but WITHOUT ANY WARRANTY; without even the implied warranty of
% MERCHANTABILITY or FITNESS FOR A PARTICULAR PURPOSE.  See the GNU
% General Public License for more details.

% Compiling this LaTeX source has the effect of generating
% a device-independent representation of a textbook, which
% can be converted to other formats and printed.  All intermediate
% representations (including DVI and Postscript), and all printed
% copies of the textbook are also covered by the GNU General
% Public License.

% This distribution includes a file named COPYING that contains the text
% of the GNU General Public License.  If it is missing, you can obtain
% it from www.gnu.org or by writing to the Free Software Foundation,
% Inc., 59 Temple Place - Suite 330, Boston, MA 02111-1307, USA.

% This is an Icelandic translation/adaptation of the orginal book by Allen B. Downey

\chapter{Skilyrði og endurkvæmni}
\label{condrecursion}

\section{Modulus virkinn}
\index{modulus}
\index{virki!modulus}

Modulus virkinn tekur tvær heiltölur sem þolendur og skilar {\em afganginum} sem myndast þegar fyrri þolandanum er deilt með þeim síðari.
Í C++ er modulus virkinn táknaður með prósentumerkinu {\tt \%}.
Málskipanin er nákvæmlega sú sama og fyrir aðra virkja: 

\begin{verbatim}
  int quotient = 7 / 3;
  int remainder = 7 % 3;
\end{verbatim}
%
Fyrsti virkinn að ofan, heiltöludeiling, skilar 2.
Seinni virkinn skilar 1.
Sem sagt, 7 deilt með 3 er 2 en afgangurinn er 1.

Það kann að koma á óvart hversu gagnlegur modulus virkinn er.
Þú getur t.d. athugað hvort ein tala sé deilanleg með annarri:
ef {\tt x \% y} er núll, þá er {\tt x} deilanleg með {\tt y}.

Þú getur líka notað modulus virkjann til að draga út tölustaf(i) sem koma fyrir lengst til hægri í tölu.
T.d. skilar {\tt x \% 10} tölustafnum lengst til hægri í {\tt x}.
Á sama hátt skilar {\tt x \% 100} þeim tveimur tölustöfum sem eru lengst til hægri.

\section{Skilyrt keyrsla}
\index{skilyrði}
\index{setning!skilyrði}

Til að skrifa gagnleg forrit þurfum við nánast alltaf að geta tékkað á ákveðnum skilyrðum og breytt flæði forritsins í samræmi við það.

{\bf Skilyrðissetningar} (e. conditional statements) gera okkur þetta kleift.
Einfaldasta form þeirra er {\tt if} setning:

\begin{verbatim}
  if (x > 0) {
    cout << "x is positive" << endl;
  }
\end{verbatim}
%
Segðin innan sviga er kallað skilyrðið.
Ef það er satt (e. true) þá er setningin innan slaufusviganna keyrð.
Ef skilyrðið er ekki satt (e. false) þá gerist ekkert.

\index{virki!samanburður}
\index{samanburður!virki}

Skilyrðið sjálft getur innihaldið sérhvern af eftirfarandi {\tt skilyrðisvirkjum} (e. comparison operators):

\begin{verbatim}
    x == y               // x equals y
    x != y               // x is not equal to y
    x > y                // x is greater than y
    x < y                // x is less than y
    x >= y               // x is greater than or equal to y
    x <= y               // x is less than or equal to y
\end{verbatim}
%
Þú þekkir vafalaust þessar aðgerðir en málskipanin sem C++ notar er aðeins frábrugðin stærðfræðitáknum eins og $=$, $\neq$ and $\le$.
Algeng forritunarvilla er að nota einfalt {\tt =} í stað tvöfalds {\tt ==}.
Þú þarft að muna að {\tt =} er gildisveitingarvirki (e. assignment operator) en {\tt ==} er samanburðarvirki (e. comparison operator).
Einnig er vert að benda á að hvorki {\tt =<} né {\tt =>} er til í C++.

Segðirnar sitt hvoru megin við samanburðarvirkja þurfa að vera af sama tagi.
Þú getur eingöngu borið saman {\tt int} og {\tt int} annars vegar og {\tt double} og {\tt double} hins vegar.
Á þessum tímapunkti getur þú ekki borið saman tvo strengi ({\tt string}).
Það er reyndar hægt og við munum fjalla um það síðar í bókinni.

\section {Varaleið}
\label{alternative}
\index{skilyrði!varaleið}

Önnur tegund af skilyrtri keyrslu er varaleið (e. alternative execution).
Í varaleið eru tveir möguleikar og skilyrðissegðin ákvarðar hvor leiðin er farin (keyrð).
Málskipanin lítur svona út:

\begin{verbatim}
  if (x%2 == 0) {
    cout << "x is even" << endl;
  } else {
    cout << "x is odd" << endl;
  }
\end{verbatim}
%
Ef afgangurinn, sem fæst með því að deila {\tt x} með 2, er núll þá vitum við að {\tt x} er slétt tala og kóðinn skrifar út skilaboð þess efnis.
Ef skilyrðið {\tt (x\%2 == 0)} er ósatt þá keyrast setningarnar í {\tt else} hlutanum. 
Þar sem skilyrðið er annað hvort satt eða ósatt (true eða false) þá mun önnur hvor leiðin verða keyrð, en ekki báðar.

Ef þú telur að þú gætir oft þurft að tékka á því hvort tiltekin tala er slétt tala eða oddatala þá væri gott að ``pakka'' (e. ``wrap'') þessum kóða inn í fall, t.d.:

\begin{verbatim}
void printParity (int x) {
  if (x%2 == 0) {
    cout << "x is even" << endl;
  } else {
    cout << "x is odd" << endl;
  }
}
\end{verbatim}
%
Þar með áttu fall með nafninu {\tt printParity} sem skrifar út viðeigandi skilaboð fyrir hvaða heiltölu sem send er inn sem viðfang.
Úr {\tt main} gætir þú kallað á þetta fall á eftirfarandi hátt:

\begin{verbatim}
    printParity (17);
\end{verbatim}
%
Mundu að þegar þú {\em kallar} á fall þá þarftu ekki að tilgreina tagið á viðföngunum því C++ þýðandinn finnur sjálfur út af hvað tagi þau eru.
Ekki freistast til að skrifa:

\begin{verbatim}
  int number = 17;
  printParity (int number);         // WRONG!!!
\end{verbatim}

\section {Skilyrðiskeðjur}
\index{skilyrði!keðjur}

Stundum þarftu að tékka á mörgum tengdum skilyrðum og velja eina af mörgum mögulegum aðgerðum.
Ein leið til að gera þetta er að tengja saman röð af {\tt if} og {\tt else} og mynda þannig {\bf skilyrðiskeðjur} (e. chained conditional):

\begin{verbatim}
  if (x > 0) {
    cout << "x is positive" << endl;
  } else if (x < 0) {
    cout << "x is negative" << endl;
  } else {
    cout << "x is zero" << endl;
  }
\end{verbatim}
%
Þessar keðjur geta verið eins langar og þú vilt hafa þær en eftir því sem þær eru lengri því ólæsilegri verða þær!
Ein leið til að gera þær læsilegri er að nota inndrátt (e. indentation) í kóðanum, eins og gert er í dæminu að ofan.
Ef þú sérð til þess að allar setningarnar og slaufusvigarnir séu með sama inndrátt þá er ólíklegra að þú gerir einhverjar málskipunarvillur og ef þú gerir villur þá finnur þú þær fljótar.

\section{Hreiðruð skilyrði}
\index{skilyrði!hreiðruð}

Í stað þess að tengja skilyrði saman þá getur þú einnig hreiðrað (e. nest) eitt skilyrði inni í öðru.
Við gætum hafa skrifað dæmið að ofan á þennan hátt:

\begin{verbatim}
  if (x == 0) {
    cout << "x is zero" << endl;
  } else {
    if (x > 0) {
      cout << "x is positive" << endl;
    } else {
      cout << "x is negative" << endl;
    }
  }
\end{verbatim}
%
Þessi kóði samanstendur af ytra skilyrði ({\tt x == 0}) sem inniheldur tvær mögulegar kvíslir (e. branches).
Fyrsta kvíslin er einfaldlega úttakssetning en seinni kvíslin samanstendur af annarri {\tt if} setningu sem sjálf hefur tvær kvíslir.
Þessar tvær kvíslir eru báðar úttakssetnignar en gætu þess vegna líka verið aðrar skilyrðissetningar.

Taktu eftir því að inndrátturinn hjálpar við að sýna uppbyggingu kóðans en samt sem áður geta hreiðruð skilyrði verið ólæsileg.
Almennt séð ættir þú að forðast að skrifa hreiðruð skilyrði ef þú kemst hjá því.

\index{nested structure}

Á hinn bóginn má segja að þar sem þessi tegund af {\bf hreiðruðum strúktúr} er algeng, og við munum sjá hana aftur, þá er gott fyrir þig að venjast þessu strax.

\section{{\tt Return} setningin}
\index{return}
\index{setning!return}

{\tt Return} setning gerir þér kleift að hætt keyrslu falls áður en komið er að lokum þess.
Ein ástæða fyrir notkun return setningar er þegar villuskilyrði (e. error condition) reynist satt:

\begin{verbatim}
#include <cmath>

void printLogarithm (double x) {
  if (x <= 0.0) {
    cout << "Positive numbers only, please." << endl;
    return;
  }

  double result = log (x);
  cout << "The log of x is " << result;
}
\end{verbatim}
%
Í þessu dæmi er skilgreining á fallinu {\tt printLogarithm} sem hefur {\tt double} nefnt {\tt x} sem lepp.
Fallið byrjar á því að athuga hvort {\tt x} er minna eða jafnt og núll og ef satt reynist þá skrifar það úr villuskilaboð
og notar síðan {\tt return} til að hætta keyrslu fallsins.
Keyrsluflæðið (e. flow of execution) færist þá sjálfkrafa til þess sem kallaði á fallið en þær setningar sem eftir standa í fallinu {\tt printLogarithm} eru ekki framkvæmdar.

Taktu eftir að ég notaði kommutölugildi á hægri hlið skilyrðisins vegna þess að það er breyta af kommutölutagi á vinstri hliðinni.

%Remember that any time you want to use one a function from the math
%library, you have to include the header file {\tt math.h}.

\section{Endurkvæmni}
\label{recursion}
\index{endurkvæmni}

Ég nefndi í síðasta kafla að eitt fall getur kallað á annað og við höfum séð nokkur dæmi um það.
Ég nefndi hins vegar ekki að fall getur einnig kallað á sjálft sig!
Það er kannski ekki augljóst af hverju það getur verið gott en svo vill reyndar til að þessi möguleiki er 
einn af þeim mest töfrandi og áhugaverðustu sem forrit getur gert.

Skoðaðu t.d. eftirfarandi fall:

\begin{verbatim}
void countdown (int n) {
  if (n == 0) {
    cout << "Blastoff!" << endl;
  } else {
    cout << n << endl;
    countdown (n-1);
  }
}
\end{verbatim}
%
Nafnið á þessu falli er {\tt countdown} og það tekur heiltölu sem viðfang.
Ef viðfangið er núll þá skrifar það út orðið ``Blastoff.''
Annars skrifar það út gildi leppsins og kallar á fallið {\tt countdown} -- sjálft sig -- og sendir gildið {\tt n-1} sem viðfang.

Hvað gerist ef við köllum á þetta fall á eftirfarandi hátt:

\begin{verbatim}
#include <iostream>

void countdown (int n) {
  if (n == 0) {
    cout << "Blastoff!" << endl;
  } else {
    cout << n << endl;
    countdown (n-1);
  }
}

int main ()
{
  countdown (3);
  return 0;
}
\end{verbatim}
%
Keyrslan á {\tt countdown} byrjar með {\tt n=3} og þar sem {\tt n} er ekki núll
þá skrifar það út gildið 3 og kallar síðan á sjálft sig ...

\begin{quote}
Keyrslan á {\tt countdown} byrjar með {\tt n=2} og þar sem {\tt n} er ekki núll
þá skrifar það út gildið 2 og kallar síðan á sjálft sig ...

\begin{quote}
Keyrslan á {\tt countdown} byrjar með {\tt n=1} og þar sem {\tt n} er ekki núll
þá skrifar það út gildið 1 og kallar síðan á sjálft sig ...

\begin{quote}
Keyrslan á {\tt countdown} byrjar með {\tt n=0} og þar sem {\tt n} er núll
þá skrifar það út orðið ``Blastoff!'' og hættir síðan keyrslu.
\end{quote}

countdown sem fékk {\tt n=1} sem viðfang hættir keyrslu.

\end{quote}

countdown sem fékk {\tt n=2} sem viðfang hættir keyrslu.

\end{quote}

countdown sem fékk {\tt n=3} sem viðfang hættir keyrslu.\\

\noindent Og þá erum við aftur komin í {\tt main} (þvílíkt ferðalag!).
Endanlegt úttak lítur því svona út:

\begin{verbatim}
3
2
1
Blastoff!
\end{verbatim}
%

Til að taka annað dæmi um endurkvæmni skulum við líta aftur á föllin {\tt newLine} og {\tt threeLine}.

\begin{verbatim}
void newLine () {
  cout << endl;
}

void threeLine () {
  newLine ();  newLine ();  newLine ();
}
\end{verbatim}
%
Þrátt fyrir að þessi föll virki þá gæti ég ekki notað þau ef ég myndi vilja skrifa út 2 nýjar (auðar) línur eða 106 nýjar línur.
Betri útfærsla væri:

\begin{verbatim}
void nLines (int n) {
  if (n > 0) {
    cout << endl;
    nLines (n-1);
  }
}
\end{verbatim}
%
Þetta fall er sambærilegt við {\tt countdown}.
Svo lengi sem {\tt n} er stærra en núll þá skrifar það út eina nýja línu og kallar svo aftur á sjálft sig til að skrifa út {\tt n-1} nýjar línur.
Fjöldi nýrra lína sem fallið skrifar út er sem sagt {\tt 1 + (n-1)} sem er auðvitað = {\tt n}.

\index{endurkvæmni}
\index{ný lína}

Það ferli þegar fall kallar á sjálft sig er kallað {\bf endurkvæmni} (e. recursion) og þess konar fall er sagt vera {\bf endurkvæmt} (e. recursive).

\section {Óendanleg endurkvæmni}

Þú hefur væntanlega tekið eftir því í dæmunum í síðasta kafla að þegar kallað var á fall endurkvæmt þá minnkaði viðfangið um einn í sérhvert sinn og varð að lokum að núlli.
Þegar leppurinn fær að lokum gildið 0 þá hættir fallið keyrslu {\em án þess að framkvæma fleiri endurkvæm köll}.
Þetta tilfelli -- þegar fall hættir án þess að framkvæma endurkvæmt kall -- er kallað {\bf grunnþrep} (e. base case).

Ef endurkvæmnin kemst aldrei í grunnþrepið þá mun viðkomandi fall halda áfram að framkvæma endurkvæm köll aftur og aftur og fallið mun ekki hættta keyrslu!
Þetta er þekkt sem {\bf óendanleg endurkvæmni} (e. infinite recursion) og er almennt séð ekki talið vera góð latína!

\index{endurkvæmni!óendanleg}
\index{óendanleg endurkvæmni}
\index{keyrsluvilla}

Í flestum stýrikerfum mun forrit, sem inniheldur óendanlega endurkvæmni, ekki keyra út í hið óendanlega.
Að lokum mun eitthvað gefa eftir með villutilkynningu.
Þetta er fyrsta dæmið sem þú sérð um svokallaða {\bf keyrsluvillu} (e. run-time error) þ.e. villu sem kemur ekki í ljós fyrr en forrit er keyrt.

Þú ættir að prófa að skrifa forrit sem inniheldur óendanlega endurkvæmni til að sjá hvað gerist þegar þú keyrir það.

\section {Staflarit fyrir endurkvæm föll}
\index{stafli}
\index{staflarit}
\index{rit!stafli}

Í kaflanum á undan notuðum við staflarit til að sýna stöðu forrits á meðan á keyrslu falls átti sér stað.
Við getum einnig notað staflarit til að sýna stöðu endurkvæms falls.

Mundu að í hvert sinn sem kallað er á fall þá býr það til nýtt tilvik af kvaðningafærslu sem inniheldur staðværar breytur og leppa.

Þessi mynd sýnir staflarit fyrir countdown með {\tt n = 3}:

\vspace{0.1in}
\centerline{\epsfig{figure=stack2.eps}}
\vspace{0.1in}
%
Hér er eitt tilvik af {\tt main} og fjögur tilvik af {\tt countdown}, sérhvert þeirra er með mismunandi gildi á leppnum {\tt n}.
Á botni staflans er {\tt countdown} með {\tt n=0} en það er einmitt grunnþrepið.
Það tilvik framkvæmir ekki endurkvæmt kall og því eru ekki fleiri tilvik af {\tt countdown} á staflanum.

Tilvikið af {\tt main} er tómt vegna þess að {\tt main} hefur hvorki leppa né staðværar breytur.
Þú ættir að prófa að teikna staflarit fyrir {\tt nLines} með leppinn {\tt n=4}.


\section{Orðalisti}

\begin{description}

\item[modulus:]  Virki, sem beitt er á tvær heiltölur, og skilar afganginum þegar einni tölu er deilt með annarri.
Í C++ er þessi virki táknaður með prósentumerkinu ({\tt \%}).

\item[skilyrðissetningar (e. conditional):]  Blokk af setningum hvers keyrsla er háð útkomu úr tilteknu skilyrði.

\item[tenging skilyrða (e. chaining):]  Leið til að tengja saman í röð nokkrar skilyrðissetningar.

\item[hreiðrun (e. nesting):] Að setja nokkrar skilyrðissetningar innan í eina eða fleiri kvíslir á annarri skilyrðissetningu.

\item[endurkvæmni (e. recursion):]  Það ferli að kalla á sama fall og nú er þegar verið að keyra.

\item[óendanleg endurkvæmni (e. infinite recursion):]  Gerist þegar fall, sem kallar á sjálft sig, kemst aldrei í grunnþrepið.
Að lokum mun óendanleg endurkvæmni hafa keyrsluvillu í för með sér.

\index{modulus}
\index{skilyrðissetning}
\index{skilyrði!tenging}
\index{skilyrði!hreiðrað}
\index{endurkvæmni}
\index{endurkvæmni!óendanleg}
\index{óendanleg endurkvæmni}

\end{description}
