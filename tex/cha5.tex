% LaTeX source for textbook ``How to think like a computer scientist''
% Copyright (C) 1999  Allen B. Downey

% This LaTeX source is free software; you can redistribute it and/or
% modify it under the terms of the GNU General Public License as
% published by the Free Software Foundation (version 2).

% This LaTeX source is distributed in the hope that it will be useful,
% but WITHOUT ANY WARRANTY; without even the implied warranty of
% MERCHANTABILITY or FITNESS FOR A PARTICULAR PURPOSE.  See the GNU
% General Public License for more details.

% Compiling this LaTeX source has the effect of generating
% a device-independent representation of a textbook, which
% can be converted to other formats and printed.  All intermediate
% representations (including DVI and Postscript), and all printed
% copies of the textbook are also covered by the GNU General
% Public License.

% This distribution includes a file named COPYING that contains the text
% of the GNU General Public License.  If it is missing, you can obtain
% it from www.gnu.org or by writing to the Free Software Foundation,
% Inc., 59 Temple Place - Suite 330, Boston, MA 02111-1307, USA.

% This is an Icelandic translation/adaptation of the orginal book by Allen B. Downey

\chapter{Föll sem skila gildi}

\section{Skilagildi}
\index{return}
\index{setning!return}
\index{fall!skilagildi}
%\index{fruitful function}
\index{skilagildi}
\index{void}
\index{fall!void}

Sum af þeim föllum sem við höfum notað, eins og t.d. stærðfræðiföllin, skila af sér einhverju gildi.
Tilgangurinn með því að kalla á þess konar fall er að búa til nýtt gildi sem er síðan notað til að gefa breytu gildi eða notað sem hluti af segð.
Dæmi:

\index{stærðfræðifall!exp}
\index{stærðfræðifall!sin}

\begin{verbatim}
  double e = exp (1.0);
  double height = radius * sin (angle);
\end{verbatim}
%
Hins vegar vill svo til að öll þau föll sem við höfum sjálf skrifað hingað til eru {\bf void} föll, þ.e. föll sem skila ekki neinu gildi.
Kall í void fall er yfirleitt gert án nokkurrar gildisveitingar (því ekkert gildi kemur til baka úr fallinu):

\begin{verbatim}
  nLines (3);
  countdown (n-1);
\end{verbatim}
%
Í þessum kafla munum við skrifa föll sem skila af sér gildum.
Það mætti segja að þessi föll beri ávöxt!
Fyrsta dæmið er {\tt area}, sem tekur {\tt double} sem viðfang, og skilar flatarmáli hrings með gefinn radíus:

\index{stærðfræðifall!acos}
\index{pi}

\begin{verbatim}
double area (double radius) {
  double pi = acos (-1.0);
  double area = pi * radius * radius;
  return area;
}
\end{verbatim}
%
Það fyrsta sem þú ættir að taka eftir er að byrjun fallaskilgreiningarinnar er öðruvísi.
Í stað {\tt void}, sem gefur til kynna void fall, þá sjáum við hér {\tt double} sem gefur til kynna að skilagildið úr þessu falli sé af taginu {\tt double}.

Taktu líka eftir að síðasta línan í fallinu er önnur útgáfa af {\tt return} setningu, return setning sem inniheldur skilagildi.
Þessi setning þýðir ``hætta strax keyrslu þessa falls og nota eftirfarandi segð sem skilagildið.''
Segðin, sem kemur á eftir lykilorðinu return, getur verið eins flókin og verða vill þannig að við gætum hafa skrifað fallið á samþjappaðri hátt:

\begin{verbatim}
double area (double radius) {
  return acos(-1.0) * radius * radius;
}
\end{verbatim}
%
Á hinn bóginn má segja að {\bf tímabundnar} (e. temporary) breytur eins og {\tt area} geri kembingu (e. debugging) oft auðveldari.
Í báðum tilvikum þarf tag segðarinnar í return setningunni að passa við skilatag fallsins.
M.ö.o., þegar þú skilgreinir að skilagildið sé af taginu {\tt double} þá ``lofar'' þú því að fallið muni að endingu skila {\tt double}.
Þýðandinn mun kvarta ef þú reynir að skila engri segð eða segð af röngu tagi. 

\index{tímabundin breyta}
\index{breyta!tímabundin}

Stundum getur verið hentug að hafa margar return setningar í falli -- eina fyrir sérhverja kvísl í skilyrðissetningu:

\begin{verbatim}
double absoluteValue (double x) {
  if (x < 0) {
    return -x;
  } else {
    return x;
  }
}
\end{verbatim}
%
Aðeins ein af þessum return setningum mun verða keyrð þar sem return setningarnar eru í mismunandi kvíslum.
Þrátt fyrir að það sé leyfilegt að hafa fleiri en eina {\tt return} setningu í falli þá skaltu muna að um leið og ein þeirra er keyrð þá hættir fallið keyrslu og mun ekki keyra þær setningar sem á eftir koma.

Kóði sem kemur á eftir {\tt return} setningu, eða á stað sem flæðið mun ekki komast í, er kallaður {\bf dauður kóði} (e. dead code).
Sumir þýðendur gefa einmitt aðvaranir ef hluti kóða er ``dauður''.

\index{dauður kóði}

Ef þú setur return setningu innan í skilyrðissetningu þá þarftu að sjá til þess að {\em sérhver möguleg leið} gegnum forritið lendi að lokum á return setningu.
Dæmi: 

\begin{verbatim}
double absoluteValue (double x) {
  if (x < 0) {
    return -x;
  } else if (x > 0) {
    return x;
  }                          // WRONG!!
}
\end{verbatim}
%
Þetta forrit er ekki rétt vegna þess að ef {\tt x} er 0 þá er hvorugt skilyrðanna satt og fallið mun þá hætta keyrslu án þess að framkvæma return setningu.
Því miður þá grípa ekki allir C++ þýðendur þessa villu.
Því má vera að forritið þýðist og keyrist en þegar {\tt x==0} þá getur skilagildið í raun verið hvað sem er og líklega mismunandi eftir ólíkum keyrsluumhverfum.

\index{tölugildi}
\index{villa!þýðing}
%\index{compile-time error}

Núna ert þú líklega orðin(n) hundleið(ur) á þýðandavillum en eftir því sem reynslan eykst þá áttar þú þig á því 
að það eina sem er verra en að fá þýðandavillu er að fá {\em ekki} þýðandavillu þegar forritið er ekki rétt!

Hér er dæmi um eitthvað sem gæti gerst: Þú prófar {\tt absoluteValue} með ýmsum mismunandi gildum á {\tt x} og það virðist virka rétt.
Þú lætur síðan einhvern annan fá forritið og viðkomandi prófar það í öðru umhverfi.
Á einhvern dularfullan hátt skilar það ekki réttu gildi og eftir nokkra daga kembingu kemstu að því að útfærslan á {\tt absoluteValue} er ekki rétt.
Bara ef þýðandinn hefði aðvarað þig!

%\index{compile-time error}
%\index{error!compile-time}
\index{aflúsun}
\index{kembing}

Framvegis skaltu ekki álasa þýðandanum ef hann bendir á villu í forritinu þínu.
Þú skalt frekar þakka honum fyrir a finna villu og spara þér nokkrar daga vinnu við aflúsun.
Sumir þýðendur hafa valkost sem segir þeim að vera sérstaklega ``strangur'' og greina frá öllum villum sem þeir finna.
Þú ættir alltaf að velja þennan valkost í þínum þýðanda.

\index{stærðfræðifall!fabs}

Sem innskot þá bendi ég á að það er fall í math safninu sem heitir {\tt fabs}.
Það reiknar tölugildið á {\tt double} -- á réttan hátt.

\section{Þróunarferli}
\label{distance}
\index{þróunarferli}

Á þessum tímapunkti ættir þú að geta skoðað C++ föll í heild sinni og sagt til um hvað þau gera.
Aftur á móti er þér kannski ekki ljóst hvernig eigi að skrifa þau.
Ég mun núna stinga upp á einni aðferð við þróun forrits sem ég kalla {\bf stigvaxandi þróun} (e. incremental development).

\index{stigvaxandi þróun}
\index{þróunarferli}

Gefum okkur t.d. að þú þurfir að skrifa fall sem reiknar fjarlægðina á milli tveggja punkta sem gefnir eru með hnitunum $(x_1, y_1)$ og $(x_2, y_2)$.
Hefðbundna skilgreiningin á fjarlægð á milli tveggja punkta er:

\begin{equation}
distance = \sqrt{(x_2 - x_1)^2 + (y_2 - y_1)^2}
\end{equation}
%
Fyrsta skrefið er að íhuga hvernig {\tt distance} fall ætti að líta út í C++.
M.ö.o., hvert er inntakið (leppar/viðföng) og hvert er úttakið (skilagildið).

Í þessu tilviki eru punktarnir tveir auðvitað viðföngin inn í fallið og það er eðlilegt að nota fjóra leppa af taginu {\tt double}.
Skilagildið er fjarlægð sem er líka eðlilegt að beri tagið {\tt double}.

Nú getum við því skrifað drög að fallinu:

\begin{verbatim}
double distance (double x1, double y1, double x2, double y2) {
  return 0.0;
}
\end{verbatim}
%
{\tt Return} setningin er þarna að forminu til svo að fallið þýðist og keyri þrátt fyrir að setningin skili ekki réttu gildi.
Á þessu stigi gerir fallið í raun ekkert gagnlegt en það er þess virði að reyna að þýða það strax svo við getum fundið málskipunarvillur áður en fallið verður flóknara.

Til að prófa nýja fallið verðum við að kalla á það með einhverjum tilraunagildum (viðföngum).
Einhversstaðar í {\tt main} myndi ég bæta við:

\begin{verbatim}
  double dist = distance (1.0, 2.0, 4.0, 6.0);
  cout << dist << endl;
\end{verbatim}
%
Ég valdi gildin þannig að lárétta fjarlægðin er 3 og lóðrétta fjarlægðin er 4.
Þá ætti niðurstaðan að vera 5 (langhliðin í 3-4-5 þríhyrningi).
Það er gagnlegt að vita rétta svarið þegar maður prófar fall!

Þegar við höfum athugað að málskipanin í fallaskilgreiningunni okkar er rétt getum við byrjað á því að bæta við setningum í fallið, einni í einu.
Svo þýðum við og keyrum forritið eftir sérhverja (stigvaxandi) breytingu.
Ef villa kemur upp þá vitum við nákvæmlega hvar hún er -- í síðustu línunni sem við bættum við.

Næsta skref í útreikningnum okkar er að finna gildin á $x_2 - x_1$ og $y_2 - y_1$.
Ég mun geyma þessi gildi (milliniðurstöður) í tímabundnum breytum sem ég nefni {\tt dx} og {\tt dy}.

\begin{verbatim}
double distance (double x1, double y1, double x2, double y2) {
  double dx = x2 - x1;
  double dy = y2 - y1;
  cout << "dx is " << dx << endl;
  cout << "dy is " << dy << endl;
  return 0.0;
}
\end{verbatim}
%
Ég bætti við úttakssetningum sem skrifa út gildin á tímabundnu breytunum. 
Eins og ég nefndi að ofan þá veit ég þegar að þessi gildi ættu að vera 3,0 og 4,0.

\index{stoðbúnaður}

Ég mun síðan fjarlægja þessar úttakssetningar þegar ég hef lokið við þróun fallsins.
Kóði eins og þessi er einsskonar {\bf stoðbúnaður} (e. scaffolding) vegna þess að hann hjálpar til við þróun fallsins en er ekki hluti af endanlegri útgáfu þess.

Stundum er reyndar gott að fjarlægja ekki stoðbúnaðinn algerlega heldur setja hann inn í athugasemdir (e. comments) ef ske kynni að við þyrftum á honum að halda síðar.

Næsta skref í þróun fallsins er þá að setja {\tt dx} og {\tt dy} í annað veldi og leggja niðurstöðurnar saman.
Við gætum notað {\tt pow} fallið en það er einfaldara og fljótvirkara að margfalda hvorn liðinn með sjálfum sér.

\begin{verbatim}
double distance (double x1, double y1, double x2, double y2) {
  double dx = x2 - x1;
  double dy = y2 - y1;
  double dsquared = dx*dx + dy*dy;
  cout << "dsquared is " << dsquared;
  return 0.0;
}
\end{verbatim}
%
Hér myndi ég aftur þýða og keyra forritið og athuga milliniðurstöðuna (tímabundna gildið) sem ætti að vera 25,0.

Að lokum notum við {\tt sqrt} fallið til að reikna og skila lokaniðurstöðunni.

\begin{verbatim}
double distance (double x1, double y1, double x2, double y2) {
  double dx = x2 - x1;
  double dy = y2 - y1;
  double dsquared = dx*dx + dy*dy;
  double result = sqrt (dsquared);
  return result;
}
\end{verbatim}
%
Í {\tt main} ættum við síðan að skrifa út og athuga gildið sem við fáum til baka úr kallinu á fallið distance.

Eftir því sem reynslan þín eykst þá muntu skrifa og kemba fleiri en eina línu í einu.
Samt sem áður mun þetta stigvaxandi þróunarferli spara þér mikinn tíma við kembingu síðar meir.

Megin þættir í þessu ferli eru:

\begin{itemize}

\item Byrjaðu með forrit sem keyrir og gerðu litlar, stigvaxtandi breytingar.
Ef einhver villa kemur upp þá veistu nákvæmlega hvar hún er.

\item Notaðu tímabundnar breytur til að geyma milliniðurstöður þannig að þú getir skrifað þær út.

\item Þegar forritið er tilbúið þá viltu kannski fjarlægja stoðbúnaðinn eða skipta mörgum setningum út fyrir samsettar segðir (e. compound expressions), en þó aðeins ef það gerir forritið þitt ekki ólæsilegra.

\end{itemize}

\section{Samsetning}
\index{samsetning}

Þegar þú hefur skilgreint nýtt fall þá getur þú notað það sem hluta af segð og þú getur einnig skilgreint ný föll með því að nota þau föll sem eru þegar til.
Hvað ef t.d. einhver gæfi þér tvo punkta, miðju hrings og punkt á hringnum, og spyrði um flatarmál hringsins?

Gefum okkur að miðjan sé geymd í breytunum {\tt xc} og {\tt yc} og punkturinn á hringnum í {\tt xp} og {\tt yp}.
Fyrsta skrefið er þá að finna radíus hringsins, þ.e. fjarlægðina á milli punktanna tveggja.
Við eigum einmitt fall, {\tt distance}, sem gerir það!

\begin{verbatim}
  double radius = distance (xc, yc, xp, yp);
\end{verbatim}
%
Næsta skref er síðan að finna flatarmál hrings með þennan radíus og skila því.

\begin{verbatim}
  double result = area (radius);
  return result;
\end{verbatim}
%
Ef við setjum þennan kóða inn í sér fall þá fáum við:

\begin{verbatim}
double fred (double xc, double yc, double xp, double yp) {
  double radius = distance (xc, yc, xp, yp);
  double result = area (radius);
  return result;
} 
\end{verbatim}
%
Nafnið á þessu falli er {\tt fred} sem er vissulega skrýtið.
Í næsta kafla mun ég skýra út af hverju nafnið er undarlegt.

Tímabundnu breyturnar {\tt radius} og {\tt area} eru gagnlegar í þróuninni og í kembun 
en þegar við erum viss um að fallið virki þá getum skrifað það á samþjappaðri hátt með því að setja saman fallaköllin tvö:

\begin{verbatim}
double fred (double xc, double yc, double xp, double yp) {
  return area (distance (xc, yc, xp, yp));
} 
\end{verbatim}

\section{Fjölbinding}
\label{overloading}
\index{fjölbinding}

Þú hefur kannski tekið eftir því, í kaflanum hér á undan, að föllin
{\tt fred} og {\tt area} hafa sama hlutverk -- að finna flatarmál hrings -- en taka samt mismunandi viðföng. 
Við sendum radíus sem viðfang í {\tt area} en tvo punkta sem viðföng í {\tt fred}.

Ef tvö föll framkvæma sömu aðgerð þá er eðlilegt að gefa þeim sama nafn.
M.ö.o, það væri eðlilegra ef fallið {\tt fred} væri kallað {\tt area}.

Í C++ er leyfilegt að hafa fleiri en eitt fall með sama nafninu svo framarlega sem sérhver útgáfa taki mismunandi viðföng.
Þessi eiginleiki er kallaður {\bf fjölbinding} (e. overloading).
Við getum því endurskýrt {\tt fred}:

\begin{verbatim}
double area (double xc, double yc, double xp, double yp) {
  return area (distance (xc, yc, xp, yp));
} 
\end{verbatim}
%
Nú lítur þetta út eins og endurkvæmt fall en svo er ekki.
Það vill svo til að þessi útgáfa af {\tt area} kallar á hina útgáfuna!
Þegar þú kallar á fjölbundið fall þá veit C++ þýðandinn hvaða útgáfu verið er að kalla á með því að skoða viðföngin sem notuð eru.
Ef þú skrifar

\begin{verbatim}
    double x = area (3.0);
\end{verbatim}
%
þá mun C++ þýðandinn leita að falli með nafninu {\tt area} sem tekur {\tt double} sem viðfang og notar því fyrri úgáfuna.
Ef þú skrifar

\begin{verbatim}
    double x = area (1.0, 2.0, 4.0, 6.0);
\end{verbatim}
%
þá notar þýðandinn seinni útgáfuna af {\tt area}.  

Margar af innbyggðu C++ aðgerðunum eru fjölbundnar sem þýðir að það eru til mismunandir útgáfur sem taka mismunandi fjölda af viðföngunum eða að tag viðfanganna er mismunandi.

Fjölbinding er gagnlegur eiginleiki en skal samt sem áður nota með varúð.
Það getur verið ruglandi ef þú kembir eina útgáfu af falli sem síðan kallar á aðra útgáfu þess.

Þetta minnir mig reyndar á eina grundvallarreglu varðandi kembingu:
{\bf Vertu viss um að útgáfan af forritun sem þú ert að skoða sé útgáfan sem er í raun keyrandi!}
Stundum lendir þú í því að gera breytingu eftir breytingu á forritinu þínu en samt sem áður sérðu alltaf sama úttak þegar þú keyrir það.
Þetta er merki um það að af einhverjum ástæðum þá ertu ekki að keyra þá útgáfu af forritun sem þú heldur að þú sért að keyra.
Til að vera viss þá getur þú skotið inn úttakssetningu (það skiptir í raun ekki máli hvað þú skrifar út) og þannig gengið úr skugga um að rétta útgáfan af forritinu sé keyrandi.

\section{Boole gildi}
\index{boolean}
\index{gildi!boole}

Tögin sem við höfum hingað til séð eru frekar stór.
Það eru ansi margar heiltölur til í heiminum og enn fleiri kommutölur.
Í samanburði er mengi stafa hins vegar frekar smátt.
Það er reyndar eitt C++ tag sem er enn smærra.
Það er kallað {\bf boole} og einu leyfilegu gildi þess eru gildin {\tt true} og {\tt false}.

Í síðustu tveimur köflum höfum við notað boole gildi án þess að fjalla sérstaklega um það.
Skilyrðið innan í {\tt if} setningu er einmitt boole segð og niðurstaðan af beitingu samanburðarvirkja er boole gildi.
Dæmi:

\begin{verbatim}
  if (x == 5) {
    // do something
  }
\end{verbatim}
%
Samanburðarvirkinn {\tt ==} ber hér saman tvö heiltölugildi og skilar boole gildi.

\index{virki!samanburður}
\index{samanburðarvirki}

Gildin {\tt true} og {\tt false} eru lykilorð í C++ 
og geta verið notuð hvar sem gert er ráð fyrir boole segð.
Dæmi:  

\begin{verbatim}
  while (true) {
    // loop forever
  }
\end{verbatim}
%
Þetta er staðlað sniðmát fyrir lykkju sem ætti að keyra endalaust (eða þangað til að komið er að {\tt return} eða {\tt break} setningu).

\section{Boole breytur}
\index{tag!{\tt bool}}

Fyrir sérhvert tag á gildi er samsvarandi tag á breytu.
Í C++ er boole tagið kallað {\bf bool}.
Breytur af taginu bool eru skilgreindar á sama hátt og breytur af öðru tagi:

\begin{verbatim}
  bool fred;
  fred = true;
  bool testResult = false;
\end{verbatim}
%
Fyrsta línan er einföld yfirlýsing á breytu.
Önnur línan er gildisveiting og þriðja línan er samsetning á yfirlýsingu og gildisveitingu, þ.e. frumstilling.

\index{frumstilling}
\index{setning!frumstilling}

Eins og ég nefndi að ofan þá er niðurstaðan úr samanburði alltaf booole gildi þannig að hægt er að geyma hana í {\tt bool} breytu

\begin{verbatim}
  bool evenFlag = (n%2 == 0);     // true if n is even
  bool positiveFlag = (x > 0);    // true if x is positive
\end{verbatim}
%
og nota breytuna síðar í skilyrðissetningu 

\begin{verbatim}
  if (evenFlag) {
    cout << "n was even when I checked it" << endl;
  }
\end{verbatim}
%
Breyta sem notuð er á þennan hátt er oft kölluð {\bf flagg} (e. flag)
vegna þess að hún gefur til kynna (``flaggar'') að eitthvað skilyrði sé til staðar eður ei.

\index{flagg}

\section{Rökvirkjar}
\index{rökvirki}
\index{virki!rök}

Í C++ eru þrír {\bf rökvirkjar} (e. logical operators): AND, OR og NOT,
sem eru táknaðir með {\tt \&\&}, {\tt ||} og {\tt !}.
Merking þessar virkja er svipuð og merking þeirra í náttúrulegu máli (ensku).
T.d. er {\tt x > 0 \&\& x < 10} true (satt) aðeins ef {\tt x} er stærra en 0 OG (AND) minna en 10.

\index{merking}

{\tt evenFlag || n\%3 == 0} er true ef {\em annað} skilyrðanna er true, þ.e. ef {\tt evenFlag} er true EÐA (OR) að talan er deilanleg með 3.

Að lokum hefur NOT virkinn þau áhrif að neita eða snúa við gildi boole segðar.
Þannig er {\tt !evenFlag} true ef {\tt evenFlag} er false,  þ.e. ef talan er oddatala.

\index{hreiðruð skilyrði}

Rökvirkjar gefa oft möguleika á því að einfalda hreiðraðar skilyrðissetningar.
Hvernig myndir þú t.d. skrifa eftirfarandi kóða með því að nota eitt skilyrði sem samsett er úr rökvirkjum?

\begin{verbatim}
  if (x > 0) {
    if (x < 10) {
      cout << "x is a positive single digit." << endl;
    }
  }
\end{verbatim}

\section{Boole föll}
\label{bool}
\index{boole}
\index{fall!boole}

Föll geta skilað {\tt bool} gildum eins og hvaða öðru gildi.
Það er oft hentugt þegar hylja þarf flókið skilyrði innan í falli:
Dæmi:

\begin{verbatim}
bool isSingleDigit (int x)
{
  if (x >= 0 && x < 10) {
    return true;
  } else {
    return false;
  }
}
\end{verbatim}
%
Nafnið á þessu falli er {\tt isSingleDigit}.
Algengt er að gefa boole föllum nöfn sem hljóma eins og já/nei spurningar.
Skilagildið er {\tt bool} sem þýðir að sérhver return setning þarf að hafa segð af taginu {\tt bool} í för með sér.

Kóðinn sjálfur er einfaldur þrátt fyrir að vera aðeins lengri en nauðsynlegt er.
Mundu að segðin {\tt x >= 0 \&\& x < 10} hefur tagið {\tt bool} þannig að það er ekkert athugavert við það að skila henni beint og þar með sleppa {\tt if} setningunni:

\begin{verbatim}
bool isSingleDigit (int x)
{
  return (x >= 0 && x < 10);
}
\end{verbatim}
%
Í {\tt main} getur þú kallað á þetta fall á venjulegan hátt:

\begin{verbatim}
  cout << isSingleDigit (2) << endl;
  bool bigFlag = !isSingleDigit (17);
\end{verbatim}
%
Fyrsta línan skrifar út gildið {\tt true} vegna þess að talan 2 samanstendur af einum tölustaf.
Það vill reyndar svo til að þegar C++ skrifar út {\tt bool} gildi þá eru orðin {\tt true} og {\tt false} ekki skrifuð út 
heldur frekar heiltölurnar {\tt 1} og {\tt 0}.
%\footnote{There is a way to fix that using the {\tt boolalpha} flag, but it is too hideous to mention.}

Í annarri línunni fær breytan {\tt bigFlag} gildið {\tt true} vegna þess að talan 17 er {\em ekki} (not) tala með einum tölustaf.

Algengasta notkun á {\tt bool} föllum er innan í skilyrðissetningu: 

\begin{verbatim}
  if (isSingleDigit (x)) {
    cout << "x is little" << endl;
  } else {
    cout << "x is big" << endl;
  }
\end{verbatim}

\section {Skilagildi úr {\tt main}}

Nú þegar við höfum kynnst föllum sem skila gildum þá get ég sagt þér frá litlu leyndarmáli.
{\tt main} á í raun ekki að vera {\tt void} fall.
Það á nefnilega að skila heiltölu:

\begin{verbatim}
int main ()
{
  return 0;
}  
\end{verbatim}
%
Yfirleitt er skilagildið úr {\tt main} 0 sem gefur til kynna að forritið endaði eðlilega.
Ef eitthvað fer úrskeiðis þá er algengt að skila -1 eða einhverju öðru gildi sem gefur til kynna að villa hafi komið upp.

Þá vaknar eðlilega spurningin:
Hver tekur við skilagildinu þar sem við köllum aldrei sjálf á {\tt main}?
Það vill svo til að þegar keyrsluumhverfið keyrir C++ forrit þá byrjar það á því að kalla á {\tt main} á sambærilegan hátt og kallað er á hvaða annað fall.

Það eru meira að segja ákveðin viðföng sem {\tt main} fær frá keyrsluumhverfinu en við munum fjalla um það síðar.

\section {Meiri endurkvæmni}
\index{endurkvæmni}
\index{mál!fullkomið}

Fram að þessu höfum við aðeins lært lítið hlutmengi af C++.
Það er hins vegar athyglisvert að þetta hlutmengi er núna {\bf fullkomið} (e. complete) forritunarmál í þeim skilningi að allt það sem hægt er að reikna (e. compute) eða framkvæma með tölvu er hægt að tjá (e. express) í þessu máli.
Sérhvert forrit sem hefur verið skrifað eða mun verða skrifað er hægt að endurskrifa með því að nota þá eiginleika C++ sem við höfum skoðað fram að þessu 
(við þyrftum reyndar nokkrar skipanir í viðbót til að stjórna jaðartækjum eins og lyklaborði, mús, hörðum diski, o.s.frv., en það er allt og sumt).

\index{Turing, Alan}

Að sanna þessa staðhæfingu er reyndar ekki einfalt en það gerði Alan Turing, einn fyrsti tölvunarfræðingurinn\footnote{Sumir myndu nú reyndar segja að Alan Turing hefði verið stærðfræðingur en margir af fyrstu tölvunarfræðingunum voru það nú reyndar.}, fyrstur manna.
Í kjölfarið nefnist þessi staðhæfing ``Turing thesis''.
Ef þú tekur námskeið í reiknanleika (e. Theory of Computation) þá færðu tækifæri til að sjá sönnunina.

Til að gefa þér hugmynd um hvað þú getur gert með þeim tólum sem við höfum lært hingað til þá ætla ég að útfæra nokkur stærðfræðiföll sem skilgreind eru endurkvæmin hátt.
Endurkvæm skilgreining svipar til hringskilgreiningar í þeim skilningi að skilgreiningin inniheldur tilvísun í það sem verið er að skilgreina!
Skilgreining sem er eingöngu hringskilgreining er yfirleitt ekki mjög gagnleg:

\begin{description}

\item[skeflulegur:] lýsingarorð sem notað til að lýsa einhverju sem er skeflulegt.

\index{frabjuous}

\end{description}

Ef þú sæir þessa skilgreiningu í orðabók þá yrðir þú væntanlega pirraður.
Á hinn bóginn, ef þú flettir upp skilgreiningu á stærðfræðifallinu {\bf hrópmerkt} (e. factorial) þá sæir þú eitthvað þessu líkt:

\begin{eqnarray*}
&&  0! = 1 \\
&&  n! = n \cdot (n-1)!
\end{eqnarray*}

(Hrópmerkt er yfirleitt táknað með $!$ sem ekki má rugla saman við C++ rökvirkjann {\tt !} sem merkir NOT.)
Þessi skilgreining segir að 0 hrópmerkt er 1 og að hrópmerkt af hvaða öðru gildi, $n$, er $n$ margfaldað með $n-1$ hrópmerkt.
$3!$ er þá 3 margfaldað með $2!$, sem er 2 margfaldað með
$1!$, sem er 1 margfaldað með 0!.
Ef við tökum þetta allt saman þá fáum við $3!$ = 3 * 2 * 1 * 1, sem er 6.

Ef þú getur skrifað endurkvæma skilgreiningu fyrir eitthvað vandamál þá getur þú yfirleitt skrifað C++ forrit sem leysir það.
Fyrst þarf að ákveða hver viðföngin eru í fallið og hvert skilagildið er.
Þú ættir að sjá að fallið factorial tekur heiltölu sem viðfang og skilar af sér heiltölu:

\begin{verbatim}
int factorial (int n)
{
}
\end{verbatim}
%
Ef leppurinn hefur gildið 0 þá þurfum við einfaldlega að skila 1 (þetta er grunnþrepið):

\begin{verbatim}
int factorial (int n)
{
  if (n == 0) {
    return 1;
  }
}
\end{verbatim}
%
Annars framkvæmum við endurkvæmt kall til að finna $n-1$ hrópmerkt og margföldum það gildi með $n$.

\begin{verbatim}
int factorial (int n)
{
  if (n == 0) {
    return 1;
  } else {
    int recurse = factorial (n-1);
    int result = n * recurse;
    return result;
  }
}
\end{verbatim}
%
Ef við skoðum keyrsluflæðið fyrir þetta fall þá sjáum við að það svipar til fallsins {\tt nLines} úr síðasta kafla.
Hvað gerist ef við köllum á {\tt factorial} með gildinu 3:

Þar sem 3 er ekki núll þá keyrist {\tt else} kvíslin sem reiknar $n-1$ hrópmerkt ...

\begin{quote}
Þar sem 2 er ekki núll þá keyrist {\tt else} kvíslin sem reiknar $n-1$ hrópmerkt ...

\begin{quote}
Þar sem 1 er ekki núll þá keyrist {\tt else} kvíslin sem reiknar $n-1$ hrópmerkt ...

\begin{quote}
Þar sem 0 {\em er} núll þá keyrist fyrsta kvíslin og gildinu 1 er skilað án þess að fleiri endurkvæm köll séu framkvæmd.

\end{quote}

Skilagildið 1 er margfaldað með {\tt n} sem er 1 og niðurstöðunni skilað.

\end{quote}

Skilagildið 1 er margfaldað með {\tt n} sem er 2 og niðurstöðunni skilað.

\end{quote}

\noindent Skilagildið 2 er margfaldað með {\tt n} sem er 3 og niðurstaðan 6 er skilað til {\tt main}, eða til þess sem kallaði á {\tt factorial (3)}.

\index{stafli}
\index{rit!staða}
\index{rit!stafli}

Hér sjáum við staflarit fyrir röð fallakallanna:

\vspace{0.1in}
\centerline{\epsfig{figure=stack3.eps}}
\vspace{0.1in}
%
Við sjáum að skilagildin eru send upp staflann.

Taktu eftir að í síðasta tilvikinu á {\tt factorial} (neðst á staflanum) eru staðværu breyturnar {\tt recurse} og {\tt result} ekki til því þegar 
{\tt n=0} þá er kvíslin (blokkin) sem býr þau til ekki keyrð.

\section{Að taka trúanlegt}
\index{taka trúanlegt}

Ein leið til að lesa forrit er að fylgja keyrsluflæði þess en eins og þú sást í kaflanum hér á undan þá getur það stundum verið eins og að finna leið út úr völundarhúsi.
Önnur leið er það sem ég kalla ``að taka trúanlegt'' (e. leap of faith).
Þegar þú kemur að fallakalli þá, í stað þess að fylgja flæðinu inn í fallið, \textit{gerir þú ráð fyrir} að fallið virki rétt og skili réttu gildi.

Reyndar vill svo til að þú tekur virkni þegar trúanlega þegar þú notar innbyggð föll.
Þegar þú kallar á {\tt cos} eða {\tt exp} fallið þá skoðar þú ekki útfærsluna á þeim.
Þú hreinlega gerir ráð fyrir að þau virki vegna þess að fólkið sem skrifaði þessi innbyggðu forritunarsöfn eru góðir forritarar.

Sama gildir um þín eigin föll.
Í kafla~\ref{bool} skrifuðum við fallið {\tt isSingleDigit} sem ákvarðar hvort tala er á milli 0 og 9.
Þegar við höfðum fullvissað okkur um að þetta fall væri rétt -- með prófunum og með því að skoða kóðann -- þá gátum við notað fallið án þess að þurfa að skoða kóðann á ný.

Sama gildir um endurkvæm föll.
Þegar komið er að endurkvæma kallinu þá ættir þú, í stað þess að fylgja keyrsluflæðinu, \textit{að gera ráð fyrir} að endurkvæma kallið virki (skili réttu gildi)
og síðan spyrja sjálfan þig:
``Að því gefnu að ég geti reiknað $n-1$ hrópmerkt, get ég reiknað $n$ hrópmerkt?''
Í þessu tilviki er ljóst að það er hægt með því að margfalda með $n$.

Það er auðvitað dálítið skrýtið að gera ráð fyrir að fall virki rétt þegar maður hefur ekki einu sinni klárað að skrifa það en þetta er einmitt ástæðan fyrir því að ég segi ``taktu það trúanlegt''!

\section{Eitt dæmi í viðbót}
\index{factorial}

Í síðasta dæmi notaði ég tímabundnar breytur í sérhverju skrefi til að gera kóðann auðveldari í kembingu en ég gæti hafa sparað nokkrar línur:

\begin{verbatim}
int factorial (int n) {
  if (n == 0) {
    return 1;
  } else {
    return n * factorial (n-1);
  }
}
\end{verbatim}
%
Héðan í frá mun ég nota samþjappaðri útgáfur af föllum en ég mæli með að þú notir ``skilmerkilegri'' (e. explicit) útgáfur á með þú ert að þróa fallið þitt.
Þegar fallið er tilbúið og þú ert viss um að það virki rétt þá getur þú gert kóðann samþjappaðri.

Annað klassískt dæmi um stærðfræðifall sem skilgreint er á endurkvæman hátt er {\tt fibonacci} -- skilgreint á eftirfarandi hátt:

\begin{eqnarray*}
&& fibonacci(0) = 1 \\
&& fibonacci(1) = 1 \\
&& fibonacci(n) = fibonacci(n-1) + fibonacci(n-2);
\end{eqnarray*}
%
Ef við þýðum þetta yfir á C++ þá fáum við

\begin{verbatim}
int fibonacci (int n) {
  if (n == 0 || n == 1) {
    return 1;
  } else {
    return fibonacci (n-1) + fibonacci (n-2);
  }
}
\end{verbatim}
%
Ef þú reynir að fylgja keyrsluflæðinu í þessu falli, meira að segja fyrir lág gildi á {\tt n}, þá mun rjúka úr höfðinu á þér!
En ef við fylgjum ``taktu það trúanlegt'' og gerum ráð fyrir að endurkvæmu köllin tvö
(já, þú getur framkvæmt tvö endurkvæm köll!) virki rétt þá er ljóst að við fáum rétta niðurstöðu þegar við leggjum saman gildin úr köllunum tveimur.

\section{Orðalisti}

\begin{description}

\item[skilatag (e. return type):]  Tagið á gildinu sem fallið skilar. 

\item[skilagildi (e. return value):]  Gildið sem fallið skilar.

\item[dauður kóði (e. dead code):]  Hluti forrits sem er aldrei keyrður, oft vegna þess að hann fylgir á eftir {\tt return} setningu.
\item[stoðbúnaður (e. scaffolding):]  Kóði sem er notaður í þróunarferlinu en er ekki hluti endanlegrar útgáfu.

\item[void:]  Sérstakt skilatag sem gefur til kynna void fall, þ.e. fall sem skilar ekki neinu gildi.

\item[fjölbinding (e. overloading):]  Það að hafa fleiri en eitt fall með sama nafni en með mismunandi leppa.  C++ þekkir hvaða útgáfu falls er verið að kalla á með því að skoða viðföngin.

\item[boole (e. boolean):]  Gildi eða breyta sem getur einungis haft tvær mögulegar stöður, $true$ eða $false$.
Í C++ eru boole gildi geymd í breytum af taginu {\tt bool}.

\item[flagg (e. flag):] Breyta (af taginu {\tt bool}) sem geymir útkomu úr tilteknu skilyrði.

\item[samanburðarvirki (e. comparison operator):]  Virki sem ber saman tvö gildi og skilar boole gildi sem gefur til kynna samband á milli þolandanna.

\item[rökvirki (e. logical operator):]  Virki sem sameinar boole gildi til að prófa samsett skilyrði.

\index{skilatag}
\index{skilagildi}
\index{dauður kóði}
\index{stoðbúnaður}
\index{void}
\index{fjölbinding}
\index{boole}
\index{virki!skilyrði}
\index{virki!rök}

\end{description}

