% LaTeX source for textbook ``How to think like a computer scientist''
% Copyright (C) 1999  Allen B. Downey

% Permission is granted to copy, distribute, and/or modify this document
% under the terms of the Creative Commons Attribution-NonCommercial 3.0 
% Unported License, which is available at
% http://creativecommons.org/licenses/by-nc/3.0

% The original form of this book is LaTeX source code.  Compiling this
% code has the effect of generating a device-independent representation
% of a textbook, which can be converted to other formats and printed.

% This book was typeset by the author using latex, dvips and ps2pdf,
% among other free, open-source programs.
% The LaTeX source for this book is available from
% http://greenteapress.com/thinkcpp
% and from the SVN repository
% http://code.google.com/p/thinkcpp/

% This is an Icelandic translation/adaptation of the orginal book by Allen B. Downey

\documentclass{book}
\usepackage{epsfig}
\usepackage{makeidx}
\usepackage{url}
\usepackage[T1]{fontenc}
\usepackage[utf8]{inputenc}
%\usepackage[icelandic]{babel}
\pssilent

\title{How to think like a computer scientist}

\author{Allen B. Downey}
\date{}

\sloppy
\setlength{\topmargin}{0.75in}
\setlength{\headsep}{0.5in}
\setlength{\oddsidemargin}{1.0in}
\setlength{\evensidemargin}{.95in}
\makeindex

\renewcommand{\chaptername}{Kafli}

\begin{document}
\hyphenation{for-rit jöfnu-hneppi setninga-fræðilega setning-ar skap-andi stærðfræð-ingar vegar villuskila-boð villuskila-boðin vísbend-ingar}
%\title {How to think like a computer scientist}
\title {Hugsaðu eins og tölvunarfræðingur}
\author {Allen B. Downey \\
Þýtt og staðfært af Hrafni Loftssyni}
\date {Maí 2014}
\maketitle

\vspace{2in}
\begin{center}
%{\Large How to think like a computer scientist}
{\Large Hugsaður eins og tölvunarfræðingur}

C++ útgáfa
\vspace{0.25in}

Copyright (C) 2012  Allen B. Downey
\end{center}
\vspace{0.25in}

Permission is granted to copy, distribute, and/or modify this document
under the terms of the Creative Commons Attribution-NonCommercial 3.0 Unported
License, which is available at \url{http://creativecommons.org/licenses/by-nc/3.
0/}.

The original form of this book is \LaTeX\ source code.  Compiling this
code has the effect of generating a device-independent representation
of a textbook, which can be converted to other formats and printed.

This book was typeset by the author using latex, dvips and ps2pdf,
among other free, open-source programs.
The LaTeX source for this book is available from
\url{http://greenteapress.com/thinkcpp} and from the SVN repository
\url{http://code.google.com/p/thinkcpp}.


\frontmatter
\tableofcontents

\mainmatter
% LaTeX source for textbook ``How to think like a computer scientist''
% Copyright (C) 1999  Allen B. Downey

% This LaTeX source is free software; you can redistribute it and/or
% modify it under the terms of the GNU General Public License as
% published by the Free Software Foundation (version 2).

% This LaTeX source is distributed in the hope that it will be useful,
% but WITHOUT ANY WARRANTY; without even the implied warranty of
% MERCHANTABILITY or FITNESS FOR A PARTICULAR PURPOSE.  See the GNU
% General Public License for more details.

% Compiling this LaTeX source has the effect of generating
% a device-independent representation of a textbook, which
% can be converted to other formats and printed.  All intermediate
% representations (including DVI and Postscript), and all printed
% copies of the textbook are also covered by the GNU General
% Public License.

% This distribution includes a file named COPYING that contains the text
% of the GNU General Public License.  If it is missing, you can obtain
% it from www.gnu.org or by writing to the Free Software Foundation,
% Inc., 59 Temple Place - Suite 330, Boston, MA 02111-1307, USA.

% This is an Icelandic translation/adaptation of the orginal book by Allen B. Downey


\chapter{Forritun}

Megin markmiðið með þessari bók er að kenna þér, lesandi góður, að hugsa eins og tölvunarfræðingur.
Tölvunarfræðileg hugsun sameinar marga af bestu eiginleikum úr stærðfræði, verkfræði og náttúruvísindum.
Tölvunarfræðingar nota, eins og stærðfræðingar, formleg mál til að lýsa hugmyndum (sérstaklega útreikningum/vinnslu (e. computation)).
Eins og verkfræðingar, þá hanna þeir hluti, setja þá saman til að mynda kerfi og meta kosti og galla mismunandi valkosta.
Eins og vísindamenn, þá fylgjast þeir með hegðun flókinna kerfa, setja fram tilgátur og prófa þær. 

Einn mikilvægasti eiginleiki tölvunarfræðinga er hæfni til að {\bf leysa vandamál/verkefni}.
Hér er átt við að geta skilgreint verkefni, hugsað á skapandi hátt um lausnir og tjáð lausnaraðferðir á skýran og nákvæman máta.
Það vill einmitt svo vel til að það að læra að forrita er mjög góð leið til að þróa með sér hæfni til að leysa vandamál.
%That's why this chapter is called ``The way of the program.''

Þessi bók er hugsuð sem kennslubók í fyrsta forritunarnámskeiðinu (Forritun) við tölvunarfræðideild Háskólans í Reykjavík (HR).
Annað markmið bókarinnar er því að kenna þér grundvallaratriði í forritun -- grunn sem er nauðsynlegur fyrir önnur námskeið í tölvunarfræðináminu, eins og Gagnaskipan, Reiknirit og Forritunarmál.
%Of course, the other goal of this book is to prepare you for the Computer Science AP Exam.
%We may not take the most direct approach to that goal, though.
%For example, there are not many exercSo the
following is illegal: ises in this book that are similar to the AP questions.
%On the other hand, if you understand the concepts in this book, along with the details of programming in C++, you will have all the tools you need to do well on the exam.

\section{Hvað er forritunarmál?}
\index{forritunarmál}
\index{forritun}
\index{mál!forritun}

Í þessari bók er forritun kennd með því að nota forritunarmálið C++ sem er mjög útbreitt forritunarmál.
C++ er {\bf æðra forritunarmál} (e. high-level language) en önnur sambærileg forritunarmál sem þú hefur kannski heyrt um eru t.d. Java, C{\#} og Python.
Einkenni æðri forritunarmála er að smáatriði varðandi innri virkni tölvunnar hafa verið dregin út (e. abstracted away) sem merkir að forritarinn getur hugsað um verkefnið á æðra plani'' í stað þess að þurfa að hugsa um eiginleika tiltekins vélbúnaðar.

Andstæðan við æðri forritunarmál eru {\bf lágtæknimál} (e. low-level languages), sem stundum eru einnig kölluð vélarmál (e. machine languages) eða smalamál (e. assembly languages).
Í stuttu máli sagt þá geta tölvur aðeins keyrt forrit sem skrifuð eru í vélarmáli.
Þess vegna þarf að þýða (e. compile) forrit sem skrifað er í æðra forritunarmáli yfir á vélarmál tiltekinnar tölvu áður en forritið er keyrt.
Ef hægt er að tala um einhvern ókost við æðri forritunarmál þá er hann sá að umrædd þýðing getur tekið nokkurn tíma.

\index{færanleiki}
\index{æðra forritunarmál}
\index{lágtæknimál}
\index{mál!æðra}
\index{mál!lágtækni}

Á hinn bóginn eru kostirnir miklir.
Í fyrsta lagi er miklu auðveldara að forrita í æðra forritunarmáli heldur en í lágtæknimáli.
Með ``auðveldara'' er hér átt við að forritunin tekur minni tíma, forritið er styttra og læsilegra og mun líklegra til að vera rétt.
Í öðru lagi eru æðri forritunarmál oftast {\bf færanleg} (e. portable), í þeim skilningi að hægt er að keyra þau á mismunandi vélum með engum eða litlum breytingum.
Forrit sem skrifað eru í lágtæknimáli er aftur á móti eingöngu hægt að keyra á einni tiltekinni vél og þarfnast endurskriftar til keyrslu á annarri vélartegund.

Vegna þessara kosta eru langflest forrit í dag skrifuð í æðri forritunarmálum.
Lágtæknimál eru notuð við sérstakar aðstæður, t.d. þegar forritið er á móti tilteknum örgjörvum.

\index{þýða}
\index{túlka}

Það eru tvær leiðir til að umbreyta forriti, sem skrifað er í æðra forritunarmáli, þannig að hægt sé að keyra það á tiltekinni vél: {\bf túlkun} (e. interpretation) eða {\bf
þýðing} (e. compilation).
{\em Túlkur} (e. interpreter) er forrit sem les annað forrit $S$, sem skrifað er í æðra forritunarmáli, og líkir eftir sérhverri skipun í $S$.
Túlkur les forrit línu fyrir línu, varpar sérhverri skipun í $S$ yfir í skipun/skipanir í því máli sem túlkurinn er skrifaður í, og framkvæmir síðan þær skipanir.
Úttakið úr túlkinum er úttakið úr $S$.

\vspace{0.1in}
\centerline{\epsfig{figure=interpret.eps}}
\vspace{0.1in}

Þýðandi er forrit sem les annað forrit (sem oSo the
following is illegal: ftast er skrifað í æðra forritunarmáli) og þýðir það yfir á annað mál (oftast lágtæknimál).
Hið þýdda forrit er þá tilbúið til keyrslu síðar meir.
Inntakið í þýðandann er kallað {\bf frumkóði} (e. source code) og úttakið er {\bf markkóði} (e. target code).

Gefum okkur t.d. að þú skrifir forrit í C++.
Þú gætir notað textaritil (e. text editor) eða sérstakt C++ þróunarumhverfi (e. development environment) til að skrifa forritið og vistað það í skrá með nafninu {\tt program.cpp}.
Hér er {\tt program} eitthvað nafn sem þú gefur forritinu og viðskeytið {\tt .cpp} gefur til kynna að umrædd skrá innihaldi C++ frumkóða.

Þegar forritið hefur verið skrifað þá gætir þú keyrt þýðanda á það.
Þýðandinn myndi lesa frumkóðann, þýða hann og búa til nýja skrá með nafninu {\tt program.o} fyrir markkóðann,
og/eða {\tt program.exe} fyrir hina endanlegu keyrsluskrá (e. executable file).

\vspace{0.1in}
\centerline{\epsfig{figure=compile.eps}}
\vspace{0.1in}

Þegar keyrsluskráin er síðan keyrð þá sér undirliggjandi stýrikerfi um að hlaða forritinu upp í minni (afrita það af diski yfir í minni) og lætur síðan tölvuna byrja að keyra forritið.

Þetta ferli virðist vera flókið en sem betur fer þá eru einstök skref í ferlinu framkvæmd á sjálfvirkan máta í flestum þróunarumhverfum.
Yfirleitt þarft þú aðeins að skrifa forritið og ýta síðan á hnapp (eða skrifa skipun) sem setur þýðingu og keyrslu af stað.
Á hinn bóginn er mikilvægt fyrir þig að vita hvaða einstöku hlutar eiga sér stað í ferlinu þannig að þú getir lagfært þá ef eitthvað bjátar á. 

\section{Hvað er forrit?}

Forrit er röð skipana sem segja til um hvernig framkvæma á tiltekna vinnslu.
Vinnslan gæti t.d. verið stærðfræðileg, eins og að leysa jöfnuhneppi eða að finna rætur á margliðu.
Hún gæti jafnframt líka verið táknræn (e. symbolic), t.d. að leita að og skipta út texta í skrá eða að þýða forrit!

\index{setning}

Skipanir eða setningar líta mismunandi út í mismunandi forritunarmálum en nokkrar grunnaðgerðir eru sameiginlegar flestum málum:

\begin{description}

\item[inntak (e. input):] Sækja gögn af lyklaborði, úr skrá eða af einhverju öðru tæki.

\item[úttak (e. output):] Sýna gögn á skjánum, skrifa gögn í skrá eða í annað tæki. 

\item[stærðfræðilegar aðgerðir:] Framkvæma stærðfræðilegar grunnaðgerðir eins og samlagningu og margföldun.

\item[prófun (e. test):] Athuga hvort tiltekið skilyrði er satt/ósatt og framkvæma síðan viðeigandi röð setninga.

\item[endurtekning (e. repitition):] Endurtaka tilteknar aðgerðir, yfirleitt með einhverjum frávikum á milli endurtekninga.

\end{description}

Það kann að hljóma ótrúlega en þetta er í raun allt og sumt!
Sérhvert forrit, sem þú hefur notað/keyrt, samanstendur af aðgerðum sem þessum.
Forritun er því í raun það ferli að brjóta stórt, flókið verkefni upp í smærri og smærri einingar þangað til að sérhver eining er orðin það einföld að hægt er að leysa hana með einhverjum af þessum aðgerðum.

\section{Hvað er kembing?}
\index{kembing}
\index{aflúsun}
\index{böggur}
\index{villa}

Forritun er flókið ferli sem framkvæmt er mönnum (en ekki vélum) og þess vegna eiga forritunarvillur oft sér stað.
Af sögulegum ástæðum eru forritunarvillur oft kallaðar {\bf böggar} (e. bugs) og ferlið við að finna og leiðrétta þær er kallað {\bf kembing} eða {\bf aflúsun} (e. debugging).

Mismunandi forritunarvillur geta komið upp í forriti og það er mikilvægt að gera greinarmun á þeim til að geta fundið þær og leiðrétt hraðar.

\subsection{Þýðingarvillur}
%\index{compile-time error}
\index{þýðingarvilla}
\index{villa!þýðing}

Þýðandinn getur aðeins þýtt (búið til markkóða) fyrir forrit sem er setningafræðilega rétt.
Að öðrum kosti mistekst þýðingin og þá er ekki hægt að keyra forritið.
{\bf Málskipan} (e. syntax) vísar til hvernig forrit er samansett og hvaða reglur gilda um samsetninguna.

\index{málskipan}

Í íslensku þarf sérhver setning t.d. að byrja á stórum staf og enda á punkti.
þessi setning inniheldur málskipunarvillu. Einnig þessi hér 

Flestir lesendur eiga ekki í erfiðleikum með að skilja texta sem inniheldur nokkrar málskipunarvillur.
Við getum t.d. lesið ljóð eftir e e cummings án þess að spýja út villuskilaboðum!

Þýðendur líta aftur á móti málskipunarvillur alvarlegum augum.
Ef þýðandi finnur t.d. aðeins eina villu í forritinu þínu þá mun hann skrifa út villuskilaboð og hætta án þess að búa til keyrsluskrá.
Í því tilviki munt þú ekki geta keyrt forritið þitt.

Til að gera illt verra þá eru margar málskipunarreglur í C++ og villuskilaboðin sem koma frá þýðandanum eru stundum ekki mjög hjálpleg.
Á fyrstu vikum forritunarferils þíns munt þú líklega eyða verulegum tíma í að finna {\bf þýðingarvillur} (e. compile-time errors).
Með reynslunni munt þú hins vegar gera færri villur og finna þær mun hraðar.

\subsection{Keyrsluvillur}
\label{run-time}
\index{keyrsluvilla}
\index{villa!keyrsla}
%\index{safe language}
%\index{language!safe}

Önnur tegund af villum eru svokallaðar {\bf keyrsluvillur} (e. run-time errors), þ.e. villur sem koma ekki upp fyrr en við keyrslu forrits.

Í þeim einföldu forritum sem við munum skrifa á næstu vikum verða keyrsluvillur sjaldgæfar þannig að líklega mun líða dálítill tími þangað til þú sérð þess konar villur.


\subsection{Rökvillur og merking}
\index{merking}
\index{rökvilla}
\index{villa!rök}

{\bf Rökvillur} (e. logical errors) eða {\bf merkingarvillur} (e. semantic errors) eru þriðja tegundin af villum.
Ef forritið þitt inniheldur rökvillur þá mun það samt sem áður þýðast og keyra án þess að villuskilboð prentist út.
Forritið mun hins vegar ekki gera það sem til stóð en þó það sem þú sagðir því að gera!

Vandamálið er sem sagt það að forritið sem þú skrifaðir er ekki forritið sem þú ætlaðir þér að skrifa.
Merking forritsins er því röng.
Það getur verið erfitt að finna rökvillur því það krefst þess að þú vinnir ``aftur á bak'' með þvi að skoða úttakið og reyna að finna út hvað geti verið að.

\subsection{Tilraunakembing}

Ein mikilvægasta færnin sem þú ættir að öðlast við að lesa og vinna með þessa bók er kembing.
Þó kembing geti stundum verið pirrandi, þá getur hún jafnframt verið sá hluti forritunar sem er hvað mest krefjandi og áhugaverður.

Að sumu leyti er kembing eins og starf rannsóknarlögreglumanns.
Vísbendingar eru gefnar og finna þarf út hvaða ferli og atburðir leiddu til niðurstöðunnar. 

Kembing er einnig eins og tilraunavísindi.
Þegar þig grunar hvað fór úrskeiðis þá breytir þú forritinu og keyrir það á ný.
Ef tilgáta þín reynist rétt þá skilar forritið þitt réttri útkomu.
Ef tilgátan er aftur á móti röng þá þarftu að finna nýja.
Eða eins og Sherlock Holmes benti á: ``When you have eliminated the
impossible, whatever remains, however improbable, must be the truth.''
(úr {\em The Sign of Four} eftir A. Conan Doyle).

\index{Holmes, Sherlock}
\index{Doyle, Arthur Conan}

Sumir líta svo á að forritun og kembing sé einn og sami hluturinn, þ.e.a.s. að forritun sé það ferli að kemba forrit þangað til að það gerir það sem til stóð.
Hugmyndin er sú að þú ættir alltaf að byrja með forrit sem virkar, þ.e. forrit sem gerir {\em eitthvað}, framkvæma síðan litlar breytingar á því, með kembingu ef þess er þörf, þannig að þú sért ávallt með forrit sem virkar.

Sem dæmi um þetta má nefna stýrikerfið Linux sem inniheldur þúsundir forritslína en byrjaði sem lítið forrit sem Linus Torvalds notaði til að kanna Intel 80386 örgjörvann.
``One of Linus's earlier projects was a program that would switch between printing AAAA and BBBB.
This later evolved to Linux'' (úr {\em The Linux Users' Guide} Beta Version 1).

\index{Linux}

Í síðari köflum þessarar bókar mun ég koma með fleiri tillögur um kembingu og aðrar forritunaraðferðir.

\section{Formleg og náttúruleg mál}
\label{formal}
\index{formlegt mál}
\index{náttúrulegt mál}
\index{mál!formlegt}
\index{mál!náttúrulegt}

{\bf Náttúruleg mál} (e. natural languages) eru mál sem fólk talar, eins og enska, spænska, franska og íslenska.
Þessi mál hafa ekki verið hönnuð af mönnum (þó svo að menn reyni oft að búa þeim einhverjar reglur) heldur hafa þau þróast á náttúrulegan hátt.

{\bf Formleg mál} (e. formal languages) eru mál sem eru hönnuð af mönnum fyrir tiltekna notkun.
Stærðfræðingar nota t.d. sérstakt formlegt mál sem hentar vel til að lýsa sambandi á milli talna og tákna.
Efnafræðingar nota formlegt mál sem lýsir efnafræðilegri byggingu sameinda.
Og síðast en ekki síst:

\begin{quote}
{\bf Forritunarmál eru formleg mál sem hafa verið hönnuð til að tjá vinnslu.} 
%{\bf Programming languages are formal languages that have been designed to express computations.}
\end{quote}

Flest formleg mál hafa stífar málfræðireglur.
$3+3=6$ er t.d. setningafræðilega rétt stærðfræðileg setning en ekki $3=+6\$$.
Sem annað dæmi má nefna að $H_2O$ er rétt nafn á frumefni en ekki $_2Zz$.

Málfræðireglurnar eru af tvennum toga.
Í fyrsta lagi eru reglur um hvað eru leyfilegir tókar (e. tokens) í viðkomandi máli. 
Tókar eru grunneiningar í öllum málum, t.d. orð, tölur og nöfn á frumefnum.

Vandamálið við {\tt 3=+6\$} er að {\tt \$} er ekki leyfilegur tóki í stærðfræði.
Á sama hátt er $_2Zz$ ekki leyfilegt því það er ekkert frumefni með skammstöfunina $Zz$.

Hin tegundin af málfræðireglum, málskipunarreglur, hefur að gera með hvernig tókar eru settir saman til að mynda setningar.
T.d. er setningin {\tt 3=+6\$} er ekki rétt sett saman því á eftir gildisveitingarvirkja (e. assignment operator) getur ekki komið virkinn plús.
%Similarly, molecular formulas have to have subscripts after the element name, not before.

Þegar þú lest setningu í þínu móðurmáli eða setningu í formlegu máli þá þarftu að finna út hvernig setningin er samansett af einstökum tókum (í náttúrulegu máli þá gerum við þetta reyndar ómeðvitað). 
Þetta ferli er kallað {\bf þáttun} (e. parsing).

\index{þáttun}

Þegar þú heyrir t.d. setninguna ``The other shoe fell'' þá áttar þú þig á því (með þáttun) að ``the other shoe'' er frumlagið (e. subject) og ``fell'' er sögnin (e. verb).
Þegar þú hefur þáttað setninguna þá getur þú fundið út hver merking hennar er.
Að því gefnu að þú vitir hvað ``shoe'' er og þú vitir hver merkingin ``to fall'' er þá skilur þú heildmerkingu setningarinnar.

Þrátt fyrir að formleg mál og náttúruleg mál eigi sér margar hliðstæður, t.d. í tókum, málskipan og merkingu, þá er jafnframt margt ólíkt með þessum tegundum mála:

\index{margræðni}
\index{ofauki}
\index{bókstafsmerking}

\begin{description}

\item[margræðni (e. ambiguity):] Margræðni er mjög algeng í náttúrlegum málum en við leysum margræðni með því að nýta okkur samhengið og aðrar upplýsingar.
Formleg mál eru aftur á móti hönnuð til að vera laus við margræðni, þ.e. sérhver setning hefur nákvæmlega eina merkingu, án tillits til samhengis.

\item[ofauki (e. redundancy):] Til að koma í veg fyrir misskilning vegna margræðni þá er orðum oft ofaukið í setningum í náttúrulegum málum.
Í formlegum málum eru hins vegar litlar sem engar málalengingar og setningar í þeim eru því gagnyrtar.

\item[bókstafsmerking (e. literalness):] Orðatiltæki og myndlíkingar eru algengar í náttúrulegum málum.
Ef ég segi t.d. ``Sjaldan er ein báran stök'', þá er ég í rauninni ekki að tala um báru.  
Merking setningar í formlegu máli er hins vegar bókstafsleg, þ.e. nákvæmlega sú sem lesa má úr setningunni. 

\end{description}

Þeir sem alast upp við náttúruleg mál (þ.e. við öll) eiga oft erfitt með að venjast formlegum málum.
Að sumu leyti er munurinn á milli náttúrulegs og formlegs máls eins og munurinn á milli bundins og óbundins máls, en þó meiri:

\index{bundið mál}
\index{óbundið mál}

\begin{description}

\item[Bundið mál:] Orð eru notuð vegna þeirra hljóða sem þau mynda og einnig vegna merkingar þeirra.
Í heild sinni hefur ljóðið einhver áhrif og veldur tilfinningalegum viðbrögðum.
Margræðni er ekki bara algeng heldur oft meðvituð.

\item[Óbundið mál:] Raunveruleg merking orða er mikilvæg og uppbygging setningar hefur mikið með merkingu að gera.
Óbundið mál er yfirleitt auðveldara að greina en bundið mál, en er samt sem áður oft margrætt. 

\item[Forrit:] Merking forrits er ekki margræð og hún er bókstafleg.
Hægt er að skilja merkinguna með því að greina tóka og málskipan. 

\end{description}

Hér fylgja nokkrar ráðleggingar við lestur forrits (eða annarra formlegra mála).
Í fyrsta lagi mundu að formleg mál eru mun ``samþjappaðri'' en náttúrleg mál og því tekur lengri tíma að lesa þau.
Í öðru lagi er uppbygging forrita mjög mikilvæg og því er yfirleitt ekki hentugt að lesa þau frá byrjun til enda, vinstri til hægri.
Í staðinn skaltu læra að þátta forritið í huganum, bera kennsl á tókana og greina uppbygginguna.
Mundu að lokum að smáatriðin skipta máli.  
Smáatriði, eins og stafsetningarvillur eða greinamerkjavillur sem hægt er að komast upp með í náttúrulegum málum, geta skipt miklu máli í formlegum málum.


\section{Fyrsta forritið}
\label{hello}
\index{hello world}

Sú venja hefur skapast að fyrsta forritið sem fólk skrifar í nýju forritunarmáli er kallað ``Hello, World'' vegna þess að eina sem það gerir er að prenta út orðin ``Hello, World''.
Svona lítur forritið út í forritunarmálinu C++: 

\begin{verbatim}
#include <iostream>
using namespace std;

// main: generate some simple output

int main ()
{
  cout << "Hello, world." << endl;
  return 0;
}
\end{verbatim}
%
Sumir dæma gæði forritunarmáls með því að skoða hversu einfalt er að skrifa ``Hello, World'' forrit í málinu.
Ef við notum þennan mælikvarða þá kemur C++ ágætlega út.
Samt sem áður inniheldur þetta einfalda forrit ýmsa eiginleika sem er erfitt að skýra út fyrir byrjendum í forritun.
Til að byrja með munum við ekki skýra suma þeirra, eins og fyrstu tvær línurnar í þessu forriti.

\index{athugasemd}
\index{setning!athugasemd}

Þriðja línan byrjar á  {\tt //} sem gefur til kynna að línan er {\bf athugasemd} (e. comment).
Athugasemd er texti (oftast skrifaður í náttúrulegu máli) sem hægt er að setja inn hvar sem er í forriti í þeim tilgangi að skýra út hvað forritið, eða hluti þess, gerir.
Þegar þýðandinn rekst á {\tt //} þá hunsar hann allan textann frá þeim stað og að enda línunnar.

Í fjórðu línunni getur þú, sem stendur, hunsað orðið (tókann) {\tt int} en taktu eftir orðinu (tókanum) {\tt main}.
sem er sérstakt nafn sem gefur til kynna staðinn þar sem keyrsla forritsins hefst.
Þegar forritið hefur keyrslu þá mun það byrja með því að keyra fyrstu setninguna í {\tt main}, heldur síðan áfram í réttri röð og hættir eftir framkvæmd síðustu setningarinnar.

\index{úttak}
\index{setning!úttak}

Það eru engin takmörk á því hversu margar setningar {\tt main} getur innihaldið en í dæminu að ofan er aðeins um eina setningu að ræða.
Sú setning er úttakssetning, þ.e. setning sem skrifar tiltekin skilaboð út á skjáinn. 

{\tt cout} er sérstakur hlutur (e. object), innbyggður í C++, sem gerir þér kleift að senda tiltekið úttak á skjáinn (cout er úttaksstraumur).
Táknið {\tt <<} er {\bf virki} (e. operator) sem beitt er á {\tt cout} og streng, og veldur því að strengurinn er skrifaður út. 

\index{virki}

{\tt endl} er sérstakt tákn sem stendur fyrir enda á línu.
Þegar þú sendir {\tt endl} á {\tt cout} þá færist bendillinn (e. cursor) í næstu línu á skjánum.
Eftir það mun texti sem skrifaður er út birtast í þessari næstu línu. 

Eins og gildir um allar setningar þá endar úttakssetningin með semikommu ({\tt ;}).

Það er nokkur önnur atriði sem vert er að hafa í huga varðandi málskipan þessa fyrsta forrits.
Í fyrsta lagi að C++ notar slaufusviga (e. curly-braces) (\{ og
\}) til að gefa til kynna að hlutir eigi saman.
Í þessu tilviki er úttakssetningin inni í slaufusvigum sem þýðir að 
hún er hluti af skilgreiningunni á {\tt main}. 
Einnig er vert að benda á að setninginn er inndregin (e. indented) sem sýnir á skýran hátt hvaða setningar eru hluti af skilgreiningunni.

Á þessum tímapunkti mæli ég með því að þú setjist fyrir framan tölvuna þína og þýðir og keyrir þetta forrit.
Hvernig það er gert er háð því forritunarumhverfi\footnote{Í fyrsta forritunarnámskeiðinu í tölvunarfræði í HR er forritunarumhverfið Code::Blocks \url{http://www.codeblocks.org/} notað.} sem þú notar en ég mun héðan í frá í þessari bók gera ráð fyrir að þú kunnir að gera það.

Eins og ég nefndi áður þá tekur C++ þýðandinn hart á málskipunarvillum.
Ef þú gerir einhverjar innsláttarvillur í forritinu þá mun forritið líklega ekki þýðast.
T.d. ef þú slærð inn {\tt oistream} í stað {\tt iostream} þá færðu villuskilaboð eins og þessi:

\begin{verbatim}
hello.cpp:1: oistream.h: No such file or directory
\end{verbatim}
%
Það eru verulegar upplýsingar fólgnar í þessum villuskilaboðum en þær eru á samþjöppuðu formi og það er ekki einfalt að túlka skilaboðin.
Notandavænni þýðandi myndi sjálfsagt segja eitthvað á þessa leið: 

\begin{quote}
``Í línu 1 í frumkóðanum með nafninu hello.cpp reyndir þú að taka inn `header' skrá með nafninu oistream.
Ég fann enga skrá með því nafni en fann aftur á móti skrá með nafninu iostream.
Getur verið að þú hafir ætlað þér að taka þá skrá inn?''
\end{quote}

Því miður eru fæstir þýðendur svona vingjarnlegir!
Þýðandi er í raun ekki mjög ``greindur'' og í mörgum tilvikum eru villuskilaboðin sem þú færð í raun bara ábending um hvað geti verið að.
Það mun taka tíma fyrir þig að læra að túlka villuskilaboðin á réttan hátt.

Þrátt fyrir þetta eru þýðendur mikilvægt tól til að læra málskipunarreglur tiltekins forritunarmáls.
Gott er að byrja með forrit sem virkar (eins og hello.cpp), breyta því á ýmsan hátt og sjá hvað gerist.
Ef þú færð villuskilaboð reyndu þá að muna hver þau eru og hvað olli þeim þannig að ef þú sérð þau aftur síðar meir þá veistu hvað þarf að gera til að lagfæra villuna.

\section{Orðalisti}

\begin{description}

\item[lausn verkefnis/vandamáls (e. problem-solving):]
Ferli sem felur í sér að skilgreina vandamál, finna lausn á því og setja lausnaraðferðina fram á skýran og nákvæman máta.

\item[æðra forritunarmál (e. high-level language):]
Forritunarmál eins og C++ sem er hannað til að fólk geti auðveldlega lesið það og skrifað.

\item[lágtæknimál (e. low-level language):]
Forritunarmál sem er hannað til að tölvur geti á auðveldan hátt keyrt forrit í málinu.
Einnig kallað vélarmál (e. machine language) eða smalamál (e. assembly language).

\item[færanleiki (e. portability):]  Einkenni forrits sem gerir það að verkum að hægt er að keyra það á fleiri en einni tegund véla.

\item[formlegt mál (e. formal language):]  Öll mál sem fólk hefur búið til í sérstökum tilgangi, eins og t.d. mál sem standa fyrir stærðfræðilegar hugmyndir eða tölvuforrit, eru formleg mál.
Öll forritunarmál eru formleg mál.

\item[náttúrlegt mál (e. natural language):]
Öll mál sem töluð eru af mönnum og hafa þróast á náttúrulegan hátt.

\item[túlka (e. interpret):]  Það að keyra forrit, sem skrifað er í tilteknu máli, með því að túlka það línu fyrir línu.

\item[þýða (e. compile):]  Það að þýða forrit yfir í lágtæknimál í þeim tilgangi að geta keyrt það síðar, aftur og aftur.

\item[frumkóði (e. source code):]  Forrit, sem er (yfirleitt) skrifað í æðra forritunarmáli, áður en það er þýtt.

\item[markkóði (e. object code):]  Úttak þýðandans, þ.e. hið þýdda forrit. 

\item[keyrsluskrá (e. executable):] Markkóði sem hægt er að keyra beint á tölvu. 

%\item[algorithm:]  A general process for solving a category of problems. 

\item[forritunarvilla (e. bug):]  Villa í forriti.

\item[málskipan (e. syntax):]  Uppbygging forrits.

\item[merking (e. semantics):]  Merking forrits.

\item[þátta (e. parse):]  Að greina forrit í einstakar setningafræðilegar einingar.

\item[málskipunarvilla (e. syntax error):]  Villa í forriti sem veldur því að ekki er hægt að þátta forritið (og þar með ekki hægt að þýða það).

\item[keyrsluvilla (e. run-time error):]  Villa í forrit sem kemur upp við keyrslu þess. 

\item[rökvilla (e. logical error):]  Villa í forriti sem veldur því að það gerir eitthvað annað en það sem forritarinn ætlaðist til.

\item[kembing (e. debugging):]  Ferlið við að finna og fjarlægja allar tegundir af forritunarvillum.

\index{lausn vandamála}
\index{æðra forritunarmál}
\index{lágtæknimál}
\index{formlegt mál}
\index{náttúrulegt mál}
\index{túlka}
\index{þýða}
\index{málskipan}
\index{merking}
\index{þátta}
\index{villa}
\index{forritun!villa}
\index{kembing}
\index{aflúsun}

\end{description}
% LaTeX source for textbook ``How to think like a computer scientist''
% Copyright (C) 1999  Allen B. Downey

% This LaTeX source is free software; you can redistribute it and/or
% modify it under the terms of the GNU General Public License as
% published by the Free Software Foundation (version 2).

% This LaTeX source is distributed in the hope that it will be useful,
% but WITHOUT ANY WARRANTY; without even the implied warranty of
% MERCHANTABILITY or FITNESS FOR A PARTICULAR PURPOSE.  See the GNU
% General Public License for more details.

% Compiling this LaTeX source has the effect of generating
% a device-independent representation of a textbook, which
% can be converted to other formats and printed.  All intermediate
% representations (including DVI and Postscript), and all printed
% copies of the textbook are also covered by the GNU General
% Public License.

% This distribution includes a file named COPYING that contains the text
% of the GNU General Public License.  If it is missing, you can obtain
% it from www.gnu.org or by writing to the Free Software Foundation,
% Inc., 59 Temple Place - Suite 330, Boston, MA 02111-1307, USA.

% This is an Icelandic translation/adaptation of the orginal book by Allen B. Downey


\chapter{Breytur og tög}

\section{Meira úttak}
\index{úttak}
\index{setning!úttak}

Eins og bent var á í síðasta kafla þá getur þú haft eins margar setningar og þú vilt í {\tt main}.
T.d., til að skrifa út fleiri en eina línu: 

\begin{verbatim}
#include <iostream>
using namespace std;
// main: generate some simple output

int main ()
{
  cout << "Hello, world." << endl;     // output one line
  cout << "How are you?" << endl;      // output another
  return 0;
}
\end{verbatim}
%
Eins og hér sést þá er löglegt að setja inn athugasemdir í enda línu sem og í sér línu.

\index{String}
\index{type!String}

Þeir hlutar setninganna að ofan sem birtir eru innan gæsalappa eru kallaðir {\bf strengir} (e. strings),
vegna þess að þeir samanstanda af röð (streng) af bókstöfum.
Reyndar vill svo til að strengir get innhaldið hvað samsetningu sem er af bókstöfum, tölum, greinarmerkjum og öðrum sérstökum táknum.

%\index{newline}
\index{lína!ný}

Oft er gagnlegt að sýna úttak úr mörgum úttakssetningum í einni og sömu línunni.
Þetta er hægt að gera með því að sleppa fyrsta {\tt endl}:

\begin{verbatim}
int main ()
{
  cout << "Goodbye, ";
  cout << "cruel world!" << endl;
  return 0
}
\end{verbatim}
%
Í þessu tilviki birtist úttakið í einni línu sem {\tt Goodbye, cruel world!}.
Athugið að í forritinu er eitt bil (e. space) á milli stafarununnar ``Goodbye,'' og seinni gæsalapparinnar.
Þetta bil birtist í úttaki forritsins og bilið í forritinu sjálfu hefur því áhrif á virkni þess.

Bil sem koma fyrir utan gæsalappa í forriti hafa aftur á móti ekki áhrif á virkni forritsins.
Ég gæti t.d. hafa skrifað:

\begin{verbatim}
int main ()
{
  cout<<"Goodbye, ";
  cout<<"cruel world!"<<endl;
  return 0;
}
\end{verbatim}
%
Þetta forrit þýðist og keyrist á sama hátt og upphaflega forritið.
Skil (e. break) í enda línu (e. newline) hafa heldur ekki áhrif á virkni forritsins.
Ég gæti því einnig hafa skrifað:

\begin{verbatim}
int main(){cout<<"Goodbye, ";cout<<"cruel world!"<<endl;return 0;}
\end{verbatim}
%
Þetta myndi líka virka en eins og þú hefur væntanlega tekið eftir þá hafa þessar breytingar gert það að verkum að forrritið hefur orðið ólæsilegra.
Skil á milli lína og bil eru einmitt gagnleg fyrir okkur því lestur forritsins verður auðveldari og gerir okkur jafnframt auðveldara um vik við að finna málskipunarvillur.

\section{Gildi}
\index{gildi}
\index{tag}

{\bf Gildi} (e. value) (t.d. bókstafur eða tala) er eitt af þeim megin atriðum sem forrit vinnur með.
Einu gildin sem við höfum unnið með hingað til eru strengjagildi, eins og {\tt "Hello, world."}.
Þú (og þýðandinn) þekkir strengjagildi frá öðrum gildum vegna þess að strengir eru runa af stöfum innan gæsalappa.

Önnur dæmi um gildi eru heiltölur (e. integers) og stafir (e. characters).
Dæmi um heiltölu er {\tt 1} eða {\tt 17}. 
Þú getur skrifað út heiltölugildi á sama hátt og þú skrifar út gildi á strengjum:

\begin{verbatim}
  cout << 17 << endl;
\end{verbatim}
%
Stafagildi er bókstafur, tölustafur eða greinarmerki innan einfaldra gæsalappa, t.d. {\tt 'a'} eða {\tt '5'}.
Þú getur skrifað út stafagildi á sama hátt:

\begin{verbatim}
  cout << '}' << endl;
\end{verbatim}
%
Í þessu dæmi er slaufusvigi skrifaður út einn og sér í línu.

Það er einfalt á ruglast á hinum ýmsu tegundum af gildum eins og 
{\tt "5"}, {\tt '5'} og {\tt 5}, en ef þú skoðar greinarmerkin þá ætti að vera ljóst að
fyrsta gildið er strengur, annað gildi er stafur og þriðja gildið er heiltala.
Ástæðan fyrir mikilvægi þessarar aðgreiningar mun bráðum verða ljós.

\section {Breytur}
\index{breyta}
\index{gildi}

Einn öflugasti eiginleiki forritunarmáls er að sá að gera forriturum kleift að vinna með {\bf breytur} (e. variables).
Breyta er nafn sem stendur fyrir minnissvæði sem geymir gildi. 

Ýmsar mismunandi tegundir af breytum eru til (á sama hátt og mismunandi tegundir af gildum).
Þegar þú býrð til nýja breytu þá þarftu (í C++) að skilgreina af hvaða {\bf tagi} (e. type) breytan er.
Í C++ er stafur t.d. kallaður (skilgreindur sem) {\tt char}.
Eftirfarandi setning býr til nýja breytu með nafninu {\tt fred} sem hefur tagið {\tt char}.

\begin{verbatim}
    char fred;
\end{verbatim}
%
Ofangreind setning er kölluð {\bf yfirlýsing} (e. declaration) því setningin lýsir yfir breytu af tilteknu tagi.

Tag breytu ákvarðar hvers konar gildi hún getur geymt.
Breyta af taginu {\tt char} getur geymt stafi og það ætti ekki að koma á óvart að {\tt int} breyta geti geymt heiltölur.
Í C++ eru nokkur tög sem geta geymt strengjagildi en við skoðum það seinna í kafla~\ref{strings}.

\index{yfirlýsing}
\index{setning!yfirlýsing}

Málskipanin sem notuð er til að lýsa yfir heiltölubreytu er: 

\begin{verbatim}
  int bob;
\end{verbatim}
%
Hér er {\tt bob} eitthvað handahófskennt nafn á heiltölubreytunni.
Almennt séð er góð regla að gefa breytum nöfn sem gefa til kynna fyrir hvað þær standa.
Ef þú sæir t.d. eftirfarandi yfirlýsingar: 

\begin{verbatim}
    char firstLetter;
    char lastLetter;
    int hour, minute;
\end{verbatim}
%
þá gætir þú væntanlega getið þér til um hvaða gildi stendur til að geyma í þessum breytum.
Þetta dæmi sýnir líka málskipanina sem notuð er til þess að lýsa yfir mörgum breytum af sama tagi: {\tt hour} og {\tt minute}
eru bæði heiltölur með tagið {\tt int}.

\section{Gildisveiting}
\index{gildisveiting}
\index{setning!gildisveiting}

Þegar við höfum lýst yfir (búið til) breytum þá viljum við gjarnan geyma einhver gildi í þeim.
Það gerum við með svokölluðum {\bf gildisveitingum} (e. assignments):

\begin{verbatim}
    firstLetter = 'a';   // give firstLetter the value 'a'
    hour = 11;           // assign the value 11 to hour
    minute = 59;         // set minute to 59
\end{verbatim}
%
Þetta dæmi sýnir þrjár gildisveitingar og athugasemdirnar sýna hvernig hægt er að tala um gildisveitingarsetningar á mismunandi hátt.
Það má vera að orðaforðinn sé ruglingslegur en hugmyndin er einföld: 

\begin{itemize}
\item Þegar þú lýsir yfir breytu þá býrð þú til minnissvæði sem geymir gildi breytunnar.
\item Í gildisveitingarsetningu er breytu gefið gildi. 
\end{itemize}

Algeng leið til að tákna breytu á pappír er að teikna kassa
með nafni breytunnar fyrir utan kassann og gildi breytunnar innan í kassanum.
Þessi tegund af mynd er kölluð {\bf stöðurit} (e. state diagram) vegna þess að hún sýnir hver staða sérhverrar breytu er (hægt er að hugsa sér að breyta hafi ``state of mind'').
Eftirfarandi mynd sýnir hvaða áhrif gildisveitingarsetningarnar þrjár hafa: 

\vspace{0.1in}
\centerline{\epsfig{figure=assign.eps}}
\vspace{0.1in}

Ég nota stundum mismunandi form til að gefa til kynna mismunandi tög breytna.
Þessi form ættu að minna þig á að ein C++ reglan krefst þess að breyta hafa sama tag og gildið sem þú gefur henni.
Þú getur t.d. ekki geymt streng í {\tt int} breytu.
C++ þýðandi samþykkir ekki eftirfarandi setningu: 

\begin{verbatim}
  int hour;
  hour = "Hello.";       // WRONG !!
\end{verbatim}
%
Þessi regla veldur stundum ruglingi því þar eru margar leiðir til að breyta gildi af einu tagi í annað og C++ breytir meira að segja tagi stundum sjálkrafa.
Sem stendur skaltu muna að almenna reglan er sú að breytur og gildi þurfa að vera af sama tagi en við munum tala um sérstök tilfelli síðar.

Annað sem getur valdið misskilningi er að sumir strengir {\em líta út} eins og heiltölur en eru það ekki.
Strengurinn {\tt "123"}, sem samanstendur af stöfunum {\tt 1}, {\tt 2} og {\tt 3}, er t.d. ekki sami hluturinn og {\em talan} {\tt 123}.
Eftirfarandi gildisveiting er ólögleg:

\begin{verbatim}
  minute = "59";         // WRONG!
\end{verbatim}
%
\section{Að skrifa út breytur}
\label{output}

Þú getur skrifað út gildi breytu með því að nota sömu skipanir og við notuðum til að skrifa út ``einföld'' gildi:

\begin{verbatim}
  int hour, minute;
  char colon;

  hour = 11;
  minute = 59;
  colon = ':';

  cout << "The current time is ";
  cout << hour;
  cout << colon;
  cout << minute;
  cout << endl;
\end{verbatim}
%
Þetta forrit býr til tvær heiltölubreytur með nöfnunum {\tt hour} og {\tt minute}, og stafabreytu með nafninu {\tt colon}.
Forritið gefur þessum breytum viðeigandi gildi og notar síðan runu af úttakssetningum til að skrifa út eftirfarandi: 

\begin{verbatim}
The current time is 11:59
\end{verbatim}

Þegar við tölum um að ``skrifa út breytu'' þá eigum við við að skrifa út 
{\em gildi} breytunnar.
Til að skrifa út {\em nafn} breytu þurfum við að setja nafnið inni í gæsalappir, t.d. {\tt cout << "hour";}

Eins og við höfum áður séð þá er hægt að hafa fleiri en eitt gildi í einni og sömu úttakssetningunni.
Það gerir forritið að ofan ``hnitmiðaðra'' (e. concise):

\begin{verbatim}
  int hour, minute;
  char colon;

  hour = 11;
  minute = 59;
  colon = ':';

  cout << "The current time is " << hour << colon << minute << endl;
\end{verbatim}
%
Í einni og sömu línunni skrifar þetta forrit út streng, tvær heiltölur, staf og sérstaka gildið {\tt endl} -- þetta er mjög áhrifamikið!

\section{Lykilorð}
\index{lykilorð}

Ég nefndi að ofan að þú gætir gefið breytum hvaða nafn sem þér dettur í hug en það er ekki alls kostar rétt.
Það eru nefnilega nokkur orð í C++ sem eru frátekin vegna þess að þau eru notuð af þýðandanum til að þátta (e. parse) forritið þitt og ef þú notar þessi nöfn fyrir breytunöfn þá ``ruglast'' þýðandinn.
Meðal þessara nafna, sem kölluð eru {\bf lykilorð} (e. keywords), eru {\tt int},
{\tt char}, {\tt void}, {\tt endl} og mörg fleiri.

Heildarlistinn yfir lykilorð má finna í C++ staðlinum, sem er obinbera skilgreiningin á málinu samþykkt af 
``International Organization for Standardization (ISO)'', þann 1. september 1998.
Hægt er að hlaða niður afriti af staðlinum frá 
\begin{verbatim}
    http://www.ansi.org/
\end{verbatim}
%
Í stað þess að læra listann af lykilorðum utan að þá legg ég til að þú nýtir þér eiginleikann sem boðið er upp á í mörgum þróunarumhverfum, þ.e. ``code highlighting''.
Þetta merkir að þegar þú forritar þá birtast mismunandi hlutar forritsins í mismunandi litum.
Til að mynda gætu lykilorð birst í bláum lit, strengir í rauðum og annar kóði í svörtum.
Ef þú býrð til breytunafn sem birtist í bláum lit þá veistu að um lykilorð er að ræða!
%Í því tilviki mun þýðandinn væntanlega ``hegða sér'' undarlega.

\section{Virkjar}
\index{virki}

{\bf Virkjar} (e. operators) eru sérstök tákn sem standa fyrir einfalda útreikninga eins og samlagningu og margföldun.
Flestir virkjar í C++ gera nákvæmalega það sem þú myndir búast við vegna þess að þeir eru þekkt stærðfræðitákn.
T.d. er {\tt +} notað til að leggja saman tvær heiltölur.

Hér má sjá dæmi um leyfilegar segðir (e. expressions) hvers merking ætti að vera augljós:

\begin{verbatim}
1+1        hour-1       hour*60 + minute     minute/60
\end{verbatim}
%
Segðir geta bæði innhaldið breytunöfn og gildi.
Í sérhverju tilviki er nafn breytu skipt út fyrir gildi hennar áður en útreikningurinn er framkvæmdur.

\index{segð}

Samlagning, frádráttur og margföldun skila því sem búast má við en deiling gæti komið á óvart!
Þetta forrit:

\begin{verbatim}
  int hour, minute;
  hour = 11;
  minute = 59;
  cout << "Number of minutes since midnight: ";
  cout << hour*60 + minute << endl;
  cout << "Fraction of the hour that has passed: ";
  cout << minute/60 << endl;
\end{verbatim}
%
skilar eftirfarandi úttaki:

\begin{verbatim}
Number of minutes since midnight: 719
Fraction of the hour that has passed: 0
\end{verbatim}
%
Fyrri línan í úttakinu er það sem við mátti búast en seinni línan er skrýtin.
Gildi breytunnar {\tt minute} er 59 og 59 deilt með 60 er 0,98333 en ekki 0.
Ástæðan fyrir þessu misræmi er sú að C++ framkvæmir það sem kallast {\bf heiltöludeiling} (e. integer division).

\index{tag!int}
\index{heiltöludeiling}
\index{útreikningur!heiltala}
\index{deiling!heiltölur}
\index{þolandi}

Þegar báðir {\bf þolendur} (e. operands) virkja eru heiltölur þá verður niðurstaðan líka heiltala.
Samkvæmt skilgreiningu er niðurstaða heiltöludeilingar rúnuð {\em niður} jafnvel þó næsta heiltala sé mjög ``nálægt''.

Í þessu dæmi væri mögulegt að reikna út prósentur í stað almenns brots:

\begin{verbatim}
  cout << "Percentage of the hour that has passed: ";
  cout << minute*100/60 << endl;
\end{verbatim}
%
Niðurstaðan er:

\begin{verbatim}
Percentage of the hour that has passed: 98
\end{verbatim}
%
Í þessu tilviki er niðurstaðan líka rúnuð niður en nú er svarið aftur á móti um það bil rétt.
Til að fá enn réttara svar getum við notað annað tag á breytu en heiltölutag, þ.e. kommutölutag (e. floating-point type).
Kommutölubreytur geta geymt brot en við munum skoða kommutölur betur í næsta kafla.

\section{Forgangur aðgerða}
\index{forgangur}
%\index{order of operations}

Þegar fleiri en einn virki er notaður í segð þá fer röð aðgerðanna eftir reglum um {\bf forgang} (e. precedence).
Nákvæm skýring á forgangsreglum getur verið flókin en við getum notað eftirfarandi reglur til að byrja með:

\begin{itemize}

\item Margföldun og deiling hafa hærri forgang en samlagning og frádráttur.
M.ö.o. margfjöldun og deiling eru framkvæmdar áður en samlagning og frádráttur.
Þannig er útkoman úr {\tt 2*3-1} 5 en ekki 4 og {\tt
2/3-1} skilar -1 en ekki 1 (útkoman úr heiltöludeilingunni {\tt 2/3} er 0).

\item Ef virkjar hafa sama forgang þá eru þeim beitt frá vinstri til hægri.
Í segðinni {\tt minute*100/60} er margfjölduninni beitt fyrst sem skilar {\tt 5900/60} og lokaniðurstaðan verður þá {\tt 98}.
Ef virkjunum væri beitt frá hægri til vinstri þá yrði niðurstaðan {\tt 59*1} = {\tt 59} sem væri rangt.

\item Ef þú vilt beita annarri forgangsröðun aðgerða en þeirri innbyggðu þá getur þú notað sviga.
Segðir í svigum eru reiknaðar fyrst og því er niðurstaðan úr {\tt 2 * (3-1)} jafngild 4.
Einnig er hægt að nota sviga til að gera segðir læsilegri eins og í {\tt (minute * 100) / 60} þó svo að niðurstaðan sé sú sama án þess að nota sviga í þessu tilfelli.

\end{itemize}

\section{Stafavirkjar}
\index{Stafavirkjar}
\index{Virki!stafir}

Það er athyglisvert að hægt er að beita sömu stærðfræðivirkjum á heiltölur og stafi.
Dæmi: 

\begin{verbatim}
  char letter;
  letter = 'a' + 1;
  cout << letter << endl;
\end{verbatim}
%
Þessi forritsbútur skrifar út stafinn {\tt b}.
Þrátt fyrir að það sé í raun setningafræðilega löglegt að margfalda stafi þá er nánast aldrei nauðsyn til að gera það.

Ég hef áður sagt að það sé eingöngu hægt að gefa heiltölubreytum heiltölugildi og stafabreytum stafagildi en það er ekki alveg rétt.
Í ákveðnum tilfellum breytir C++ sjálfkrafa um tög.
Eftirfarandi forritsbútur er til að mynda löglegur:

\begin{verbatim}
  int number;
  number = 'a';
  cout << number << endl;
\end{verbatim}
%
Útkoman er 97 en það er sú tala sem notuð er í C++ til að tákna stafinn {\tt 'a'}.
Samt sem áður er almennt séð góð regla að meðhöndla stafi sem stafi og heiltölur sem heiltölur og aðeins að breyta einu tagi í annað ef nauðsyn krefur.

Sjálfvirk tagbreyting (e. type conversion) er dæmi um þekkt vandamál við hönnun forritunarmáls --
annars vegar er áhersla á {\bf formhyggju} (e. formalism) (skilyrði um að formlegt mál skuli hafa einfaldar reglur með fáum undantekningum) 
og hins vegar áhersla á {\bf þægindi} (e. convenience) (skilyrði um að forritunarmál skuli vera þægilegt í notkun).

Oftar en ekki vinna þægindin sem er yfirleitt gott fyrir vana forritara sem er þá hlíft við strangri formhyggju.
Aftur á móti er það stundum slæmt fyrir byrjendur í forritun sem finnast oft reglurnar vera flóknar og með fjölda undantekninga.
Í þessari bók hef ég reynt að einfalda hlutina með því að leggja áherslu á reglurnar og sleppa að fjalla um margar undantekningar.

\section{Samsetning}
\index{samsetning}
\index{segð}

Hingað til höfum við skoðað einstakar einingar forritunarmáls -- breytur, segðir og setningar -- án þess að ræða hvernig þær eru settar saman.

Einn gagnlegasti eiginleiki forritunarmála er geta þeirra til að setja saman litlar forritseiningar og mynda þannig heildstætt forrit.
Við vitum t.d. hvernig á að margfalda saman heiltölur og við vitum hvernig á að skrifa út gildi; svo vill reyndar til að við getum gert hvort tveggja á sama tíma:

\begin{verbatim}
    cout << 17 * 3;
\end{verbatim}
%
Reyndar ætti ég ekki að segja ``á sama tíma'' því í raun á margföldunin sér stað áður en niðurstaðan er skrifuð út 
en punkturinn er sá að að sérhver segð, sem inniheldur tölur, stafi og breytur, getur verið notuð í úttakssetningu.
Við höfum þegar séð eitt dæmi um þetta:

\begin{verbatim}
  cout << hour*60 + minute << endl;
\end{verbatim}
%
Hvaða segð sem er getur einnig komið fyrir á hægri hlið gildisveitingarsetningar (e. assignment statement): 

\begin{verbatim}
  int percentage;
  percentage = (minute * 100) / 60;
\end{verbatim}
%
Þessi eiginleiki virðist kannski ekki mikilvægur núna en við munum sjá önnur dæmi þar sem samsetning gerir okkur kleift að tjá (e. express) útreikninga á ``snyrtilegan'' og hnitmiðaðan hátt.

VIÐVÖRUN: 
Það eru takmarkanir á því hvar hægt er að nota segðir.
Þetta á sérstaklega við um vinstri hlið gildisveitingarsetningar sem verður að vera nafn á {\bf breytu} en ekki segð.
Ástæðan er sú að vinstri hliðin stendur fyrir minnissvæði sem geyma mun gildi segðarinnar á hægri hlið.
Aftur á móti standa segðir ekki fyrir minnissvæði heldur eingöngu gildi.
Eftirfarandi er því ekki löglegt: {\tt minute+1 = hour;}.

\section{Orðalisti}

\begin{description}

\item[breyta (e. variable):] Nafn sem stendur fyrir minnissvæði sem geymir gildi. 
Allar breytur í C++ hafa tag sem ákvarðar hvers konar gildi breytan getur geymt. 

\item[gildi (e. value):] Stafur, tala eða aðrir hlutir sem hægt er að geyma í breytu.

\item[tag (e. type):] Mengi af gildum. Tögin sem við höfum þegar séð eru t.d. heiltölur ({\tt int}) og stafir ({\tt char}).

\item[lykilorð (e. keyword):]  Frátkeið orð sem notað er af þýðandanum til að þátta forritið.
Dæmi um lykilorð: {\tt int}, {\tt void} og {\tt endl}.

\item[setning (e. statement):] Forritseining sem stendur fyrir tiltekna skipun eða aðgerð.
Þær setningar sem við höfum séð hingað til eru yfirlýsingar, gildisveitingar og úttakssetningar.

\item[yfirlýsing (e. declaration):] Setning sem býr til nýja breytu og tilgreinir tag hennar.

\item[gildisveiting (e. assignment):] Setning sem gefur tiltekinni breytu gildi.

\item[segð (e. expression):] Samsetning breytna, virkja og gilda sem í heild sinni stendur fyrir eitt gildi (niðurstöðu).
Segð hefur tag sem ákvarðast af þeim virkjum og þolendum sem koma fyrir í segðinni. 

\item[virki (e. operator):] Sérstakt tákn sem stendur fyrir tiltekna aðgerð eins og samlagningu eða margföldun.

\item[þolandi (operand):] Eitt af þeim gildum sem virkja er beitt á. 

\item[forgangur (e. precedence):] Segir til um í hvaða röð virkjum er beitt. 

\item[samsetning (e. composition):] Sá eiginleiki að geta sett saman einfaldar segðir og setningar til að tjá flóknar aðgerðir á samþjappaðan hátt.

\index{breyta}
\index{gildi}
\index{tag}
\index{lykilorð}
\index{setning}
\index{gildisveting}
\index{segð}
\index{virki}
\index{þolandi}
\index{samsetning}

\end{description}

% LaTeX source for textbook ``How to think like a computer scientist''
% Copyright (C) 1999  Allen B. Downey

% This LaTeX source is free software; you can redistribute it and/or
% modify it under the terms of the GNU General Public License as
% published by the Free Software Foundation (version 2).

% This LaTeX source is distributed in the hope that it will be useful,
% but WITHOUT ANY WARRANTY; without even the implied warranty of
% MERCHANTABILITY or FITNESS FOR A PARTICULAR PURPOSE.  See the GNU
% General Public License for more details.

% Compiling this LaTeX source has the effect of generating
% a device-independent representation of a textbook, which
% can be converted to other formats and printed.  All intermediate
% representations (including DVI and Postscript), and all printed
% copies of the textbook are also covered by the GNU General
% Public License.

% This distribution includes a file named COPYING that contains the text
% of the GNU General Public License.  If it is missing, you can obtain
% it from www.gnu.org or by writing to the Free Software Foundation,
% Inc., 59 Temple Place - Suite 330, Boston, MA 02111-1307, USA.

% This is an Icelandic translation/adaptation of the orginal book by Allen B. Downey

\chapter{Föll}

\section{Kommutölur}
\index{kommutala}
\index{tag!double}
%\index{double (floating-point)}

Í síðasta kafla áttum við í nokkrum vandræðum með tölur sem eru ekki heiltölur.
Við leystum vandamálið að hluta til með því að reikna út prósentur í stað brots.
Almennari lausn er hins vegar að nota kommutölur (e. floating-point) sem geta staðið fyrir bæði brot og heiltölur.
Tvenns konar kommutölur eru í C++, {\tt float} og {\tt double}.
Í þessari bók munum við eingöngu nota {\tt double} tölur.

Þú getur búið til kommutölur og gefið þeim gildi með sömu málskipan og notuð er fyrir önnur tög.
Dæmi:

\begin{verbatim}
  double pi;
  pi = 3.14159;
\end{verbatim}
%
Það er einnig mögulegt að lýsa yfir breytu og gefa henni gildi á sama tíma:

\begin{verbatim}
  int x = 1;
  String empty = "";
  double pi = 3.14159;
\end{verbatim}
%
Reyndar vill svo til að þessi leið er mjög algeng.
Samsett yfirlýsing (e. declaration) og gildisveiting (e. asssignment) er stundum kölluð {\bf upphafsstilling} (e. initialization).

\index{upphafsstilling}

Þó svo að kommutölur séu gagnlegar þá valda þær stundum ruglingi vegna þess að svo virðist sem heiltölur og kommutölur skarist. 
T.d., er gildið {\tt 1} heiltala, kommutala eða hvoru tveggja?

C++ gerir greinarmun á heiltölugildinu {\tt 1} og kommutölugildinu {\tt 1.0} þrátt fyrir að þau virðast standa fyrir sömu töluna.
Ástæðan er sú að gildin tvö tilheyra mismunandi tögum og almennt gildir að ekki er leyfilegt að framkvæma gildisveitingu ``á milli'' taga.
Þetta er t.d. ekki leyfilegt í C++

\begin{verbatim}
    int x = 1.1;
\end{verbatim}
%
vegna þess að breytan á vinstri hlið gildisveitingarinnar er af taginu {\tt int} og gildið á hægri hlið er {\tt double}.
Það er aftur á móti auðvelt að gleyma þessari reglu, sérstaklega vegna þess að í sumum tilvikum breytir C++ þýðandi einu tagi í annað á sjálfvirkan hátt.
T.d. ætti 

\begin{verbatim}
    double y = 1;
\end{verbatim}
%
tæknilega séð ekki að vera leyfilegt en C++ þýðandi leyfir þetta með því að breyta sjálfur
{\tt int} (gildinu 1) í {\tt double}.
Þessi ``linkind'' af hálfu þýðandans getur verið þægileg fyrir forritarann en getur jafnframt leitt til vandræða.
Dæmi:

\begin{verbatim}
    double y = 1 / 3;
\end{verbatim}
%
Hér gætir þú búist við því að breytan {\tt y} fái gildið {\tt 0,333333}, sem er leyfilegt kommutölugildi, en í raun fær hún gildið {\tt 0.0}.
Ástæðan er sú að segðin á hægri hlið gildisveitingarinnar er hlutfall tveggja heiltalna og því framkvæmir C++ {\em heiltöludeilingu}, hvers niðurstaða er heiltölugildið {\tt 0}.
Þegar því gildi er breytt (af þýðandanum) í kommutölu þá er niðurstaðan {\tt 0,0}.

Ein leið til að leysa þetta vandamál er að gera gildi segðarinnar á hægri hlið að kommutölu:

\begin{verbatim}
    double y = 1.0 / 3.0;
\end{verbatim}
%
Þetta veldur því að {\tt y} fær gildið {\tt 0,333333} eins og við var að búast.

\index{útreikningur!kommutala}

Allar þær reikniaðgerðir sem við höfum séð hingað til -- samlagning, frádráttur, margföldun og deiling -- virka á kommutölum sem og á heiltölum.
Það er hins vegar athyglisvert að undirliggjandi vélarmálsútreikningur er mismunandi eftir því hvort um kommutölur eða heiltölur er að ræða.
Flestir örgjörvar hafa einmitt sérstakan búnað til að framkvæma aðgerðir á kommutölum.

\section{Breyting á {\tt double} í {\tt int}}
\label{rounding}
\index{afrúnun}
\index{tagmótun}

Eins og ég nefndi áður þá breytir C++ þýðandinn {\tt int}
í {\tt double} á sjálfvirkan hátt ef þörf er á vegna þess að engar upplýsingar tapast í breytingunni.
Á hinn bóginn þá krefst breyting á {\tt double} í {\tt int} afrúnunar (e. rounding off).
C++ framkvæmir þá aðgerð ekki sjálfvirkt og því þarft þú, sem forritari, að vera meðvitaður um brotið sjálft (þ.e. sá hluti sem kemur á eftir kommunni) tapast.

Einfaldasta leiðin til að breyta kommutölu í heiltölu er að nota {\bf tagmótun} (e. typecast).
Tagmótun dregur nafn sitt af því að það gefur þér kost á því að ``móta'' gildi sem tilheyrir einu tagi yfir í annað tag.

Málskipan fyrir tagmótun er eins og málskipan fyrir fallakall.
Dæmi:

\begin{verbatim}
  double pi = 3.14159;
  int x = int (pi);
\end{verbatim}
%
{\tt int} fallið skilar heiltölu þannig að {\tt x} fær gildið 3.
Að móta kommutölu í heiltölu hefur í för með sér að talan er rúnuð niður (e. rounded down) jafnvel þó svo að brotið sé 0.99999999.

Fyrir sérhvert tag í C++ er til samsvarandi fall sem tagmótar sitt viðfang í viðeigandi tag.

\section{Stærðfræðiföll}
\index{Math function}
\index{fall!stærðfræði}
\index{segð}
\index{viðfang}

Í stærðfræði hefur þú væntanlega séð föll eins og $\sin$ og $\log$ og hefur lært að reikna út gildi segða eins og $\sin(\pi/2)$ og $\log(1/x)$.
Fyrst reiknar þú út gildi segðar innan sviga en það er kallað {\bf viðfang} (e. argument) fallsins.
T.d. er $\pi/2$ u.þ.b. 1,571 og $1/x$ er 0,1 (ef $x$ hefur gildið 10).

Eftir þetta getur þú ákvarðar gildi fallsins sjálfs, annað hvort með því að fletta upp í töflu eða með því að framkvæma ýmsa útreikninga.
$\sin$ af 1,571 is 1 og $\log$ af 0,1 er -1 (ef við gerum ráð fyrir því að $\log$ standi fyrir 
lógariþma með grunn 10).

Þetta ferli er hægt að endurtaka til að ákvarða gildi flóknari segða eins og $\log(1/\sin(\pi/2))$.
Fyrst ákvörðum við gildi viðfangs innsta fallsins (þ.e. ($\pi/2$)), síðan reiknum við út gildi fallsins (þ.e. $\sin$), o.s.frv.

C++ býður upp á mengi af innbyggðum (e. built-in) föllum sem inniheldur flestar þær stærðfræðilegu aðgerðir sem þú getur ímyndað þér.
Kallað er á þessi stærðfræðiföll með því að nota málskipan sem er sambærileg við stærðfræðilega táknun: 

\begin{verbatim}
     double log = log (17.0);
     double angle = 1.5;
     double height = sin (angle);
\end{verbatim}
%
\index{stærðfræðifall!log}
\index{stærðfræðifall!sin}

Fyrsta setningin að ofan gefur breytunni {\tt log} gildið lógariþmi af 17 (grunnur $e$).
Það er einnig til fall {\tt log10} sem reiknar út lógariþma miðað við grunn 10.

Þriðja setningin reiknar út sínus af gildinu sem geymt er í breytunni {\tt angle}.
C++ gerir ráð fyrir því að gildin sem notuð eru með sínus fallinu, og öðrum hornaföllum ({\tt cos}, {\tt tan}), séu í {\em radian}.
Til að breyta gráðum í radian getur þú deilt með 360 og margfaldað með $2 \pi$.  

Ef þú þekkir ekki gildið á $\pi$ (með 15 aukastöfum!) þá getur þú reiknað það út með því að nota {\tt acos} fallið.
Arccosínus (eða andhverfa cosínus) af -1 er $\pi$ vegna þess að cosínus af $\pi$ er -1.

\begin{verbatim}
  double pi = acos(-1.0);
  double degrees = 90;
  double angle = degrees * 2 * pi / 360.0;
\end{verbatim}
\index{pi}
%
Áður en þú getur notað eitthvað af stærðfræðiföllunum þarftu að taka inn (e. include) sérstaka {\bf hausaskrá} (e. header file) í forritið þitt.
Hausaskrá inniheldur upplýsingar um föll, sem eru skilgreind annars staðar en í þínu eigin forriti, og sem þýðandinn þarf á að halda.
Í ``Hello, world!'' forritinu tókum við t.d. inn haus nefndan {\tt iostream} með því að nota {\bf include} setningu:

\begin{verbatim}
#include <iostream>
using namespace std;
\end{verbatim}
%
{\tt iostream} inniheldur upplýsingar um inntaks- og úttaksstrauma (e. I/O streams), þ.m.t. um úttaksstrauminn {\tt cout}.
C++ notar öflugan eiginleika sem kallaður er {\bf nafnasvið} (e. namespaces) sem gerir þér t.d. kleift að útfæra {\tt cout} á þinn eigin máta.
Í flestum tilvikum notum við reyndar hina stöðluðu útfærslu sem skilgreind er í nafnasviðinu std.
Við látum þýðandann vita af þessu með línunni

\begin{verbatim}
using namespace std;
\end{verbatim}

Þumalputtareglan er sú að þú átt að skrifa {\tt using namespace std;} í hvert sinn sem þú ætlar að nota {\tt iostream}.

\index{header skrá}
\index{cmath}
\index{iostream}

Á sambærilegan hátt inniheldur {\tt cmath} hausaskráin upplýsingar um stærðfræðiföll.
Þú getur tekið þá skrá inn, ásamt {\tt iostream}, í upphafi forritsins þíns: 

\begin{verbatim}
#include <cmath>
\end{verbatim}

Hausaskrár sem byrja á `c' gefa til kynna að þær hafi upphaflega verið búnar til fyrir forritunarmálið {\bf C}.

\section {Samsetning}
\label{composition}
\index{samsetning}
\index{segð}

Föll í C++ geta verið samsett á sama hátt og stærðfræðiföll.
Þetta merkir að þú getur notað eina segð sem hluta af annarri.
Þú getur t.d. notað hvaða segð sem er sem viðfang í fall:

\begin{verbatim}
    double x = cos (angle + pi/2);
\end{verbatim}
\index{stærðfræðifall!cos}
%
Í þessari setningu er deilt í gildið {\tt pi} með tveimur og gildinu á breytunni {\tt angle} bætt við niðurstöðuna.
Summan er síðan send sem viðfang í fallið {\tt cos}.

Þú getur einnig sent niðurstöðuna úr einu falli sem viðfang í annað fall:

\begin{verbatim}
    double x = exp (log (10.0));
\end{verbatim}
\index{stærðfræðifall!exp}
%
Hér er tekinn lógariþmi (með grunn $e$) af 10 og niðurstaðan (nefnum hana $t$) síðan send inn í exp fallið sem reiknar $e$ í veldinu $t$.
Breytan {\tt x} fær að lokum gildið úr heildarniðurstöðunni sem ég vona að þú vitir hver er!

\section{Nýjum föllum bætt við}
\index{fall!skilgreining}
\index{main}
\index{fall!main}

Hingað til höfum við eingöngu notað föll sem eru innbyggð í C++ en við getum einnig bætt við nýjum föllum.
Reyndar vill svo til að við höfum þegar bætt við einu nýju falli: {\tt main}.
Fallið {\tt main} er sérstakt að því leyti til að það gefur til kynna hvar keyrsla forritsins á að byrja en málskipan fyrir {\tt main} er sú sama og fyrir hvaða aðra fallaskilgreiningu sem er:

\begin{verbatim}
  void NAFN ( LISTI AF VIÐFÖNGUM ) {
    SETNINGAR
  }
\end{verbatim}
%
Þú getur gefið fallinu þínu hvaða nafn sem er með þeirri undantekningu að þú getur hvorki kallað það 
{\tt main} né notað annað C++ lykilorð.
Listinn af viðföngum skilgreinir hvaða upplýsingar (ef nokkrar) þarf að gefa fallinu þegar það er notað (þegar {\bf kallað} (e. call)) er á það.

{\tt main} tekur engin viðföng eins og sjá má með tómum svigum {\tt ()} í skilgreiningunni á fallinu.
Fyrstu tvö föllin sem við munum skrifa taka heldur engin viðföng -- málskipanin lítur þá svona út:

\begin{verbatim}
  void newLine () {
    cout << endl;
  }
\end{verbatim}
%
Þessu falli hefur verið gefið nafnið {\tt newLine}.
Það inniheldur aðeins eina setningu, þ.e. skrifar út stafinn {\tt endl} sem stendur fyrir nýja línu.

Úr {\tt main} getum við kallað á þetta nýja fall með því að nota málskipan sem er svipuð þeirri sem við notum þegar við köllum á innbyggð föll:

\begin{verbatim}
int main ()
{
  cout << "First Line." << endl;
  newLine ();
  cout << "Second Line." << endl;
  return 0;
}
\end{verbatim}
%
Úttakið úr forritinu er:

\begin{verbatim}
First line.

Second line.
\end{verbatim}
%
Taktu eftir auka (tómu) línunni á milli línanna tveggja.
En hvað ef við þyrftum á meira bili að halda á milli línanna?
Við gætum kallað á þetta sama fall nokkrum sinnum:

\begin{verbatim}
int main ()
{
  cout << "First Line." << endl;
  newLine ();
  newLine ();
  newLine ();
  cout << "Second Line." << endl;
  return 0;
}
\end{verbatim}
%
Annar möguleiki væri sá að skrifa nýtt fall, t.d. með nafninu {\tt threeLine}, sem skrifar út þrjár nýjar línur: 

\begin{verbatim}
void threeLine ()
{
  newLine ();  newLine ();  newLine ();
}

int main ()
{
  cout << "First Line." << endl;
  threeLine ();
  cout << "Second Line." << endl;
  return 0;
}
\end{verbatim}
%
Hér eru nokkur atriði sem vert er að gefa gaum:

\begin{itemize}

\item Hægt er að kalla á sama fallið aftur og aftur. Reyndar vill svo til að það er einmitt algengt og gagnlegt.

\item Hægt er að láta eitt fall kalla á annað.
Í forritinu að ofan kallar {\tt main} á {\tt threeLine} og {\tt threeLine} kallar á {\tt newLine}.
Þetta er einnig algengt og gagnlegt. 

\item Í {\tt threeLine} skrifaði ég þrjár setningar í einni og sömu línunni sem er setningafræðilega rétt (mundu að bil og tómar línur breyta ekki merkingu forrits í C++).
Á hinn bóginn bendi ég á að það er yfirleitt betra að hafa eina setningu í hverri línu því þannig verður forritið læsilegra.
Í þessari bók brýt ég stundum þessa reglu til að spara pláss.

\end{itemize}

Á þessum tímapunkti er kannski ekki ljóst hvað ávinnst með því að búa til öll þessi föll.
Fyrir því eru margar ástæður en forritið að ofan sýnir fram á tvær þeirra:

\begin{enumerate}

\item Með því að búa til nýtt fall þá getur þú gefið safni setninga nafn.
Föll geta einfaldað forrit með því að hylja flókna reikninga/aðgerðir ``á bak við'' eina skipun og með því að nota orð sem eru okkur töm í stað ``skrýtinna'' tákna.
Hvort er læsilegra, {\tt newLine} eða {\tt cout << endl}?

\item Það að búa til fall getur eytt endurteknum kóða og þar með gert forrit styttra.
Einföld leið til að prenta t.d. níu nýjar línur í röð væri að kalla á {\tt threeLine} þrisvar sinnum.
Hvernig myndir þú prenta út 27 nýjar línur?

\end{enumerate}

\section {Skilgreiningar og notkun}

Ef við tökum saman alla kóðabútana úr kaflanum hér á undan þá lítur forritið í heild sinni svona út:

\begin{verbatim}
#include <iostream>
using namespace std;

void newLine ()
{
  cout << endl;
}

void threeLine ()
{
  newLine ();  newLine ();  newLine ();
}

int main ()
{
  cout << "First Line." << endl;
  threeLine ();
  cout << "Second Line." << endl;
  return 0;
}
\end{verbatim}

Þetta forrit inniheldur þrjár fallaskilgreiningar: {\tt newLine}, {\tt threeLine} og {\tt main}.

Skilgreiningin á {\tt main} inniheldur setningu sem kallar á {\tt threeLine}.
Á sama hátt kallar {\tt threeLine} þrisvar sinnum á {\tt newLine}.

Taktu eftir að skilgreiningin á sérhverju falli kemur á undan þeim stað þar sem fallið er notað (þar sem kallað er á það).
C++ krefst þess að skilgreiningin á falli komi á undan fyrstu noktun þess.
Þú ættir að prófa að þýða þetta forrit með föllunum í annarri röð til að sjá hvers konar villumeldingar þú færð.

\section {Forrit með mörgum föllum}

Þegar þú skoðar forrit sem inniheldur nokkur föll þá er ruglandi að lesa forritið frá ``toppi til táar''
því það endurspeglar ekki {\bf keyrsluröð} (e. order of execution) forritsins.

Keyrslan hefst alltaf í fyrstu setningunni í {\tt main}, burtséð frá því hvar {\tt main} er staðsett í forritinu (það er reyndar oft neðst í kóðanum).
Setningar eru keyrðar, ein í einu, í röð þangað til komið er að fallakalli.
Fallaköll eru eins og krókur í flæði keyrslunnar.
Í stað þess að fara í næstu setningu þá er stokkið í fyrstu línu fallsins sem kallað er á, allar setningar fallsins keyrðar og síðan stokkið til baka og þráðurinn tekinn upp þar sem frá var horfið.

Þetta hljómar svo sem einfalt en mundu að eitt fall getur kallað á annað.
Þannig að þegar við eru í miðri keyrslu á {\tt main} gætum við þurft að stökkva burt og keyra setningarnar í {\tt threeLine}.
En meðan setningarnar í {\tt threeLine} eru keyrðar þá eru við ``trufluð'' (e. interrupted) þrisvar sinnum til að keyra {\tt newLine}.

Sem betur fer þurfum við sem forritarar ekki að hafa áhyggjur af þessum stökkvum í föll því C++ þýðandinn býr til kóða sem sér um þetta fyrir okkur.
Þegar {\tt newLine} hættir heldur forritið áfram á réttum stað í {\tt threeLine} og kemst að lokum til baka í {\tt main} þar sem keyrslunni lýkur.

Boðskapurinn er sem sagt sá að þegar þú lest forritið þá skaltu ekki lesa það frá toppi til táar heldur fylgja keyrsluflæðinu (e. flow of execution).

\section {Leppar og viðföng}
\index{leppur}
\index{viðfang}

Sum af þeim innbyggðu föllum sem við höfum skoðað hafa {\bf leppa} (e. formal parameters)
en þeir geyma þau gildi sem við látum fall hafa til að það geti gert það sem það á að gera.
Ef þú þarft t.d. að finna sínus af einhverri tölu þá þarftu að gefa til kynna hver talan er.
Þ.e. {\tt sin} tekur {\tt double} gildi sem viðfang.

Sum föll hafa fleiri en einn lepp.
Dæmi um það er fallið {\tt pow} sem tekur tvö {\tt double} gildi, grunninn og veldisvísinn.

Í sérhverju þessara tilfella þarf bæði að taka fram hversu margir lepparnir eru og af hvaða tagi þeir eru.
Það ætti því ekki að koma á óvart að þegar þú skrifar fallaskilgreiningu þá inniheldur listinn yfir leppana einnig tag þeirra:
Dæmi:

\begin{verbatim}
  void printTwice (char phil) {
    cout << phil << phil << endl;
  }
\end{verbatim}
%
Þetta fall er með lepp með nafninu {\tt phil} sem er af taginu {\tt char}.
Gildið sem kemur inn, hvert svo sem það er (á þessum tímapunkti vitum við það ekki), er prentað tvisvar og ný lína á eftir.
Ég notaði hér nafnið {\tt phil} til að gefa til kynna að þú getur notað hvaða nafn sem er á leppum en almennt séð þá ættir þú að velja eitthvað meira lýsandi nafn en {\tt phil}.

Til að kalla á þetta fall verðum við að gefa því {\tt char} gildi.
Við gætum t.d. skrifað {\tt main} fallið svona:

\begin{verbatim}
  int main () {
    printTwice ('a');
    return 0;
  }
\end{verbatim}
%
{\tt char} gildið í kallinu á {\tt printTwice} er kallað {\bf viðfang} (e. actual parameter/argument).
Talað er um að senda viðfangið (e. to pass the argument) til fallsins.
Í þessu tilviki {\tt 'a'} sent sem viðfang í {\tt printTwice} sem prentar gildið út tvisvar.

Ef við hefðum breytu af taginu {\tt char} þá gætum við notað hana sem viðfang í staðinn:

\begin{verbatim}
  int main () {
    char argument = 'b';
    printTwice (argument);
    return 0;
  }
\end{verbatim}
%
Hér er eitt mikilvægt atriðið: Nafnið á breytunni sem við sendum sem viðfang hefur ekkert að gera með nafnið á leppnum ({\tt phil}).
Ég endurtek:
\begin{quote}

{\bf Nafnið á breytunni sem við sendum sem viðfang hefur ekkert að gera með nafnið á leppnum.}

\end{quote}

Nöfnin geta verið þau sömu en þau geta líka verið mismunandi.
Það er mikilvægt að gera sér grein fyrir því að leppurinn og viðfangið eru ekki sami hluturinn
þó svo að þau hafi sama gildið (í þessu tilviki stafinn {\tt 'b'}).

Gildið sem sent er sem viðfang verður að hafa sama tagið og leppurinn í fallinu sem kallað er á.
Þessi regla er mikilvæg en getur verið ruglandi því C++ breytir stundum einu tagi í annað á sjálfvirkan hátt.
Á þessum tímapunkti skaltu muna þessa almennu reglu en við munum ræða undantekningar frá henni síðar.

\section {Leppar og breytur eru staðværar}

Gildissvið (e. scope) leppa og breytna er aðeins innan í eigin föllum.
Innan marka {\tt main} er ekki neinn hlutur með nafninu {\tt phil} aðgengilegur.
Þýðandinn mun kvarta ef þú reynir að nota það nafn innan í {\tt main}.
Á sama hátt er enginn hlutur með nafninu {\tt argument} aðgengilegur inni í {\tt printTwice}.

Breytur eins og þessar eru sagðar vera {\bf staðværar} (e. local).
Það getur verið gott að teikna {\bf staflarit} (e. stack diagram) til að gera sér grein fyrir leppum og staðværum breytum.
Staflarit sýna (eins og stöðurit) gildi á sérhverri breytu en breyturnar eru innan í stærri boxum sem gefa til kynna hvaða föllum þær tilheyra.

Staflarit fyrir {\tt printTwice} lítur svona út:

\vspace{0.1in}
\centerline{\epsfig{figure=stack.eps}}
\vspace{0.1in}
%
Í hvert skipti sem kallað er á fall til nýtt {\bf tilvik} af upplýsingum um fallið (kallað {\bf kvaðningafærsla} (e. activation record)).
Sérhvert tilvik af fallinu inniheldur leppana og staðværu breyturnar.
Í myndinni er tilvik af fallinu sýnt sem box með nafni fallsins að utanverðu en að innanverðu eru breytur og leppar.

Í dæminu hefur {\tt main} eina staðværa breytu, {\tt argument}, og enga leppa.
{\tt printTwice} hefur engar staðværar breytur en einn lepp með nafninu {\tt phil}.

\section {Föll með marga leppa}
\index{leppur!margir}
\index{fall!margir leppar}
%\index{class!Time}

Málskipanin sem notuð er til að lýsa yfir og kalla á föll með mörgum leppum veldur of villum við þýðingu.
Í fyrsta lagi þarf að muna að það þarf að lýsa yfir tagi á sérhverjum lepp.
Dæmi:
\begin{verbatim}
  void printTime (int hour, int minute) {
    cout << hour;
    cout << ":";
    cout << minute;
  }
\end{verbatim}
%
Það er freistandi að skrifa {\tt (int hour, minute)} en sá ritháttur er aðeins löglegur fyrir yfirlýsingar á breytum en ekki á leppum!

Annað sem getur valdið ruglingi er að þú þarft ekki að lýsa yfir tagi á viðföngunum.
Eftirfarandi er rangt:

\begin{verbatim}
    int hour = 11;
    int minute = 59;
    printTime (int hour, int minute);   // WRONG!
\end{verbatim}
%
Ástæðan er sú að þýðandinn getur séð hvert tagið á viðföngunum, breytunum {\tt hour} og {\tt minute}, er með því að fletta upp í skilgreiningunni á þeim.
Það er sem sagt óþarfi og óleyfilegt að taka fram tagið á viðföngunum.
Rétta málskipanin er {\tt printTime (hour, minute)}.

\section {Föll sem skila gildi}
\index{frjósöm föll}
\index{föll!frjósöm}

Þú ættir að hafa tekið eftir því að sum af þeim föllum sem við höfum notað, t.d. stærðfræðiföll, skila af sér gildi.
Önnur föll, eins og {\tt newLine}, framkvæma aðgerð en skila ekki gildi.
Þetta vekur upp nokkrar spurningar:

\begin{itemize}

\item Hvað gerist ef þú kallar á fall og þú gerir ekkert við niðurstöðuna (þ.e. þú setur niðurstöðuna hvorki inn í breytu né notar hana sem hluta af stærri segð)?

\item Hvað gerist ef þú nota fall sem ekki skilar gildi sem hluta af segð, eins og {\tt newLine() + 7}?

\item Getum við skrifað föll sem skila gildum eða sitjum við uppi með að skrifa föll eins og {\tt newLine} og {\tt printTwice}?

\end{itemize}

Svarið við þriðju spurningunni er ``já'', þ.e. við getum skrifað föll sem skila gildum og við munum einmitt gera það í næstu köflum.
Ég mun láta þér eftir að svara fyrstu tveimur spurningunum með prófunum.
Það er góð leið að spyrja þýðandann í sérhvert sinn sem þig vantar svar varðandi það hvort tiltekið atriði er leyfilegt eður ei í C++.

\section{Orðalisti}

\begin{description}

\item[kommutala (e. floating-point):] Tag breytu (eða gildi) sem getur geymt brot sem og heiltölur.
Það eru nokkur kommutölutög í C++ en við munum nota {\tt double} í þessari bók.

\item[upphafsstilling (e. initialization):]  Setning sem lýsir yfir breytu og gefur henni gildi á sama tíma.

\item[fall (e. function):]  Röð setninga sem bera tiltekið nafn og framkvæma tiltekna(r) aðgerð(ir).
Föll taka 0 eða fleiri viðföng og geta skilað af sér gildi.

\item[leppur (e. parameter):]  Geymir upplýsingar sem gefnar eru upp þegar kallað er á fall.
Leppar eru eins og breytur í þeim skilningi að þeir innihelda bæði gildi og eru af tilteknu tagi.

\item[viðfang (e. argument):]  Gildi sem gefið er upp þegar kallað er á fall.
Gildið verður að vera af sama tagi og viðkomandi leppur.

\item[fallakall (e. function call):]  Veldur því að fall er keyrt. 

\index{kommutala}
\index{fall}
\index{leppur}
\index{viðfang}
\index{fallakall}
\index{upphafsstilling}

\end{description}
% LaTeX source for textbook ``How to think like a computer scientist''
% Copyright (C) 1999  Allen B. Downey

% This LaTeX source is free software; you can redistribute it and/or
% modify it under the terms of the GNU General Public License as
% published by the Free Software Foundation (version 2).

% This LaTeX source is distributed in the hope that it will be useful,
% but WITHOUT ANY WARRANTY; without even the implied warranty of
% MERCHANTABILITY or FITNESS FOR A PARTICULAR PURPOSE.  See the GNU
% General Public License for more details.

% Compiling this LaTeX source has the effect of generating
% a device-independent representation of a textbook, which
% can be converted to other formats and printed.  All intermediate
% representations (including DVI and Postscript), and all printed
% copies of the textbook are also covered by the GNU General
% Public License.

% This distribution includes a file named COPYING that contains the text
% of the GNU General Public License.  If it is missing, you can obtain
% it from www.gnu.org or by writing to the Free Software Foundation,
% Inc., 59 Temple Place - Suite 330, Boston, MA 02111-1307, USA.



\chapter{Conditionals and recursion}
\label{condrecursion}

\section{The modulus operator}
\index{modulus}
\index{operator!modulus}

The modulus operator works on integers (and integer expressions)
and yields the {\em remainder} when the first operand is divided
by the second.  In C++, the modulus operator is a percent sign,
{\tt \%}.  The syntax is exactly the same as for other operators:

\begin{verbatim}
  int quotient = 7 / 3;
  int remainder = 7 % 3;
\end{verbatim}
%
The first operator, integer division, yields 2.  The second
operator yields 1.  Thus, 7 divided by 3 is 2 with 1 left over.

The modulus operator turns out to be surprisingly useful.  For
example, you can check whether one number is divisible by
another: if {\tt x \% y} is zero, then {\tt x} is divisible
by {\tt y}.

Also, you can use the modulus operator to extract the rightmost
digit or digits from a number.  For example, {\tt x \% 10} yields
the rightmost digit of {\tt x} (in base 10).  Similarly
{\tt x \% 100} yields the last two digits.

\section{Conditional execution}
\index{conditional}
\index{statement!conditional}

In order to write useful programs, we almost always need the ability
to check certain conditions and change the behavior of the program
accordingly.  {\bf Conditional statements} give us this ability.  The
simplest form is the {\tt if} statement:

\begin{verbatim}
  if (x > 0) {
    cout << "x is positive" << endl;
  }
\end{verbatim}
%
The expression in parentheses is called the condition.
If it is true, then the statements in brackets get executed.
If the condition is not true, nothing happens.

\index{operator!comparison}
\index{comparison!operator}

The condition can contain any of the {\tt comparison operators}:

\begin{verbatim}
    x == y               // x equals y
    x != y               // x is not equal to y
    x > y                // x is greater than y
    x < y                // x is less than y
    x >= y               // x is greater than or equal to y
    x <= y               // x is less than or equal to y
\end{verbatim}
%
Although these operations are probably familiar to you, the
syntax C++ uses is a little different from mathematical
symbols like $=$, $\neq$ and $\le$.  A common error is
to use a single {\tt =} instead of a double {\tt ==}.  Remember
that {\tt =} is the assignment operator, and {\tt ==} is
a comparison operator.  Also, there is no such thing as
{\tt =<} or {\tt =>}.

The two sides of a condition operator have to be the same
type.  You can only compare {\tt ints} to {\tt ints} and
{\tt doubles} to {\tt doubles}.  Unfortunately, at this
point you can't compare {\tt String}s at all!  There is
a way to compare {\tt String}s, but we won't get to it for a couple
of chapters.

\section {Alternative execution}
\label{alternative}
\index{conditional!alternative}

A second form of conditional execution is alternative execution,
in which there are two possibilities, and the condition determines
which one gets executed.  The syntax looks like:

\begin{verbatim}
  if (x%2 == 0) {
    cout << "x is even" << endl;
  } else {
    cout << "x is odd" << endl;
  }
\end{verbatim}
%
If the remainder when {\tt x} is divided by 2 is zero, then
we know that {\tt x} is even, and this code displays a message
to that effect.  If the condition is false, the second
set of statements is executed.  Since the condition must
be true or false, exactly one of the alternatives will be
executed.

As an aside, if you think you might want to check the parity
(evenness or oddness) of numbers often, you might want to
``wrap'' this code up in a function, as follows:

\begin{verbatim}
void printParity (int x) {
  if (x%2 == 0) {
    cout << "x is even" << endl;
  } else {
    cout << "x is odd" << endl;
  }
}
\end{verbatim}
%
Now you have a function named {\tt printParity} that will display
an appropriate message for any integer you care to provide.
In {\tt main} you would call this function as follows:

\begin{verbatim}
    printParity (17);
\end{verbatim}
%
Always remember that when you {\em call} a function, you do
not have to declare the types of the arguments you provide.
C++ can figure out what type they are.  You should resist the
temptation to write things like:

\begin{verbatim}
  int number = 17;
  printParity (int number);         // WRONG!!!
\end{verbatim}

\section {Chained conditionals}
\index{conditional!chained}

Sometimes you want to check for a number of related conditions
and choose one of several actions.  One way to do this is by
{\bf chaining} a series of {\tt if}s and {\tt else}s:

\begin{verbatim}
  if (x > 0) {
    cout << "x is positive" << endl;
  } else if (x < 0) {
    cout << "x is negative" << endl;
  } else {
    cout << "x is zero" << endl;
  }
\end{verbatim}
%
These chains can be as long as you want, although they can
be difficult to read if they get out of hand.  One way to
make them easier to read is to use standard indentation,
as demonstrated in these examples.  If you keep all the
statements and squiggly-braces lined up, you are less
likely to make syntax errors and you can find them more
quickly if you do.

\section{Nested conditionals}
\index{conditional!nested}

In addition to chaining, you can also nest one conditional
within another.  We could have written the previous example
as:

\begin{verbatim}
  if (x == 0) {
    cout << "x is zero" << endl;
  } else {
    if (x > 0) {
      cout << "x is positive" << endl;
    } else {
      cout << "x is negative" << endl;
    }
  }
\end{verbatim}
%
There is now an outer conditional that contains two branches.  The
first branch contains a simple output statement, but the second
branch contains another {\tt if} statement, which has two branches
of its own.  Fortunately, those two branches are both output
statements, although they could have been conditional statements as
well.

Notice again that indentation helps make the structure
apparent, but nevertheless, nested conditionals get difficult to read
very quickly.  In general, it is a good idea to avoid them when you
can.

\index{nested structure}

On the other hand, this kind of {\bf nested structure} is common, and
we will see it again, so you better get used to it.

\section{The {\tt return} statement}
\index{return}
\index{statement!return}

The {\tt return} statement allows you to terminate the execution
of a function before you reach the end.  One reason to use it
is if you detect an error condition:

\begin{verbatim}
#include <cmath>

void printLogarithm (double x) {
  if (x <= 0.0) {
    cout << "Positive numbers only, please." << endl;
    return;
  }

  double result = log (x);
  cout << "The log of x is " << result;
}
\end{verbatim}
%
This defines a function named {\tt printLogarithm} that takes
a {\tt double} named {\tt x} as a parameter.  The first thing
it does is check whether {\tt x} is less than or equal to
zero, in which case it displays an error message and then uses
{\tt return} to exit the function.  The flow of execution
immediately returns to the caller and the remaining lines of
the function are not executed.

I used a floating-point value on the right side of the condition
because there is a floating-point variable on the left.

Remember that any time you want to use one a function from the math
library, you have to include the header file {\tt math.h}.

\section{Recursion}
\label{recursion}
\index{recursion}

I mentioned in the last chapter that it is legal for one function to
call another, and we have seen several examples of that.  I neglected
to mention that it is also legal for a function to call itself.  It
may not be obvious why that is a good thing, but it turns out to be
one of the most magical and interesting things a program can do.

For example, look at the following function:

\begin{verbatim}
void countdown (int n) {
  if (n == 0) {
    cout << "Blastoff!" << endl;
  } else {
    cout << n << endl;
    countdown (n-1);
  }
}
\end{verbatim}
%
The name of the function is {\tt countdown} and it takes a single
integer as a parameter.  If the parameter is zero, it outputs
the word ``Blastoff.''  Otherwise, it outputs the parameter and
then calls a function named {\tt countdown}---itself---passing
{\tt n-1} as an argument.

What happens if we call this function like this:

\begin{verbatim}
#include <iostream>
#include <cmath>
using namespace std;
void printLogarithm (double x) {
  if (x <= 0.0) {
    cout << "Positive numbers only, please." << endl;
    return;
  }

  double result = log (x);
  cout << "The log of x is " << result;
}

void countdown (int n) {
  if (n == 0) {
    cout << "Blastoff!" << endl;
  } else {
    cout << n << endl;
    countdown (n-1);
  }
}

int main ()
{
  countdown (3);
  return 0;
}
\end{verbatim}
%
The execution of {\tt countdown} begins with {\tt n=3}, and
since {\tt n} is not zero, it outputs the value 3, and then
calls itself...

\begin{quote}
The execution of {\tt countdown} begins with {\tt n=2}, and
since {\tt n} is not zero, it outputs the value 2, and then
calls itself...

\begin{quote}
The execution of {\tt countdown} begins with {\tt n=1}, and
since {\tt n} is not zero, it outputs the value 1, and then
calls itself...

\begin{quote}
The execution of {\tt countdown} begins with {\tt n=0}, and
since {\tt n} is zero, it outputs the word ``Blastoff!''
and then returns.
\end{quote}

The countdown that got {\tt n=1} returns.

\end{quote}

The countdown that got {\tt n=2} returns.

\end{quote}

The countdown that got {\tt n=3} returns.

\noindent And then you're back in {\tt main} (what a trip).  So the
total output looks like:

\begin{verbatim}
3
2
1
Blastoff!
\end{verbatim}
%
As a second example, let's look again at the functions
{\tt newLine} and {\tt threeLine}.

\begin{verbatim}
void newLine () {
  cout << endl;
}

void threeLine () {
  newLine ();  newLine ();  newLine ();
}
\end{verbatim}
%
Although these work, they would not be much help if I wanted
to output 2 newlines, or 106.  A better alternative would be

\begin{verbatim}
void nLines (int n) {
  if (n > 0) {
    cout << endl;
    nLines (n-1);
  }
}
\end{verbatim}
%
This program is similar to {\tt countdown}; as long as {\tt n} is
greater than zero, it outputs one newline, and then calls itself to
output {\tt n-1} additional newlines.  Thus, the total number of
newlines is {\tt 1 + (n-1)}, which usually comes out to roughly {\tt
n}.

\index{recursive}
\index{newline}

The process of a function calling itself is called {\bf recursion}, and
such functions are said to be {\bf recursive}.

\section {Infinite recursion}

In the examples in the previous section, notice that each time the
functions get called recursively, the argument gets smaller by one, so
eventually it gets to zero.  When the argument is zero, the function
returns immediately, {\em without making any recursive calls}.
This case---when the function completes without making a recursive
call---is called the {\bf base case}.

If a recursion never reaches a base case, it will go on making recursive
calls forever and the program will never terminate.  This is known as
{\bf infinite recursion}, and it is generally not considered a good
idea.

\index{recursion!infinite}
\index{infinite recursion}
\index{run-time error}

In most programming environments, a program with an infinite
recursion will not really run forever.  Eventually, something
will break and the program will report an error.  This is the
first example we have seen of a run-time error (an error that
does not appear until you run the program).

You should write a small program that recurses forever and run
it to see what happens.

\section {Stack diagrams for recursive functions}
\index{stack}
\index{diagram!state}
\index{diagram!stack}

In the previous chapter we used a stack diagram to represent the
state of a program during a function call.  The same kind
of diagram can make it easier to interpret a recursive function.

Remember that every time a function gets called it creates
a new instance that contains
the function's local variables and parameters.

This figure shows a stack diagram for countdown, called
with {\tt n = 3}:

\vspace{0.1in}
\centerline{\epsfig{figure=stack2.eps}}
\vspace{0.1in}
%
There is one instance of {\tt main} and four instances of
{\tt countdown}, each with a different value for the parameter
{\tt n}.  The bottom of the stack, {\tt countdown} with {\tt n=0}
is the base case.  It does not make a recursive call, so there
are no more instances of {\tt countdown}.

The instance of {\tt main} is empty because {\tt main} does not
have any parameters or local variables.  As an exercise, draw a
stack diagram for {\tt nLines}, invoked with the parameter {\tt n=4}.


\section{Glossary}

\begin{description}

\item[modulus:]  An operator that works on integers and yields
the remainder when one number is divided by another.  In C++
it is denoted with a percent sign ({\tt \%}).

\item[conditional:]  A block of statements that may or may not
be executed depending on some condition.

\item[chaining:]  A way of joining several conditional statements
in sequence.

\item[nesting:] Putting a conditional statement inside one or both
branches of another conditional statement.

\item[recursion:]  The process of calling the same function you
are currently executing.

\item[infinite recursion:]  A function that calls itself
recursively without every reaching the base case.  Eventually
an infinite recursion will cause a run-time error.

\index{modulus}
\index{conditional}
\index{conditional!chained}
\index{conditional!nested}
\index{recursion}
\index{recursion!infinite}
\index{infinite recursion}

\end{description}



% LaTeX source for textbook ``How to think like a computer scientist''
% Copyright (C) 1999  Allen B. Downey

% This LaTeX source is free software; you can redistribute it and/or
% modify it under the terms of the GNU General Public License as
% published by the Free Software Foundation (version 2).

% This LaTeX source is distributed in the hope that it will be useful,
% but WITHOUT ANY WARRANTY; without even the implied warranty of
% MERCHANTABILITY or FITNESS FOR A PARTICULAR PURPOSE.  See the GNU
% General Public License for more details.

% Compiling this LaTeX source has the effect of generating
% a device-independent representation of a textbook, which
% can be converted to other formats and printed.  All intermediate
% representations (including DVI and Postscript), and all printed
% copies of the textbook are also covered by the GNU General
% Public License.

% This distribution includes a file named COPYING that contains the text
% of the GNU General Public License.  If it is missing, you can obtain
% it from www.gnu.org or by writing to the Free Software Foundation,
% Inc., 59 Temple Place - Suite 330, Boston, MA 02111-1307, USA.

% This is an Icelandic translation/adaptation of the orginal book by Allen B. Downey

\chapter{Föll sem skila gildi}

\section{Skilagildi}
\index{return}
\index{setning!return}
\index{fall!skilagildi}
%\index{fruitful function}
\index{skilagildi}
\index{void}
\index{fall!void}

Sum af þeim föllum sem við höfum notað, eins og t.d. stærðfræðiföllin, skila af sér einhverju gildi.
Tilgangurinn með því að kalla á þess konar fall er að búa til nýtt gildi sem er síðan notað til að gefa breytu gildi eða sem hluta af segð.
Dæmi:

\index{stærðfræðifall!exp}
\index{stærðfræðifall!sin}

\begin{verbatim}
  double e = exp (1.0);
  double height = radius * sin (angle);
\end{verbatim}
%
Hins vegar vill svo til að öll þau föll sem við höfum sjálf skrifað hingað til eru {\bf void} föll, þ.e. föll sem skila ekki neinu gildi.
Kall í void fall er yfirleitt gert án nokkurrar gildisveitingar (því ekkert gildi kemur til baka úr fallinu!):

\begin{verbatim}
  nLines (3);
  countdown (n-1);
\end{verbatim}
%
Í þessum kafla munum við skrifa föll sem skila af sér gildum.
Það mætti segja að þessi föll beri ávöxt!
Fyrsta dæmið er {\tt area}, sem tekur {\tt double} sem viðfang, og skilar flatarmáli hrings með gefinn radíus:

\index{stærðfræðifall!acos}
\index{pi}

\begin{verbatim}
double area (double radius) {
  double pi = acos (-1.0);
  double area = pi * radius * radius;
  return area;
}
\end{verbatim}
%
Það fyrsta sem þú ættir að taka eftir er að byrjun fallaskilgreiningarinnar er öðruvísi.
Í stað {\tt void}, sem gefur til kynna void fall, þá sjáum við hér {\tt double} sem gefur til kynna að skilagildið úr þessu falli sé af taginu {\tt double}.

Taktu líka eftir að síðasta línan í fallinu er önnur útgáfa af {\tt return} setningu sem inniheldur skilagildi.
Þessi setning þýðir ``hætta strax keyrslu þessa falls og nota eftirfarandi segð sem skilagildið.''
Segðin. sem kemur á eftir lykilorðinu return. getur verið eins flókin og verða vill þannig að við gætum hafa skrifað fallið á samþjappaðri hátt:

\begin{verbatim}
double area (double radius) {
  return acos(-1.0) * radius * radius;
}
\end{verbatim}
%
Á hinn bóginn má segja að, {\bf tímabundnar} (e. temporary) breytur eins og {\tt area} gera aflúsun oft auðveldari.
Í báðum tilvikum þarf tag segðarinnar í return setningunni að passa við skilatag fallsins.
M.ö.o., þegar þú skilgreinir að skilagildið sé af taginu {\tt double} þá ert ``lofar'' þú því að fallið muni að endingu skila {\tt double}.
Þýðandinn mun kvarta ef þú reynir að skila engri segð eða segð af röngu tagi. 

\index{tímabundin breyta}
\index{breyta!tímabundin}

Stundum getur verið hentug að hafa margar return setningar í falli -- eina fyrir sérhverja kvísl í skilyrðissetningu:Sometimes it is useful to have multiple return

\begin{verbatim}
double absoluteValue (double x) {
  if (x < 0) {
    return -x;
  } else {
    return x;
  }
}
\end{verbatim}
%
Aðeins ein af þessum return setningum mun verða keyrð þar sem return setningarnar eru í mismunandi kvíslum.
Þrátt fyrir að það sé leyfilegt að hafa fleiri en eina {\tt return} setningu í falli þá skaltu muna að um leið og ein þeirra er keyrð þá hættir fallið keyrslu og mun ekki keyra þær setningar sem á eftir koma.

Kóði sem kemur á eftir {\tt return} setningu, eða á stað sem flæðið mun ekki komast í, er kallaður {\bf dauður kóði} (e. dead code}.
Sumir þýðendur gefa einmitt aðvaranir ef hluti kóða er ``dauður''.

\index{dauður kóði}

Ef þú setur return setningu innan í skilyrðissetningu þá þarftu að sjá til þess að {\em sérhver möguleg leið} gegnum forritið lendi að lokum á return setningu.
Dæmi: 

\begin{verbatim}
double absoluteValue (double x) {
  if (x < 0) {
    return -x;
  } else if (x > 0) {
    return x;
  }                          // WRONG!!
}
\end{verbatim}
%
Þetta forrit er ekki rétt vegna þess að ef {\tt x} er 0 þá er hvorugt skilyrðanna satt og fallið mun þá hætta keyrslu án þess að framkvæma return setningu. then
Því miður þá grípa ekki allir C++ þýðendur þessa villu.
Því má vera að forritið þýðist og keyrist en þegar {\tt x==0} þá getur skilagildið í raun verið hvað sem er og líklega mismunandi eftir ólíkum umhverfum.

\index{tölugildi}
\index{villa!á þýðandatíma}
%\index{compile-time error}

Núna ert þú líklega orðin(n) hundleið(ur) á þýðandavillum en eftir því sem reynslan eykst þá áttar þú þig á því 
að það eina sem er verra en að fá þýðandavillu er að fá {\em ekki} þýðandavillu þegar forritið er ekki rétt!

Hér er dæmi um eitthvað sem gæti gerst: Þú prófar {\tt absoluteValue} með ýmsum mismunandi gildum á {\tt x} og það virðist virka rétt.
Þú lætur síðan einhvern annan fá forritið og viðkomandi prófar það í öðru umhverfi.
Á einhvern dularfullan hátt skilar það ekki réttu gildi og eftir nokkra daga aflúsun kemstu að því að útfærslan á {\tt absoluteValue} er ekki rétt.
Bara ef þýðandinn hefði aðvarað þig!

%\index{compile-time error}
%\index{error!compile-time}
\index{aflúsun}

Framvegis skaltu ekki álasa þýðandanum ef hann bendir á villu í forritinu þínu.
Þú skalt frekar þakka honum fyrir a finna villu og spara þér nokkrar daga vinnu við aflúsun.
Sumum þýðendur hafa valkost sem segir þeiim að vera sérstaklega ``strangur'' og greina frá öllum villum sem þeir finna.
Þú ættir alltaf að velja þennan valkost í þínum þýðanda.

\index{stærðfræðifall!fabs}

Sem innskot þá bendi ég á að það er fall í math safninu sem heitir {\tt fabs}.
Það reiknar tölugildið á {\tt double} -- á réttan hátt.

\section{Program development}
\label{distance}
\index{program development}

At this point you should be able to look at complete C++ functions
and tell what they do.  But it may not be clear yet how to go
about writing them.  I am going to suggest one technique that
I call {\bf incremental development}.

\index{incremental development}
\index{program development}

As an example, imagine you want to find the distance between two
points, given by the coordinates $(x_1, y_1)$ and $(x_2, y_2)$.  By
the usual definition,

\begin{equation}
distance = \sqrt{(x_2 - x_1)^2 + (y_2 - y_1)^2}
\end{equation}
%
The first step is to consider what a {\tt distance} function
should look like in C++.  In other words, what are the inputs
(parameters) and what is the output (return value).

In this case, the two points are the parameters, and it is natural to
represent them using four {\tt double}s.  The return value is the
distance, which will have type {\tt double}.

Already we can write an outline of the function:

\begin{verbatim}
double distance (double x1, double y1, double x2, double y2) {
  return 0.0;
}
\end{verbatim}
%
The {\tt return} statement is a placekeeper so that the function will
compile and return something, even though it is not the right answer.
At this stage the function doesn't do anything useful, but it is
worthwhile to try compiling it so we can identify any syntax errors
before we make it more complicated.

In order to test the new function, we have to call it with
sample values.  Somewhere in {\tt main} I would add:

\begin{verbatim}
  double dist = distance (1.0, 2.0, 4.0, 6.0);
  cout << dist << endl;
\end{verbatim}
%
I chose these values so that the horizontal
distance is 3 and the vertical distance is 4; that way,
the result will be 5 (the hypotenuse of a 3-4-5 triangle).
When you are testing a function, it is useful to know the right
answer.

Once we have checked the syntax of the function definition, we
can start adding lines of code one at a time.  After each
incremental change, we recompile and run the program.  That
way, at any point we know exactly where the error must be---in
the last line we added.

The next step in the computation is to find the differences
$x_2 - x_1$ and $y_2 - y_1$.  I will store those values in
temporary variables named {\tt dx} and {\tt dy}.

\begin{verbatim}
double distance (double x1, double y1, double x2, double y2) {
  double dx = x2 - x1;
  double dy = y2 - y1;
  cout << "dx is " << dx << endl;
  cout << "dy is " << dy << endl;
  return 0.0;
}
\end{verbatim}
%
I added output statements that will let me check the intermediate
values before proceeding.  As I mentioned, I already know that they
should be 3.0 and 4.0.

\index{scaffolding}

When the function is finished I will remove the output statements.  Code
like that is called {\bf scaffolding}, because it is helpful for
building the program, but it is not part of the final product.
Sometimes it is a good idea to keep the scaffolding around, but
comment it out, just in case you need it later.

The next step in the development is to square {\tt dx} and {\tt dy}.
We could use the {\tt pow} function, but it is simpler and
faster to just multiply each term by itself.

\begin{verbatim}
double distance (double x1, double y1, double x2, double y2) {
  double dx = x2 - x1;
  double dy = y2 - y1;
  double dsquared = dx*dx + dy*dy;
  cout << "dsquared is " << dsquared;
  return 0.0;
}
\end{verbatim}
%
Again, I would compile and run the program at this stage
and check the intermediate value (which should be 25.0).

Finally, we can use the {\tt sqrt} function to compute and
return the result.

\begin{verbatim}
double distance (double x1, double y1, double x2, double y2) {
  double dx = x2 - x1;
  double dy = y2 - y1;
  double dsquared = dx*dx + dy*dy;
  double result = sqrt (dsquared);
  return result;
}
\end{verbatim}
%
Then in {\tt main}, we should output and check the value of the result.

As you gain more experience programming, you might find yourself
writing and debugging more than one line at a time.  Nevertheless,
this incremental development process can save you a lot of
debugging time.

The key aspects of the process are:

\begin{itemize}

\item Start with a working program and make small, incremental
changes.  At any point, if there is an error, you will know
exactly where it is.

\item Use temporary variables to hold intermediate values so
you can output and check them.

\item Once the program is working, you might want to remove
some of the scaffolding or consolidate multiple statements into
compound expressions, but only if it does not make the program
difficult to read.

\end{itemize}

\section{Composition}
\index{composition}

As you should expect by now, once you define a new function,
you can use it as part of an expression, and you can build
new functions using existing functions.  For example, what if someone
gave you two points, the center of the circle and a point on
the perimeter, and asked for the area of the circle?

Let's say the center point is stored in the variables {\tt xc}
and {\tt yc}, and the perimeter point is in {\tt xp} and
{\tt yp}.  The first step is to find the radius of the circle, which
is the distance between the two points.  Fortunately, we have
a function, {\tt distance}, that does that.

\begin{verbatim}
  double radius = distance (xc, yc, xp, yp);
\end{verbatim}
%
The second step is to find the area of a circle with that
radius, and return it.

\begin{verbatim}
  double result = area (radius);
  return result;
\end{verbatim}
%
Wrapping that all up in a function, we get:

\begin{verbatim}
double fred (double xc, double yc, double xp, double yp) {
  double radius = distance (xc, yc, xp, yp);
  double result = area (radius);
  return result;
} 
\end{verbatim}
%
The name of this function is {\tt fred}, which may seem odd.  I will
explain why in the next section.

The temporary variables {\tt radius} and {\tt area} are
useful for development and debugging, but once the program is
working we can make it more concise by composing
the function calls:

\begin{verbatim}
double fred (double xc, double yc, double xp, double yp) {
  return area (distance (xc, yc, xp, yp));
} 
\end{verbatim}

\section{Overloading}
\label{overloading}
\index{overloading}

In the previous section you might have noticed that {\tt fred}
and {\tt area} perform similar functions---finding
the area of a circle---but take different parameters.  For
{\tt area}, we have to provide the radius; for {\tt fred}
we provide two points.

If two functions do the same thing, it is natural to give them
the same name.  In other words, it would make more sense if
{\tt fred} were called {\tt area}.

Having more than one function with the same name, which is called {\bf
overloading}, is legal in C++ {\em as long as each version takes
different parameters}.  So we can go ahead and rename {\tt fred}:

\begin{verbatim}
double area (double xc, double yc, double xp, double yp) {
  return area (distance (xc, yc, xp, yp));
} 
\end{verbatim}
%
This looks like a recursive function, but it is not.  Actually,
this version of {\tt area} is calling the other version.
When you call an overloaded function, C++ knows which version you
want by looking at the arguments that you provide.  If you write:

\begin{verbatim}
    double x = area (3.0);
\end{verbatim}
%
C++ goes looking for a function named {\tt area} that
takes a {\tt double} as an argument, and so it uses the
first version.  If you write

\begin{verbatim}
    double x = area (1.0, 2.0, 4.0, 6.0);
\end{verbatim}
%
C++ uses the second version of {\tt area}.  

Many of the built-in C++ commands are overloaded, meaning that there
are different versions that accept different numbers or types of
parameters.

Although overloading is a useful feature, it should be used with
caution.  You might get yourself nicely confused if you are trying to
debug one version of a function while accidently calling a different
one.

Actually, that reminds me of one of the cardinal rules of debugging:
{\bf make sure that the version of the program you are looking at is
the version of the program that is running!}  Some time you may find
yourself making one change after another in your program, and seeing
the same thing every time you run it.  This is a warning sign that for
one reason or another you are not running the version of the program
you think you are.  To check, stick in an output statement (it
doesn't matter what it says) and make sure the behavior of the
program changes accordingly.

\section{Boolean values}
\index{boolean}
\index{value!boolean}

The types we have seen so far are pretty big.  There are a lot
of integers in the world, and even more floating-point numbers.
By comparison, the set of characters is pretty small.  Well, there
is another type in C++ that is even smaller.  It is called
{\bf boolean}, and the only values in it are
{\tt true} and {\tt false}.

Without thinking about it, we have been using boolean values for the
last couple of chapters.  The condition inside an {\tt if}
statement or a {\tt while} statement is a boolean expression.
Also, the result of a comparison operator is a boolean value.
For example:

\begin{verbatim}
  if (x == 5) {
    // do something
  }
\end{verbatim}
%
The operator {\tt ==} compares two integers and produces a
boolean value.

\index{operator!comparison}
\index{comparison operator}

The values {\tt true} and {\tt false} are keywords in C++,
and can be used anywhere a boolean expression is called for.
For example, 

\begin{verbatim}
  while (true) {
    // loop forever
  }
\end{verbatim}
%
is a standard idiom for a loop that should run forever (or
until it reaches a {\tt return} or {\tt break} statement).

\section{Boolean variables}
\index{type!{\tt bool}}

As usual, for every type of value, there is a corresponding
type of variable.  In C++ the boolean type is called {\bf bool}.
Boolean variables work just like the other types:

\begin{verbatim}
  bool fred;
  fred = true;
  bool testResult = false;
\end{verbatim}
%
The first line is a simple variable declaration;
the second line is an assignment, and the third line is a
combination of a declaration and as assignment, 
called an initialization.

\index{initialization}
\index{statement!initialization}

As I mentioned, the result of a comparison operator is a boolean,
so you can store it in a {\tt bool} variable

\begin{verbatim}
  bool evenFlag = (n%2 == 0);     // true if n is even
  bool positiveFlag = (x > 0);    // true if x is positive
\end{verbatim}
%
and then use it as part of a conditional statement later

\begin{verbatim}
  if (evenFlag) {
    cout << "n was even when I checked it" << endl;
  }
\end{verbatim}
%
A variable used in this way is called a {\bf flag},
since it flags the presence or absence of some condition.

\index{flag}

\section{Logical operators}
\index{logical operator}
\index{operator!logical}

There are three {\bf logical operators} in C++: AND, OR and NOT,
which are denoted by the symbols {\tt \&\&}, {\tt ||} and
{\tt !}.  The semantics (meaning) of these operators is similar
to their meaning in English.  For example {\tt x > 0 \&\& x < 10}
is true only if {\tt x} is greater than zero AND less than 10.

\index{semantics}

{\tt evenFlag || n\%3 == 0} is true if {\em either} of
the conditions is true, that is, if {\tt evenFlag} is true OR the
number is divisible by 3.

Finally, the NOT operator has the effect of negating or
inverting a bool expression, so {\tt !evenFlag} is true
if {\tt evenFlag} is false; that is, if the number is odd.

\index{nested structure}

Logical operators often provide a way to simplify nested
conditional statements.  For example, how would you write
the following code using a single conditional?

\begin{verbatim}
  if (x > 0) {
    if (x < 10) {
      cout << "x is a positive single digit." << endl;
    }
  }
\end{verbatim}

\section{Bool functions}
\label{bool}
\index{bool}
\index{function!bool}

Functions can return {\tt bool} values just like any other type,
which is often convenient for hiding complicated tests inside
functions.  For example:

\begin{verbatim}
bool isSingleDigit (int x)
{
  if (x >= 0 && x < 10) {
    return true;
  } else {
    return false;
  }
}
\end{verbatim}
%
The name of this function is {\tt isSingleDigit}.  It is common
to give boolean functions names that sound like yes/no questions.
The return type is {\tt bool}, which means that every return
statement has to provide a {\tt bool} expression.

The code itself is straightforward, although it is a bit longer than
it needs to be.  Remember that the expression {\tt x >= 0 \&\& x < 10}
has type {\tt bool}, so there is nothing wrong with returning it
directly, and avoiding the {\tt if} statement altogether:

\begin{verbatim}
bool isSingleDigit (int x)
{
  return (x >= 0 && x < 10);
}
\end{verbatim}
%
In {\tt main} you can call this function in the usual ways:

\begin{verbatim}
  cout << isSingleDigit (2) << endl;
  bool bigFlag = !isSingleDigit (17);
\end{verbatim}
%
The first line outputs the value {\tt true} because 2 is a
single-digit number.  Unfortunately, when C++ outputs {\tt bool}s, it
does not display the words {\tt true} and {\tt false}, but rather the
integers {\tt 1} and {\tt 0}.\footnote{There is a way to fix that
using the {\tt boolalpha} flag, but it is too hideous to mention.}

The second line assigns the value {\tt true} to {\tt bigFlag}
only if 17 is {\em not} a single-digit number.

The most common use of {\tt bool} functions is inside conditional
statements

\begin{verbatim}
  if (isSingleDigit (x)) {
    cout << "x is little" << endl;
  } else {
    cout << "x is big" << endl;
  }
\end{verbatim}

\section {Returning from {\tt main}}

Now that we have functions that return values, I can let you in
on a secret.  {\tt main} is not really supposed to be a {\tt void}
function.  It's supposed to return an integer:

\begin{verbatim}
int main ()
{
  return 0;
}  
\end{verbatim}
%
The usual return value from {\tt main} is 0, which indicates that
the program succeeded at whatever it was supposed to to.  If something
goes wrong, it is common to return -1, or some other value that
indicates what kind of error occurred.

Of course, you might wonder who this value gets returned to, since
we never call {\tt main} ourselves.  It turns out that when the
system executes a program, it starts by calling {\tt main}
in pretty much the same way it calls all the other functions.

There are even some parameters that are passed to {\tt main}
by the system, but we are not going to deal with them for a little
while.

\section {More recursion}
\index{recursion}
\index{language!complete}

So far we have only learned a small subset of C++, but you
might be interested to know that this subset is now
a {\bf complete} programming language, by which I
mean that anything that can be computed can be expressed in this
language.  Any program ever written could be rewritten
using only the language features we have used so far (actually, we
would need a few commands to control devices like the keyboard, mouse,
disks, etc., but that's all).

\index{Turing, Alan}

Proving that claim is a non-trivial exercise first
accomplished by Alan Turing, one of the first computer scientists
(well, some would argue that he was a mathematician, but a lot of the
early computer scientists started as mathematicians).  Accordingly, it
is known as the Turing thesis.  If you take a course on the Theory of
Computation, you will have a chance to see the proof.

To give you an idea of what you can do with the tools we have learned
so far, we'll evaluate a few recursively-defined
mathematical functions.  A recursive definition is similar to a
circular definition, in the sense that the definition contains a
reference to the thing being defined.  A truly circular definition is
typically not very useful:

\begin{description}

\item[frabjuous:] an adjective used to describe
something that is frabjuous.

\index{frabjuous}

\end{description}

If you saw that definition in the dictionary, you might be
annoyed.  On the other hand, if you looked up the definition
of the mathematical function {\bf factorial}, you might
get something like:

\begin{eqnarray*}
&&  0! = 1 \\
&&  n! = n \cdot (n-1)!
\end{eqnarray*}

(Factorial is usually denoted with the symbol $!$, which is
not to be confused with the C++ logical operator {\tt !} which
means NOT.)  This definition says that the factorial of 0 is 1,
and the factorial of any other value, $n$, is $n$ multiplied
by the factorial of $n-1$.  So $3!$ is 3 times $2!$, which is 2 times
$1!$, which is 1 times 0!.  Putting it all together, we get
$3!$ equal to 3 times 2 times 1 times 1, which is 6.

If you can write a recursive definition of something, you can usually
write a C++ program to evaluate it.  The first step is to decide what
the parameters are for this function, and what the return type is.
With a little thought, you should conclude that factorial takes an
integer as a parameter and returns an integer:

\begin{verbatim}
int factorial (int n)
{
}
\end{verbatim}
%
If the argument happens to be zero, all we have to do is
return 1:

\begin{verbatim}
int factorial (int n)
{
  if (n == 0) {
    return 1;
  }
}
\end{verbatim}
%
Otherwise, and this is the interesting part, we have to make
a recursive call to find the factorial of $n-1$, and then
multiply it by $n$.

\begin{verbatim}
int factorial (int n)
{
  if (n == 0) {
    return 1;
  } else {
    int recurse = factorial (n-1);
    int result = n * recurse;
    return result;
  }
}
\end{verbatim}
%
If we look at the flow of execution for this program,
it is similar to {\tt nLines} from the previous chapter.
If we call {\tt factorial} with the value 3:

Since 3 is not zero, we take the second branch and calculate
the factorial of $n-1$...

\begin{quote}
Since 2 is not zero, we take the second branch and calculate
the factorial of $n-1$...

\begin{quote}
Since 1 is not zero, we take the second branch and calculate
the factorial of $n-1$...

\begin{quote}
Since 0 {\em is} zero, we take the first branch and return
the value 1 immediately without making any more recursive
calls.

\end{quote}

The return value (1) gets multiplied by {\tt n}, which is 1,
and the result is returned.

\end{quote}

The return value (1) gets multiplied by {\tt n}, which is 2,
and the result is returned.

\end{quote}

\noindent The return value (2) gets multiplied by {\tt n}, which is 3,
and the result, 6, is returned to {\tt main}, or whoever
called {\tt factorial (3)}.

\index{stack}
\index{diagram!state}
\index{diagram!stack}

Here is what the stack diagram looks like for this sequence of
function calls:

\vspace{0.1in}
\centerline{\epsfig{figure=stack3.eps}}
\vspace{0.1in}
%
The return values are shown being passed back up the stack.

Notice that in the last instance of {\tt factorial}, the local
variables {\tt recurse} and {\tt result} do not exist because
when {\tt n=0} the branch that creates them does not execute.

\section{Leap of faith}
\index{leap of faith}

Following the flow of execution is one way to read programs, but as
you saw in the previous section, it can quickly become labarynthine.
An alternative is what I call the ``leap of faith.''  When you come to
a function call, instead of following the flow of execution, you
{\em assume} that the function works correctly and returns the
appropriate value.

In fact, you are already practicing this leap of faith
when you use built-in functions.  When you call {\tt cos}
or {\tt exp}, you don't examine the implementations of 
those functions.  You just assume that they work, because the people
who wrote the built-in libraries were good programmers.

Well, the same is true when you call one of your own functions.
For example, in Section~\ref{bool} we wrote a function called
{\tt isSingleDigit} that determines whether a number is between
0 and 9.  Once we have convinced ourselves that this function
is correct---by testing and examination of the code---we can
use the function without ever looking at the code again.

The same is true of recursive programs.  When you get to the recursive
call, instead of following the flow of execution, you should {\em
assume} that the recursive call works (yields the correct result), and
then ask yourself, ``Assuming that I can find the factorial of $n-1$,
can I compute the factorial of $n$?''  In this case, it is clear that
you can, by multiplying by $n$.

Of course, it is a bit strange to assume that the function works
correctly when you have not even finished writing it, but that's why
it's called a leap of faith!

\section{One more example}
\index{factorial}

In the previous example I used temporary variables to spell out the
steps, and to make the code easier to debug, but I could have saved a
few lines:

\begin{verbatim}
int factorial (int n) {
  if (n == 0) {
    return 1;
  } else {
    return n * factorial (n-1);
  }
}
\end{verbatim}
%
From now on I will tend to use the more concise version, but
I recommend that you use the more explicit version while you
are developing code.   When you have it working, you can
tighten it up, if you are feeling inspired.

After {\tt factorial}, the classic example of a recursively-defined
mathematical function is {\tt fibonacci}, which has the
following definition:

\begin{eqnarray*}
&& fibonacci(0) = 1 \\
&& fibonacci(1) = 1 \\
&& fibonacci(n) = fibonacci(n-1) + fibonacci(n-2);
\end{eqnarray*}
%
Translated into C++, this is

\begin{verbatim}
int fibonacci (int n) {
  if (n == 0 || n == 1) {
    return 1;
  } else {
    return fibonacci (n-1) + fibonacci (n-2);
  }
}
\end{verbatim}
%
If you try to follow the flow of execution here, even for fairly small
values of {\tt n}, your head explodes.  But according to the leap of
faith, if we assume that the two recursive calls (yes, you can make
two recursive calls) work correctly, then it is clear that we get the
right result by adding them together.

\section{Glossary}

\begin{description}

\item[return type:]  The type of value a function returns.

\item[return value:]  The value provided as the result of a function
call.

\item[dead code:]  Part of a program that can never be executed,
often because it appears after a {\tt return} statement.

\item[scaffolding:]  Code that is used during program development
but is not part of the final version.

\item[void:]  A special return type that indicates a void function;
that is, one that does not return a value.

\item[overloading:]  Having more than one function with the same name
but different parameters.  When you call an overloaded function,
C++ knows which version to use by looking at the arguments you
provide.

\item[boolean:]  A value or variable that can take on one of
two states, often called $true$ and $false$.  In C++, boolean
values can be stored in a variable type called {\tt bool}.

\item[flag:]  A variable (usually type {\tt bool}) that records
a condition or status information.

\item[comparison operator:]  An operator that compares two values
and produces a boolean that indicates the relationship between the
operands.

\item[logical operator:]  An operator that combines boolean values
in order to test compound conditions.

\index{return type}
\index{return value}
\index{dead code}
\index{scaffolding}
\index{void}
\index{overloading}
\index{bool}
\index{operator!conditional}
\index{operator!logical}

\end{description}


% LaTeX source for textbook ``How to think like a computer scientist''
% Copyright (C) 1999  Allen B. Downey

% This LaTeX source is free software; you can redistribute it and/or
% modify it under the terms of the GNU General Public License as
% published by the Free Software Foundation (version 2).

% This LaTeX source is distributed in the hope that it will be useful,
% but WITHOUT ANY WARRANTY; without even the implied warranty of
% MERCHANTABILITY or FITNESS FOR A PARTICULAR PURPOSE.  See the GNU
% General Public License for more details.

% Compiling this LaTeX source has the effect of generating
% a device-independent representation of a textbook, which
% can be converted to other formats and printed.  All intermediate
% representations (including DVI and Postscript), and all printed
% copies of the textbook are also covered by the GNU General
% Public License.

% This distribution includes a file named COPYING that contains the text
% of the GNU General Public License.  If it is missing, you can obtain
% it from www.gnu.org or by writing to the Free Software Foundation,
% Inc., 59 Temple Place - Suite 330, Boston, MA 02111-1307, USA.

% This is an Icelandic translation/adaptation by Hrafn Loftsson of the orginal book by Allen B. Downey.

\chapter{Ítrun}

\section{Fjölgildisveiting}
\index{gildisveiting}
\index{setning!gildisveiting}
\index{fjölgildisveiting}

Ég hef ekki nefnt það áður að í C++ er leyfilegt að gefa breytu gildi oftar en einu sinni.
Tilgangur gildisveitingar nr. 2 er að skipta út gömlu gildi tiltekinnar breytu fyrir nýtt gildi.

\begin{verbatim}
  int fred = 5;
  cout << fred;
  fred = 7;
  cout << fred;
\end{verbatim}
%
Úttakið úr þessu forriti er {\tt 57} vegna þess að í fyrra skiptið sem við skrifum út {\tt fred}
þá er gildi breytunnar 5 en í síðari skiptið er gildið 7.

Þessi tegund af {\bf fjölgildisveitingu} (e. multiple assignment) er ástæðan fyrir því að
ég lýsti breytum sem einskonar {\em gámi} (e. container) fyrir gildi.
Þegar þú gefur breytu gildi þá breytir þú innihaldi gámsins eins og sést á eftirfarandi mynd:

\vspace{0.1in}
\centerline{\epsfig{figure=assign2.eps}}
\vspace{0.1in}

Þegar um fjölgildisveitingu er að ræða þá er sérstaklega mikilvægt að gera greinarmun á gildisveitingu og samanburði.
C++ notar {\tt =} táknið fyrir gildisveitingu en það er freistandi að túlka setningu eins og {\tt a = b} sem samanburð (jöfnuð).
Svo er hins vegar ekki!

Í fyrsta lagi þá er samanburður víxlin (e. commutative) en gildisveiting er það ekki.
Í stærðfræði ef t.d. $a = 7$ þá er $7 = a$.
En í C++ þá er setningin {\tt a = 7;} lögleg en {\tt 7 = a;} aftur á móti ekki. Af hverju ekki?

Jafnframt gildir í stærðfræði að setning um jöfnuð (e. statement of equality) er sönn án tillits til tímasetningar.
Ef $a = b$ núna, þá verður $a$ alltaf jafnt $b$.
Í C++ getur gildisveiting gert tvær breytur jafnar en þær þurfa hins vegar ekki alltaf að vera jafnar eftir það!

\begin{verbatim}
  int a = 5;
  int b = a;     // a and b are now equal
  a = 3;         // a and b are no longer equal
\end{verbatim}
%
Þriðja línan breytir gildinu á {\tt a} en breytir hins vegar ekki gildinu á {\tt b} og því eru breyturnar tvær ekki lengur jafnar.
Í mörgum forritunarmálum er annað tákn notað fyrir gildisveitingu, t.d. {\tt <-} eða {\tt :=}, til að koma í veg fyrir rugling.

Þrátt fyrir að fjölgildisveiting sé oft gagnleg þá skaltu nota hana með varúð.
Ef gildi breytna eru stöðugt að breytast á mismunandi stöðum í forritunu þínu þá getur það orðið ólæsilegra og erfiðara að kemba.

\section{Ítrun}
\index{ítrun}

Eitt af því sem tölvur eru oft notaðar í er að sjálfvirknivæða einhæf verkefni.
Tölvur er, ólíkt fólki, góðar í því að endurtaka sama eða svipað verkefni án þess að gera villur.

Við höfum hingað til séð forrit sem beitir endurkvæmni til að framkvæma endurtekningar, t.d. {\tt nLines} og {\tt countdown}.
Þessi tegund af endurtekningum er kölluð {\bf ítrun} (e. iteration) og C++ hefur ýmsa innbyggða eiginleika sem gera okkur auðveldara að skrifa forrit sem notar ítranir.

%Við ætlum nú að skoða tvo af þessum eiginleikum:
%{\tt while} setningu og {\tt for} setningu.

Við ætlum nú að skoða einn af þessum eiginleikum: {\tt while} setningu.

\section{{\tt While} setning}
\index{setning!while}
\index{while setning}

Með því að nota {\tt while} setningu getum við endurskrifað {\tt countdown} fallið:

\begin{verbatim}
int countdown (int n) {
  while (n > 0) {
    cout << n << endl;
    n = n-1;
  }
  cout << "Blastoff!" << endl;
  return 0;
}
\end{verbatim}
%
Þú getur nánast lesið {\tt while} setningu eins og um enskan texta væri að ræða.
Merkingin hér er: ``While {\tt n} is greater than
zero, continue displaying the value of {\tt n} and then reducing
the value of {\tt n} by 1.  When you get to zero, output the
word `Blastoff!'''

Við getum lýst keyrsluflæðinu í {\tt while} setningu á formlegri hátt:

\begin{enumerate}

\item Gildið á skilyrðinu innan í svigunum er ákvarðað. Útkoman er annaðhvort {\tt true} eða {\tt false}.

\item Ef skilyrðið er ósatt (false) þá er strax hætt í {\tt while} setningunni og haldið áfram keyrslu í næstu setningu á eftir.

\item Ef skilyrðið er satt (true) þá er sérhver setning í blokkinni sem afmarkast af slaufusvigunum keyrð og síðan farið til baka í skref 1.

\end{enumerate}

Þessi tegund af flæði er kölluð {\bf lykkja} (e. loop) vegna þess að þriðja skrefið stekkur til baka í fyrsta skrefið.
Athugaðu að ef skilyrðið er ósatt (false) strax í upphafi þá eru setningarnar innan lykkjunnar aldrei keyrðar.
Setningarnar innan í lykkju eru kallaðar {\bf meginmál} (e. body) lykkjunnar.

\index{lykkja}
\index{lykkja!meginmál}
\index{lykkja!óendanleg}
\index{meginmál!lykkja}
\index{óendanleg lykkja}

Það er mikilvægt að meginmál lykkjunnar breyti gildi einnar eða fleiri breytna þannig að skilyrðið verði false á einhverjum tímapunkti og lykkjan hætti þá keyrslu.
Að öðrum kosti mun lykkjan halda áfram sínum ítrunum (að eilífu!) og í því tilviki er um {\bf óendanlega lykkju} (e. infinite loop) að ræða.
%An endless source of amusement for computer scientists is the observation
%that the directions on shampoo, ``Lather, rinse, repeat,'' are an infinite loop.

Í tilviki {\tt countdown} getum við sannað að lykkjan muni klárast.
Við vitum að gildið á {\tt n} minnkar um 1 í hverri {\bf ítrun} lykkjunnar þannig að {\tt n} fær gildið 0 að lokum.
Í öðrum tilfellum er kannski ekki svo auðvelt að segja til um þetta:

\begin{verbatim}
  void sequence (int n) {
    while (n != 1) {
      cout << n << endl;
      if (n%2 == 0) {           // n is even
        n = n / 2;
      } else {                  // n is odd
        n = n*3 + 1;
      }
    }
  }
\end{verbatim}
%
Skilyrði fyrir áframhaldi þessarar lykkju er {\tt n != 1} og hún heldur því áfram þangað til {\tt n} er 1 (sem gerir skilyrðin false).
Í sérhverri ítrun skrifar fallið út gildið á {\tt n} og athugar síðan hvort það er slétt tala eða oddatala.
Ef {\tt n} er slétt þá er deilt í gildið með tveimur.
Ef það er oddatala þá er gildinu skipt út fyrir $3n+1$.
Ef upphafsgildið (þ.e. gildi viðfangsins sem sent er í {\tt sequence}) á {\tt n} er t.d. 3 þá skrifast út eftirfarandi röð:
3, 10, 5, 16, 8, 4, 2, 1.

Þar sem {\tt n} hækkar stundum eða lækkar þá er ekki auðvelt að sanna að {\tt n} fái að lokum gildið 1 og þar með að lykkjan klárist að lokum.
Fyrir tiltekin gildi á {\tt n} getum við sannað að lykkjan klárist.
Ef upphafsgildið á {\tt n} er t.d. veldi af tveimur þá mun gildið á {\tt n} vera slétt tala í sérhverri ítrun lykkjunnar þangað til það verður að lokum 1.
Dæmið að ofan endar með þannig röð, þ.e. röðinni sem byrjar á 16.

Það er hins vegar athyglisvert vandamál að reyna að sanna að lykkjan hætti keyrslu fyrir {\em öll} gildi á {\tt n}.
Hingað til hefur engum tekist að sanna eða afsanna það!

\section{Töflur}
\index{tafla}
\index{lógariþmi}

Eitt af því sem hentugt er að nota lykkur í er að búa til töflugögn.
Fyrir tíma tölvunnar þurfti fólk að reikna lógariþma, sínus og cósínus, og önnur stærðfræðiföll, í höndunum.
Til að gera þetta auðveldara þá voru til bækur sem innhéldu langar töflur þar sem hægt var að fletta upp gildum fyrir ýmis föll.
Það var tímafrekt og leiðinlegt að búa þessar töflur til og sumar þeirra innihéldu fullt af villum.

Ein fyrstu viðbrögðin við því þegar tölvur komu á sjónarsviðið voru:
``Þetta er frábært!  Við getum notað tölvurnar til að búa til töflur sem innihalda engar villur.''
Það reyndist vera (næstum því) rétt en skammsýnt.
Brátt urðu nefnilega tölvur og reiknivélar svo útbreiddar að það var engin þörf á þessum töflum lengur.

Jæja, næstum því.
Það vill reyndar svo til að fyrir tilteknar aðgerðir nota tölvur töflur til að nálga svörin og nota síðan útreikninga til að bæta nálganirnar.
Í sumum tilvikum hafa verið villur í undirliggjandi töflum -- sú þekktasta í töflunni sem fyrsta útgáfa af Intel Pentium örgjörvanum notaði til að framkvæma kommutöludeilingu.

\index{deiling!kommutala}

Þrátt fyrir að ``log tafla'' sé ekki eins gagnleg og áður fyrr þá má nota hana sem gott dæmi um ítrun.
Eftirfarandi forrit skrifar út röð gilda í vinstri dálki og samsvarandi lógariþma í hægri dálki:

\begin{verbatim}
  double x = 1.0;
  while (x < 10.0) {
    cout << x << "\t" << log(x) << "\n";
    x = x + 1.0;
  }
\end{verbatim}
%
Stafarunan \verb+\t+ stendur fyrir {\bf dálkastafinn} (e. tab character).
Runan \verb+\n+ stendur fyrir {\bf newline} (ný lína) stafinn.
Báðar þessar runur er hægt að setja hvar sem er inn í streng en í þessu dæmi mynda runurnar heilan streng án nokkurra annara stafa.

Dálkastafurinn veldur því að bendillinn ``stekkur'' til hægri þangað til hann lendir á einum af {\bf dálkastoppum} (e. tab stops) sem erum yfirleitt eftir hverja átta stafi.

Eins og við sjáum hér rétt á eftir þá er dálkastafurinn hentugur til að skrifa út texta sem passar í dálka.

Stafurinn {\bf newline} virkar á nákvæmlega sama hátt og {\tt endl}, þ.e. veldur því að bendillinn hoppar í næstu línu fyrir neðan.
Ég nota yfirleitt {\tt endl} ef newline stafurinn stendur einn og sér en annars nota ég \verb+\n+ ef hann er hluti strengs.

Úttak þessa forrits er:

\begin{verbatim}
1      0
2      0.693147
3      1.09861
4      1.38629
5      1.60944
6      1.79176
7      1.94591
8      2.07944
9      2.19722
\end{verbatim}
%
Ef þér finnst þessi gildi vera undarleg, mundu þá að {\tt log} fallið notar grunn $e$.
Vegna þess hversu veldi af tveimur eru mikilvæg í tölvunarfræði þá viljum við oft reikna út lógariþma með grunn 2.
Það getum við gert með því að nota eftirfarandi formúlu:

\[ \log_2 x = \frac {log_e x}{log_e 2} \]
%
Með því að breyta úttakssetningunni í 

\begin{verbatim}
      cout << x << "\t" << log(x) / log(2.0) << endl;
\end{verbatim}
%
þá fáum við:

\begin{verbatim}
1      0
2      1
3      1.58496
4      2
5      2.32193
6      2.58496
7      2.80735
8      3
9      3.16993
\end{verbatim}
%
Hér sjáum við að 1, 2, 4 og 8 eru veldi af tveimur vegna þess að lógariþminn með grunn 2 er heil tala.
Ef við vildum finna lógariþma af öðrum tölum af veldinu tveimur þá gætum við breytt forritinu á þennan hátt:

\begin{verbatim}
  double x = 1.0;
  while (x < 100.0) {
    cout << x << "\t" << log(x) / log(2.0) << endl;
    x = x * 2.0;
  }
\end{verbatim}
%
Í stað þess að bæta einhverju við {\tt x} í sérhverri ítrun lykkjunnar, og fá þannig {\bf jafnmunarunu} (e. arithmetic sequence),
þá margföldum við {\tt x} með einhverju og fáum út {\bf kvótarunu} (e. geometric sequence).
Niðurstaðan er:

\begin{verbatim}
1      0
2      1
4      2
8      3
16     4
32     5
64     6
\end{verbatim}
%
Við notum dálkastafinn á milli dálka og því er staðsetning seinni dálksins ekki háð fjölda tölustafa í fyrri dálkinum.

Það má vera að log töflur séu ekki gagnlegar nú til dags en hins vegar er mikilvægt fyrir tölvunarfræðinga að þekkja veldi af tveimur!
Prófaðu að breyta forritinu þannig að það skrifi út veldi af tveimur upp í 65536 (þ.e. $2^{16}$).
Prentaðu það út og mundu það!


\section{Tvívíðar töflur}
\index{tafla!tvívíð}

Í tvívíðri töflu (e. two-dimensional table) getur þú valið röð og dálk og lesið gildið þar sem röðin og dálkurinn mætast.
Gott dæmi um tvívíða töflu er margföldunartafla.
Segjum að þig langi til að prenta út margföldunartöflu fyrir gildin 1 til 6.

Þú gætir byrjað með því að skrifa einfalda lykkju sem prentar út margfeldi af 2, öll í sömu línu:

\begin{verbatim}
  int i = 1;
  while (i <= 6) {
    cout << 2*i << "   ";
    i = i + 1;
  }
  cout << endl;
\end{verbatim}
%
Fyrsta línan frumstillir breytuna {\tt i} en hún er hér notuð sem teljari eða {\bf lykkjubreyta} (e. loop variable).
Þegar lykkjan keyrir þá hækkar gildið á {\tt i} úr 1 í 6 og lykkjan hættir keyrslu þegar {\tt i} er 7.
Í sérhverri ítrun lykkjunnar prentum við út gildið {\tt 2*i} ásamt þremur bilum.
Við skrifum allt úttakið í einni og sömu línunni því við sleppum {\tt endl} úr fyrstu úttakssetningunni.

\index{lykkjubreyta}
\index{breyta!lykkja}

Úttak þessa forrits er:

\begin{verbatim}
2   4   6   8   10   12
\end{verbatim}
%
Næsta skref er að {\bf hjúpa} (e. encapsulate) og {\bf alhæfa} (e. generalize).

\section {Hjúpun og alhæfing}

Hjúpun (e. encapsulation) merkir yfirleitt að taka hluta af kóða og ``pakka'' (e. wrap) honum inn í fall og þannig nýta kostina sem almennt fylgja föllum.
Við höfum séð tvö dæmi um hjúpun, þ.e. þegar við skrifuðum {\tt printParity} í kafla~\ref{alternative} og {\tt
isSingleDigit} í kafla~\ref{bool}.

Alhæfing merkir að taka eitthvað einstakt tilfelli, eins og að prenta út margfeldi af tveimur, og gera það almennara, eins og að prenta út margfeldi af hvaða heiltölu sem er.

\index{hjúpun}
\index{alhæfing}

Hér er fall sem hjúpar lykkjuna að ofan og gerir hana almennari þannig að hún prenti út margeldi af {\tt n}.

\begin{verbatim}
void printMultiples (int n)
{
  int i = 1;
  while (i <= 6) {
    cout << n*i << "   ";
    i = i + 1;
  }
  cout << endl;
}
\end{verbatim}
%
Eina sem ég þurfti að gera til að hjúpa var að bæta fyrstu línunni við, þ.e. að tilgreina nafn fallsins, lepp og skilagildi.
Til að alhæfa þurfti ég aðeins að skipta út gildinu 2 fyrir leppinn {\tt n}.

Ef við köllum á þetta fall með viðfanginu 2 þá fáum við sama úttak og áður.
Ef við köllum með viðfanginu 3 fæst úttakið: 

\begin{verbatim}
3   6   9   12   15   18
\end{verbatim}
%
og með viðfanginu 4 fæst úttakið:

\begin{verbatim}
4   8   12   16   20   24 
\end{verbatim}
%
Á þessum tímapunkti ættir þú að sjá hvernig við förum að því að prenta út margföldunartöflu:
Við munum kalla endurtekið á {\tt printMultiples} með mismunandi viðfögnum.
Reyndar vill svo til að við munum nota aðra lykkjum til að ítra yfir raðirnar:

\begin{verbatim}
  int i = 1;
  while (i <= 6) {
    printMultiples (i);
    i = i + 1;
  }    
\end{verbatim}
%
Taktu eftir hversu svipuð þessi lykkja er þeirri inni í fallinu {\tt printMultiples}.
Eina sem ég gerði var að skipta út úttakssetningunni fyrir fallakall.

Úttakið úr þessu forriti er

\begin{verbatim}
1   2   3   4   5   6   
2   4   6   8   10   12   
3   6   9   12   15   18   
4   8   12   16   20   24   
5   10   15   20   25   30   
6   12   18   24   30   36   
\end{verbatim}
%
sem er (örlítið bjöguð) margföldunartafla.
Ef þessi bjögun angrar þig þá ættir þú að prófa að skipta út bilum fyrir dálkastafinn og athuga hvernig taflan lítur út eftir þá breytingu.

\section{Föll}
\index{fall}

Í síðasta kafla nefndi ég ``og þannig nýta kostina sem almennt fylgja föllum''.
Á þessum tímapunkti veltir þú kannski fyrir þér hvaða kostir þetta eru. 
Hér er nokkrar ástæður fyrir gagnsemi falla:

\begin{itemize}

\item Með því að gefa röð setninga tiltekið nafn þá gerir þú forritið þitt bæði læsilegra og auðveldara að kemba.

\item Með því að brjóta langt forrit upp í föll þá skiptir þú því í mismunandi hluta, getur unnið í einstökum hlutum þess án tillits til annarra hluta,
og síðan sett þessa hluta saman til að mynda eina heild.

\item Föll styðja bæði við endurkvæmni og ítrun.

\item Vel hönnuð föll eru oft gagnleg fyrir mörg forrit. Þegar þú hefur skrifað fall þá getur þú oft endurnýtt það í öðrum forritum.

\end{itemize}

\section{Meira um hjúpun}
\index{hjúpun}
\index{forritun!hjúpun}

Í þeim tilgangi að sýna hjúpun aftur þá ætla ég núna að taka kóðann úr síðasta kafla og pakka honum inn í fallið {\tt printMultTable}:

\begin{verbatim}
void printMultTable () {
  int i = 1;
  while (i <= 6) {
    printMultiples (i);
    i = i + 1;
  }
}
\end{verbatim}
%
Þetta ferli sem ég hef sýnt þér er algengt þróunarferli í forritun.
Þú þróar forrit smám saman með því að bæta setningu við {\tt main} eða á einhverjum öðrum stað, 
og þegar forritið virkar þá dregur þú kóðann út og pakkar honum inn í fall.

Ástæðan fyrir því að þetta er gagnlegt er að stundum veistu ekki í upphafi hvernig best er að skipta forritinu þínu upp í einstök föll.
Þessi aðferð gerir þér kleift að hanna um leið og þú skrifar forritið.

\section{Staðværar breytur}

Á þessum tímapunkti gætir þú velt því fyrir þér hvernig standi á því að hægt sé að nota sömu breytuna {\tt i}
í bæði {\tt printMultiples} og {\tt printMultTable}.
Töluðum við ekki um að eingöngu er hægt að lýsa yfir breytu einu sinni?
Og veldur það ekki vandræðum ef eitt fall breytir gildinu á breytunni?

Svarið við báðum þessum spurningum er ``nei''.
Ástæðan er sú að {\tt i} í {\tt printMultiples} og {\tt i} í {\tt printMultTable} eru {\em ekki sama breytan}.
Þær bera vissulega sama nafnið en minnissvæðið sem þær standa fyrir er ekki það sama og þess vegna hefur það ekki áhrif á aðra breytuna að breyta gildi hinnar.

\index{staðvær breyta}
\index{breyta!staðvær}

Mundu að breyta sem er lýst yfir í falli er staðvær (e. local).
Það er ekki hægt að nálgast gildi staðværrar breytu utan frá (þ.e. utan fallsins sem skilgreinir breytuna)
og því getur þú haft margar breytur með sama nafni svo framarlega sem þær eru ekki skilgreindar í sama fallinu.

Staflaritið fyrir þetta forrit sýnir glögglega að breyturnar tvær með nafnið {\tt i} standa ekki fyrir sama minnissvæði.
Gildi þeirra getur verið mismunandi og breyting á annarri breytunni hefur engin áhrif á hina breytuna.

\vspace{0.1in}
\centerline{\epsfig{figure=stack4.eps}}
\vspace{0.1in}
%
Taktu eftir að gildið á leppnum {\tt n} í {\tt printMultiples} er það sama og gildið á {\tt i} í {\tt printMultTable}.
Á hinn bóginn hleypur gildið á {\tt i} í {\tt printMultiple} úr 1 upp í {\tt n}.
Á myndinni þá vill svo til að þetta gildi er 3 en í næstu ítrun lykkjunnar verður það 4.

Til að minnka líkur á ruglingi þá er yfirleitt góð forritunarvenja að nota mismunandi nöfn á breytum í mismunandi föllum.
Aftur á móti eru líka ástæður fyrir því að nota sömu breytunöfn.
Það er t.d. algengt að nota nöfnin {\tt i}, {\tt j} og {\tt k} fyrir lykkjubreytur.
Ef þú forðast að nota þau í einu falli vegna þess að þú notar þau í öðru falli þá verður forritið þitt líklega bara ólæsilegra.

\index{lykkjubreyta}
\index{breyta!lykkja}

\section{Meira um alhæfingu}
\index{alhæfing}

Tökum annað dæmi um alhæfingu.
Gerum ráð fyrir að þú viljir skrifa forrit sem prentar út margföldunartöflu af hvaða stærð sem er, þ.e. ekki bara 6x6 töflu.
Þú gætir bætt leppi við {\tt printMultTable}:

\begin{verbatim}
void printMultTable (int high) {
  int i = 1;
  while (i <= high) {
    printMultiples (i);
    i = i + 1;
  }
}
\end{verbatim}
%
Hér hef ég skipt út gildinu 6 með leppnum {\tt high}.
Með því að kalla á {\tt printMultTable} með viðfanginu 7 þá fæst:

\begin{verbatim}
1   2   3   4   5   6   
2   4   6   8   10   12   
3   6   9   12   15   18   
4   8   12   16   20   24   
5   10   15   20   25   30   
6   12   18   24   30   36   
7   14   21   28   35   42   
\end{verbatim}
%
Þetta er ágætt en gott væri ef taflan gæti sýnt jafnmargar raðir og dálka.
Ég þarf því að bæta öðrum lepp við {\tt printMultiples} sem tilgreinir þann fjölda dálka sem ég vil sjá.

Svona til að pirra þig þá ætla ég líka að kalla þennan lepp {\tt high} og þar með sýna að mismunandi föll geta haft leppa (eins og staðværar breytur) með sama nafni:

\begin{verbatim}
void printMultiples (int n, int high) {
  int i = 1;
  while (i <= high) {
    cout << n*i << "   ";
    i = i + 1;
  }    
  cout << endl;
}

void printMultTable (int high) {
  int i = 1;
  while (i <= high) {
    printMultiples (i, high);
    i = i + 1;
  }
}

int main() {
  printMultTable(7);
}

\end{verbatim}
%
Taktu eftir því að þegar ég bætti nýjum leppi við þá þurfti ég að breyta fyrstu línunni í fallinu og einnig þeim stað þar sem kallað er á fallið í {\tt printMultTable}.
Eins og gera mátti ráð fyrir þá skrifar þetta forrit út 7x7 margföldunartöflu:

\begin{verbatim}
1   2   3   4   5   6   7   
2   4   6   8   10   12   14   
3   6   9   12   15   18   21   
4   8   12   16   20   24   28   
5   10   15   20   25   30   35   
6   12   18   24   30   36   42   
7   14   21   28   35   42   49
\end{verbatim}
%
Þegar maður gerir fall almennara þá sér maður stundum að niðurstaðan hefur eiginleika sem ekki voru fyrirséðir.
Þú tekur kannski eftir því núna að margföldunartaflan er samhverf (e. symmetric) því $ab = ba$ og því birtast öll gildi í töflunni tvisvar.
Við gætum sparað okkur blekið (að því gefnu að úttakið væri sent á prentara) með því að skrifa bara út helming töflunnar.
Við þurfum aðeins að breyta einni línu í {\tt printMultTable} til að gera það.
Breyttu

\begin{verbatim}
      printMultiples (i, high);
\end{verbatim}
%
í

\begin{verbatim}
      printMultiples (i, i);
\end{verbatim}
%
og þá færðu

\begin{verbatim}
1   
2   4   
3   6   9   
4   8   12   16   
5   10   15   20   25   
6   12   18   24   30   36   
7   14   21   28   35   42   49  
\end{verbatim}
%
Ég læt þig um að finna út hvernig þetta virkar!

\section{Orðalisti}

\begin{description}

\item[lykkja (e. loop):]  Setning sem er endurtekin á meðan að tiltekið skilyrði er satt.

\item[óendanleg lykkja (e. infinite loop):]  Lykkja, hvers skilyrði er ávallt satt.

\item[meginmál (e. body):]  Setningarnar innan í lykkju.

\item[ítrun (e. iteration):]  Ein umferð í gegnum meginmál lykkju, þ.m.t. ákvörðun á gildi skilyrðisins.

\item[dálkalykill (e. tab):] Sérstakur stafur, skrifaður sem \verb+\t+ í C++, sem gerir það að verkum að 
bendillinn færist í næsta dálkastopp í núverandi línu.

\item[hjúpa (e. encapsulate):]  Að setja tiltekna virkni inn í sérstaka einingu, t.d. fall, og einangra virknina (t.d. með því að nota staðværar breytur) frá öðrum hluta forritsins.

\item[staðvær breyta (e. local variable):]  Breyta sem er skilgreind inni í falli og hvers líftimi er aðeins innan í fallinu.
Ekki er hægt að nálgast staðværar breytur utan frá og þær hafa engin áhrif á virkni annarra falla.

\item[alhæfa (e. generalize):]  Að breyta einhverju sem er óþarflega sérstakt (eins og fast gildi) yfir í eitthvað sem er almennara (eins og breytu eða lepp)
Alhæfing gerir kóðann fjölhæfara og líklegri til að verða endurnýtanlegur.

\item[þróunarferli (e. development process):]  Ferli til að þróa forrit.
Í þessum kafla hef ég sýnt sérstaka aðferð sem byggir á því að skrifa kóða sem framkvæmir einfalda, sérstaka hluti og hef síðan beitt hjúpun og alhæfingu.

\index{lykkja}
\index{óendanleg lykkja}
\index{meginmál}
\index{dálkalykill}
\index{lykkja!óendanleg}
\index{ítrun}
\index{hjúpun}
\index{alhæfing}
\index{staðvær breyta}
\index{breyta!staðvær}
\index{þróunarferli}

\end{description}


% LaTeX source for textbook ``How to think like a computer scientist''
% Copyright (C) 1999  Allen B. Downey

% This LaTeX source is free software; you can redistribute it and/or
% modify it under the terms of the GNU General Public License as
% published by the Free Software Foundation (version 2).

% This LaTeX source is distributed in the hope that it will be useful,
% but WITHOUT ANY WARRANTY; without even the implied warranty of
% MERCHANTABILITY or FITNESS FOR A PARTICULAR PURPOSE.  See the GNU
% General Public License for more details.

% Compiling this LaTeX source has the effect of generating
% a device-independent representation of a textbook, which
% can be converted to other formats and printed.  All intermediate
% representations (including DVI and Postscript), and all printed
% copies of the textbook are also covered by the GNU General
% Public License.

% This distribution includes a file named COPYING that contains the text
% of the GNU General Public License.  If it is missing, you can obtain
% it from www.gnu.org or by writing to the Free Software Foundation,
% Inc., 59 Temple Place - Suite 330, Boston, MA 02111-1307, USA.

% This is an Icelandic translation/adaptation of the orginal book by Allen B. Downey

\chapter{Strengir}
\label{strings}

\section{Geymsla fyrir strengi}

Við höfum hingað til séð fimm tegundir af gildum -- boole (e. booleans), stafi (e. characters), heiltölur (e. integers),
kommutölur (e. floating-point numbers) og strengi (e. strings) -- en aðeins fjórar tegundir af breytum -- {\tt bool}, {\tt char}, {\tt int} og {\tt double}.
Við höfum því ekki séð neina leið til að geyma streng í breytu og framkvæma aðgerðir á strengjum.

Það eru reyndar nokkrar tegundir af breytum í C++ sem geta geymt strengi.
Einn þeirra er grunntag, hluti af C++ málinu, sem er stundum kallað ``a native C-string.''
Málskipanin fyrir C-strengi er dálítið skrýtin og þar sem notkun C-strengja krefst þekkingar á atriðum sem við höfum ekki fjallað um þá munum við að mestu leyti sleppa C-strengjum.

Strengjatagið sem við munum nota er kallað {\tt string} en það er einn af þeim klösum (e. classes) sem tilheyra C++ Standard Library.\footnote{Þú veltir því vafalaust fyrir þér hvað ég á við með {\bf klasa}. Við eigum eftir að skoða nokkra kafla í viðbót áður en ég get gefið fullnægjandi skilgreiningu
en að svo stöddu getum við sagt að klasi sé safn falla sem skilgreina aðgerðir sem hægt er að framkvæma á tilteknu tagi.
{\tt string} klasinn inniheldur öll föll sem hægt er að beita á tagið {\tt string}.}

Því miður vill reyndar svo til að við getum ekki algerlega hunsað C-strengi.
Á nokkrum stöðum í þessum kafla mun ég benda þér á vandamál sem hægt er að rekast á með því að nota {\tt string} í stað C-strengja.

\section{{\tt string} breytur}

Þú býrð til breytu af taginu {\tt string} á hefðbundin hátt:

\begin{verbatim}
  string first;
  first = "Hello, ";
  string second = "world.";
\end{verbatim}
%
Fyrsta línan lýsir yfir breytu af taginu {\tt string} án þess að gefa henni gildi.
Önnur línan gefur henni strengjagildið \verb+"Hello."+
Þriðja línan er bæði yfirlýsing (e. declaration) og gildisveiting (e. assignment), þ.e. upphafsstilling (e. initialization).

Þegar strengjagildi eins og \verb+"Hello, "+ eða \verb+"world."+ koma fyrir þá eru þau venjulega meðhöndluð sem C-strengir.
Í þessu tilviki er þeim hins vegar breytt sjálfvirkt í {\tt string} gildi vegna þess að við notum breytur af taginu {\tt string} til að geyma gildin.

Við getum skrifað út strengi á hefðbundin máta: 

\begin{verbatim}
  cout << first << second << endl;
\end{verbatim}
%

Þú þarft að setja hausaskrá (e. header file) fyrir {\tt string} klasann inn í forritið þitt til að geta þýtt þennan kóða:
\begin{verbatim}
  #include <string>
\end{verbatim}

%In order to compile this code, you will have to include the
%header file for the {\tt string} class, and you will have
%to add the file {\tt string} to the list of files you
%want to compile.  The details of how to do this depend on your
%programming environment.

Áður en við höldum lengra þá skaltu slá ofangreint forrit inn og vera viss um að þú getir þýtt það og keyrt.

\section{Útdráttur stafa úr streng}

Strengir eru kallaðir ``strengir'' vegna þess er þeir eru settir saman af röð eða streng af stöfum.
Fyrsta aðgerðin sem við ætlum að framkvæma á streng er að draga út einn af stöfunum.
C++ notar hornklofa (e. square brackets) ({\tt [} og {\tt ]}) fyrir þessa aðgerð:

\begin{verbatim}
  string fruit = "banana";
  char letter = fruit[1];
  cout << letter << endl;
\end{verbatim}
%
Segðin {\tt fruit[1]} gefur til kynna að ég vilji fá staf nr. 1 í strengnum með nafninu {\tt fruit}.
Niðurstaðan er geymd í {\tt char} breytu með nafninu {\tt letter}.
Þegar ég síðan skrifa út gildið á {\tt letter} þá kemur niðurstaðan á óvart:

\begin{verbatim}
a
\end{verbatim}
%
{\tt a} er ekki fyrsti stafurinn í \verb+"banana"+.
Nema að þú sért tölvunarfræðingur!
Af sérstökum ástæðum byrja tölvunarfræðingar alltaf að telja frá núlli.
Núllti stafurinn í \verb+"banana"+ er {\tt b}.
Fyrsti stafurinn í er {\tt a} og annar stafurinn er {\tt n}.

Þannig að ef þú vilt fá núllta stafinn úr streng þá þarftu að setja núll inn í hornklofana: 

\begin{verbatim}
  char letter = fruit[0];
\end{verbatim}

\section{Lengd}
\index{strengur!lengd}
\index{lengd!strengur}

Við getum notað {\tt length} fallið til að finna lengd (fjölda stafa) strengs.
Málskipanin sem notuð er til að kalla á þetta fall er dálítið frábrugðin þeirri sem við höfum séð hingað til:

\begin{verbatim}
  int length;
  length = fruit.length();
\end{verbatim}
%
Til að lýsa þessu fallakalli getum við sagt að við séum að {\bf kalla á} (e. invoking) þetta lengdarfall sem tilheyrir streng með nafninu {\tt fruit}.
Þetta hjómar kannski undarlega en við munum sjá mörg önnur dæmi um að kalla á fall sem tilheyrir hlut (e. object).
Málskipanin fyrir þetta er kölluð ``punktatáknun'' (e. dot notation) vegna þess að punktur (e. dot) kemur á milli nafns hlutarins
{\tt fruit}, og nafns fallsins {\tt length}.

{\tt length} tekur engin viðfögn eins og sjá má með tómu svigunum {\tt ()}.
Skilagildið er heiltala, í þessu tilviki 6.
Taktu eftir að það er leyfilegt að hafa breytu með sama nafn og fall. 

Það gæti verið freistandi að reyna eftirfarandi til að draga út síðasta staf í streng:

\begin{verbatim}
  int length = fruit.length();
  char last = fruit[length];       // WRONG!!
\end{verbatim}
%
Þetta gengur ekki vegna þess að það er enginn stafur nr. 6 í \verb+"banana"+.
Þar sem við byrjum að telja í 0 þá eru stafirnir 6 númeraðir frá 0 upp í 5.
Við þurfum að draga 1 frá {\tt length} til að draga út síðasta stafinn.

\begin{verbatim}
  int length = fruit.length();
  char last = fruit[length-1];
\end{verbatim}

\section{Að ferðast eftir}
\index{ferðast eftir}

Þegar unnið er með strengi þá er algengt að byrja á byrjuninni, velja síðan næsta staf, 
framkvæma einhverja aðgerða á honum, og endurtaka þetta þangað til að komið er að enda strengsins.
Þetta vinnslumynstur er kallað {\bf að ferðast eftir} (e. traverse).
Það er einfalt að kóða svona ``rölt'' með {\tt while} setningu:

\begin{verbatim}
  int index = 0;
  while (index < fruit.length()) {
    char letter = fruit[index];
    cout << letter << endl;
    index = index + 1;
  }
\end{verbatim}
%
Þessi lykkja ferðast eftir strengnum og skrifar út sérhvern staf í sér línu.
Taktu eftir því að skilyrðið er {\tt index < fruit.length()} sem merkir að þegar {\tt index} er jafnt lengd strengsins þá
er skilyrðið ósatt og meginmál lykkjunnar er þá ekki keyrt.
Síðasti stafurinn sem við drögum út er sá með vísinn (e. index) {\tt fruit.length()-1}.

\index{lykkjubreyta}
\index{breyta!lykkja}
\index{vísir}

Nafn lykkjubreytunnar er {\tt index}.
{\bf index} (eða vísir á íslensku) er breyta eða gildi sem notuð er til að tilgreina eitt stak í röðuðu mengi, í þessu tilviki mengi af stöfum í streng.
Vísirinn tilgreinir hvaða stak maður hefur áhuga á.
Mengið þarf að vera raðað þannig að sérhver stafur hafi vísi og að visir standi fyrir einn tiltekinn staf.

Þú ættir núna, svona til að æfa þig, að skrifa fall sem tekur {\tt string} sem viðfang og skrifar út stafi strengsins í öfugri röð, alla í einni og sömu línunni.

\section{Keyrsluvilla}
\index{villa!keyrsla}
\index{keyrsluvilla}

Í kafla~\ref{run-time} talaði ég um keyrsluvillur, þ.e. villur sem koma ekki fram fyrr en við keyrslu forrits.

Hingað til hefur þú líklega ekki séð margar keyrsluvillur vegna þess að við höfum ekki gert marga hluti sem gætu orsakað keyrsluvillur.
Það breytist núna.
Ef þú notar {\tt []} virkjann og gefur upp vísi sem er negatífur eða stærri en {\tt length-1} þá færðu keyrsluvillu með villuskilaboðum eitthvað á þessa leið:

\begin{verbatim}
index out of range: 6, string: banana
\end{verbatim}
%
Prófaðu þetta í þínu þróunarumhverfi til að sjá hvernig villuskilaboðin líta út.

\section{{\tt find} fallið}
\index{find}

{\tt string} klasinn býður upp á ýmis önnur föll sem þú getur beitt á strengi.
{\tt find} fallið er eins og andhverfan við {\tt []} virkjann.
Í stað þess að gefa upp vísi og draga út stafinn sem tengist þeim vísi þá tekur {\tt find} fallið staf sem viðfang
og finnur vísinn þar sem viðkomandi stafur finnst.

\begin{verbatim}
  string fruit = "banana";
  int index = fruit.find('a');
\end{verbatim}
%
Þetta dæmi finnur vísinn fyrir stafinn {\tt 'a'} í strengnum {\tt fruit}.
Í þessu tilviki birtist stafurinn þrisvar sinnum í strengnum þannig að það er ekki augljóst hvað {\tt find} eigi að gera.
Samkvæmt skjölun (e. documentation) um fallið þá skilar það vísinum á {\em fyrsta} tilvikinu, þannig að hér er niðurstaðan 1.
Ef uppgefinn stafur finnst ekki í strengnum þá skilar {\tt find} -1.

Önnur útgáfa af {\tt find} er til sem tekur annan {\tt string} (hlutstreng) sem viðfang og finnur vísinn sem samsvarar því hvar hlutstrengurinn byrjar í strengnum.
Dæmi:

\begin{verbatim}
  string fruit = "banana";
  int index = fruit.find("nan");
\end{verbatim}
%
Í þessu dæmi er niðurstaðan 2.

Þú ættir að muna frá kafla~\ref{overloading} að hægt er að vera með fleiri en eitt fall með sama nafni,
svo framarlega sem þau taka mismunandi fjölda viðfanga eða mismunandi tög.
Í þessu tilvik veit C++ þýðandinn hvora útgáfuna af {\tt find} verið er að nota með því að skoða tög viðfanganna í kallinu.

\section{Okkar eigin útgáfa af {\tt find}}

Ef við erum að leita að tilteknum staf í streng þá getur verið að við viljum ekki endilega byrja að leita frá upphafi strengsins. 
Ein leið til að gera {\tt find} fallið almennara er að skrifa útgáfu af því sem tekur eitt auka viðfang -- vísi sem stendur fyrir þann stað sem við viljum byrja leitina frá.
Hér er útfærsla af þessu falli:

\begin{verbatim}
int find (string s, char c, int i)
{
  while (i<s.length()) {
    if (s[i] == c) return i;
    i = i + 1;
  }
  return -1;
}
\end{verbatim}
%
Í stað þess að kalla á þetta fall með því að nota punktatáknun eins og við gerðum fyrir fyrstu útgáfuna af {\tt find}
þá þurfum við að senda {\tt string} sem fyrsta viðfang.
Hin viðföngin eru stafurinn sem við erum að leita að og vísirinn sem gefur til kynna hvar við byrjum leitina.

\section{Talning}
\label{loopcount}
\index{ferðast um!talning}
\index{lykkja!talning}

Eftirfarandi forrit telur fjölda tilvika af stafnum {\tt 'a'} í streng:

\begin{verbatim}
  string fruit = "banana";
  int length = fruit.length();
  int count = 0;

  int index = 0;
  while (index < length) {
    if (fruit[index] == 'a') {
      count = count + 1;
    }
    index = index + 1;
  }
  cout << count << endl;
\end{verbatim}
%
Þetta forrit er gott dæmi um notkun á {\bf teljara} (e. counter).
Breytan {\tt count} er upphafsstillt með núlli og síðan hækkuð í sérhvert sinn sem við finnum stafinn {\tt 'a'}.
(Á ensku er talað um {\bf to increment}, þ.e. að hækka um einn, sem er andstæðan við {\bf to decrement}, að lækka um einn.)
Þegar við hættum í lykkjunni þá mun {\tt count} innihalda niðurstöðuna: fjöldann af stafnum {\tt 'a'}.

\index{teljari}
\index{hækka}
\index{lækka}

Svona til að æfa þig þá skaltu núna hjúpa þennan kóða sem fall með nafninu {\tt countLetters} og beita síðan alhæfingu (gera fallið almennt)
þannig að það taki við streng og staf sem viðföngum.

\index{hjúpun}
\index{alhæfing}

%As a second exercise, rewrite this function so that instead
%of traversing the string, it uses the version of
%{\tt find} we wrote in the previous section.

\section{Hækkunar- og lækkunarvirkjar}
\index{virki!auki}
\index{virki!frádrag}

Það að bæta við einum (e. incrementing) og draga einn frá (e. decrementing) eru svo algengar aðgerðir að C++ býður upp á sértaka virkja fyrir þær.
{\tt ++} virkinn (sem kalla má ``auki'' á íslensku) bætir einum við núverandi gildi á {\tt int}, {\tt char} eða {\tt double}, og
\verb+--+ (sem kalla má ``frádrag'' á íslensku) dregur einn frá.
Hvorugur virkjanna virkar á {\tt string} og hvorugum þeirra {\em ætti} að beita á {\tt bool}.

Tæknilega séð er löglegt að hækka breytu og nota hana í segð á sama tíma.
Þú gætir t.d. séð svona kóða: 

\begin{verbatim}
  cout << i++ << endl;
\end{verbatim}
%
Þegar við horfum á þetta þá er ekki ljóst hvort hækkun breytunnar muni eiga sér stað fyrir eða eftir að gildi hennar er skrifað út.
Ég mæli ekki með að þú notir {\tt ++} eða \verb+--+ á þennan hátt því þessi notkun getur valdið ruglingi.
Reyndar ætla ég ekki að segja hver niðurstaðan er.  
Þú verður bara að prófa þetta ef þú iðar í skinninu að vita það!

Með því að nota hækkunarvirkjann getum við endurskrifað kóðann sem telur tilvik af staf:

\begin{verbatim}
  int index = 0;
  while (index < length) {
    if (fruit[index] == 'a') {
      count++;
    }
    index++;
  }
\end{verbatim}
%
Algeng villa er að skrifa eitthvað á þessa leið:

\begin{verbatim}
  index = index++;             // WRONG!!
\end{verbatim}
%
Því miður vill svo til að þetta er málfræðilega rétt og því mun þýðandinn ekki gera neinar athugasemdir.
Niðurstaðan af þessari setningu er sú að gildið á {\tt index} helst óbreytt.
Það er oft erfitt að finna þessa villu.

Mundu að þú getur skrifað {\tt index = index +1;} eða {\tt index++;}.
Þú ættir hins vegar ekki að blanda þessu tvennu saman.

\section{Samskeyting strengja}

Það er athyglisvert að hægt er að beita virkjanum {\tt +} á strengi en þá framkvæmir virkinn {\bf samskeytingu} (e. concatenation).
Samskeyting merkir að skeyta saman tveimur (eða fleiri) strengjum.
Dæmi:

\begin{verbatim}
  string fruit = "banana";
  string bakedGood = " nut bread";
  string dessert = fruit + bakedGood;
  cout << dessert << endl;
\end{verbatim}
%
Úttakið úr þessu forriti er {\tt banana nut bread}.

Því miður virkar {\tt +} virkinn hins vegar ekki á ``native'' C-strengi og því er ekki hægt að skrifa

\begin{verbatim}
  string dessert = "banana" + " nut bread";
\end{verbatim}
%
vegna þess að báðir þolendur {\tt +} virkjans eru hér C-strengir (\verb+"banana"+ og \verb+" nut bread"+).
Svo framarlega sem annar þolandanna er {\tt string} þá mun C++ reyndar breyta hinum sjálfvirkt í {\tt string}.

Það er líka mögulegt að skeyta stökum staf við byrjun eða enda á streng.
Í eftirfarandi dæmi notum við samskeytingu og stafareikning (e. character arithmetic) til að skrifa út ``abecedarian'' röð.

``Abecedarian'' vísar til raðar eða lista hvers stök eru í stafrófsröð.
Í bók Robert McCloskey's {\em Make Way for Ducklings} eru t.d. nöfn andarunganna Jack,
Kack, Lack, Mack, Nack, Ouack, Pack og Quack.
Hér er lykkja sem skrifar út þessi nöfn í réttri röð: 

\begin{verbatim}
  string suffix = "ack";

  char letter = 'J';
  while (letter <= 'Q') {
    cout << letter + suffix << endl;
    letter++;
  }
\end{verbatim}
%
Úttak forritsins er:

\begin{verbatim}
Jack
Kack
Lack
Mack
Nack
Oack
Pack
Qack
\end{verbatim}
%
Auðvitað er þetta ekki alveg rétt því ``Ouack'' og ``Quack'' skrifast ekki alveg rétt út.
Þú ættir að breyta forritinu og leiðrétta þessa villu.

Hér þurfum við aftur að passa okkur að nota samskeytingu aðeins með {\tt string} taginu en ekki með ``native'' C-strengjum.
Því miður er segð eins og {\tt letter} + \verb+"ack"+ málfræðilega rétt í C++ en útkoman er mjög skrýtin, a.m.k. í mínu þróunarumhverfi.

\section{Strengir eru breytanlegir}
\index{klasi!string}
\index{string!breytanlegir}
\index{string}

Þú getur breytt einstökum stöfum í streng með því að nota {\tt []} virkjann á vinstri hlið gildisveitingar:
Þetta dæmi,

\begin{verbatim}
  string greeting = "Hello, world!";
  greeting[0] = 'J';
  cout << greeting << endl;
\end{verbatim}
%
skrifar út {\tt Jello, world!}.


\section{Strengir eru samanburðarhæfir}
\label{incomparable}
\index{klasi!string}
\index{samanburður!string}
\index{string}

Allir samanburðarvirkjarnir sem virka fyrir {\tt int} og {\tt double} virka líka fyrir {\tt string}.
Ef þú vilt t.d. vita hvort tveir strengir eru jafnir:

\begin{verbatim}
  if (word == "banana") {
    cout << "Yes, we have no bananas!" << endl;
  }
\end{verbatim}
%
Hinir samanburðarvirkjarnir eru t.d. gagnlegir til að raða orðum í stafrófsröð:

\begin{verbatim}
  if (word < "banana") {
    cout << "Your word, " << word << ", comes before banana." << endl;
  } else if (word > "banana") {
    cout << "Your word, " << word << ", comes after banana." << endl;
  } else {
    cout << "Yes, we have no bananas!" << endl;
  }
\end{verbatim}
%
Athugaðu þó að {\tt string} klasinn meðhöndlar ekki hástafi og lágstafi á sama hátt og við gerum.
Allir hástafir koma á undan öllum lágstöfum.
Þess vegna,
\begin{verbatim}
Your word, Zebra, comes before banana.
\end{verbatim}
%
Algeng leið til að bregðast við þessu vandamáli er að breyta strengjum yfir á staðlað form, t.d. yfir í lágstafi, áður en samanburður er framkvæmdur.
Næsti kafli skýrir hvernig það er gert.
%I will not address the
%more difficult problem, which is making the program realize that
%zebras are not fruit.

\section{Flokkun stafa}

Það er oft gagnlegt að skoða staf og athuga hvort hann er hástafur eða lágstafur eða hvort hann er bókstafur eða tölustafur.
C++ fylgir safn falla (e. library functions) sem getur flokkað stafi á þennan hátt.
Þú þarf að setja hausaskrána {\tt ctype.h} inn í forritið þitt til að geta notað þessi föll.

\begin{verbatim}
  char letter = 'a';
  if (isalpha(letter)) {
    cout << "The character " << letter << " is a letter." << endl;
  }
\end{verbatim}
%
Hér er eðlilegt að gera ráð fyrir því að skilagildið úr fallinu {\tt isalpha} sé {\tt bool}, 
en af skrýtnum ástæðum er skilagildið reyndar heiltala sem er 0 ef viðfangið er ekki bókstafur en einhver önnur tala ef viðfangið er bókstafur.

Þessi undarlegheit eru reyndar ekki eins óþægileg eins og kannski virðist því það er leyfilegt að nota svona heiltölu í skilyrði eins og sést í dæminu.
Gildið 0 er meðhöndlað sem {\tt false} en allar aðrar tölur eru meðhöndlaðar sem {\tt true}.

Tæknilega séð ætti svona lagað ekki að vera leyfilegt -- heiltölugildi eru frábrugðin boole gildum.
Samt sem áður getur þessi C++ venja, að breyta sjálfvirkt á milli taga, stundum verið gagnleg.

Önnur föll sem flokka stafi eru t.d. {\tt isdigit}, sem ber kennsl á tölustafina 0 til 9, 
og {\tt isspace}, sem ber kennsl á allar tegundir af ``hvítum'' bilum (e. white spaces), þ.m.t. bil, dálkastaf og nýja línu.
Einnig má nefna föllin {\tt isupper} og {\tt islower} sem gera greinarmun á hástöfum og lágstöfum.

Að lokum nefni ég tvö föll sem breyta stöfum úr eini formi í annað, {\tt toupper} og {\tt tolower}.
Bæði þessi föll taka einn staf sem viðfang og skila (hugsanlega) breyttum staf. 

\begin{verbatim}
  char letter = 'a';
  letter = toupper (letter);
  cout << letter << endl;
\end{verbatim}
%
Úttakið úr þessum kóða er {\tt A}.

Til að æfa þig ættir þú að nota ofangreind föll til að skrifa föllin {\tt stringToUpper} og
{\tt stringToLower} sem bæði taka einn {\tt string} sem viðfang og breyta honum þannig að öllum stöfum er breytt í hástafi eða lágstafi.
Skilagildi fallanna ætti að vera {\tt void}.

\section{Önnur strengjaföll}

Í þessum kafla höfum við ekki talað um öll strengjaföllin.
Við munum ræða tvö í viðbót, {\tt c\_str} í kafla ~\ref{finput} og {\tt substr} í kafla~\ref{parsing}.

\section{Orðalisti}

\begin{description}

\item[hlutur (e. object):] Tilvik af tilteknum klasa. Tilvikinu fylgja mengi af föllum sem framkvæma aðgerðir á því.
Hlutirnir sem við höfum notað hingað til eru {\tt cout} og tilvik af {\tt string}.

\item[vísir (e. index):]  Breyta eða gildi sem notuð er til að velja eitt stak í röðuðu mengi, t.d. einn staf úr streng.

\item[ferðast eftir (e. traverse):]  Að ítra í gegnum öll stök mengis og framkvæma sömu aðgerð á sérhverju staki.

\item[teljari (e. counter):]  Breyta sem er notuð til að telja eitthvað. Breytan er yfirleitt upphafsstillt með 0 og síðan hækkuð.

\item[hækka (e. increment):]  Að hækka gildi breytu um einn.
Hækkunarvirkinn í C++ er {\tt ++}.  Þetta er einmitt ástæðan fyrir því að C++ er kallað C++, því markmiðið var að það væri einum betra en C!

\item[lækka (e. decrement):]  Að lækka gildi breytu um einn.
Lækkunarvirkinn í C++ er \verb+--+.

\item[skeyta saman (e. concatenate):] Að sameina tvo þolendur.

\index{hlutur}
\index{vísir}
\index{ferðast um}
\index{teljari}
\index{hækka}
\index{lækka}
\index{skeyta saman}

\end{description}

% LaTeX source for textbook ``How to think like a computer scientist''
% Copyright (C) 1999  Allen B. Downey

% This LaTeX source is free software; you can redistribute it and/or
% modify it under the terms of the GNU General Public License as
% published by the Free Software Foundation (version 2).

% This LaTeX source is distributed in the hope that it will be useful,
% but WITHOUT ANY WARRANTY; without even the implied warranty of
% MERCHANTABILITY or FITNESS FOR A PARTICULAR PURPOSE.  See the GNU
% General Public License for more details.

% Compiling this LaTeX source has the effect of generating
% a device-independent representation of a textbook, which
% can be converted to other formats and printed.  All intermediate
% representations (including DVI and Postscript), and all printed
% copies of the textbook are also covered by the GNU General
% Public License.

% This distribution includes a file named COPYING that contains the text
% of the GNU General Public License.  If it is missing, you can obtain
% it from www.gnu.org or by writing to the Free Software Foundation,
% Inc., 59 Temple Place - Suite 330, Boston, MA 02111-1307, USA.

% This is an Icelandic translation/adaptation of the orginal book by Allen B. Downey

\chapter{Strúktúrar}
\label{structs}
\index{struct}

%\section{Compound values}
\section{Samsett gildi}

Flest af þeim gagnatögum sem við höfum unnið með hingað til standa fyrir eitt tiltekið
gildi -- heiltölu, kommutölu og boole gildi.
Strengir eru öðruvísi að því leyti til að þeir eru settir saman úr smærri gildum, þ.e. stöfum.
Strengir eru því dæmi um {\bf samsett} gildi (e. compound type).

Við gætum þurft að meðhöndla samsett gildi sem einn tiltekinn hlut og við gætum þurft að komast í einstaka hluta (tilvikabreytur (e instance variables)) samsetta gildisins.
%This ambiguity is useful.

Það er einnig gagnlegt fyrir þig að geta búið til þín eigin samsett gildi.
C++ býður upp á tvær leiðir til að gera það: {\bf strúktúra} (e. structures) og {\bf klasa} (e. classes).
Við munum byrja á því að fjalla um strúktúra og förum síðan í klasa í kafla~\ref{class} (það er ekki mikill munur á milli þeirra.

\section{{\tt Point} hlutir}
\index{Point}
\index{struct!Point}

Við skulum skoða stærðfræðilegan punkt (eins og í hnitakerfi) sem dæmi um samsett gildi.
Punktur er í raun tvær tölur (hnit) sem við meðhöndlum sem einn tiltekinn hlut.
Í stærðfræði eru punktar oft skrifaðir innan sviga með kommu á milli hnitanna.
T.d. gefur $(0, 0)$ til kynna upphafspunkt (hnitamiðju) og $(x, y)$ stendur fyrir punkt sem er $x$ einingar til hægri og $y$ einingar upp miðað við upphafspunktinn.

Eðlileg leið til að tákna punkt í C++ er að nota tvær kommutölur, {\tt double}.
Spurningin er hins vegar hvernig hægt er að setja þessi tvö gildi saman í samsettan hlut eða strúktúr.
Það er hægt með því að nota {\tt struct} skilgreiningu:

\begin{verbatim}
struct Point {
  double x, y;
};  
\end{verbatim}
%
{\tt struct} skilgreining kemur yfirleitt fyrir utan fallaskilgreininga, í upphafi forrits (á eftir {\tt include} setningum).

Þessi skilgreining gefur til kynna að í strúktúrnum eru tvö stök (gildi), nefnd {\tt x} og {\tt y}.
Þessi stök eru kölluð {\bf tilvikabreytur} (e. instance variables), en ég mun skýra síðar hver ástæðan er fyrir því.

Það er algeng villa að gleyma semíkommunni í enda strúktúrskilgreiningar.
Það virðist skrýtið að setja semíkommu á eftir slaufusviga en þú venst því fljótt.

Þegar þú hefur skilgreint nýjan strúktúr þá getur þú búið til breytur af því tagi:

\begin{verbatim}
  Point blank;
  blank.x = 3.0;
  blank.y = 4.0;   
\end{verbatim}
%
Fyrsta línan er hefðbundin yfirlýsing á breytu: {\tt blank} er af taginu {\tt Point}.
Næstu tvær línur upphafsstilla tilvikabreytur strúktúrsins.
Hér er punktatáknun notuð á svipaðan hátt og þegar kallað er á fall sem tilheyrir tilteknum hlut, eins og í {\tt fruit.length()}.
Munurinn er auðvitað sá að fallakalli fylgir viðfangalisti, jafnvel þó hann sé tómur.

\index{yfirlýsing}
\index{setning!yfirlýsing}
\index{tilvísun}
\index{stöðurit}
\index{staða}

Niðurstaðan af þessum gildisveitingum sést í eftirfarandi stöðuriti:

\vspace{0.1in}
\centerline{\epsfig{figure=point.eps}}
\vspace{0.1in}

Að venju birtist nafn breytunnar {\tt blank} utan kassans en gildi hennar innan hans.
Í þessu tilviki er gildið samsettur hlutur með tveimur tilvikabreytum.

\section{Aðgangur að tilvikabreytum}
\index{struct!tilvikabreytur}

Þú getur lesið gildi tilvikabreytu með því að nota sömu málskipan og við notuðum til að gefa henni gildi:

\begin{verbatim}
    int x = blank.x;
\end{verbatim}
%
Segðin {\tt blank.x} merkri ``farðu í hlutinn með nafninu {\tt blank} og náðu í gildið á {\tt x}.''
Í þessu tilviki gefum við staðværri breytu með nafninu {\tt x} það gildi.
Taktu eftir því að þá er enginn ``árekstur'' á milli staðværu breytunnar {\tt x} og tilvikabreytunnar {\tt x}.
Tilgangur punktatáknunar er einmitt sá að gefa til kynna hvaða breytu þú ert að vísa í án þess að einhver margræðni sé til staðar.

Hægt er að nota punktatáknun sem hluta af hvaða C++ segð sem er, þannig að eftirfarandi er t.d. löglegt:

\begin{verbatim}
  cout << blank.x << ", " << blank.y << endl;
  double distance = blank.x * blank.x + blank.y * blank.y;
\end{verbatim}
%
Fyrri línar skrifar út {\tt 3, 4} og seinni línan reiknar út gildið 25.

\section{Aðgerðir á strúktúrum}
\index{struct!aðgerðir}

Flestum af þeim aðgerðum sem við höfum notað á önnur tög, eins og stærðfræðivirkjarnir (e. mathematical operators) ( {\tt +}, {\tt \%}, o.s.frv.)
og samanburðarvirkjarnir (e. comparison operators) ({\tt ==}, {\tt >}, o.s.frv.) er ekki hægt að beita á strúktúra.
Reyndar er hægt að breyta merkingu þessara virkja fyrir ný tög en við munum ekki fjalla um það í þessari bók.

Gildisveitingarvirkjanum (e. assignment operator) er aftur á móti hægt að beita á strúktúra.
Hægt er að nota hann á tvo vegu: til að upphafsstilla tilvikabreytu strúktúrs eða til að afrita gildi tilvikabreytu úr einum strúktúr í annan.
Upphafsstilling lítur svona út:

\begin{verbatim}
  Point blank = { 3.0, 4.0 };
\end{verbatim}
%
Tilvikabreytur breytunnar {\tt blank} fá hér gildin úr slaufusviganum, í þeirri röð sem þau eru sett fram.
Þannig að hér fær {\tt blank.x} fyrsta gildið (3.0) og {\tt blank.y} annað gildið (4.0). 

Því miður er eingöngu hægt að nota þessa málskipan í upphafsstillingu en ekki í gildisveitingarsetningu.
Eftirfarandi er því ekki löglegt:

\begin{verbatim}
  Point blank;
  blank = { 3.0, 4.0 };       // WRONG !!
\end{verbatim}
%
Það er eðlilegt að velta því fyrir sér hver ástæðan er fyrir því að jafn eðlileg setning sé óleyfileg.
Ég er ekki alveg viss en vandamálið gæti verið það að þýðandinn veit ekki hvert tagið á hægri hliðinni er.
Ef þú bætir við tagmótun (e. typecast) þá er allt í fína:

\begin{verbatim}
  Point blank;
  blank = (Point){ 3.0, 4.0 };
\end{verbatim}
%

Það er jafnframt leyfilegt að gefa einum stŕúktúr gildi annars strúkturs. Dæmi: 

\begin{verbatim}
  Point p1 = { 3.0, 4.0 };
  Point p2 = p1;
  cout << p2.x << ", " <<  p2.y << endl;
\end{verbatim}
%
Úttakið úr þessu forriti er {\tt 3, 4}.

\section{Strúktúrar sem viðföng}
\index{viðfang}
\index{struct!viðfang}

Þú getur haft strúktúr sem lepp í falli, t.d. :

\begin{verbatim}
void printPoint (Point p) {
  cout << "(" << p.x << ", " << p.y << ")" << endl;
}
\end{verbatim}
%
{\tt printPoint} tekur punkt sem viðfang og skrifar gildi hans út á staðlaðan hátt.
Ef þú kallar á fallið með {\tt printPoint (blank)} þá skrifast út {\tt (3, 4)}.

Tökum annað dæmi.  Við getum endurskrifað {\tt distance} fallið úr kafla~\ref{distance} þannig að það taki tvo punkta sem viðföng í stað fjögurrra kommutalna:

\begin{verbatim}
double distance (Point p1, Point p2) {
  double dx = p2.x - p1.x;
  double dy = p2.y - p1.y;
  return sqrt (dx*dx + dy*dy);
}
\end{verbatim}

\section{Kallað með gildi}
\index{stikun færibreytna}
\index{kall!með gildi}

Það er mikilvægt að gera sér grein fyrir því að þegar strúktúr er sendur sem viðfang í fall þá er viðfangið (e. argument/actual parameter) og leppurinn (e. formal parameter) ekki sama breytan.
Um er að ræða tvær breytur (önnur í þeim sem kallar (e. caller) og hin í þeim sem kallað er á (e. callee)) sem hafa sama gildið, a.m.k. í upphafi.
Þegar við t.d. köllum á {\tt printPoint} þá lítur stöðuritið svona út: 

\vspace{0.1in}
\centerline{\epsfig{figure=point2.eps}}
\vspace{0.1in}
%
Ef {\tt printPoint} breytir annarri (eða báðum) tilvikabreytum {\tt p} þá mun það ekki hafa nein áhrif á {\tt blank} (auðvitað er engin ástæða fyrir {\tt printPoint} að breyta leppnum sínum).

Þessi tegund af {\bf stikun færibreytna} (e. parameter passing) er kölluð ``kall með gildi'' (e. ``pass by value'')
vegna þess að gildi (e. value) strúktúrsins (eða hvaða tags sem er) er sent til fallsins.

\section{Kallað með tilvísun}
\index{stikun færibreytna}
\index{kall!með tilvísun}
\index{tilvísun}

Önnur aðferð við stikun færibreytna í C++ er ``kall með tilvísun'' (e. ``pass by reference'').
Þessi aðferð gerir þér kleift að senda strúktúr í fall og breyta gildum hans!

Þú gætir t.d. speglað punkti um 45-gráðu línuna með því að skipta gildum hnitanna tveggja.
Augljósasta leiðin (en ekki sú rétta) er að útfæra {\tt reflect} fallið á þennan hátt:

\begin{verbatim}
void reflect (Point p)      // WRONG !!
{
  double temp = p.x;
  p.x = p.y;
  p.y = temp;
}
\end{verbatim}
%
Þetta virkar ekki vegna þess að það að gera breytingar á leppnum í {\tt reflect} hefur engin áhrif á þann sem kallaði.

Í staðinn verðum við að tilgreina að við ætlum að senda viðfangið með tilvísun (e. by reference).
Það gerum við með því að bæta tákninu {\tt \&} við í skilgreiningu á leppunum: 

\begin{verbatim}
void reflect (Point& p)
{
  double temp = p.x;
  p.x = p.y;
  p.y = temp;
}
\end{verbatim}
%
Núna getum við kallað á fallið á hefðbundin hátt:

\begin{verbatim}
  printPoint (blank);
  reflect (blank);
  printPoint (blank);
\end{verbatim}
%
Úttak forritsins er eins og við gerum ráð fyrir: 

\begin{verbatim}
(3, 4)
(4, 3)
\end{verbatim}
%
Svona myndi síðan stöðurit líta út fyrir þetta forrit:

\vspace{0.1in}
\centerline{\epsfig{figure=point3.eps}}
\vspace{0.1in}
%
Leppurinn {\tt p} er tilvísun í strúktúr með nafnið {\tt blank}.
Hefðbundin táknun fyrir tilvísun er punktur með ör sem bendir á það sem tilvísunin vísar á.

Það mikilvæga sem hægt er að lesa úr þessu stöðuriti er að allar breytingar sem 
{\tt reflect} gerir á {\tt p} munu einnig hafa áhrif á {\tt blank}.

Það að stika strúktúr með tilvísun (e. by reference) er sveigjanlegra heldur en stika hann með gildi (e. by value)
vegna þess að sá sem kallað er á getur breytt strúktúrnum.
Það er jafnframt hraðvirkara vegna þess að kerfið þarf ekki að afrita heilan strúktúr.
Á hinn bóginn má segja að það sé ekki eins öruggt vegna þess að það er erfiðara að gera sér grein fyrir hvaða breytingar eru gerðar hvar.
Samt sem áður eru strúktúrar yfirleitt stikaðir með tilvísun í C++ forritum og ég mun fylgja þeirri venju í þessari bók.

\section{Rectangles}
\index{Rectangle}
\index{struct!Rectangle}

Now let's say that we want to create a structure to represent
a rectangle.  The question is, what information do I have to
provide in order to specify a rectangle?  To keep things simple
let's assume that the rectangle will be oriented vertically or
horizontally, never at an angle.

There are a few possibilities: I could specify the center of
the rectangle (two coordinates) and its size (width and height),
or I could specify one of the corners and the size, or I
could specify two opposing corners.

The most common choice in existing programs is to specify the
upper left corner of the rectangle and the size.  To do that
in C++, we will define a structure that contains a {\tt Point}
and two doubles.

\begin{verbatim}
struct Rectangle {
  Point corner;
  double width, height;
};  
\end{verbatim}
%
Notice that one structure can contain another.  In fact, this
sort of thing is quite common.  Of course, this means that in
order to create a {\tt Rectangle}, we have to create a {\tt Point}
first:

\begin{verbatim}
  Point corner = { 0.0, 0.0 };
  Rectangle box = { corner, 100.0, 200.0 };
\end{verbatim}
%
This code creates a new {\tt Rectangle} structure and initializes the
instance variables.  The figure shows the effect of this assignment.

\vspace{0.1in}
\centerline{\epsfig{figure=rectangle.eps}}
\vspace{0.1in}
%
We can access the {\tt width} and {\tt height} in the usual way:

\begin{verbatim}
  box.width += 50.0;
  cout << box.height << endl;
\end{verbatim}
%
In order to access the instance variables of {\tt corner}, we can use a
temporary variable:

\begin{verbatim}
  Point temp = box.corner;
  double x = temp.x;
\end{verbatim}
%
Alternatively, we can compose the two statements:

\index{composition}

\begin{verbatim}
  double x = box.corner.x;
\end{verbatim}
%
It makes the most sense to read this statement from right to
left: ``Extract {\tt x} from the {\tt corner} of the {\tt box},
and assign it to the local variable {\tt x}.''

While we are on the subject of composition, I should point
out that you can, in fact, create the {\tt Point} and the
{\tt Rectangle} at the same time:

\begin{verbatim}
  Rectangle box = { { 0.0, 0.0 }, 100.0, 200.0 };
\end{verbatim}
%
The innermost squiggly braces are the coordinates of the
corner point; together they make up the first of the three
values that go into the new {\tt Rectangle}.  This statement
is an example of {\bf nested structure}.

\index{nested structure}


\section{Structures as return types}
\index{struct!as return type}
\index{return}
\index{statement!return}

You can write functions that return structures.  For example,
{\tt findCenter} takes a {\tt Rectangle} as an argument and
returns a {\tt Point} that contains the coordinates of the
center of the {\tt Rectangle}:

\begin{verbatim}
Point findCenter (Rectangle& box)
{
  double x = box.corner.x + box.width/2;
  double y = box.corner.y + box.height/2;
  Point result = {x, y};
  return result;
}
\end{verbatim}
%
To call this function, we have to pass a box as an argument
(notice that it is being passed by reference), and assign the
return value to a {\tt Point} variable:

\begin{verbatim}
  Rectangle box = { {0.0, 0.0}, 100, 200 };
  Point center = findCenter (box);
  printPoint (center);
\end{verbatim}
%
The output of this program is {\tt (50, 100)}.

\section {Passing other types by reference}
\index{parameter passing}
\index{call by reference}
\index{reference}

It's not just structures that can be passed by reference.
All the other types we've seen can, too.  For example, to swap
two integers, we could write something like:

\begin{verbatim}
void swap (int& x, int& y)
{
  int temp = x;
  x = y;
  y = temp;
}
\end{verbatim}
%
We would call this function in the usual way:

\begin{verbatim}
  int i = 7;
  int j = 9;
  swap (i, j);
  cout << i << j << endl;
\end{verbatim}
%
The output of this program is {\tt 97}.  Draw a stack
diagram for this program to convince yourself this is true.
If the parameters {\tt x} and {\tt y} were declared as
regular parameters (without the {\tt \&}s), {\tt swap} would
not work.  It would modify {\tt x} and {\tt y} and have no
effect on {\tt i} and {\tt j}.

When people start passing things like integers by reference,
they often try to use an expression
as a reference argument.  For example:

\begin{verbatim}
  int i = 7;
  int j = 9;
  swap (i, j+1);         // WRONG!!
\end{verbatim}
%
This is not legal because the expression {\tt j+1} is not
a variable---it does not occupy a location that the reference
can refer to.  It is a little tricky to figure out exactly
what kinds of expressions can be passed by reference.  For now
a good rule of thumb is that reference arguments have to be
variables.

\section{Getting user input}
\label{input}
\index{input!keyboard}

The programs we have written so far are pretty predictable;
they do the same thing every time they run.  Most of the time,
though, we want programs that take input from the user and
respond accordingly.

There are many ways to get input, including keyboard
input, mouse movements and button clicks, as well as more exotic
mechanisms like voice control and retinal scanning.  In this
text we will consider only keyboard input.

\index{stream}
\index{cin}
\index{cout}

In the header file {\tt iostream},
C++ defines an object named {\tt cin} that handles input in
much the same way that {\tt cout} handles output.  To get an
integer value from the user:

\begin{verbatim}
  int x;
  cin >> x;
\end{verbatim}
%
The {\tt >>} operator causes the program to stop executing and
wait for the user to type something.  If the user types a valid
integer, the program converts it into an integer value and
stores it in {\tt x}.

\index{operator!{\tt >>}}

If the user types something other than an integer,
C++ doesn't report an error, or anything sensible like that.
Instead, it puts some meaningless value in {\tt x} and continues.

Fortunately, there is a way to check and see if an input
statement succeeds.  We can invoke the {\tt good} function on
{\tt cin} to check what is called the {\bf stream state}.
{\tt good} returns a {\tt bool}: if true, then the last input
statement succeeded.  If not, we know that some previous operation
failed, and also that the next operation will fail.

Thus, getting input from the user might look like this:

\begin{verbatim}
#include <iostream>

using namespace std;

int main ()
{
  int x;

  // prompt the user for input
  cout << "Enter an integer: ";

  // get input
  cin >> x;

  // check and see if the input statement succeeded
  if (cin.good() == false) {
    cout << "That was not an integer." << endl;
    return -1;
  }

  // print the value we got from the user
  cout << x << endl;
  return 0;
}
\end{verbatim}
%
{\tt cin} can also be used to input a {\tt string}:

\begin{verbatim}
  string name;

  cout << "What is your name? ";
  cin >> name;
  cout << name << endl;
\end{verbatim}
%
Unfortunately, this statement only takes the first word of
input, and leaves the rest for the next input statement.
So, if you run this program and type your full name, it
will only output your first name.

Because of these problems (inability to handle errors and
funny behavior), I avoid using the {\tt >>} operator altogether,
unless I am reading data from a source that is known to be
error-free.

Instead, I use a function in the header {\tt string} called {\tt getline}.

\begin{verbatim}
  string name;

  cout << "What is your name? ";
  getline (cin, name);
  cout << name << endl;
\end{verbatim}
%
The first argument to {\tt getline} is {\tt cin}, which is
where the input is coming from.  The second argument is the
name of the {\tt string} where you want the result to be
stored.

{\tt getline} reads the entire line until the user hits
Return or Enter.  This is useful for inputting strings that
contain spaces.

In fact, {\tt getline} is generally useful for getting input
of any kind.  For example, if you wanted the user to type an
integer, you could input a string and then check to see if
it is a valid integer.  If so, you can convert it to an integer
value.  If not, you can print an error message and ask the user
to try again.

To convert a string to an integer you can use the {\tt atoi}
function defined in the header file {\tt cstdlib}.  We will
get to that in Section~\ref{parsing}.

\section{Glossary}

\begin{description}

\item[structure:]  A collection of data grouped together and
treated as a single object.

\item[instance variable:]  One of the named pieces of data that make up
a structure.

\item[reference:]  A value that indicates or refers to a variable
or structure.  In a state diagram, a reference appears as an arrow.

\item[pass by value:]  A method of parameter-passing in which the
value provided as an argument is copied into the corresponding
parameter, but the parameter and the argument occupy distinct
locations.

\item[pass by reference:]  A method of parameter-passing in which
the parameter is a reference to the argument variable.  Changes
to the parameter also affect the argument variable.

\index{structure}
\index{instance variable}
\index{reference}
\index{pass by value}
\index{pass by reference}

\end{description}


% LaTeX source for textbook ``How to think like a computer scientist''
% Copyright (C) 1999  Allen B. Downey

% This LaTeX source is free software; you can redistribute it and/or
% modify it under the terms of the GNU General Public License as
% published by the Free Software Foundation (version 2).

% This LaTeX source is distributed in the hope that it will be useful,
% but WITHOUT ANY WARRANTY; without even the implied warranty of
% MERCHANTABILITY or FITNESS FOR A PARTICULAR PURPOSE.  See the GNU
% General Public License for more details.

% Compiling this LaTeX source has the effect of generating
% a device-independent representation of a textbook, which
% can be converted to other formats and printed.  All intermediate
% representations (including DVI and Postscript), and all printed
% copies of the textbook are also covered by the GNU General
% Public License.

% This distribution includes a file named COPYING that contains the text
% of the GNU General Public License.  If it is missing, you can obtain
% it from www.gnu.org or by writing to the Free Software Foundation,
% Inc., 59 Temple Place - Suite 330, Boston, MA 02111-1307, USA.

% This is an Icelandic translation/adaptation by Hrafn Loftsson of the orginal book by Allen B. Downey.

\chapter{Meira um strúktúra}
\label{time}
\index{struct}

\section{Time}
\index{struct!Time}
\index{Time}

Í þeim tilgangi að sýna annað dæmi um strúktur munum við nú skilgreina tag með nafnið {\tt Time} sem notað er til að halda utan um tíma innan dags.
Tími samanstendur af klukkustund (hour), mínútu (minute) og sekúndu (second) þannig að þetta munu verða tilvikabreytur strúktúrsins.

Fyrsta skrefið er þá að ákveða hvert tag sérhverrar tilvikabreytu á að vera.
Það virðist ljóst að {\tt hour} og {\tt minute} ættu að vera heiltölur.
Til að hafa þetta áhugaverðara þá skulum við láta {\tt second} vera {\tt double} þannig að við getum haldið utan um brot af sekúndu.

Svona lítur þá strúktúrinn okkar út: 

\begin{verbatim}
struct Time {
  int hour, minute;
  double second;
};
\end{verbatim}
%
Við getum þá búið til {\tt Time} hlut á venjulegan hátt: 

\begin{verbatim}
  Time time = { 11, 59, 3.14159 };
\end{verbatim}
%
Stöðuritið fyrir þennan hlut lítur svona út:

\vspace{0.1in}
\centerline{\epsfig{figure=time.eps}}
\vspace{0.1in}

Orðið ``tilvik'' (e. instance) er stundum notað þegar við tölum um hluti (e. objects) vegna þess að sérhver hlutur er tilvik af einhverju tagi.
Ástæðan fyrir því að tilvikabreytur bera það nafn er að sérhvert tilvik af einhverju tagi eiga sér afrit af breytum þess tags.

\section{{\tt printTime}}
\label{printobject}
\index{úttak}
\index{setning!úttak}
\index{hlutur!úttak}

Þegar við skilgreinum nýtt tag þá er góð regla að skrifa fall sem skrifar út tilvikabreyturnar á læsilegan hátt.
Dæmi:

\begin{verbatim}
void printTime (Time& t) {
  cout << t.hour << ":" << t.minute << ":" << t.second << endl;
}
\end{verbatim}
%
Ef við sendum inn {\tt time} sem viðfang í þetta fall þá er úttakið {\tt 11:59:3.14159}.

\begin{verbatim}
#include <iostream>

using namespace std;

struct Time {
  int hour, minute;
  double second;
};


void printTime (Time& t) {
  cout << t.hour << ":" << t.minute << ":" << t.second << endl;
  cout << "Time is " << t.hour << " hour " << t.minute << " minutes " 
       << t.second << "  seconds  "<<endl;
}


int main ()
{
 Time time = { 11, 59, 3.14159 };
 printTime(time);
 
 return 0;
}
\end{verbatim}
%

\section{Föll fyrir hluti}
\label{objectops}
\index{hlutur}
\index{föll!fyrir hluti}

Í næstu köflum mun ég sýna dæmi um nokkur möguleg skil (e. interfaces) fyrir föll sem vinna með hluti.
Þú munt hafa val um nokkur möguleg skil fyrir tilteknar aðgerðir þannig að þú ættir að meta kosti og galla sérhvers möguleika:

\begin{description}

\item[hreint fall (e. pure function):]  Tekur hlut og/eða grunntag sem viðföng en breytir ekki hlutnum.
Skilagildið er annaðhvort grunntag eða nýr hlutur sem búinn er til í fallinu. 

\item[breytir (e. modifier):]  Tekur hluti sem viðföng og breytir sumum eða öllum þeirra.  Skilar oft void. \index{void}

\item[fyllir (e. fill-in function):]  Eitt af viðföngunum er ``tómur'' hlutur sem fallið fyllir inn í.
Tæknilega séð er þetta þá í raun tegund af breyti. 

\end{description}

\section{Hrein föll}
\index{hreint fall}
\index{fall}
\index{fall!hreint}

Fall er sagt vera hreint ef skilagildi þess er eingöngu háð viðföngunum og að það hafi ekki neinar aukaverkanir (e. side effects)
eins og að breyta viðfangi eða skrifa eitthvað út.
Það eina sem gerist þegar kallað er á hreint fall er að skilagildi þess kemur til baka.

Eitt dæmi um hreint fall er {\tt after} sem ber saman tvo {\tt Time} hluti og skilar {\tt bool} sem gefur til kynna hvort fyrra viðfangið komi á eftir því seinna (tímalega séð):

\begin{verbatim}
bool after (Time& time1, Time& time2) {
  if (time1.hour > time2.hour) return true;
  if (time1.hour < time2.hour) return false;

  if (time1.minute > time2.minute) return true;
  if (time1.minute < time2.minute) return false;

  if (time1.second > time2.second) return true;
  return false;
}
\end{verbatim}
%
Hvert er skilagildi þessa falls ef tímarnir tveir eru jafnir?
Finnst þér það vera viðeigandi skilagildi fyrir þetta fall?
Ef þú værir að skjala þetta fall, myndir þú nefna þetta tilfelli sérstaklega?

Annað dæmi er {\tt addTime} sem reiknar summuna af tveimur tímasetningum.
Ef tíminn er t.d. {\tt 9:14:30} og brauðgerðin þín tekur 3 klukkustundir og 35 mínútur þá gætir þú notað {\tt addTime} til að finna út hvenær brauðið verður tilbúið.

Hér eru drög, sem eru reyndar ekki alveg rétt, af þessu falli:

\begin{verbatim}
Time addTime (Time& t1, Time& t2) {
  Time sum;
  sum.hour = t1.hour + t2.hour;
  sum.minute = t1.minute + t2.minute;
  sum.second = t1.second + t2.second;
  return sum;
}
\end{verbatim}
%
Hér er dæmi um hvernig hægt er að nota þetta fall.
Ef {\tt currentTime} inniheldur núverandi tíma og {\tt breadTime} inniheldur tímann sem tekur að baka brauðið þá getur þú notað {\tt addTime} til að reikna út hvenær brauðið verður tilbúið.

\begin{verbatim}
  Time currentTime = { 9, 14, 30.0 };
  Time breadTime = { 3, 35, 0.0 };
  Time doneTime = addTime (currentTime, breadTime);
  printTime (doneTime);
\end{verbatim}
%
Úttak þessa forrits er {\tt 12:49:30} sem er rétt.
Á hinn bóginn eru tilvik þar sem niðurstaðan er ekki rétt.
Getur þú fundið dæmi um þess konar tilvik?

Vandamálið er að þetta fall ræður ekki við þau tilvik þar sem summa sekúndna eða mínútna er stærri en 60.
Þegar það gerirst þá þurfum við að ``færa'' auka sekúndur yfir í mínútudálkinn eða auka mínútur yfir í klukkustundadálkinn.

Hér er önnur, nú rétt, útgáfa af fallinu:

\begin{verbatim}
Time addTime (Time& t1, Time& t2) {
  Time sum;
  sum.hour = t1.hour + t2.hour;
  sum.minute = t1.minute + t2.minute;
  sum.second = t1.second + t2.second;

  if (sum.second >= 60.0) {
    sum.second -= 60.0;
    sum.minute += 1;
  }
  if (sum.minute >= 60) {
    sum.minute -= 60;
    sum.hour += 1;
  }
  return sum;
}
\end{verbatim}
%
Þó svo að þessi útgáfa sé rétt þá er hún orðin dálítið löng.
Ég mun síðar leggja til aðra lausnaraðferð sem er mun styttri.

\index{increment}
\index{decrement}
\index{auki}
\index{frádrag}
\index{virki!auki}
\index{virki!frádrag}

Kóðinn að ofan sýnir dæmi um tvo virkja sem við höfum ekki séð áður, {\tt +=} og {\tt -=}.
Þessir virkjar eru notaðir sem samþjöppuð leið til að hækka eða lækka breytur.
Setningin {\tt sum.second -= 60.0;} er t.d. jafngild setningunni {\tt sum.second = sum.second - 60;}

\section{{\tt const} leppar}

Þú hefur væntanlega tekið eftir því að viðföngin í föllin {\tt after}
og {\tt addTime} eru send með tilvísun (e. by reference).
Þar sem um er að ræða hrein föll þá breyta þau ekki viðföngunum og því hefði ég alveg eins getað sent þau sem gildi (e. by value).

Kosturinn við að senda viðföng sem gildi er að fallið sem kallað er á og sá sem kallar eru hjúpuð á viðeigandi hátt -- breyting í öðrum þeirra getur ekki leitt til breytingar í hinum, nema í tengslum við skilagildið.

Á hinn bóginn er yfirleitt skilvirkara að senda viðföng með tilvísun vegna þess að þá þarf ekki að afrita nein gildi úr viðföngunum yfir í leppana.
Svo vill líka til að C++ inniheldur eiginleika sem kallaður er {\tt const} og gerir það að verkum að það er jafn öruggt að senda tilvísunarviðföng eins og gildisviðföng.

Ef þú skrifar fall og ætlar ekki að breyta viðfangi þá getur þú skilgreint leppinn sem {\bf constant reference parameter} (fast tilvísunarviðfang).
Málskipanin lítur svona út:

\begin{verbatim}
void printTime (const Time& time) ...
Time addTime (const Time& t1, const Time& t2) ...
\end{verbatim}
%
Hér sýni ég aðeins fyrstu línuna í föllunum.
Ef þú lætur þýðandann vita að þú ætlir ekki að breyta viðfangi í falli þá getur hann minnt þig á það!
Ef þú reynir að breyta viðfangi þá mun þýðandinn kvarta.

\index{keyrsluvilla}
\index{villa!við keyrslu}

\section{Breytiföll}
\index{modifier}
\index{fall!breytir}

Auðvitað vill svo til stundum að þú vilt einmitt breyta viðfangi.  Fall sem gerir það eru kallað breytir (e. modifier).

Skoðum fallið {\tt increment} sem dæmi um breyti en það bætir tilteknum fjölda sekúndna við {\tt Time} hlut.
Gróf drög fyrir þetta fall gæti litið svona út:

\begin{verbatim}
void increment (Time& time, double secs) {
  time.second += secs;

  if (time.second >= 60.0) {
    time.second -= 60.0;
    time.minute += 1;
  }
  if (time.minute >= 60) {
    time.minute -= 60;
    time.hour += 1;
  }
}
\end{verbatim}
%
Fyrsta línan framkvæmir grunnaðgerð en það sem á eftir kemur sér um sérstök tilfelli, eins og við sáum áður.

Er þetta fall rétt? Hvað gerist ef leppurinn {\tt secs} er mikið stærri en 60?
Í því tilfelli er ekki nóg að draga 60 frá einu sinni -- við þurfum að gera það þangað til {\tt second} er lægra en 60.

Við getum gert það með því að skipta {\tt if} setningum út fyrir {\tt while} setningar:

\begin{verbatim}
void increment (Time& time, double secs) {
  time.second += secs;

  while (time.second >= 60.0) {
    time.second -= 60.0;
    time.minute += 1;
  }
  while (time.minute >= 60) {
    time.minute -= 60;
    time.hour += 1;
  }
}
\end{verbatim}
%
Þessi lausn er rétt en ekki mjög skilvirk.
Getur þú séð fyrir þér lausn sem þarfnast ekki ítrunar?

\section{Fylliföll}
\index{fyllir}
\index{fall!fyllir}

Af og til sérð þú fall eins og {\tt addTime} skrifað með því að nota önnur skil (e. interface), þ.e. önnur viðföng og annað skilagildi.
Í stað þess að búa til nýjan hlut í sérhvert sinn sem kallað er á {\tt addTime} þá gætum við krafist þess að sá sem kallar útvegi ``tóman'' hlut sem {\tt addTime} getur sett upplýsingar inn í.
Berðu saman eftirfarandi útgáfu og þá fyrri:

\begin{verbatim}
void addTimeFill (const Time& t1, const Time& t2, Time& sum) {
  sum.hour = t1.hour + t2.hour;
  sum.minute = t1.minute + t2.minute;
  sum.second = t1.second + t2.second;

  if (sum.second >= 60.0) {
    sum.second -= 60.0;
    sum.minute += 1;
  }
  if (sum.minute >= 60) {
    sum.minute -= 60;
    sum.hour += 1;
  }
}
\end{verbatim}
%
Kosturinn við þessa útfærslu er að sá sem kallar hefur möguleika á að endurnýta sama hlutinn aftur og aftur til að framkvæma röð af samlagningum á {\tt Time} hlutum.
Þetta getur verið aðeins skilvirkara þó að þetta geti verið ruglandi og valdið lúmskum villum.
Fyrir flest forritunarverkekfni getur verið gott að eyða meiri keyrslutíma til að koma í veg fyrir langan kembitíma síðar meir.

Athugaðu að hægt er að skilgreina fyrstu tvo leppana sem {\tt const} en ekki þann þriðja.

\section{Hvað er best?}
\index{forritunarstíll}

Það sem hægt er að gera með breyti eða fylli er líka hægt að gera með hreinu falli.
Það vill reyndar svo til að ákveðin tegund forritunarmála, {\bf fallaforritunarmál}, leyfa eingöngu hrein föll.
Sumir forritarar trúa því að fljótlegra sé að skrifa forrit sem nota eingöngu hrein föll og þau séu líka áreiðanlegri (innihaldi færri villur) en þau sem nota breytiföll.
Aftur á móti geta breytiföll stundum verið hentug og tilvik koma upp þar sem fallaforrit eru ekki eins skilvirk.

Almennt séð mæli ég með því að þú skrifir hrein föll þar sem það liggur beint við og notir aðeins breytiföll ef það er augljós kostur.
Þetta viðhorf mætti kalla fallaforritunarstíl (e. functional programming style).

\section{Stigvaxandi þróun vs. áætlunargerð}
\index{stigvaxandi þróun}
\index{frumgerð}
\index{forritunarþróun!stigvaxandi}
\index{forritunarþróun!áætlunargerð}

Ég hef í þessum kafla sýnt dæmi um aðferð við forritunarþróun sem leiðir af sér {\bf skjóta frumgerð með stigvaxandi endurbótum} (e. rapid prototyping with iterative improvement).
Í sérhverju tilviki skrifaði ég drög (frumgerð) sem framkvæmdi grunnútreikninga, síðan prófaði ég þá á nokkrum tilvikum og leiðrétti villur þegar þær komu í ljós.

Þó svo að þessi aðferð geti verið markvirk (e. effective) þá getur kóðinn orðið óþarflega flókinn, vegna þess að hann meðhöndlar mörg mismunandi tilvik,
og óáreiðanlegur, vegna þess að það getur verið erfitt að fullvissa sig um að maður hafi fundið allar villurnar.

Önnur aðferð er hönnun eða áætlunargerð sem felur í sér að smá innsýn í vandamálið getur gert forritunina mikið einfaldari.
Í þessu tilviki er innsýnin sú að {\tt Time} er í raun tala með þremur tölustöfum með grunn 60!
Sekúndan ({\tt second}) er ``ones column'', mínútan ({\tt minute}) er ``60's column'', og klukkustundin ({\tt hour}) er ``3600's column''.

Þegar við skrifuðum {\tt addTime} og {\tt increment} þá gerðum við í raun samlagningu með grunn 60, sem er einmitt ástæðan fyrir því að við þurftum að ``færa'' frá einum dálki yfir í annan.

\index{útreikningur!grunnur 60}
\index{útreikningur!kommutala}

Önnur leið til að leysa vandamálið er því að breyta {\tt Time} hlutum yfir í {\tt double} breytur og 
nýta sér það að tölvan veit þegar hvernig gera á útreikninga með {\tt double}.
Hér er fall sem breytir {\tt Time} yfir í {\tt double}:

\begin{verbatim}
double convertToSeconds (const Time& t) {
  int minutes = t.hour * 60 + t.minute;
  double seconds = minutes * 60 + t.second;
  return seconds;
}
\end{verbatim}
%
Allt sem við þurfum þá til viðbótar er leið til að breyta {\tt double} yfir í {\tt Time} hlut:

\begin{verbatim}
Time makeTime (double secs) {
  Time time;
  time.hour = int (secs / 3600.0);
  secs -= time.hour * 3600.0;
  time.minute = int (secs / 60.0);
  secs -= time.minute * 60;
  time.second = secs;
  return time;
}
\end{verbatim}
%
Þú þarft líklega að hugsa dálítið um þetta til að fullvissa þig um að aðferðin sem ég nota hér til að breyta úr einum grunn yfir í annan sé rétt.
Að því gefnu að þú sért sannfærð(ur) þá getum við notað þessi föll til að endurskrifa {\tt addTime}:

\begin{verbatim}
Time addTime (const Time& t1, const Time& t2) {
  double seconds = convertToSeconds (t1) + convertToSeconds (t2);
  return makeTime (seconds);
}
\end{verbatim}
%
Þessi útgáfa er miklu styttri en upphaflega útgáfan og það er líka mun einfaldara að sýna að hún sé rétt.
Þú ættir núna að æfa þig með því að endurskrifa {\tt increment} á sama hátt.

\section{Alhæfing}
\index{alhæfing}

Á vissan hátt er erfiðara að breyta úr grunni 60 yfir í grunn 10, og til baka, en að meðhöndla tíma.
Breyting á grunni er meira abstrakt, tilfinning okkar fyrir tíma er meiri.

Á hinn bóginn má segja að ef við meðhöndlum tíma sem tölur með grunn 60 og skrifum umbreytingaföllin
({\tt convertToSeconds} og {\tt makeTime}) þá endum við með forrit sem er styttra, læsilegra og einfaldara að kemba og jafnframt áreiðanlegra.

Það er jafnframt auðveldara að bæta eiginleikum við forritið síðar.
Gefum okkur t.d. að við þurfum að geta dregið einn tíma frá öðrum til að finna tímann á milli þeirra.
Bein leið væri að útfæra frádráttinn með ``láni'' (e. borrowing).
Það væri hins vegar auðveldara að nota umbreytingaföllin og jafnframt líklegra til að vera rétt.

Það er kaldhæðnislegt að stundum verður vandamál auðveldara (færri sérstök tilvik og færri möguleikar á villum) ef það er gert erfiðara (almennara)!

\section{Reiknirit}
\label{algorithm}
\index{reiknirit}

Þegar þú skrifar almenna lausn fyrir safn af vandamálum, í stað þess að skrifa sérstaka lausn fyrir eitt tiltekið vandamál, þá hefur þú útfært 
{\bf reiknirit} (e. algorithm).
Ég nefndi þetta orð í kafla 1 en skilgreindi það ekki nákvæmlega.
Það er reyndar ekki auðvelt að skilgreina en hér mun ég reyna það með tveimur aðferðum.

Í fyrsta lagi skaltu hugsa um eitthvað sem er ekki reiknirit.
Þegar þú lærðir t.d. að margfalda saman tölur með einum tölustaf þá lærðir þú væntanlega margföldunartöfluna utan að.
Í rauninni settir þú 100 mismunandi lausnir á minnið!
Þess konar þekking er í raun ekki reikniritanleg.

En ef þú varst ``löt/latur'' þá svindlaðir þú líklega og lærðir nokkur trikk.
Til að finna t.d. margfeldið af $n$ and 9 getur þú skrifað $n-1$ sem fyrri tölustafinn og $10-n$ seinni stafinn.
Þetta trikk er almenn lausn til að margfalda tölu (með einum tölustaf) með 9.
Þetta er reiknirit!

Á sama hátt eru aðferðirnar sem þú lærðir til að leggja saman, með því færa yfir, og að draga frá, með því að taka að láni, líka reiknirit.
Eitt sem einkennir reiknirit er að þau krefjast ekki neinnar sérstakar greindar til að framkvæma þau.
Þau eru sjálfvirk ferli þar sem sérhvert skref er framkvæmt á eftir undanfarandi skrefi í samræmi við einfalt mengi af reglum.

Það er mín skoðun að það sé í raun vandræðalegt hversu löngum tíma við eyðum í skóla að keyra reiknirit sem þarfnast í raun engrar greindar.

Á hinn bóginn er ferlið við að hanna reiknirit mjög áhugavert og krefjandi og í raun hryggjarstykkið í því sem við köllum forritun.

Sumt af því sem manneskjan gerir á náttúrulegan hátt, án vandræða eða meðvitaðrar hugsunar, er hvað erfiðast að tjá með reikniriti.
Það að greina og skilja náttúrulegt tungumál er gott dæmi.
Við gerum það öll en hingað til hefur engum tekist að skýra út {\em hvernig} við gerum það, a.m.k. ekki í formi reiknirits.

Seinna í þessari bók færðu tækifæri til að skrifa einföld reiknirit fyrir ýmis vandamál.
Ef þú ert í tölvunarfræðinámi þá muntu síðar taka námskeiðið Reiknirit (e. Data Structures) og kynnast
mörgum áhugaverðum, snjöllum og gagnlegum reikniritum sem tölvunarfræðingar hafa þróað.

\section{Orðalisti}

\begin{description}

\item[tilvik (e.instance):]  Eitt dæmi úr tilteknum flokki.  Kötturinn minn er tilvik úr flokknum ``læður''. 
Sérhver hlutur (e. object) er tilvik af einhverju tagi.

\item[tilvikabreyta (e. instance variable):]  Ein af þeim breytum sem mynda strúktúr. Sérhvert tilvik af strúktúr á sér eigin afrit af tilvikabreytunum fyrir viðkomandi tag.

\item[fast tilvísunarviðfang (e. constant reference parameter):]  Viðfang sem er sent sem tilvísun í fall en er samt sem áður ekki hægt að breyta í fallinu.

\item[hreint fall (e. pure function):]  Fall, hvers skilagildi er eingöngu háð viðföngum þess, og sem hefur engar aukaverkanir en þær að skila gildi.

\item[fallaforritunarstíll (e. functional programming style):]  Forritunarstíll sem leggur áherslu á að meginhluti falla séu hrein.

\item[breytir (e. modifier):]  Fall sem breytir einu eða fleiri viðföngum og skilar yfirleitt {\tt void}.

\item[fyllir (e. fill-in function):]  Fall sem tekur ``tóman'' hlut sem viðfang og fyllir inn í tilvikabreytur hlutarins í stað þess að skila gildi.

\item[reiknirit (e. algorithm):]  Eins konar uppskrift til að leysa tiltekna tegund af vandamálum með sjálfvirku ferli.

\index{tilvik}
\index{tilvikabreyta}
\index{hreint fall}
\index{fallaforritun}
\index{breytir}
\index{fyllir}
\index{reiknirit}

\end{description}


% LaTeX source for textbook ``How to think like a computer scientist''
% Copyright (C) 1999  Allen B. Downey

% This LaTeX source is free software; you can redistribute it and/or
% modify it under the terms of the GNU General Public License as
% published by the Free Software Foundation (version 2).

% This LaTeX source is distributed in the hope that it will be useful,
% but WITHOUT ANY WARRANTY; without even the implied warranty of
% MERCHANTABILITY or FITNESS FOR A PARTICULAR PURPOSE.  See the GNU
% General Public License for more details.

% Compiling this LaTeX source has the effect of generating
% a device-independent representation of a textbook, which
% can be converted to other formats and printed.  All intermediate
% representations (including DVI and Postscript), and all printed
% copies of the textbook are also covered by the GNU General
% Public License.

% This distribution includes a file named COPYING that contains the text
% of the GNU General Public License.  If it is missing, you can obtain
% it from www.gnu.org or by writing to the Free Software Foundation,
% Inc., 59 Temple Place - Suite 330, Boston, MA 02111-1307, USA.

% This is an Icelandic translation/adaptation by Hrafn Loftsson of the orginal book by Allen B. Downey.

\chapter{Vektorar}
\label{vectors}
\index{vektor}
\index{tag!vektor}

{\bf Vektor} (e. vector) er mengi gilda þar sem hvert þeirra er auðkennt með tölu sem kölluð er vísir (e. index).
Strengur ({\tt string}) er svipaður og vektor því hann samanstendur af mengi af stöfum sem hægt er að vísa í með tölu.
Vektorar í C++ eru hentugir í notkun því þeir geta geymt ýmiss konar gögn, þ.m.t. grunntög eins og {\tt int} og {\tt double}, 
eða tög sem skilgreind eru af notanda, eins og {\tt Point} og {\tt Time}.

Tagið {\tt vector} er skilgreint í C++ ``Standard Template Library'' (STL).
Til að nota það þarf að taka inn (e. include) hausaskrána {\tt vector} -- hvernig það er gert er háð þínu forritunarumhverfi.

Þú getur búið til vektor á sama hátt og þú býrð til breytu af hvaða öðru tagi:

\begin{verbatim}
  vector<int> count;
  vector<double> doubleVector;
\end{verbatim}
%
Tagið sem vektorinn mun geyma kemur fram í hornsvigunum (e. angle brackets) ({\tt <} og {\tt >}).
Fyrri línan býr til vektor af heiltölum með nafnið {\tt count} en seinni línan býr til vektor af {\tt double}.
Þrátt fyrir að þessar setningar séu löglegar þá eru þær ekki mjög gagnlegar vegna þess að þær búa til vektora sem hafa engin stök (stærð þeirra er núll).
Það er því mun algengara að skilgreina stærð vektorsins innan sviga: 

\begin{verbatim}
  vector<int> count (4);
\end{verbatim}
%
Málskipanin er hér dálítið skrýtin því hún lítur út eins og samsetning breytuyfirlýsingar og fallakalls.
Það er reyndar nákvæmlega það sem hún er!
Fallið sem við erum að kalla á er svokallaður smiður í {\tt vector} klasanum.
{\bf Smiður} (e. constructor) er sérstakt fall sem býr til nýtt tilvik og frumstillir tilvikabreytur þess.
Í þessu tilviki tekur smiðurinn eitt viðfang sem er stærð vektorsins.

\index{smiður}

Eftirfarandi mynd sýnir hvernig vektorar eru táknaðir á stöðuritum:

\vspace{0.1in}
\centerline{\epsfig{figure=vector.eps}}
\vspace{0.1in}

Stóru tölurnar innan í boxunum eru {\bf stök} (e. elements) vektorsins. 
Litlu tölurnar fyrir utan boxin eru vísarnir (e. indices) sem notaðir eru til að einkenna sérhvert box.
Stök vektors eru ekki frumstillt þegar nýr vektor er búinn til og því gætu þau gætu í raun innihaldið hvaða gildi sem er.

Það er til annar smiður fyrir {\tt vector} sem tekur tvö viðföng; seinna viðfangið er ``frumstillingargildi'', þ.e. gildið sem sérhvert stak vektorsins fær í upphafi.

\begin{verbatim}
  vector<int> count (4, 0);
\end{verbatim}
%
Þessi setning býr til vektor með fjórum stökum og frumstillir öll stökin með núlli.

\section{Aðgangur að stökum}
\index{stak}
\index{vektor!stak}

Virkjann {\tt []} er hægt að nota til að lesa og skrifa stök vektors á sambærilegan hátt og gert er með strengi.
Vísarnir byrja í núlli, þannig að {\tt count[0]} vísar til ``núllta'' staks vektorsins {\tt count} og
{\tt count[1]} vísar til ``fyrsta'' staksins.
Þú getur notað virkjann {\tt []} hvar sem er í segð:

\begin{verbatim}
  count[0] = 7;
  count[1] = count[0] * 2;
  count[2]++;
  count[3] -= 60;
\end{verbatim}
%
Allar þessar setningar eru löglegar og áhrif þessa kóða á vektorinn er: 

\vspace{0.1in}
\centerline{\epsfig{figure=vector2.eps}}
\vspace{0.1in}

Ekkert stak er með vísinn 4 þar sem stök þessa vektors eru númeruð frá 0 to 3.
Það er algeng villa að reyna að vísa ``út fyrir'' vektor og það veldur keyrsluvillu.
Í því tilfelli skrifar forritið út villuskilaboð eins og ``Illegal vector index'', og hættir síðan keyrslu.

\index{keyrsluvilla}
\index{vísir}
\index{segð}

Þú getur notað hvaða segð sem er sem vísi svo framarlega sem hún hefur tagið {\tt int}.
Ein algengasta leiðin til að vísa í stök vektors er að nota lykkjubreytu (e. loop variable).
Dæmi:

\begin{verbatim}
  int i = 0;
  while (i < 4) {
    cout << count[i] << endl;
    i++;
  }
\end{verbatim}
%
Þessi {\tt while} lykkja ``hleypur'' frá 0 í 4.
Þegar lykkjubreytan {\tt i} fær gildið 4 þá verður skilyrðið {\tt false} og lykkjan hættir.
Meginmál lykkjunnar er því eingöngu keyrt þegar {\tt i} er 0, 1, 2 og 3.

\index{lykkja}
\index{lykkjubreyta}
\index{breyta!lykkja}

Í sérhverri ítrun lykkjunnar notum við {\tt i} sem vísi inn í vektorinn og skrifum út {\tt i}-ta stakið.
Þessi tegund af ``vektorrölti'' er mjög algeng enda vinna vektorar og lykkjur vel saman.

\section{Afritun vektora}
\index{vektor!afritun}

Það er til einn smiður í viðbót fyrir {\tt vector} sem kallaður er ``afritatökusmiður'' (e. copy constructor)
vegna þess að hann tekur einn {\tt vector} sem viðfang og býr til nýjan vektor af sömu stærð og með sömu stök.

\begin{verbatim}
  vector<int> copy (count);
\end{verbatim}
%
Þrátt fyrir að þessi málskipan sé lögleg þá er hún sjaldan notuð fyrir vektorar því til er betri leið:

\begin{verbatim}
  vector<int> copy = count;
\end{verbatim}
%
Gildisveitingarvirkinn {\tt =} virkar fyrir vektorar á þann hátt sem ætla má.

\section{{\tt for} lykkjur}

Lykkjurnar sem við höfum skrifað hingað til eiga ýmislegt sameiginlegt.
Allar byrja þær á því að frumstilla breytur.
Þær innihalda jafnframt skilyrði sem er háð breytunni og meginmál lykkjunnar gerir eitthvað við breytuna, eins og að hækka hana. 

\index{lykkja!for}
\index{for}
\index{setning!for}

Þessi tegund af lykkju er svo algeng að það er til annars konar lykkjusetning, kölluð {\tt for}-setning, sem tjáir þetta á samþjappaðri hátt.
Málskipanin lítur svona út:

\begin{verbatim}
  for (INITIALIZER; CONDITION; INCREMENTOR) {
    BODY
  }
\end{verbatim}
%
Ofangreind setning er nákvæmlega jafngild þessu:

\begin{verbatim}
  INITIALIZER;
  while (CONDITION) {
    BODY
    INCREMENTOR
  }
\end{verbatim}
%
{\tt for}-setningin er aftur á móti gagnyrtari og læsilegri því í henni eru allar lykkjusetningarnar á einum stað.
Dæmi:

\begin{verbatim}
  int i;
  for (i = 0; i < 4; i++) {
    cout << count[i] << endl;
  }
\end{verbatim}
%
er jafngilt:

\begin{verbatim}
  int i = 0;
  while (i < 4) {
    cout << count[i] << endl;
    i++;
  }
\end{verbatim}

\section{Stærð vektors}
\index{stærð!vektor}
\index{vektor!stærð}

Það eru margvísleg föll sem hægt er að keyra fyrir {\tt vector}.
Eitt af þeim er þó sérstaklega gagnlegt: {\tt size()}.
Þetta fall skilar stærð vektors, þ.e. fjölda staka.

Það er góð regla að nota þetta gildi sem efri mörk (e. upper bound) lykkju frekar heldur en einhvern fasta.
Þannig þarf ekki að breyta lykkjunni ef stærð vektorsins breytist.

\begin{verbatim}
  int i;
  for (i = 0; i < count.size(); i++) {
    cout << count[i] << endl;
  }
\end{verbatim}
%
Í síðasta skiptið sem meginmál lykkjunnar er keyrt er gildið á {\tt i} jafnt {\tt count.size() - 1}, sem er vísirinn á síðasta stakinu.
Þegar {\tt i} er jafnt {\tt count.size()} mun skilyrðið verða {\tt false} og meginmál lykkjunnar verður því ekki keyrt lengur.
Það er gott því annars myndi keyrsluvilla koma upp!

Taktu eftir því að kallað er á {\tt size()} fallið í hverri ítrun lykkjunnar.
Það að kalla á fall aftur og aftur hefur áhrif á keyrslutímann þannig að í raun væri betra að geyma stærð vektorsins í einhverri breytu með því að kalla á 
{\tt size()} áður en lykkjan hefst og nota síðan breytuna til að tékka á síðasta gildinu.
Þú ættir að æfa þig með því að prófa að gera þessa breytingu.

\section{Vektorföll}
\index{föll!vektor}
\index{vektor!föll}

Einn besti eiginleiki vektors er geta hans til að stækka eða minnka ef þurfa þykir.
Eftir að vektor hefur verið búinn til þá er hægt að stækka hann eða minnka hvar sem er í forritinu.
Gerum t.d. ráð fyrir að við lesum inn tölur frá notanda inn í vektor þangað til notandi slær inn {\tt -1} en þá skrifum við tölurnar út.
Í svona tilviki vitum við ekki stærðina á vektornum fyrirfram.
Við þurfum því að geta bætt nýjum stökum við enda vektorsins um leið og notandinn slær inn ný gildi.
Við getum notað fallið {\tt push\_back()} í þessum tilgangi:

\begin{verbatim}
  #include<iostream>
  #include<vector>
  using namespace std;
  int main()
  {
    vector<int> values;
    int c,i,len;
    cin>>c;
    
    while(c != -1) {
      values.push_back(c);
      cin >> c;
    }
    len=values.size();
    for(i = 0; i < len; i++) {
      cout << values[i] << endl;
    }
  }

\end{verbatim}

\section{Slembitölur}
\label{random}
\label{pseudorandom}
\index{slembitala}
%\index{deterministic}
\index{löggengur}
%\index{nondeterministic}
\index{brigðgengur}

Flestar tölvur gera það sama í hvert sinn sem þær eru keyrðar með sama forritinu og eru kallaðar {\bf löggengar} (e. deterministic). 
Löggengni er yfirleitt kostur því við viljum jú að sömu útreikningar skili alltaf af sér sömu niðurstöðu.
Í sumum tilvikum gætum við aftur á móti viljað að tölvur væru óútreiknanlegar.  Tölvuleikir eru gott dæmi um þetta.

Það að gera forrit algerlega {\bf brigðgengt} (e. nondeterministic) er ekki auðvelt en það eru til leiðir til að láta það líta svo út.
Ein leið er að búa til hermislembitölur (e. pseudo random numbers) og nota þær til að stýra útkomunni úr forriti.
Hermislembitölur eru ekki algerlega handahófskenndar í stærðfræðilegum skilningi en þær duga fyrir það sem við ætlum að gera.

Í hausaskránni {\tt cstdlib} (sem inniheldur ýmis konar ``standard library'' föll) er fallið {\tt random} skilgreint en það býr til hermislembitölur.

Skilagildið úr {\tt random} er heiltala á milli 0 og {\tt RAND\_MAX}
en {\tt RAND\_MAX} er stór tala (um það bil 2 milljarðar á minni tölvu) sem einnig er skilgreind í hausaskránni.
Í sérhvert sinn sem þú kallar á {\tt random} færðu nýja slembitölu.
Til að sjá dæmi um þetta skaltu keyra eftirfarandi lykkju:

\begin{verbatim}
#include <iostream>
#include <cstdlib>
using namespace std;

int main ()
{
  for (int i = 0; i < 4; i++) {
    int x = random ();
    cout << x << endl;
  }
  return 0;
}

  
\end{verbatim}
%
Ég fæ eftirfarandi úttak á minni vél: 

\begin{verbatim}
1804289383
846930886
1681692777
1714636915
\end{verbatim}
%
Þú færð væntanlega eitthvað annað, en þó svipað, á þinni tölvu.

Auðvitað viljum við ekki alltaf vinna með svona stórar heiltölur.
Það er algengara að búa til heiltölur á milli 0 og einhvers efri marks.
Einföld leið til að gera það er að nota ``modulus'' virkjann:
Dæmi:

\begin{verbatim}
  int x = random ();
  int y = x % upperBound;
\end{verbatim}
%
Þar sem {\tt y} er afgangurinn sem fæst með þvi að deila {\tt x} með {\tt upperBound} þá liggja gildin fyrir {\tt y} á bilinu 0 og {\tt upperBound - 1} (að báðum meðtöldum).
Hafðu í huga að {\tt y} er aldrei jafnt og {\tt upperBound}.

Það er einnig oft gagnlegt að búa til slembikommutölur.
Algeng leið til að gera það er að deila með {\tt RAND\_MAX}.
Dæmi:

\begin{verbatim}
  int x = random ();
  double y = double(x) / RAND_MAX;
\end{verbatim}
%
Þessi kóði gefur {\tt y} gildi slembikommutölu á milli 0,0 and 1,0 (að báðum meðtöldum).
Þú ættir núna að íhuga hvernig hægt er að búa til slembikommutölu á ákveðnu bili, t.d. á milli 100,0 og 200,0.

\section{Tölfræði}
\index{statistics}
\index{dreifing}
\index{meðaltal}

Tölurnar sem {\tt random} fallið býr til eiga að dreifast jafnt.
Það þýðir að sérhvert gildi á hinu valda bili ætti að vera jafn líklegt.
Fjöldinn af sérhverju gildi ætti að vera nokkurn veginn sá sami að því gefnu að við látum {\tt random} búa til nógu mörg gildi fyrir okkur.

Hér á eftir munum við skrifa forrit sem mynda röð af slembitölum og athugar hvort þessi staðhæfing sé rétt.

\section{Vektor af slembitölum}

Fyrsta skrefið er að búa til mikinn fjölda af slembitölum og geyma þær í vektor.
Með ``mikinn fjölda'' á ég auðvitað við 20!  
Það er góð regla að byrja með lítinn fjölda (sem auðveldar kembun) og fjölga tölum síðar.

Eftirfarandi fall tekur eitt viðfang, stærð vektors.
Það úthlutar minni fyrir nýjan vektor af heiltölum og fyllir hann með slembitölum á bilinu 0 og {\tt upperBound-1}.

\begin{verbatim}
vector<int> randomVector (int n, int upperBound) {
  vector<int> vec (n);
  for (int i = 0; i<vec.size(); i++) {
    vec[i] = random () % upperBound;
  }
  return vec;
}
\end{verbatim}
%
Skilagildið er {\tt vector<int>}, þ.e. vektor af heiltölum.
Til að prófa þetta fall er þægilegt að hafa fall sem skrifar út innihalds vektors:

\begin{verbatim}
void printVector (const vector<int>& vec) {
  for (int i = 0; i<vec.size(); i++) {
    cout << vec[i] << " ";
  }
}
\end{verbatim}
%
Taktu eftir því að það er löglegt að senda {\tt vector} með tilvísun.
Það er einmitt mjög algengt því þá þarf ekki að afrita öll stök vektorsins (eins og gera þyrfti ef vektorinn væri sendur sem gildi).
Við skilgreinum leppinn sem {\tt const} þar sem {\tt printVector} breytir ekki viðfangi sínu. 

Eftirfarandi kóði býr til vektor og skrifar innihald hans út: 

\begin{verbatim}
  int numValues = 20;
  int upperBound = 10;
  vector<int> vector = randomVector (numValues, upperBound);
  printVector (vector);
\end{verbatim}
%
Á minni vél er úttakið 

\begin{verbatim}
3 6 7 5 3 5 6 2 9 1 2 7 0 9 3 6 0 6 2 6 
\end{verbatim}
%
sem virðist vera nokkuð handahófskennt.  Þitt úttak er væntanlega öðruvísi.

Ef þessar tölur eru raunverulegar slembitölur þá megum við gera ráð fyrir því að sérhver tala komi jafn oft fyrir, þ.e. tvisvar sinnum.
Það vill reyndar svo til að talan 6 kemur fyrir fimm sinnum og tölurnar 4 og 8 koma ekki fyrir.

En þýðir þetta þá að gildin fylgi ekki jafnri dreifingu?
Það er erfitt að segja til um því þegar gildin eru svona fá þá eru afskaplega litlar líkur á því að fá nákvæmlega það sem við eigum von á.
Þegar gildum fjölgar þá ætti útkoman, aftur á móti, að vera fyrirsjáanlegri.

Til að prófa þessa tilgátu munum við skrifa forrit sem telur hversu oft sérhvert gildi kemur fyrir og síðan athuga hvað gerist
þegar við hækkum {\tt numValues}.

\section{Talning}
\label{counting}
\index{ferðast um!talning}
\index{lykkja!talning}
\index{teljari}

Góð leið til að leysa vandamál eins og þetta er að skrifa einföld föll sem er auðvelt að skrifa og gera má ráð fyrir að verði gagnleg.
Þá er hægt að skrifa heildarlausnina með því að nýta sér þessi einföldu föll.
Þessi forritunaraðferð er stundum kölluð {\bf neðansækin hönnun} (e. bottom-up deisgn).
Auðvitað er það svo að það er ekki auðvelt að vita fyrirfram hvaða föll koma til með að verða gagnleg
en eftir því sem reynslan eykst því auðveldara verður fyrir þig að átta þig á því.

\index{neðansækin hönnun}
\index{forritunarþróun!neðansækin}

Að auki er ekki alltaf augljóst hvers konar föll er auðvelt að skrifa en 
það er góð leið að leita að hlutvandamálum sem passa við eitthvað mynstur sem þú hefur séð áður.

\index{mynstur!teljari}

Í kafla~\ref{loopcount} skoðuðum við lykkju sem ``ferðaðist um'' í streng og taldi hversu oft tiltekinn stafur kom fyrir í strengnum.
Þú getur litið á það forrit sem dæmi um mynstur sem kallast ``ferðast um og telja''.
Einstakir hlutar þessa mynsturs eru:

\begin{itemize}

\item Mengi eða gámur (e. container) sem hægt er að ferðast um í, t.d. strengur eða vektor. 

\item Próf (e. test) sem hægt er að beita á sérhvert stak í gámnum.

\item Teljari sem heldur utan um hversu mörg stök standast prófið. 

\end{itemize}

Í þessu tilviki hef ég fall í huga sem ég kalla {\tt howMany} sem telur fjölda staka í vektor sem eru jöfn tilteknu gildi. 
Viðföngin í fallið eru vektor og heiltalan sem leitað er að.
Skilagildið er gildi sem segir til um hversu oft heiltalan fannst í vektornum.

\begin{verbatim}
int howMany (const vector<int>& vec, int value) {
  int count = 0;
  for (int i=0; i< vec.size(); i++) {
    if (vec[i] == value) count++;
  }
  return count;
}
\end{verbatim}


\section{Athugun á öðrum gildum}

{\tt howMany} telur aðeins fjölda tilvika á af tilteknu gildi í vektornum en við höfum áhuga á hversu oft sérhvert gildi kemur fyrir.
Það getum við leyst með því að nota lykkju:

\begin{verbatim}
  int numValues = 20;
  int upperBound = 10;
  vector<int> vector = randomVector (numValues, upperBound);

  cout << "value\thowMany" << endl;

  for (int i = 0; i<upperBound; i++) {
    cout << i << '\t' << howMany (vector, i) << endl;
  }
\end{verbatim}
%
Taktu eftir því að það er löglegt að skilgreina breytu innan í {\tt for}-setningu.
Þessi málskipan er stundum hentug en athugaðu að breyta sem skilgreind er innan í lykkju er eingöngu lifandi í lykkjunni.
Þú færð villu frá þýðandanum ef þú reynir að vísa í {\tt i} utan lykkjunnar.

Þessi kóði sendir lykkjubreytuna sem viðfang í {\tt howMany} í þeim tilgangi að tékka á sérhverju gildi á milli 0 og 9 í réttri röð.
Niðurstaðan er:

\begin{verbatim}
value   howMany
0       2
1       1
2       3
3       3
4       0
5       2
6       5
7       2
8       0
9       2
\end{verbatim}
%
Það er erfitt að segja hvort tölurnar birtast í raun jafn oft.
Ef við hækkum hins vegar {\tt numValues} í 100.000 þá fæst:

\begin{verbatim}
value   howMany
0       10130
1       10072
2       9990
3       9842
4       10174
5       9930
6       10059
7       9954
8       9891
9       9958
\end{verbatim}
%
Í sérhverju tilviki er fjöldi tilvika innan við 1\% af væntu gildi (10.000) þannig að við getum ályktað sem svo að tölurnar dreifist líklega jafnt.

\section {Súlurit}
\index{histogram}

Það er oft gagnlegt að geyma gögn, eins og úr töflunni hér að ofan, og sækja þau síðar í stað þess að prenta þau bara út.
Til þess þurfum við einhverja leið til að geyma 10 heiltölur.
Við gætum búið til 10 heiltölubreytur með nöfn eins og {\tt howManyOnes}, {\tt howManyTwos}, o.s.frv.
Þá þyrftum við hins vegar að slá mikið inn og það yrði erfitt fyrir okkur að breyta forritinu ef bilið (nú 0-9) breytist.

Miklu betri lausn er að nota vektor af stærðinni 10.
Þannig getum við búið til minnið fyrir allar 10 heiltölurnar í einu og getum nálgast þær með því að nota vísa (e. indices) í stað 10 mismunandi breytunafna.
Hér er dæmi um þetta:

\begin{verbatim}
  int numValues = 100000;
  int upperBound = 10;
  vector<int> vector = randomVector (numValues, upperBound);
  vector<int> histogram (upperBound);

  for (int i = 0; i<upperBound; i++) {
    int count = howMany (vector, i);
    histogram[i] = count;
  }
\end{verbatim}
%
Ég kallaði vektorinn {\bf histogram} (súlurit) því það er tölfræðilegt hugtak fyrir vektor af tölum sem heldur utan um fjölda tilvika af gildum á ákveðnu bili.

\index{súlurit}

Taktu eftir að hér nota ég lykkjubreytuna á tvo mismunandi vegu.
Í fyrsta lagi sem viðfang í {\tt howMany} fallið til að tilgreina hvaða gildi ég hef áhuga á í hverri ítrun.
Í öðru lagi er breytan notuð sem vísir inn í súluritið, þ.e. til að tilgreina hvar (í súluritinu) geyma eigi fjöldann.

\section{Skilvirkari lausn}

Þrátt fyrir að þessi kóði virki þá er hann ekki eins skilvirkur eins og hann gæti verið.
Ferðast er um í öllum vektornum í hvert sinn sem kallað er á {\tt howMany}.
Í þessu dæmi er því ferðast um í vektornum 10 sinnum!

Það væri miklu betra að renna aðeins einu sinni (e. make a single pass) í gegnum vektorinn.
Fyrir sérhvert gildi í vektornum gætum við fundið samsvarandi teljari og hækkað hann.
Við getum, m.ö.o., notað gildið í vektornum sem vísi inn í súluritið.
Svona myndi þetta líta út:

\begin{verbatim}
  vector<int> histogram (upperBound, 0);

  for (int i = 0; i<numValues; i++) {
    int index = vector[i];
    histogram[index]++;
  }
\end{verbatim}
%
Fyrsta línan frumstillir stök súluritsins með núlli.
Það þýðir að þegar við notum virkjann {\tt ++} innan í lykkjunni þá vitum við að byrjað er í 0.
Það er einmitt algeng villa að gleyma að frumstilla með núlli.

Þú ættir núna að hjúpa þennan kóða í fall sem kallað er {\tt histogram}.
Fallið tekur vektor og bil (í þessu tilviki 0-9) sem viðföng og skilar súluriti yfir gildin í vektornum.

\section{Slembifræ}
\index{fræ}
\index{slembitölur}

Ef þú hefur keyrt kóðann í þessum kafla nokkrum sinnum þá hefur þú kannski tekið eftir því að þú hefur fengið sömu ``slembitölurnar'' aftur og aftur.
Það er ekki mjög handahófskennt!

Einn af eiginleikum hermislembitalna er að ef byrjað er á sama stað þá myndast alltaf sama talnaröðin.
Byrjunarpunkturinn er kallað ``fræ'' (e. seed) og C++ notar allaf sama fræið í hvert sinn sem þú keyrir forritið.

Það getur oft verið gagnlegt að sjá sömu niðurstöðu (talnaröðina) aftur og aftur á meðan þú ert að kemba forrit.
Þannig getur þú auðveldlega séð hvort breyting á forritinu þínu hefur í raun áhrif á úttak þess.

Ef þú vilt hins vegar nota mismunandi fræ fyrir myndun slembitalna þá getur þú notað {\tt srand} fallið.
Það tekur eitt viðfang sem er heiltala á milli 0 og {\tt RAND\_MAX}.

Í mörgum forritum, t.d. leikjum, er þörf á því að fá mismunandi slembitölurunur í hvert sinn sem forritið er keyrt.
Algeng leið til að gera það er að nota fall eins og {\tt gettimeofday} sem býr til eitthvað sem er tiltölulega ófyrrisjáanlegt og ekki auðvelt að endurtaka,
eins og fjöldi millisekúndna síðan kallað var síðast, og nota það gildi sem fræ.

\section{Orðalisti}

\begin{description}

\item[vektor (e. vector):]  Safn af gildum sem öll hafa sama tag. Sérhvert gildi er einkennt með vísi (e. index).

\item[stak (e. element):]  Eitt af gildunum í vektor. Virkinn {\tt []} velur stök í vektor.

\item[vísir (e. index):]  Heiltölubreyta eða gildi sem notað er til að einkenna stak í vektor. 

\item[smiður (e. constructor):]  Sérstakt fall sem býr til nýtt tilvik og frumstillir tilvikabreytur þess.

\item[löggengt forrit (e. deterministic program):]  Forrit sem gerir það sama í hvert skipti sem það er keyrt.

\item[hermislembitöluruna (e. pseudo random sequence):]  Runa af tölum sem virðist vera handahófskennd en er í raun niðurstaðan af löggengum útreikningum.

\item[fræ (e. seed):]  Gildi sem notað er til að frumstilla slembitölurunu.
Ef sama fræ er notað þá ætti að fást sama slembitöluruna.

\item[neðansækin hönnun (e. bottom-up design):]  Forritunaraðferð sem byggir á því að skrifa fyrst lítil, gagnleg föll, sem síðan eru nýtt í stærri heildarlausn.

\item[súlurit (e. histogram):]  Vektor af heiltölugildum þar sem hver tala stendur fyrir fjölda gilda á ákveðnu bili.

\index{vektor}
\index{stak}
\index{vísir}
\index{smiður}
\index{löggeng}
\index{slembitöluruna}
\index{fræ}
\index{súlurit}

\end{description}

% LaTeX source for textbook ``How to think like a computer scientist''
% Copyright (C) 1999  Allen B. Downey

% This LaTeX source is free software; you can redistribute it and/or
% modify it under the terms of the GNU General Public License as
% published by the Free Software Foundation (version 2).

% This LaTeX source is distributed in the hope that it will be useful,
% but WITHOUT ANY WARRANTY; without even the implied warranty of
% MERCHANTABILITY or FITNESS FOR A PARTICULAR PURPOSE.  See the GNU
% General Public License for more details.

% Compiling this LaTeX source has the effect of generating
% a device-independent representation of a textbook, which
% can be converted to other formats and printed.  All intermediate
% representations (including DVI and Postscript), and all printed
% copies of the textbook are also covered by the GNU General
% Public License.

% This distribution includes a file named COPYING that contains the text
% of the GNU General Public License.  If it is missing, you can obtain
% it from www.gnu.org or by writing to the Free Software Foundation,
% Inc., 59 Temple Place - Suite 330, Boston, MA 02111-1307, USA.

% This is an Icelandic translation/adaptation of the orginal book by Allen B. Downey

\chapter{Meðlimaföll}

\section{Hlutir og föll}
\index{meðlimafall}
\index{fall!meðlimur}

C++ er hlutbundið forritunarmál sem merkir að það hefur eiginleika sem styðja við hlutbundna forritun.

Það er ekki auðvelt að skilgreina hlutbundna forritun en við höfum þó þegar séð suma eiginleika hennar:

\begin{enumerate}

\item Forrit samanstanda af safni af strúktúrskilgreiningum og fallaskilgreiningum 
þar sem flest föllin vinna með tiltekna tegundur af strúktúrum (eða hlutum).

\item Sérhver strúktúrskilgreining stendur fyrir einhvern hlut eða hugtak í raunveruleikanum og föllin sem framkvæma aðgerðir á viðkomandi strúktúr
líkja eftir því hvernig raunverulegir hlutir eiga samskipti.

\end{enumerate}

{\tt Time} strúktúrinn, t.d., sem við skilgreindum í kafla~\ref{time}
samsvarar augljóslega hvernig fólk skráir tíma innan dagsins og 
aðgerðirnar sem við skilgreindum samsvara því hvað fólk gerir við hugtakið tími.

Á sambærilegan hátt má segja að {\tt Point} og
{\tt Rectangle} hlutirnir samsvari stærðfræðilegu hugtökunum hnit og rétthyrningur.

Hingað til höfum við ekki nýtt okkur þá eiginleika sem C++ býður upp á til að styðja við hlutbundna forritun.
Strangt til tekið eru þessir eiginleikar reyndar ekki nauðsynlegir.
Að mestu leyti gera þeir forriturum kleift að nota aðra málskipan til að framkvæma eitthvað sem við höfum þegar gert áður
en í mörgum tilvikum er þessi nýja málskipan samanþjappaðri (e. more concise) og sýnir á nákvæmari hátt högun forritsins. 
 
Í {\tt Time} forritinu er t.d. ekkert augljóst samband á milli strúktúrskilgreiningarinnar og fallanna sem á eftir koma.
Við nánari athugun sést hins vegar er ljóst að sérhvert fallanna tekur a.m.k. einn {\tt Time} hlut sem viðfang og það leiðir okkur að hugtakinu  {\bf meðlimafall} (e. member function).
Meðlimaföll eru ólík öðrum föllum sem við höfum skrifað á tvennan hátt:

\begin{enumerate}

\item Þegar við köllum á meðlimafall þá tengjum við kallið við tiltekinn hlut (e. object).
Stundum er sagt að kall á meðlimafall hafi í för með sér aðgerð á hlut (e. operation on an object) 
eða verið sé að senda boð til hlutar (e. sending a message to an object).

\item {\em Yfirlýsing} (e. declaration) meðlimafallsins á sér stað innan {\tt struct}
skilgreiningar í þeim tilgangi að gera sambandið á milli strúktúrsins og fallsins skýrt (e. explicit). 

\end{enumerate}

Í því sem á eftir kemur munum við taka föllin úr kafla~\ref{time} og umbreyta þeim í meðlimaföll.
Athugaðu að þessi umbreyting er algerlega vélræn, m.ö.o. hægt er að framkvæma hana með því að fylgja ákveðinni forskrift.

%\index{nonmember function}
\index{fall!nonmember}

Eins og ég nefndi að ofan þá getur allt það sem hægt er að gera með meðlimafalli líka verið gert með falli sem stendur eitt og sér (e. nonmember function / free-standing function).
Oft eru hins vegar kostir/ókostir sem fylgja notkun annars í stað hins.
Ef þú getur á auðveldan hátt umbreytt öðru forminu yfir í hitt þá munt þú geta valið betri leiðina fyrir það sem þú vinnur að sérhverju sinni.

\section{{\tt print}}

Í kafla~\ref{time} skilgreindum við strúktúrinn {\tt Time}
og skrifuðum fallið {\tt printTime}

\begin{verbatim}
struct Time {
  int hour, minute;
  double second;
};

void printTime (const Time& time) {
  cout << time.hour << ":" << time.minute << ":" << time.second << endl;
}
\end{verbatim}
%
Til að kalla á þetta fall þurftum við að senda {\tt Time} hlut sem viðfang. 

\begin{verbatim}
  Time currentTime = { 9, 14, 30.0 };
  printTime (currentTime);
\end{verbatim}
%
Fyrsta skrefið í áttina að gera {\tt printTime} að meðlimafalli er að breyta nafninu á fallinu úr {\tt printTime} í {\tt Time::print}.
Virkinn {\tt ::} kemur á milli nafns strúktúrsins og nafns fallsins.
Til saman gefa þessi nöfn til kynna að um sé að ræða fallið {\tt print} sem tilheyrir {\tt Time} strúktúr.

Næsta skref felst í því að fjarlægja leppinn.
Í stað þess að senda hlut inn sem viðfang þá vekjum við fallið með tilteknum hlut (e. invoke the function on an object).

Niðurstaðan er sú að inni í fallinu er ekki lengur leppur með nafnið {\tt time}.
Í staðinn er um að ræða {\bf núverandi hlut} (e. current object) sem er hluturinn sem fallið er keyrt á.
Hægt er að vísa í núverandi hlut með því að nota C++ lykilorðið {\tt this}.

\index{núverandi hlutur}
\index{hlutur!núverandi}
\index{bendir}
\index{this}

Það flækir aðeins málið að {\tt this} er í raun {\bf bendir} á strúktúr frekar en strúktúrinn sjálfur.
Bendir er svipaður og tilvísun en ég vil þó ekki fjalla nákvæmlega um bendanotkun enn sem komið er.
Eina bendaaðgerðin sem við þurfum á að halda núna er {\tt *} virkinn (e. dereference operator) sem breytir bendi á strúktúr yfir í strúktur.
Í eftirfarandi falli notum við þennan virkja til að gefa staðværu (e. local) breytunnni {\tt time} gildið á {\tt this}:

\begin{verbatim}
void Time::print () {
  Time time = *this;
  cout << time.hour << ":" << time.minute << ":" << time.second << endl;
}
\end{verbatim}
%
Fyrstu tvær línurnar í þessu falli breyttust þó nokkuð þegar við umbreyttum fallinu yfir í meðlimafall 
en taktu eftir því að úttakssetningin sjálf breyttist ekki neitt.

Þessa nýju útgáfu af {\tt print} keyrum við síðan á {\tt Time} hlut (taktu eftir punktatáknuninni):

\begin{verbatim}
  Time currentTime = { 9, 14, 30.0 };
  currentTime.print ();
\end{verbatim}
%
Síðasta skrefið í umbreytingarferlinu er að við þurfum að lýsa yfir fallinu inni í strúktúrskilgreiningunni:

\begin{verbatim}
struct Time {
  int hour, minute;
  double second;

  void print ();
};
\end{verbatim}
%
{\bf Yfirlýsing} falls lítur eins út og fyrsta línan í skilgreiningu fallsins nema að því leyti að yfirlýsingin er með semíkommu í endann og ekki þarf að tiltaka nafn strúktúrsins sem fallið tilheyrir (í þessu tilfelli Time).
Yfirlýsingin lýsir {\bf skilum} (e. interface) fallsins, þ.e. fjölda og tagi leppa og tagi skilagildisins.

Þegar þú lýsir yfir falli þá ``lofar'' þú þýðandanum að þú munir, á einhverjum öðrum stað í forritinu, setja skilgreiningu fallsins fram.
Skilgreiningin sjálf er {\bf útfærsla) (e. implementation) fallsins því hún inniheldur nákvæmar upplýsingar um hvernig fallið virkar.
Þýðandinn mun kvarta ef þú sleppir skilgreiningunni eða setur fram skilgreiningu sem inniheldur önnur skil en þú lofaðir.

\section {Dulinn aðgangur að breytum (e. Implicit variable access)}

Nýja útgáfan okkar af {\tt Time::print} er reyndar flóknari en hún þarf að vera.
Við þurfum í raun ekki að búa til staðværa breytu í þeim tilgangi að vísa í tilvikabreytur núverandi hlutar.

Ef fallið vísar í {\tt hour}, {\tt minute} eða {\tt second}, án þess að nota punktatáknun, þá veit C++ þýðandinn að um er að ræða tilvísun í núverandi hlut.
Við gætum því skrifað fallið á þennan hátt: 

\begin{verbatim}
void Time::print ()
{
  cout << hour << ":" << minute << ":" << second << endl;
}
\end{verbatim}
%
Þessi tegund af aðgangi að breytum er kallaður ``dulinn'' (e. implicit) vegna þess að nafn hlutarins er ekki ljóst (e. explicit). 
Þetta er ein ástæða þess að meðlimaföll eru oft samþjappaðri en föll sem eru ekki meðlimaföll.

\section {Annað dæmi}

Nú skulum við umbreyta {\tt increment} í meðlimafall.
Við munum breyta einu viðfanginu í dulda viðfangið {\tt this}.
Síðan getum við gert allan aðgang að einstökum breytum dulinn. 

\begin{verbatim}
void Time::increment (double secs) {
  second += secs;

  while (second >= 60.0) {
    second -= 60.0;
    minute += 1;
  }
  while (minute >= 60) {
    minute -= 60.0;
    hour += 1;
  }
}
\end{verbatim}
%
Athugaðu að þessi útfærsla fallsins er ekki sú skilvirkasta.
Þú ættir núna að skrifa skilvirkari útfærslu ef þú gerðir það ekki þegar þú last kafla~\ref{time}

Til að lýsa fallinu yfir þurfum við aðeins að afrita fyrstu línuna úr skilgreiningu þess (og sleppa tilvísuninni í Time strúktúrinn):

\begin{verbatim}
struct Time {
  int hour, minute;
  double second;

  void print ();
  void increment (double secs);
};
\end{verbatim}
%
Nú getum við kallað á fallið með því að vekja það með tilteknum {\tt Time} hlut:

\begin{verbatim}
  Time currentTime = { 9, 14, 30.0 };
  currentTime.increment (500.0);
  currentTime.print ();
\end{verbatim}
%
Útttakið úr þessu forriti er {\tt 9:22:50}.

\section{Enn eitt dæmi}

Upphaflega útgáfan af {\tt convertToSeconds} leit svona út:

\begin{verbatim}
double convertToSeconds (const Time& time) {
  int minutes = time.hour * 60 + time.minute;
  double seconds = minutes * 60 + time.second;
  return seconds;
}
\end{verbatim}
%
Það er einfalt að umbreyta þessu í meðlimafall: 

\begin{verbatim}
double Time::convertToSeconds () const {
  int minutes = hour * 60 + minute;
  double seconds = minutes * 60 + second;
  return seconds;
}
\end{verbatim}
%
Hér er athyglisvert að dulda viðfanginu er lýst yfir sem 
{\tt const} vegna þess að við breytum því ekki inni í fallinu. 
Það er reyndar ekki augljóst hvar setja á upplýsingar um viðfang sem er ekki til!
Eins og sést í þessu dæmi felst lausnin í því að setja þetta á eftir viðfangalistanum (sem er reyndar tómur í þessu tilviki).

Fallið {\tt print} í fyrri kafla hefði líka á að lýsa dulda viðfanginu yfir sem {\tt const}.

\section {Flóknara dæmi}

Þrátt fyrri að ferlið við að umbreyta föllum í meðlimaföll sé vélrænt þá geta aðstæður verið sérkennilegar.
Fallið {\tt after}, t.d., vinnur á tveimur {\tt Time} hlutum (ekki bara einum) og við getum ekki gert þá báða að duldum viðföngum.
Í staðinn köllum við á fallið í gegnum annan þeirra en sendum hinn sem viðfang.

Inni í fallinu getum við notað dulinn aðgang að breytum hlutarins sem notaður var til að vekja fallið en fáum síðan aðgang að breytum hins hlutarins með því að nota punktatáknun.

\begin{verbatim}
bool Time::after (const Time& time2) const {
  if (hour > time2.hour) return true;
  if (hour < time2.hour) return false;

  if (minute > time2.minute) return true;
  if (minute < time2.minute) return false;

  if (second > time2.second) return true;
  return false;
}
\end{verbatim}
%
Svona köllum við þá á þetta fall:

\begin{verbatim}
  if (doneTime.after (currentTime)) {
    cout << "The bread will be done after it starts." << endl;
  }
\end{verbatim}
%
Það er nánast hægt að lesa þennan kóða eins og ensku:
``If the done-time is after the current-time, then...''

\section{Smiðir}

Í kafla~\ref{time} skrifuðum við fallið {\tt makeTime}:

\begin{verbatim}
Time makeTime (double secs) {
  Time time;
  time.hour = int (secs / 3600.0);
  secs -= time.hour * 3600.0;
  time.minute = int (secs / 60.0);
  secs -= time.minute * 60.0;
  time.second = secs;
  return time;
}
\end{verbatim}
%
Fyrir sérhvert nýtt tag (e. type) þurfum við að geta búið til nýja hluti af því tagi.
Það vill einmitt reyndar svo til að föll eins og {\tt makeTime} eru svo algeng að sérstök málskipan er notuð fyrir þau.
Þessi föll eru kölluð {\bf smiðir} (e. constructors} og málskipanin er þessi: 

\begin{verbatim}
Time::Time (double secs) {
  hour = int (secs / 3600.0);
  secs -= hour * 3600.0;
  minute = int (secs / 60.0);
  secs -= minute * 60.0;
  second = secs;
}
\end{verbatim}
%
Hér eru tvö atriði sem vert er að minnast á.
Í fyrsta lagi ber smiðurinn sama nafn og strúktúrinn sjálfur og hann hefur ekkert skilagildi.
Viðföngin hafa hins vegar ekkert breyst.

Í öðru lagi þurfum við ekki að búa til nýjan Time hlut og við þurfum ekki að skila neinu.
Bæði þessi skref eru gerð fyrir okkur af þýðandanum.
Við getum vísað í nýja hlutinn -- þann sem við erum að smíða -- með því að nota lykilorðið
{\tt this}, eða á dulinn hátt eins og sést að hér. 
Þegar við notum nöfnin {\tt hour}, {\tt minute}
og {\tt second} þá veit þýððandinn að við erum að vísa í meðlimabreytur (tilvikabreytur) nýja hlutarins. 

Til að vekja (kalla á) smiðinn þá notum við sérstaka málskipan sem er einhvers konar millivegur á milli breytuyfirlýsingar og fallakalls:

\begin{verbatim}
  Time time (seconds);
\end{verbatim}
%
Þessi setning lýsir yfir að breytan {\tt time} hafi tagið {\tt Time} og vekur smiðinn, sem við vorum að skrifa, 
með því að senda {\tt seconds} sem viðfang.
Kerfið úthlutar minni fyrir nýja hlutinn og smiðurinn upphafsstillir meðlimabreyturnar.
Niðurstaðan er gildi sem breytan {\tt time} fær.


\section {Að upphafsstilla eða smíða?}

Í fyrri kafla lýstum við yfir og upphafsstilltum {\tt Time} strúktúr með því að nota slaufusviga (e. squiggly-braces):

\begin{verbatim}
  Time currentTime = { 9, 14, 30.0 };
  Time breadTime = { 3, 35, 0.0 };
\end{verbatim}
%
Með því að nota smiði getum við nú notað aðra aðferð við yfirlýsingu og upphafsstillingu:

\begin{verbatim}
  Time time (seconds);
\end{verbatim}
%
Þessar tvær aðferðir standa fyrir mismunandi forritunarstíl og reyndar mismunandi tíma í sögu C++.
Það er kannski ástæðan fyrir því að C++ þýðandinn krefst þess að þú notar aðra aðferðina, en ekki báðar, í sama forritinu.

Ef þú skilgreinir smið fyrir strúktúr þá verður þú að nota smiðinn til að upphafsstilla öll ný tilvik af viðkomandi tagi.
Hin málskipanin, þ.e. að nota slaufusviga við upphafsstillingu, er þá ekki leyfð.

Sem betur fer er leyfilegt að fjölbinda (e. overload) smiði og sama hátt og hægt er að fjölbinda föll.
M.ö.o, það geta verið fleiri en einn smiður með sama nafn svo framarlega sem smiðirnir taki mismunandi viðföng.
Þegar við upphafsstillum nýtjan hlut mun þýðandinn reyna að finna þann smið sem tekur viðeigandi viðföng.
Fortunately, it is legal to overload constructors in the same

Það er t.d. algengt að hafa einn smið sem tekur eitt viðfang fyrir sérhverja meðlimabreytu og gefur meðlimabreytunum gildi viðfanganna:

\begin{verbatim}
Time::Time (int h, int m, double s)
{
  hour = h;  minute = m;  second = s;
}
\end{verbatim}
%
Við notum sömu skrýtnu málskipanina og áður til að vekja þennan smið nema að nú verða viðföngin að vera tvær heiltölur og ein {\tt double}:

\begin{verbatim}
  Time currentTime (9, 14, 30.0);
\end{verbatim}

\section {Eitt dæmi að lokum}

Síðasta dæmið sem við skoðum er {\tt addTime}:

\begin{verbatim}
Time addTime2 (const Time& t1, const Time& t2) {
  double seconds = convertToSeconds (t1) + convertToSeconds (t2);
  return makeTime (seconds);
}
\end{verbatim}
%
Við þurfum að gera nokkrar breytingar á þessu falli, þ.m.t.:

\begin{enumerate}

\item Breyta nafninu á fallinu úr {\tt addTime} í {\tt Time::add}.

\item Fjarlægja fyrsta viðfangið og gera það að duldu {\tt const} viðfangi.

\item Skipta út kalli í {\tt makeTime} fyrir vakningu á smið. 

\end{enumerate}
%
Hér er niðurstaðan:

\begin{verbatim}
Time Time::add (const Time& t2) const {
  double seconds = convertToSeconds () + t2.convertToSeconds ();
  Time time (seconds);
  return time;
}
\end{verbatim}
%
Þegar við köllum á {\tt convertToSeconds} í fyrsta sinn þá virðist vanta hlutinn sem vekja á fallið með!
Inni í meðlimafalli (eins og Time::add) gerir þýðandinn ráð fyrir þvi að við viljum vekja upp föll með núverandi hlut (þ.e. {\tt this}). 
Þess vegna á fyrsta kallið í {\tt convertToSeconds} við {\tt this} en í seinna kallinu er {\tt convertToSeconds} vakið upp með {\tt t2}.

Í næstu línu fallsins er smiður vakinn upp sem tekur eitt {\tt double} sem viðfang.
Í síðustu línunni er síðan nýja hlutnum (time) skilað til baka úr fallinu.

\section {Header files}

It might seem like a nuisance to declare functions inside
the structure definition and then define the functions later.
Any time you change the interface to a function, you have
to change it in two places, even if it is a small change
like declaring one of the parameters {\tt const}.

There is a reason for the hassle, though, which is that it
is now possible to separate the structure definition and the
functions into two files: the {\bf header file},
which contains the structure definition, and the implementation
file, which contains the functions.

Header files usually have the same name as the implementation
file, but with the suffix {\tt .h} instead of {\tt .cpp}.  For
the example we have been looking at, the header file is called
{\tt Time.h}, and it contains the following:

\begin{verbatim}
struct Time {
  // instance variables
  int hour, minute;
  double second;

  // constructors
  Time (int hour, int min, double secs);
  Time (double secs);

  // modifiers
  void increment (double secs);

  // functions
  void print () const;
  bool after (const Time& time2) const;
  Time add (const Time& t2) const;
  double convertToSeconds () const;
};
\end{verbatim}
%
Notice that in the structure definition I don't really have
to include the prefix {\tt Time::} at the beginning of every
function name.  The compiler knows that we are declaring functions
that are members of the {\tt Time} structure.

{\tt Time.cpp} contains the definitions of the member functions
(I have elided the function bodies to save space):

\begin{verbatim}
#include <iostream>
using namespace std;
#include "Time.h"

Time::Time (int h, int m, double s)  ...

Time::Time (double secs) ...

void Time::increment (double secs) ...

void Time::print () const ...

bool Time::after (const Time& time2) const ...

Time Time::add (const Time& t2) const ...

double Time::convertToSeconds () const ...
\end{verbatim}
%
In this case the definitions in {\tt Time.cpp} appear in the
same order as the declarations in {\tt Time.h}, although it
is not necessary.

On the other hand, it is necessary to include the header
file using an {\tt include} statement.  That way, while the
compiler is reading the function definitions, it knows enough
about the structure to check the code and catch errors.

Finally, {\tt main.cpp} contains the function {\tt main} along
with any functions we want that are not members of the {\tt Time}
structure (in this case there are none):

\begin{verbatim}
#include <iostream>
using namespace std;
#include "Time.h"

int main ()
{
  Time currentTime (9, 14, 30.0);
  currentTime.increment (500.0);
  currentTime.print ();

  Time breadTime (3, 35, 0.0);
  Time doneTime = currentTime.add (breadTime);
  doneTime.print ();

  if (doneTime.after (currentTime)) {
    cout << "The bread will be done after it starts." << endl;
  }
  return 0;
}

\end{verbatim}
%
Again, {\tt main.cpp} has to include the header file.

It may not be obvious why it is useful to break such a small
program into three pieces.  In fact, most of the advantages come
when we are working with larger programs:

\begin{description}

\item[Reuse:]  Once you have written a structure like {\tt Time},
you might find it useful in more than one program.  By separating
the definition of {\tt Time} from {\tt main.cpp}, you make is easy
to include the {\tt Time} structure in another program.

\item[Managing interactions:]  As systems become large, the number
of interactions between components grows and quickly becomes
unmanageable.  It is often useful to minimize these interactions
by separating modules like {\tt Time.cpp} from the programs that
use them.

\item[Separate compilation:]  Separate files can be compiled
separately and then linked into a single program later.  The details
of how to do this depend on your programming environment.  As
the program gets large, separate compilation can save a lot of time,
since you usually need to compile only a few files at a time.

\end{description}

For small programs like the ones in this book, there is
no great advantage to splitting up programs.  But it is good
for you to know about this feature, especially since it explains
one of the statements that appeared in the first program we
wrote:

\begin{verbatim}
#include <iostream>
using namespace std;
\end{verbatim}
%
{\tt iostream} is the header file that contains declarations
for {\tt cin} and {\tt cout} and the functions that operate on
them.  When you compile your program, you need the information
in that header file.

The implementations of those functions are stored in a library,
sometimes called the ``Standard Library'' that gets linked to
your program automatically.  The nice thing is that you don't
have to recompile the library every time you compile a program.
For the most part the library doesn't change, so there is no
reason to recompile it.

\section{Glossary}

\begin{description}

\item[member function:]  A function that operates on an object
that is passed as an implicit parameter named {\tt this}.

\item[nonmember function:]  A function that is not a member
of any structure definition.  Also called a ``free-standing''
function.

\item[invoke:] To call a function ``on'' an object, in order to
pass the object as an implicit parameter.

\item[current object:]  The object on which a member function
is invoked.  Inside the member function, we can refer to the
current object implicitly, or by using the keyword {\tt this}.

\item[this:]  A keyword that refers to the current object.
{\tt this} is a pointer, which makes it difficult to use, since
we do not cover pointers in this book.

\item[interface:] A description of how a function is used, including
the number and types of the parameters and the type of the return
value.

\item[function declaration:] A statement that declares the interface
to a function without providing the body.  Declarations of
member functions appear inside structure definitions even if the
definitions appear outside.

\item[implementation:] The body of a function, or the details of how
a function works.

\item[constructor:] A special function that initializes the instance
variables of a newly-created object.

\index{member function}
\index{nonmember function}
\index{function!member}
\index{function!nonmember}
\index{interface}
\index{implementation}
\index{invoke}
\index{constructor}

\end{description}


% LaTeX source for textbook ``How to think like a computer scientist''
% Copyright (C) 1999  Allen B. Downey

% This LaTeX source is free software; you can redistribute it and/or
% modify it under the terms of the GNU General Public License as
% published by the Free Software Foundation (version 2).

% This LaTeX source is distributed in the hope that it will be useful,
% but WITHOUT ANY WARRANTY; without even the implied warranty of
% MERCHANTABILITY or FITNESS FOR A PARTICULAR PURPOSE.  See the GNU
% General Public License for more details.

% Compiling this LaTeX source has the effect of generating
% a device-independent representation of a textbook, which
% can be converted to other formats and printed.  All intermediate
% representations (including DVI and Postscript), and all printed
% copies of the textbook are also covered by the GNU General
% Public License.

% This distribution includes a file named COPYING that contains the text
% of the GNU General Public License.  If it is missing, you can obtain
% it from www.gnu.org or by writing to the Free Software Foundation,
% Inc., 59 Temple Place - Suite 330, Boston, MA 02111-1307, USA.

% This is an Icelandic translation/adaptation of the orginal book by Allen B. Downey

\chapter{Vektorar af hlutum}

\section{Samsetning}
\index{samsetning}
\index{hreiðruð skipan}

Hingað til höfum við séð nokkur dæmi um samsetningu (e. composition), þ.e. þegar einstakir eiginleikar forritunarmálsins eru settir saman á ýmsan máta.
Eitt af fyrstu dæmunum sem við sáum var að nota fallakall sem hluta af segð (e. expression).
Annað dæmi er hreiðruð skipan setninga: hægt er að setja {\tt if} setningu inn í {\tt while} lykkju eða inn í aðra {\tt if} setningu, o.s.frv.

Nú þegar við höfum séð þessi samsetningarmynstur, og lært um vektorara og hluti, þá ætti ekki að koma á óvart að hægt er að búa til vektora af hlutum.
Reyndar vill svo til að það er líka hægt að búa til hluti sem innihalda vektora (sem meðlimabreytur) vektora sem innihalda vektora; hluti sem innihalda aðra hluti, o.s.frv.

Í næstu tveimur köflum munum við skoða nokkur dæmi um svona samsetningar með því að nota {\tt Card} hluti sem sýnidæmi. 

\section{{\tt Card} hlutir}
\index{Card}
\index{hlutur!Card}

Ef þú þekkir ekki handspil (e. playing cards) þá er nú góður tímapunktur að sækja spilastokk (e. deck) því annars er hætt við því að efni þessa kafla fari fyrir ofan garð og neðan.
Það eru 52 spil í spilastokki og sérhvert spil tilheyrir einum af fjórum litum (e. suit) og 13 gildum (e. rank).
Litirnir eru (í lækkandi röð í bridds): Spaðar (e. Spades), hjörtu (e. Hearts), tíglar (e. Diamonds) og lauf (e. Clubs).
Gildin eru ás (e. Ace), 2, 3, 4, 5, 6, 7, 8, 9, 10, gosi (e. Jack), drottning (e. Queen) og kóngur (e. King).
Það fer eftir því hvaða spil þú ert að spila hvort ásinn er hærri en kóngur eða lægri en tvistur.

\index{gildi}
\index{rank}
\index{litur}
\index{suit}

Það er nokkuð augljóst hverjar meðlimabreyturnar eiga að vera ef við viljum skilgreina nýjan hlut sem stendur fyrir tiltekið spil: {\tt rank} og {\tt suit}.
Það er hins vegar ekki jafn augljóst hvert tag meðlimabreytnanna ætti að vera.
Einn möguleiki er að nota {\tt string} sem þá myndi innihalda t.d. \verb+"Spade"+ fyrir lit og \verb+"Queen"+ fyrir gildi.
Eitt vandamálið við þá útfærslu er að það yrði ekki einfalt að bera saman tvö spil, þ.e. að finna út hvort þeirra er með hærri lit eða hærra gildi.

\index{kóta}
\index{varpa}

Annar möguleiki er sá að nota heiltölur til að {\bf kóta} (e. encode) gildin og litina. 
Það sem tölvunarfræðingar eiga við með ``að kóta'' er t.d. að skilgreina vörpun á milli talnarunu og hlutanna sem sérhvert tala raðarinnar stendur fyrir.
Dæmi:

\vspace{0.1in}
\begin{tabular}{l c l}
Spades & $\mapsto$ & 3 \\
Hearts & $\mapsto$ & 2 \\
Diamonds & $\mapsto$ & 1 \\
Clubs & $\mapsto$ & 0
\end{tabular}
\vspace{0.1in}

Táknið $\mapsto$ er stærðfræðileg tákn fyrir ``varpast í''. 
Það má augljóslega lesa úr þessari vörpun að litirnir varpast í heiltölur í ákveðinni röð.
Þannig getum við borið saman liti með því að bera saman heiltölur.
Vörpunin fyrir gildi eru nokkuð augljós: sérhvert spil með talnagildi varpast yfir í viðkomandi heiltölu en fyrir mannspil notum við eftirfarandi vörpun:

\vspace{0.1in}
\begin{tabular}{l c l}
Jack & $\mapsto$ & 11 \\
Queen & $\mapsto$ & 12 \\
King & $\mapsto$ & 13 \\
\end{tabular}
\vspace{0.1in}

Ástæðan fyrir því að ég nota stærðfræðilega táknun fyrir þessar varpanir er sú að þær eru ekki hluti af C++ forritunarmálinu.
Varpanirnar eru hluti af hönnun forritsins en þær koma ekki beint fyrir í forritskóðanum.
Strúktúrskilgreiningin fyrir {\tt Card} tagið lítur svona út:

\begin{verbatim}
struct Card
{
  int suit, rank;

  Card ();
  Card (int s, int r);
};

Card::Card () { 
  suit = 0;  rank = 0;
}

Card::Card (int s, int r) { 
  suit = s;  rank = r;
}
\end{verbatim}
%
Það eru tveir smiðir (e. constructors) fyrir {\tt Card}.
Þú sérð að um er að ræða smiði vegna þess að þeir hafa ekkert skilagildi og nafn þeirra er það sama og nafn strúktúrsins.
Fyrri smiðurinn tekur engin viðföng og upphafsstillir meðlimabreyturnar í raun með gagnlausum gildum (0 í laufi!).

Seinni smiðurinn er gagnlegri.
Hann tekur tvö viðföng, lit og gildi spilsins.

\index{smiður}

Eftirfarandi kóði býr til hlut með nafninu {\tt threeOfClubs} sem stendur fyrir laufaþrist: 

\begin{verbatim}
   Card threeOfClubs (0, 3);
\end{verbatim}
%
Fyrsta viðfangið, {\tt 0}, stendur fyrir litinn lauf og seinna viðfangið fyrir gildið 3.

\section{{\tt printCard} fallið}
\index{printCard}
\index{print!Card}

Fyrsta skrefið í því að búa til nýtt tag felst yfirleitt í því að lýsa yfir meðlimabreytum og skrifa smiði.
Annað skrefið felst oft í því að skrifa út viðkomandi hlut á læsilegan máta (e. human-readable).

\index{string!vektor af}
\index{vector!af string}

Í tilviki {\tt Card} hlutanna þá merkir ``læsilegur máti'' að við verðum að varpa innri framsetningu litar og gildis í orð.
Eðlileg leið til að gera það er að nota vektor af strengjum ({\tt string}).
Þú getur búið til vektor af strengjum á sama hátt og þú býrð til vektor af hvaða öðru tagi sem er:

\begin{verbatim}
  vector<string> suits (4);
\end{verbatim}
%
%Of course, in order to use {\tt apvector}s and {\tt apstring}s, you
%will have to include the header files for both\footnote{{\tt apvector}s
%are a little different from {\tt apstring}s in this regard.
%The file {\tt apvector.cpp} contains a template that allows the
%compiler to create vectors of various kinds.  The first time you
%use a vector of integers, the compiler generates code
%to support that kind of vector.  If you use a vector of {\tt apstring}s,
%the compiler generates different code to handle that kind of
%vector.  As a result, it is usually sufficient to include the
%eader file {\tt apvector.h}; you do not have to compile
%{\tt apvector.cpp} at all!  Unfortunately, if you do, you are
%likely to get a long stream of error messages.  I hope this
%footnote helps you avoid an unpleasant surprise, but the details
%in your development environment may differ.}.

Við getum notað röð gildisveitingarsetninga (e. assignment statements) til að upphafsstilla stök vektors:

\begin{verbatim}
  suits[0] = "Clubs";
  suits[1] = "Diamonds";
  suits[2] = "Hearts";
  suits[3] = "Spades";
\end{verbatim}
%
Stöðurit frir þennan vektor lítur þá svona út: 

\index{stöðurit}

\vspace{0.1in}
\centerline{\epsfig{figure=apstringvector.eps}}
\vspace{0.1in}

Við getum búið til sambærilegan vektor til að afkóta (e. decode) gildin.
Þá getum við valið viðeigandi stök með því að nota {\tt suit} og {\tt rank} sem vísa (e. indices).
Að lokum getum við síðan skrifað fallið {\tt print} sem skrifar út spilið:

\begin{verbatim}
void Card::print () const
{
  vector<string> suits (4);
  suits[0] = "Clubs";
  suits[1] = "Diamonds";
  suits[2] = "Hearts";
  suits[3] = "Spades";

  vector<string> ranks (14);
  ranks[1] = "Ace";
  ranks[2] = "2";
  ranks[3] = "3";
  ranks[4] = "4";
  ranks[5] = "5";
  ranks[6] = "6";
  ranks[7] = "7";
  ranks[8] = "8";
  ranks[9] = "9";
  ranks[10] = "10";
  ranks[11] = "Jack";
  ranks[12] = "Queen";
  ranks[13] = "King";

  cout << ranks[rank] << " of " << suits[suit] << endl;
}
\end{verbatim}
%
Segðin {\tt suits[suit]} merkir að ``nota meðlimabreytuna {\tt suit} í núverandi hlut sem vísi inn í vektorinn {\tt suits}
og velja viðeigandi streng.''

Vegna þess að {\tt print} er meðlimafall í {\tt Card} þá getur það vísað í meðlimbreytur núverandi hlutar á dulinn hátt (án þess að nota punktatáknun og ``this'' til að tilgreina hlutinn).
Úttakið úr þessum kóða

\begin{verbatim}
  Card card (1, 11);
  card.print ();
\end{verbatim}
%
er {\tt Jack of Diamonds}.

Þú tekur kannski eftir því að við notum ekki núllta stakið í {\tt ranks} vektornum.
Ástæðan er sú að leyfileg gildi eru aðeins þau á bilinu 1--13.
Með því að skilja fyrsta stakið (núllta stakið) eftir ónotað í vektornum þá fáum við vörpun sem varpar 2 í ``2'', 3 í ``3'', o.s.frv.
Notandi verður ekkert var við þessa vörpun (eða innri framsetningu) sem við notum í forritinu því allt inntak og úttak er sett fram á læsilegan hátt.
Á hinn bóginn er það oft þægilegt fyrir forritarann ef vörpun er sett fram á máta sem auðvelt er að muna.

\section{Fallið {\tt equals}}
\index{samanburður}

Tvö spil eru jöfn ef þau eru með sama gildi og bera sama lit.
Því miður getum við ekki notað {\tt ==} virkjann til að bera saman tvö spil
því hann virkar ekki fyrir tög sem notandinn skilgreinir (e. user-defined types) eins og {\tt Card}.
Við verðum því að skrifa fall sem ber saman tvö spil.
Köllum það {\tt equals}.
Það er reyndar hægt að fjölbinda (e. overload) {\tt ==} virkjann en við munu ekki fjalla um þann möguleika í þessari bók. 

Það er ljóst að skilagildið úr {\tt equals} ætti að vera bool gildi sem gefur til kynna hvort tvö spil eru jöfn eður ei.
Það er jafnframt ljóst að fallið þarf að taka tvo {\tt Card} hluti sem viðföng. 
Nú stöndum við frammi fyrir eftirfarandi vali: Á {\tt equals} að vera meðlimafall (e. member function) eða fall sem stendur eitt og sér (e. free-standing function)?

Sem meðlimafall lítur {\tt equals} svona út:

\begin{verbatim}
bool Card::equals (const Card& c2) const
{
  return (rank == c2.rank && suit == c2.suit);
}
\end{verbatim}
%
Til að nota fallið vekjum við það upp með einu spili og sendum hitt spilið sem viðfang: 

\begin{verbatim}
  Card card1 (1, 11);
  Card card2 (1, 11);

  if (card1.equals(card2)) {
    cout << "Yup, that's the same card." << endl;
  }
\end{verbatim}
%
Mér finnst þessi aðferð við vakningu (e. invocation) alltaf líta dálítið undarlega út
þegar um er að ræða fall eins og {\tt equals} þar sem viðföngin tvö eru samhverf (e. symmetric).
Það sem ég á við með samhverf viðföng er að það skiptir ekki máli hvort ég spyr 
``Is A equal to B?'' eða ``Is B equal to A?''.
Í þessu tilviki mér finnst því eðlilegra að skrifa {\tt equals} sem ``ekki-meðlimafall'' (e. nonmember function). 

\begin{verbatim}
bool equals (const Card& c1, const Card& c2)
{
  return (c1.rank == c2.rank && c1.suit == c2.suit);
}
\end{verbatim}
%
Þegar við köllum á þessa útgáfu fallsins þá birtast viðföngin hlið við hlið á máta sem mér finnst eðlilegri.

\begin{verbatim}
  if (equals (card1, card2)) {
    cout << "Yup, that's the same card." << endl;
  }
\end{verbatim}
%
Þetta er samt auðvitað bara spurning um smekk.
Punkturinn hér er sá að þú ættir að geta skrifað hvort sem er meðlimafall eða ``ekki-meðlimafall'' þannig að þú getir ákveðið þau skil (e. interface)
sem eru hentugust því verkefni sem liggur fyrir hverju sinni.

\section{Fallið {\tt isGreater}}
\index{isGreater}
\index{virki!samanburður}
\index{samanburðarvirki}

Fyrir grunntög eins og {\tt int} og {\tt double} eru til innbyggðir samanburðarvirkjar
sem bera saman gildi og ákvarða hvort eitt gildi er stærra eða minna en annað.
Þessir virkjar ({\tt <} og {\tt >} og fleiri) virka hins vegar ekki fyrir tög sem skilgreind eru af notanda.
Við þurfum því, á sama hátt og við gerðum í tilviki {\tt ==} virkjans, að skrifa samanburðarfall sem líkir eftir {\tt >} virkjanum.
Við munum síðar nota þetta fall til að raða spilum spilastokks.

\index{röðun}
%\index{complete ordering}
%\index{partial ordering}
\index{fullröðun}
\index{hlutröðun}

Sum mengi eru fullröðuð (e. totally ordered) sem merkir að hægt er að bera saman hvaða tvö stök sem er og segja til um hvort er stærra.
Heiltölur og kommutölur eru t.d. fullraðaðar.
Sum mengi eru hins vegar ekki röðuð sem merkir að það er engin nærtæk leið til að segja til um hvort eitt stak sé stærra en annað.
Ávextir eru t.d. ekki raðaðir sem er einmitt ástæðan fyrir því að við getum ekki borið saman epli og appelsínur!
Annað dæmi um mengi sem er ekki raðað er {\tt bool} tagið -- við getum ekki sagt að {\tt true} sé stærra en {\tt false}.

Mengi handspila er hlutraðað (e. partially ordered), þ.e. stundum er hægt að bera saman spil og stundum ekki.
Ég veit t.d. að laufaþristur er hærri en laufatvistur vegna þess að sá fyrrnefndi er með hærra gildi en sá síðarnefndi
og að tígulþristur er hærri en laufaþristur vegna þess að sá fyrrnefndi er með hærri lit.
En hvort er laufaþristur eða tígultvistur betra spil?  Annað er með hærra gildi en hitt en með hærri lit.

%\index{comparable}
\index{samanburðarhæfni}

Til að tvö spil séu samanburðarhæf verðum við að ákveða hvort litur eða gildi er mikilvægara 
Ég ætla að segja að litur sé mikilvægari og ástæðan er sú að þegar þú kaupir nýjan spilastokk þá er hann raðaður eftir litum,
þ.e. öll laufin saman, síðan tíglarnir, o.s.frv.

Nú þegar þetta er ákveðið þá getum við skrifað fallið {\tt isGreater}.
Aftur ætti að vera augljóst að viðföngin eru tveir {\tt Card} hlutir og að skilagildið er {\tt bool}.
Við þurfum á ný að velja á milli meðlimafalls og ``ekki-meðlimafalls''.
Í þetta skiptið eru viðföngin ekki samhverf.
Það skipir máli hvort við viljum vita ``Is A greater than B?'' eða ``Is B greater than A?''.
Þess vegna finnst mér eðlilegra að útfæra {\tt isGreater} sem meðlimafall:

\begin{verbatim}
bool Card::isGreater (const Card& c2) const
{
  // first check the suits
  if (suit > c2.suit) return true;
  if (suit < c2.suit) return false;

  // if the suits are equal, check the ranks
  if (rank > c2.rank) return true;
  if (rank < c2.rank) return false;

  // if the ranks are also equal, return false
  return false;
}
\end{verbatim}
%
Þegar við vekjum fallið upp þá er augljóst út frá málskipaninni hvora af hinum tveimur spurningum að ofan við erum að setja fram:

\begin{verbatim}
  Card card1 (2, 11);
  Card card2 (1, 11);

  if (card1.isGreater (card2)) {
    card1.print ();
    cout << "is greater than" << endl;
    card2.print ();
  }
\end{verbatim}
%
Það er nánast hægt að lesa þetta eins og ensku: ``If card1 isGreater card2 ...''
Úttak forritsins er:

\begin{verbatim}
Jack of Hearts
is greater than
Jack of Diamonds
\end{verbatim}
%
Samkvæmt {\tt isGreater} eru ásar lægri en tvistar. 
Þú ættir núna að laga fallið þannig að ásar séu hærri en kóngar eins og þeir eru í flestum spilum.

\section{Vektor af spilum}
\index{vector!af hlutum}
\index{hlutur!vektor af}
\index{spilastokkur}

Ástæðan fyrir því að ég valdi spil ({\tt Cards}) sem hluti í þessum kafla er að það er augljóst notkun fyrir vektor af spilum, þ.e. spilastokkur.
Hér er forritskóði sem býr til nýjan stokk af 52 spilum:

\begin{verbatim}
  vector<Card> deck (52);
\end{verbatim}
%
Og hér er stöðurit fyrir þennan hlut: 

\index{state diagram}

\vspace{0.1in}
\centerline{\epsfig{figure=cardvector.eps}}
\vspace{0.1in}

Punktarnir þrír standa fyrir spilin 48 sem ég sleppti að teikna.
Taktu eftir að við höfum ekki upphafsstillt meðlimabreytur sérhvers spils.
Í sumum keyrsluumhverfum munu þær verða upphafsstilltar sjálfvirkt með núllum (eins og sýnt er á myndinni)
en annars staðar gætu þær fengið hvaða gildi sem er.

Ein leið til að upphafsstilla meðlimabreyturnar er að senda {\tt Card} hlut sem annað viðfang í smiðinn:

\begin{verbatim}
  Card aceOfSpades (3, 1);
  vector<Card> deck (52, aceOfSpades);
\end{verbatim}
%

Þessi forritskóði býr til stokk me 52 eins spilum, svona eins og gæti verið notaður í töfrabragði!
Það er auðvitað eðlilegra að búa til stokk með 52 mismunandi spilum í.
Við getum gert það með því að nota hreiðraða (e. nested) lykkju.

\index{lykkja!hreiðruð}
\index{hreiðruð lykkja}

Ytri lykkjan ``telur upp'' (e. enumerates) litina, frá 0 til 3.
Fyrir hvern lit telur innri lykkjan upp gildin, frá 1 til 13.
Þar sem ytri lykkjan keyrir fjórum sinnum og innri lykkjan 13 sinnum þá keyrir meginmál (e. body) lykkjunnar samtals 52 sinnum (13*4).

\begin{verbatim}
  int i = 0;
  for (int suit = 0; suit <= 3; suit++) {
    for (int rank = 1; rank <= 13; rank++) {
      deck[i].suit = suit;
      deck[i].rank = rank;
      i++;
    }
  }
\end{verbatim}
%
Ég notaði breytuna {\tt i} til að halda utan um í hvaða sæti í spilastokknum næsta spil ætti að fara. 

\index{vísir}

Taktu eftir því að við getum sett saman málskipanina til að velja stak úr fylki (með {\tt []} virkjanum)
og málskipanina sem velur meðlimabreytu úr hlut (punkta táknið).
Segðin {\tt deck[i].suit} merkir þá ``the suit of the ith card in the deck''.

\index{hjúpun}

Til að æfa þig ættir þú nú að hjúpa ofangreindan kóða í fall sem kallast {\tt buildDeck}.
Fallið tekur engin viðföng og skilar vektor af {\tt Card} sem hefur verið upphafsstilltur með 52 spilum.

\section{Fallið {\tt printDeck}}
\label{printdeck}
\index{printDeck}
\index{print!vektor af spilum}

Þegar unnið er með vektor þá er þægilegt að eiga fall sem skrifar út innihald vektorsins.
Við höfum nokkrum sinnum séð mynstur til að ``ferðast um'' í vektor þannig að eftirfarandi fall ætti að vera kunnuglegt:

\begin{verbatim}
void printDeck (const vector<Card>& deck) {
  for (int i = 0; i < deck.length(); i++) {
    deck[i].print ();
  }
}
\end{verbatim}
%
Það ætti heldur ekki að koma á óvart að við getum sett saman málskipanina til að nálgast stak í vektor og málskipanina til að vekja upp fall.

Þar sem {\tt deck} hefur tagið {\tt vector<Card>} þá hefur stak í {\tt deck} tagið {\tt Card}.
Þess vegna er leyfilegt að vekja upp fallið {\tt print} með {\tt deck[i]}.

\section{Searching}
\label{find}
\index{searching}
\index{linear search}
\index{find}

The next function I want to write is {\tt find}, which searches
through a vector of {\tt Card}s to see whether it contains a certain
card.  It may not be obvious why this function would be useful, but it
gives me a chance to demonstrate two ways to go searching for things,
a {\tt linear} search and a {\tt bisection} search.

\index{traverse}
\index{loop!search}

Linear search is the more obvious of the two; it involves traversing
the deck and comparing each card to the one we are looking for.  If we
find it we return the index where the card appears.  If it is not in
the deck, we return -1.

\begin{verbatim}
int find (const Card& card, const apvector<Card>& deck) {
  for (int i = 0; i < deck.length(); i++) {
    if (equals (deck[i], card)) return i;
  }
  return -1;
}
\end{verbatim}
%
The loop here is exactly the same as the loop in {\tt printDeck}.
In fact, when I wrote the program, I copied it, which saved me
from having to write and debug it twice.

Inside the loop, we compare each element of the deck to
{\tt card}.  The function returns as soon as it discovers
the card, which means that we do not have to traverse the entire
deck if we find the card we are looking for.  If the loop terminates
without finding the card, we know the card is not in the deck
and return {\tt -1}.

\index{pattern!eureka}
\index{statement!return}
\index{return!inside loop}

To test this function, I wrote the following:

\begin{verbatim}
  apvector<Card> deck = buildDeck ();

  int index = card.find (deck[17]);
  cout << "I found the card at index = " << index << endl;
\end{verbatim}
%
The output of this code is

\begin{verbatim}
I found the card at index = 17
\end{verbatim}
%


\section{Bisection search}
\index{bisection search}

If the cards in the deck are not in order, there is no way to search
that is faster than the linear search.  We have to look at every card,
since otherwise there is no way to be certain the card we want is not
there.

But when you look for a word in a dictionary, you don't search
linearly through every word.  The reason is that the words are in
alphabetical order.  As a result, you probably use an algorithm that
is similar to a bisection search:

\begin {enumerate}

\item Start in the middle somewhere.

\item Choose a word on the page and compare it to the word you
are looking for.

\item If you found the word you are looking for, stop.

\item If the word you are looking for comes after the word on
the page, flip to somewhere later in the dictionary and go to
step 2.

\item If the word you are looking for comes before the word on
the page, flip to somewhere earlier in the dictionary and go to
step 2.

\end {enumerate}

If you ever get to the point where there are two adjacent words on the
page and your word comes between them, you can conclude that your word
is not in the dictionary.  The only alternative is that your word has
been misfiled somewhere, but that contradicts our assumption that the
words are in alphabetical order.

In the case of a deck of cards, if we know that the cards are in
order, we can write a version of {\tt find} that is much faster.  The
best way to write a bisection search is with a recursive function.
That's because bisection is naturally recursive.

\index{findBisect}

The trick is to write a function called {\tt findBisect} that takes
two indices as parameters, {\tt low} and {\tt high}, indicating the
segment of the vector that should be searched (including both
{\tt low} and {\tt high}).

\begin{enumerate}

\item To search the vector, choose an index between {\tt low} and {\tt
high}, and call it {\tt mid}.  Compare the card at {\tt mid} to the
card you are looking for.

\item If you found it, stop.

\item If the card at {\tt mid} is higher than your card, search
in the range from {\tt low} to {\tt mid-1}.

\item If the card at {\tt mid} is lower than your card, search
in the range from {\tt mid+1} to {\tt high}.

\end{enumerate}
%
Steps 3 and 4 look suspiciously like recursive
invocations.  Here's what this all looks like translated into
C++:

\begin{verbatim}
int findBisect (const Card& card, const apvector<Card>& deck,
                int low, int high) {
  int mid = (high + low) / 2;

  // if we found the card, return its index
  if (equals (deck[mid], card)) return mid;

  // otherwise, compare the card to the middle card
  if (deck[mid].isGreater (card)) {
    // search the first half of the deck
    return findBisect (card, deck, low, mid-1);
  } else {
    // search the second half of the deck
    return findBisect (card, deck, mid+1, high);
  }
}
\end{verbatim}
%
Although this code contains the kernel of a bisection search, it
is still missing a piece.  As it is currently written,
if the card is not in the deck, it will recurse forever.  We
need a way to detect this condition and deal with it properly
(by returning {\tt -1}).

\index{recursion}

The easiest way to tell that your card is not in the deck
is if there are {\em no} cards in the deck, which is the
case if {\tt high} is less than {\tt low}.  Well, there are
still cards in the deck, of course, but what I mean is that
there are no cards in the segment of the deck indicated by
{\tt low} and {\tt high}.

With that line added, the function works correctly:

\begin{verbatim}
int findBisect (const Card& card, const apvector<Card>& deck,
                int low, int high) {

  cout << low << ", " << high << endl;

  if (high < low) return -1;

  int mid = (high + low) / 2;

  if (equals (deck[mid], card)) return mid;

  if (deck[mid].isGreater (card)) {
    return findBisect (card, deck, low, mid-1);
  } else {
    return findBisect (card, deck, mid+1, high);
  }
}
\end{verbatim}
%
I added an output statement at the beginning so I could watch
the sequence of recursive calls and convince myself
that it would eventually reach the base case.  I tried out the
following code:

\begin{verbatim}
  cout << findBisect (deck, deck[23], 0, 51));
\end{verbatim}
%
And got the following output:

\begin{verbatim}
0, 51
0, 24
13, 24
19, 24
22, 24
I found the card at index = 23
\end{verbatim}
%
Then I made up a card that is not in the deck (the 15 of Diamonds),
and tried to find it.  I got the following:

\begin{verbatim}
0, 51
0, 24
13, 24
13, 17
13, 14
13, 12
I found the card at index = -1
\end{verbatim}
%
These tests don't prove that this program is correct.  In fact, no
amount of testing can prove that a program is correct.  On the other
hand, by looking at a few cases and examining the code, you might be
able to convince yourself.

\index{testing}
\index{correctness}

The number of recursive calls is fairly small, typically 6 or 7.  That
means we only had to call {\tt equals} and {\tt isGreater} 6 or 7
times, compared to up to 52 times if we did a linear search.  In
general, bisection is much faster than a linear search, especially for
large vectors.

Two common errors in recursive programs are forgetting to include a
base case and writing the recursive call so that the base case is never
reached.  Either error will cause an infinite recursion, in which case
C++ will (eventually) generate a run-time error.

\index{recursion!infinite}
\index{infinite recursion}
\index{run-time error}

\section{Decks and subdecks}
\index{deck}
\index{subdeck}

Looking at the interface to {\tt findBisect}

\begin{verbatim}
int findBisect (const Card& card, const apvector<Card>& deck,
		int low, int high) {
\end{verbatim}
%
it might make sense to treat three of the parameters, {\tt deck}, {\tt
low} and {\tt high}, as a single parameter that specifies a {\bf
subdeck}.

\index{parameter!abstract}
\index{abstract parameter}

This kind of thing is quite common, and I sometimes think of it as an
{\bf abstract parameter}.  What I mean by ``abstract,'' is something
that is not literally part of the program text, but which describes the
function of the program at a higher level.

For example, when you call a function and pass a vector and the bounds
{\tt low} and {\tt high}, there is nothing that prevents the called
function from accessing parts of the vector that are out of bounds.  So
you are not literally sending a subset of the deck; you are really
sending the whole deck.  But as long as the recipient plays by the
rules, it makes sense to think of it, abstractly, as a subdeck.

There is one other example of this kind of abstraction that you might
have noticed in Section~\ref{objectops}, when I referred to an
``empty'' data structure.  The reason I put ``empty'' in quotation
marks was to suggest that it is not literally accurate.  All variables
have values all the time.  When you create them, they are given
default values.  So there is no such thing as an empty object.

But if the program guarantees that the current value of a variable is
never read before it is written, then the current value is irrelevant.
Abstractly, it makes sense to think of such a variable as ``empty.''

This kind of thinking, in which a program comes to take on meaning
beyond what is literally encoded, is a very important part of thinking
like a computer scientist.  Sometimes, the word ``abstract'' gets used
so often and in so many contexts that it is hard to interpret.
Nevertheless, abstraction is a central idea in computer science (as
well as many other fields).

\index{abstraction}

A more general definition of ``abstraction'' is ``The process of
modeling a complex system with a simplified description in order to
suppress unnecessary details while capturing relevant behavior.''

\section{Glossary}

\begin{description}

\item[encode:]  To represent one set of values using another
set of values, by constructing a mapping between them.

\item[abstract parameter:]  A set of parameters that act together
as a single parameter.

\index{encode}
\index{abstract parameter}

\end{description}



% LaTeX source for textbook ``How to think like a computer scientist''
% Copyright (C) 1999  Allen B. Downey

% This LaTeX source is free software; you can redistribute it and/or
% modify it under the terms of the GNU General Public License as
% published by the Free Software Foundation (version 2).

% This LaTeX source is distributed in the hope that it will be useful,
% but WITHOUT ANY WARRANTY; without even the implied warranty of
% MERCHANTABILITY or FITNESS FOR A PARTICULAR PURPOSE.  See the GNU
% General Public License for more details.

% Compiling this LaTeX source has the effect of generating
% a device-independent representation of a textbook, which
% can be converted to other formats and printed.  All intermediate
% representations (including DVI and Postscript), and all printed
% copies of the textbook are also covered by the GNU General
% Public License.

% This distribution includes a file named COPYING that contains the text
% of the GNU General Public License.  If it is missing, you can obtain
% it from www.gnu.org or by writing to the Free Software Foundation,
% Inc., 59 Temple Place - Suite 330, Boston, MA 02111-1307, USA.

% This is an Icelandic translation/adaptation by Hrafn Loftsson of the orginal book by Allen B. Downey.

%\chapter{Objects of Vectors}
\chapter{Hlutir með vektorum}

%\section{Enumerated types}
\section{Upptalningartög}
\index{tag!upptalning}
\index{upptalningartag}
\index{vörpun}

Í kaflanum hér á undan ræddi ég um vörpun á milli raunverulegra gilda, eins og gilda eða lita tiltekinna spila, 
og innri framsetningar, eins og heiltalna og strengja.
Þó svo að við höfum annars vegar búið til vörpun á milli gilda og heiltalna og hins vegar á milli lita og heiltalna
þá benti ég á að vörpunin sjálf kemur ekki beint fram í forritskóðanum.

Reyndar vill svo til að C++ hefur eiginleika sem kallast {\bf upptalningartag} (e. enumerated type)
sem gerir kleift (1) að gera vörpun sem hluta af forritskóða og (2) að skilgreina mengi af gildum sem vörpunin samanstendur af.
Hér er t.d. skilgreiningin á upptalningartagi fyrir {\tt Suit} og {\tt Rank}:

\begin{verbatim}
enum Suit { CLUBS, DIAMONDS, HEARTS, SPADES };

enum Rank { ACE=1, TWO, THREE, FOUR, FIVE, SIX, SEVEN, EIGHT, NINE,
TEN, JACK, QUEEN, KING };
\end{verbatim}
%
Sjálfgefin vörpun er sú að fyrsta gildið í upptalningartagi varpast í heiltöluna 0, annað gildið í 1, o.s.frv.
Í {\tt Suit} taginu stendur því gildið {\tt CLUBS} fyrir heiltöluna 0, {\tt DIAMONDS} fyrir 1, o.s.frv.

Skilgreiningin á {\tt Rank} yfirskrifar sjálfgefnu vörpunina og tilgreinir að {\tt ACE} standi fyrir heiltöluna 1.
Þar með varpast {\tt TWO} í 2,  {\tt THREE} í 3, o.s.frv.

Um leið og við höfum skilgreint þessi tög getum við notað þau hvar sem er.
Meðlimabreytunum {\tt rank} og {\tt suit} getur t.d. verið lýst yfir með tögunum {\tt Rank} og {\tt Suit}:

\begin{verbatim}
struct Card
{
  Rank rank;
  Suit suit;

  Card (Suit s, Rank r);
};
\end{verbatim}
%
Taktu eftir að tag leppanna í smiðnum hefur líka breyst.
Til að ``smíða'' card hlut getum við þá notað gildi úr upptalningartögunum sem viðföng:

\begin{verbatim}
  Card card (DIAMONDS, JACK);
\end{verbatim}
%
Það er hefð fyrir því að gildi úr upptalningartögum séu nöfn með stórum stöfum.
Þessi yfirlýsing á card hlut er miklu læsilegri en sú sem við notuðum áður með heiltölum:

\begin{verbatim}
  Card card (1, 11);
\end{verbatim}
%
Við getum notað gildi upptalningartaga sem vísi inn í vektora vegna þess að við vitum 
að þessi gildi standa fyrir heiltölur.
Af þessum sökum mun gamla {\tt print} fallið okkar virka á nokkurra breytinga.
Við þurfum hins vegar að gera breytingar á {\tt buildDeck} fallinu:

\begin{verbatim}
  int index = 0;
  for (Suit suit = CLUBS; suit <= SPADES; suit = Suit(suit+1)) {
    for (Rank rank = ACE; rank <= KING; rank = Rank(rank+1)) {
      deck[index].suit = suit;
      deck[index].rank = rank;
      index++;
    }
  }
\end{verbatim}
%
Með því að nota upptalningartögin er kóðinn læsilegri en það er ein flækja.
Strangt til tekið getum við ekki beitt útreikningum á upptalningartög þannig að {\tt suit++} er ekki löglegt.
Hins vegar breytir C++ upptalningargildinu suit í segðinni {\tt suit+1} í heiltölu.
Þar með getum við tekið niðurstöðuna og notað tagmótun (e. typecast) til að ``kasta'' gildinu til baka í upptalningartag: 

\begin{verbatim}
  suit = Suit(suit+1);
  rank = Rank(rank+1);
\end{verbatim}
%
Það er reyndar til betri leið til að gera þetta með því að yfirskrifa {\tt ++} virkjann fyrir
upptalningartögin en við fjöllum ekki um það í þessari bók. 

\section{{\tt switch} setning}
\index{switch setning}
\index{setning!switch}

Það er erfitt að ræða upptalningartög án þess að nefna {\tt switch} setningar því oft haldast þær í hendur.
{\tt switch} setning er önnur leið til að setja fram keðjur af if-skilyrðissetningum og oft reyndar bæði læsilegri og skilvirkari. 
Hún lítur svona út:

\begin{verbatim}
  switch (symbol) {
  case '+':
    perform_addition ();
    break;
  case '*':
    perform_multiplication ();
    break;
  default:
    cout << "I only know how to perform addition and multiplication" << endl;
    break;
  }
\end{verbatim}
%
Þessi {\tt switch} setning er jafngild eftirfarandi keðju af if-setningum:

\begin{verbatim}
  if (symbol == '+') {
    perform_addition ();
  } else if (symbol == '*') {
    perform_multiplication ();
  } else {
    cout << "I only know how to perform addition and multiplication" << endl;
  }
\end{verbatim}
%
{\tt break} setningarnar eru nauðsynlegar í sérhverri ``grein'' í 
{\tt switch} setningu vegna þess að án {\tt break} setningar þá ``fellur'' keyrsluflæðið í næsta tilvik (e. case).
Án {\tt break} setninga myndi táknið {\tt +} verða þess valdandi að forritið myndi framkvæma
samlagningu (e. addition), síðan margföldun (e. multiplication) og að lokum prenta út villuskilaboðin.
Af og til er þessi eiginleiki reyndar gagnlegur en í flestum tilvikum koma upp villur í keyrslu þegar 
{\tt break} setningar gleymast.

\index{break setning}
\index{setning!break}

{\tt switch} setningar virka fyrir heiltölur, stafi og upptalningatög.
Til að breyta {\tt Suit} í samsvarandi streng þá getum við t.d. gert eftirfarandi: 

\begin{verbatim}
  switch (suit) {
  case CLUBS:     return "Clubs";
  case DIAMONDS:  return "Diamonds";
  case HEARTS:    return "Hearts";
  case SPADES:    return "Spades";
  default:        return "Not a valid suit";
  }
\end{verbatim}
%
Í þessu dæmi þurfum við ekki á {\tt break} setningu að halda vegna þess að 
{\tt return} setningar valda því að keyrsluflæðið fer til baka til þess sem kallaði í stað þess
að ``falla'' í næsta tilvik. 

\index{default}

Almennt séð er góður forritunarstíll að hafa sjálfgefið ({\tt default}) tilvik í sérhverri     
{\tt switch} setningu til að meðhöndla villur eða óvænt gildi. 

\section{Stokkar}
\label{deck}
\index{stokkur}
\index{vektor!af spilum}

Í köflunum hér á undan unnum við með vektor af hlutum 
en ég nefndi það líka að það væri hægt að búa til hluti 
sem innihalda vektor sem meðlimabreytu.
Í þessum kafla ætla ég að búa til nýjan hlut, {\tt Deck},
sem inniheldur vektor af {\tt Card}.

\index{meðlimabreyta}
\index{breyta!meðlimur}

Strúktúrskilgreiningin lítur svona út:

\begin{verbatim}
struct Deck {
  vector<Card> cards;

  Deck (int n);
};

Deck::Deck (int size)
{
  vector<Card> temp (size);
  cards = temp;
}
\end{verbatim}
%
Nafnið á meðlimabreytunni er {\tt cards} sem hjálpar okkur að gera
greinarmun á {\tt Deck} (stokknum) sjálfum og vektor af {\tt Card} sem stokkurinn inniheldur.


\index{smiður}

Sem stendur er aðeins um einn smið að ræða.
Sá nýtir sér staðværa breytu með nafninu {\tt temp} sem er upphafsstillt
með því að vekja upp smiðinn í {\tt vector} klasanum (stærðin, size, er send sem viðfang).
Síðan afritar smiðurinn {\tt temp} yfir í meðlimabreytuna {\tt cards}.

Nú getum við þá búið til stokk af spilum á eftirfarandi hátt: 

\begin{verbatim}
  Deck deck (52);
\end{verbatim}
%
Stöðurit fyrir {\tt Deck} hlut lítur svona út:

\index{stöðurit}
\index{smiður}

\vspace {0.1in}
\centerline{\epsfig{figure=deckobject.eps}}
\vspace {0.1in}

Hluturinn með nafnið {\tt deck} á sér eina meðlimabreytu með nafnið {\tt cards}
sem er vektor af {\tt Card} hlutum.
Til að nálgast spil í stokknum þurfum við að setja saman málskipanina fyrir að nálgast meðlimabreytu
og málskipanina fyrir að velja stak úr vektor.
Segðin {\tt deck.cards[i]} er t.d. i-ta spilið í stokknum og
{\tt deck.cards[i].suit} er litur þess.

Eftirfarandi lykkja

\begin{verbatim}
  for (int i = 0; i<52; i++) {
    deck.cards[i].print();
  }
\end{verbatim}
%
sýnir hvernig er hægt að ferðast um í stokknum og prenta út sérhvert spil.

\section {Annar smiður}
\index{smiður}

Það getur verið gagnlegt að upphafsstilla spilin í {\tt Deck} hlutnum.
Við gætum notað fallið {\tt buildDeck} úr fyrri kafla (með smávægilegum breytingum) 
en það er líklega eðlilegra að skrifa annan smið í {\tt Deck}.

\index{lykkja!hreiðruð}

\begin{verbatim}
Deck::Deck ()
{
  vector<Card> temp (52);
  cards = temp;

  int i = 0;
  for (Suit suit = CLUBS; suit <= SPADES; suit = Suit(suit+1)) {
    for (Rank rank = ACE; rank <= KING; rank = Rank(rank+1)) {
      cards[i].suit = suit;
      cards[i].rank = rank;
      i++;
    }
  }
}
\end{verbatim}
%
Taktu eftir því hversu líkt þetta fall er {\tt buildDeck} -- við þurftum aðeins að breyta málskipaninni til að gera þetta að smið.
Nú getum við búið til hefðbundinn 52-spila stokk með einfaldri yfirlýsingu: {\tt Deck deck;}

\section {Meðlimaföll í {\tt Deck}}
\index{meðlimaföll}
%\index{function!member}

Nú þegar við höfum sérstakan {\tt Deck} hlut þá er eðlilegt að gera öll föll, sem hafa með stokk að gera, að
meðlimaföllum í {\tt Deck}.
Einn augljós kandídat er fallið {\tt printDeck} (kafli~\ref{printdeck}).
Svona lítur það út eftir að því hefur verið breytt í meðlimafall í {\tt Deck}:

\index{printDeck}

\begin{verbatim}
void Deck::print () const {
  for (int i = 0; i < cards.length(); i++) {
    cards[i].print ();
  }
}
\end{verbatim}
%
Að venju getum við vísað í meðlimabreytu núverandi hlutar (t.d. {\tt cards}) án þess að nota punktatáknun (e. dot notation).

Það er hins vegar ekki augljóst fyrir sum önnur föll hvort þau ættu að vera
meðlimaföll í {\tt Card}, í {\tt Deck} eða föll sem standa eitt og sér (e. nonmember function)
og taka {\tt Card} og {\tt Deck} hluti sem viðföng.
Útgáfan af {\tt find} úr fyrri kafla tekur t.d. {\tt Card} og {\tt Deck} hluti sem viðföng
en við gætum gert fallið að meðlimafalli annars hvors hlutarins. 
Til að æfa þig skaltu núna endurskrifa {\tt find} sem meðlimafall í {\tt Deck} sem tekur
{\tt Card} hlut sem viðfang. 

Það að skrifa {\tt find} sem meðlimafall í {\tt Card} er aftur á móti dálítið snúið.
Hér er mín útgáfa: 

\begin{verbatim}
int Card::find (const Deck& deck) const {
  for (int i = 0; i < deck.cards.length(); i++) {
    if (equals (deck.cards[i], *this)) return i;
  }
  return -1;
}
\end{verbatim}
%
Fyrsta ``vandamálið'' er að við verðum að nota lykilorðið {\tt this}
til að vísa í það spil sem fallið er vakið upp með.

\index{strúktúrskilgreining}

Annað vandamálið er það að C++ gerir ekki auðvelt að skrifa strúktúrskilgreiningar
sem vísa hvor á aðra.
Vandamálið er að þegar þýðandinn les fyrri skilgreininguna þá hefur það ekki séð þá seinni.

Ein lausn er að lýsa {\tt Deck} yfir (e. declare) á undan {\tt Card} og síðan skilgreina (e. define) {\tt Deck} eftir það:

\begin{verbatim}
// declare that Deck is a structure, without defining it
struct Deck;

// that way we can refer to it in the definition of Card
struct Card
{
  int suit, rank;

  Card ();
  Card (int s, int r);

  void print () const;
  bool isGreater (const Card& c2) const;
  int find (const Deck& deck) const;
};

// and then later we provide the definition of Deck
struct Deck {
  vector<Card> cards;

  Deck ();
  Deck (int n);
  void print () const;
  int find (const Card& card) const;
};
\end{verbatim}


\section{Að stokka}
\label{shuffle}
\index{stokka}

Í flestum spilum þarf að vera hægt að stokka spilin, þ.e. setja þau í handahófskennda röð í stokknum.
Við sáum í kafla~\ref{random} hvernig hægt er að búa til slembitölur en það er ekki augljóst hvernig hægt er 
að nota þær til að stokka spilin.

Ein leið er sú að líkja eftir því hvernig við sjálf stokkum, t.d. að skipta stokknum í tvennt
og setja hann svo saman á ný með því að velja eitt spil í einu frá hvorum helmingnum.
Þetta er kallað fullkomin stokkun (e. perfect shuffle).
Það getur tekið okkur um 7 ítranir þangað til spilin eru vel stokkuð með þessari aðferð.
Forrit hefur hins vegar þann pirrandi eiginleika að stokka fullkomlega í sérhvert sinn sem er í raun ekki mjög handahófskennt!
Það vill reyndar svo til að eftir 8 fullkomnar stokkanir þá er stokkurinn í sömu röð og þegar byrjað var.
Þú getur séð umfjöllun um þessa fullyrðingu á {\tt http://www.wiskit.com/marilyn/craig.html} eða leitað á vefnum
með leitarorðunum ``perfect shuffle.''

Betra stokkunaralgrím er að ferðast um í stokknum, spil fyrir spil, og skipta á núverandi spili og öðru spili (sem valið er af handahófi) í sérhverri ítrun. 

\index{sauðakóði}

Hér má sjá drög af því hvernig þetta algrím virkar.
Til að setja fram forritið nota ég sambland af C++ setningum og ensku -- þetta er stundum kallað 
called {\bf sauðakóði} (e. pseudocode):

\begin{verbatim}
  for (int i=0; i<cards.length(); i++) {
    // choose a random number between i and cards.length()
    // swap the ith card and the randomly-chosen card
  }
\end{verbatim}
%
Kosturinn við að nota sauðakóða er oft sá að hann gefur skýrt til kynna hvaða föll eru nauðsynleg.
Í þessu tilviki þurfum við fall eins og {\tt randomInt} sem velur, á handahófskenndan hátt, heiltölu á milli
viðfanganna {\tt low} og {\tt high}, og fall eins og {\tt swapCards} sem tekur tvo vísa og skiptir á spilunum í viðkomandi sætum.

\index{slembitölur}

Þú ættir að geta fundið út hvernig skrifa á {\tt randomInt} með því að kíkja til baka í kafla~\ref{random}.
Þú þarft þó að passa þig að búa ekki til vísa sem eru út fyrir bilið (e. out of range). 

%\index{swapCards}
%\index{reference}

Þú ættir líka að geta skrifað {\tt swapCards} sjálf(ur). 

%I will leave the remaining implementation of these functions as an exercise to the reader.

\section{Röðun}
\label{sorting}
\index{röðun}

Nú þegar við getum stokkað spilin þurfum við líka að geta raðað þeim í rétta röð.
Það vill reyndar svo til að algrímið fyrir röðun er mjög svipað stokkunaralgríminu.

Við munum aftur ferðast um í spilastokknum og skipta á núverandi spili og öðru spili í sérhverri ítrun.
Eini munurinn er sá að í stað þess að velja annað spil af handahófi þá munum við finna lægsta spilið sem eftir er í stokknum.

Það sem ég á við með ``eftir er í stokknum'' eru spil sem eru jöfn eða hærri heldur er núverandi vísir {\tt i}.

\begin{verbatim}
  for (int i=0; i<cards.length(); i++) {
    // find the lowest card at or to the right of i
    // swap the ith card and the lowest card
  }
\end{verbatim}
%
Sauðakóðinn hjálpar hér aftur við hönnunina á {\bf hjálparföllum}.
Í þessu tilviki getum við aftur notað {\tt swapCards} fallið og við þurfum því einungis eitt nýtt fall, {\tt findLowestCard},
sem tekur vektor af spilum og vísi sem gefur til kynna hvar byrja á að leita í vektornum.

Þetta ferli að nota sauðakóða til að finna út hvaða hjálparföll eru nauðsynleg eru stundum kallað 
{\bf ofansækin hönnun} (e. top-down design), sem er andstaðan við {\bf neðansækna hönnun} (e. bottom-up design) sem ég ræddi um í kafla~\ref{counting}.

\index{ofansækin hönnun}
\index{hönnun!ofansækin}
\index{neðansækin hönnun}
\index{hönnun!bottom-up}
\index{hjálparfall}
%\index{fall!helper}

Þú ættir núna að spreyta þig á útfærslunni á þessum sauðakóða.

\section {Hlutstokkur}
\index{hlutstokkur}

Hvernig ættum við að tákna hönd eða annað hlutmengi af spilastokki?
Ein einföld leið er að búa til {\tt Deck} hlut sem hefur færri en 52 spil. 

Við gætum skrifað fall, {\tt subdeck}, sem tekur vektor af spilum og vísa (sem tákna bil)
og skilar nýjum vektor af spilum sem innheldur viðkomandi hlutmengi úr stokknum:

\begin{verbatim}
Deck Deck::subdeck (int low, int high) const {
  Deck sub (high-low+1);
	
  for (int i = 0; i<sub.cards.length(); i++) {
    sub.cards[i] = cards[low+i];
  }
  return sub;
}
\end{verbatim}
%
Til að búa til staðværu breytuna {\tt subdeck} notum við hér {\tt Deck} smiðinn sem tekur stærðina á nýja stokknum (hlutstokknum)
sem viðfang og upphafsstillir ekki spilin í stokknum.
Spilin eru í raun upphafsstillt þegar þau eru afrituð úr upphaflega stokknum.

Stærðin á hlutstokknum er {\tt high-low+1} vegna þess að spilin í bæði {\tt low} og {\tt high} eru innifalin.
Svona útreikningur getur verið ruglandi og hefur oft í för með sér ``off-by-one'' villur.
Það er oft gott að teikna mynd til að koma í veg fyrir þessar villur.

\index{smiður}
%\index{overloading}

%As an exercise, write a version of {\tt findBisect} that takes a
%subdeck as an argument, rather than a deck and an index range.  Which
%version is more error-prone?  Which version do you think is more
%efficient?

\section{Að stokka og gefa}
\index{stokka}
\index{gefa}

Í kafla~\ref{shuffle} skrifaði ég sauðakóða fyrir stokkunaralgríminu.
Ef við gerum nú ráð fyrir því að eiga fallið {\tt shuffle}, sem tekur stokk sem viðfang og stokkar hann, þá getum við búið til og stokkað spilastokk: 

\begin{verbatim}
  Deck deck;               // create a standard 52-card deck
  deck.shuffle ();         // shuffle it
\end{verbatim}
%
Síðan getum við gefið spil með því að nota {\tt subdeck}:

\begin{verbatim}
  Deck hand1 = deck.subdeck (0, 4);
  Deck hand2 = deck.subdeck (5, 9);
  Deck pack = deck.subdeck (10, 51);
\end{verbatim}
%
Þessi kóði gefur fyrstu 5 spilin til einnar handar, næstu 5 spil til annnarra handar og afgangurinn fer í bunkann.

Hvarflaði að þér að við ættum að gefa eitt spil í einu til sérhverrar handar eins og algengt er í raunverulegum spilum?
Ég hugsaði um það en áttaði mig svo á því að það væri ekki nauðsynlegt í forriti.
Það að gefa eitt spil í einu er yfirleitt gert til að leiðrétta ófullkomna stokkun og gera það að verkum að erfiðara sé fyrir þann sem gefur að svindla.
Í tilviki tölvu þarf ekki að hugsa um þessi atriði. 

%This example is a useful reminder of one of the dangers of engineering
%metaphors: sometimes we impose restrictions on computers that are
%unnecessary, or expect capabilities that are lacking, because we
%unthinkingly extend a metaphor past its breaking point.  Beware of
%misleading analogies.


\section {Mergesort}
\index{skilvirkni}
\index{röðun}
\index{mergesort}

Í kafla~\ref{sorting} sáum við einfalt röðunaralgrím sem vill svo til að er ekki mjög skilvirkt.
Til að raða $n$ hlutum (spilum) þarf algrímið að ferðast um í vektornum $n$ sinnum (for-lykkjan) og sérhver umferð tekur tíma sem er í réttu hlutfalli við $n$ (finna lægsta spil).
Heildartíminn sem algrímið tekur er því í réttu hlutfalli við $n^2$.

Í þessum kafla mun ég setja fram mun skilvirkara röðunaralgrím sem kallast {\bf mergesort}.
Tíminn sem mergesort tekur að raða $n$ hlutum er í réttu hlutfalli við $n \log n$.
Þetta virðist ekki vera mikil bæting en eftir því sem $n$ stækkar því meiri munur verður á $n^2$ og $n \log n$.
Prófaðu nokkur gildi fyrir $n$ og sjáðu muninn. 

Grundvallarhugmyndin á bak við mergesort er eftirfarandi:
Ef þú hefur tvo hlutstokka, sem hvor um sig er þegar raðaður, þá er einfalt (og skilvirkt) að setja þá saman (e. merge) í einn raðaðan stokk.
Prófaðu þetta með raunverulegum spilastokki:

\begin{enumerate}

\item Búðu til tvo hlutstokka með um 10 spilum hvor og raðaðu þeim þannig að þegar spilin snúa upp þá er lægsta spilið efst.
Settu báða stokkana fyrir framan þig (þannig að spilin snúi upp). 

\item Berðu saman efstu spilin í hvorum stokki og veldu það lægra. 
Snúðu því við og bættu því við nýja samsetta stokkinn (sem upphaflega er tómur). 

\item Endurtaktu skref tvö þangað til annar stokkurinn er tómur. 
Bættu þá restinni af spilunum við samsetta stokkinn. 

\end{enumerate}

Niðurstaðan ætti að vera einn raðaaður stokkur.
Svona lítur þetta út í sauðakóða:

\begin{verbatim}
  Deck merge (const Deck& d1, const Deck& d2) {
    // create a new deck big enough for all the cards
    Deck result (d1.cards.length() + d2.cards.length());

    // use the index i to keep track of where we are in
    // the first deck, and the index j for the second deck
    int i = 0;
    int j = 0;
		
    // the index k traverses the result deck
    for (int k = 0; k<result.cards.length(); k++) {
			
      // if d1 is empty, d2 wins; if d2 is empty, d1 wins;
      // otherwise, compare the two cards
			
      // add the winner to the new deck
    }
    return result;
  }
\end{verbatim}
%
Ég ákvað að gera {\tt merge} að falli sem stendur eitt og sér (e. nonmember function) vegna þess að viðföngin tvö eru samhverf (e. symmetric). 

Besta leiðin til að prófa {\tt merge} er að búa til stokk, nota subdeck til að búa til tvær (litlar) hendur
og nota síðan röðunaralgrímið úr síðasta kafla til að raða höndunum tveimur.
Eftir það getur þú sent hendurnar tvær (hlutstokkana) inn í {\tt merge} til að sjá hvort það virkar. 

\index{prófanir}

Ef þú færð þetta til að virka ættir þú að prófa eftirfarandi útfærslu af {\tt mergeSort}:

\begin{verbatim}
Deck Deck::mergeSort () const {
  // find the midpoint of the deck
  // divide the deck into two subdecks
  // sort the subdecks using sort
  // merge the two halves and return the result
}
\end{verbatim}
%
Taktu eftir því að núverandi hlutur er skilgreindur sem {\tt const} vegna þess að 
{\tt mergeSort} breytir honum ekki. Í staðinn býr fallið til nýjan {\tt Deck} hlut og skilar honum. 

Fjörið byrjar fyrst núna ef þetta gengur hjá þér! 
Það sem er töfrandi við mergesort er að fallið er endurkvæmt.
Af hverju ættir þú að kalla á gömlu, óskilvirku útgáfunum af {\tt sort} þegar þú ætlar að raða hlutstokkunum?
Af hverju ekki að kalla á {\tt mergeSort} sem þú ert einmitt að skrifa? 

\index{endurkvæmni}

Það er ekki einungis góð hugmynd heldur er það {\em nauðsynlegt} til að algrímið í heild sinni verði skilvirkara eins og ég lofaði.
Til að fá það til að virka verðum við þó að bæta við grunnþrepi þannig að ekki verði um óendanlega endurkvæmni að ræða.
Einfalt grunnþrep er hlutstokkur með 0 eða einu spili.
Ef {\tt mergesort} fær svona lítinn hlutstokk þá getur fallið einfaldlega skilað stokknum óbreyttum til baka því hann er þá sannarlega þegar raðaður.

Endurkvæma útgáfan af {\tt mergesort} lítur einhvern veginn svona út: 

\begin{verbatim}
Deck Deck::mergeSort (Deck deck) const {
  // if the deck is 0 or 1 cards, return it

  // find the midpoint of the deck
  // divide the deck into two subdecks
  // sort the subdecks using mergesort
  // merge the two halves and return the result
}
\end{verbatim}
%
Að venju eru tvær leiðir til að hugsa um endurkvæm föll:
Hægt er fylgja keyrsluflæð þess í heild sinni eða ``taka það trúanlegt'' (e. make the leap of faith).
Ég setti þetta dæmi fram vísvitandi til að hvetja þig til að ``taka það trúanlegt''.

\index{taka trúanlegt}

Þegar þú notaðir {\tt sort} til að raða hlutstokkum þá fannst þér ekki ástæða til að fylgja keyrsluflæði þess í heild sinni, ekki satt?
Þú gerðir einfaldlega ráð fyrir því að {\tt sort} fallið virkaði vegna þess að þú hafðir áður aflúsað það. 
Eina sem þú þurftir að breyta til að gera {\tt mergeSort} endurkvæmt var að skipta út einu röðunaralgrími fyrir annað. 
Það er engin ástæða til að líta á nýja forritið á annan hátt. 

Reyndar þurftir þú að hugsa um að ná grunnþrepinu réttu og að það myndi á einhverjum tímapunkti reyna á grunnþrepið en að öðru leyti ætti ekki að vera neitt sérstakt vandamál að skrifa endurkvæmu lausnina.  Gangi þér vel!

\section{Glossary}

\begin{description}

\item[pseudocode:]  A way of designing programs by writing
rough drafts in a combination of English and C++.

\item[helper function:]  Often a small function that does not
do anything enormously useful by itself, but which helps
another, more useful, function.

\item[bottom-up design:]  A method of program development that
uses pseudocode to sketch solutions to large problems and
design the interfaces of helper functions.

\item[mergesort:]  An algorithm for sorting a collection of
values.  Mergesort is faster than the simple algorithm in
the previous chapter, especially for large collections.

\index{pseudocode}
\index{helper function}
\index{bottom-up design}
\index{program development!bottom-up}
\index{function!helper}
\index{mergesort}


\end{description}


% LaTeX source for textbook ``How to think like a computer scientist''
% Copyright (C) 1999  Allen B. Downey

% This LaTeX source is free software; you can redistribute it and/or
% modify it under the terms of the GNU General Public License as
% published by the Free Software Foundation (version 2).

% This LaTeX source is distributed in the hope that it will be useful,
% but WITHOUT ANY WARRANTY; without even the implied warranty of
% MERCHANTABILITY or FITNESS FOR A PARTICULAR PURPOSE.  See the GNU
% General Public License for more details.

% Compiling this LaTeX source has the effect of generating
% a device-independent representation of a textbook, which
% can be converted to other formats and printed.  All intermediate
% representations (including DVI and Postscript), and all printed
% copies of the textbook are also covered by the GNU General
% Public License.

% This distribution includes a file named COPYING that contains the text
% of the GNU General Public License.  If it is missing, you can obtain
% it from www.gnu.org or by writing to the Free Software Foundation,
% Inc., 59 Temple Place - Suite 330, Boston, MA 02111-1307, USA.


\chapter{Classes and invariants}
\label{class}

\section{Private data and classes}
\index{private}
\index{class}
\index{data encapsulation}
\index{encapsulation!data}
\index{encapsulation!functional}

I have used the word ``encapsulation'' in this book to refer
to the process of wrapping up a sequence of instructions in
a function, in order to separate the function's interface (how
to use it) from its implementation (how it does what it does).

This kind of encapsulation might be called ``functional
encapsulation,'' to distinguish it from ``data encapsulation,'' which
is the topic of this chapter.  Data encapsulation is based on the idea
that each structure definition should provide a set of functions that
apply to the structure, and prevent unrestricted access to the
internal representation.

\index{interface}
\index{implementation}
\index{representation}

One use of data encapsulation is to hide implementation details 
from users or programmers that don't need to know them.

For example, there are many possible representations for a {\tt Card},
including two integers, two strings and two enumerated types.  The
programmer who writes the {\tt Card} member functions needs to
know which implementation to use, but
someone using
the {\tt Card} structure should not have to know anything about
its internal structure.

As another example, we have been using {\tt apstring} and
{\tt apvector} objects without ever discussing their implementations.
There are many possibilities, but as ``clients'' of these
libraries, we don't need to know.

\index{client programs}
\index{detail hiding}

In C++, the most common way to enforce data encapsulation is
to prevent client programs from accessing the instance variables
of an object.  The keyword {\tt private} is used to protect parts
of a structure definition.  For example, we could have written
the {\tt Card} definition:

\begin{verbatim}
struct Card
{
private:
  int suit, rank;

public:
  Card ();
  Card (int s, int r);

  int getRank () const { return rank; }
  int getSuit () const { return suit; }
  void setRank (int r) { rank = r; }
  void setSuit (int s) { suit = s; }
};
\end{verbatim}
%
There are two sections of this definition, a private part and
a public part.  The functions are public, which means that they
can be invoked by client programs.  The instance variables are
private, which means that they can be read and written only by
{\tt Card} member functions.

\index{accessor function}
\index{function!accessor}

It is still possible for client programs to read and
write the instance variables using the {\bf accessor functions}
(the ones beginning with {\tt get} and {\tt set}).
On the other hand, it is now easy to control which
operations clients can perform on which instance variables.
For example, it might be a good idea to make cards ``read only''
so that after they are constructed, they cannot be changed.
To do that, all we have to do is remove the {\tt set} functions.

Another advantage of using accessor functions is that we
can change the internal representations of cards without
having to change any client programs.

\section{What is a class?}
\index{class}
\index{struct}
\index{object-oriented programming}

In most object-oriented programming languages, a {\bf class} is
a user-defined type that includes a set of functions.  As
we have seen, structures in C++ meet the general definition of
a class.

But there is another feature in C++ that also meets this definition;
confusingly, it is called a {\tt class}.  In C++, a class
is just a structure whose instance variables are private by
default.  For example, I could have written the {\tt Card}
definition:

\begin{verbatim}
class Card
{
  int suit, rank;

public:
  Card ();
  Card (int s, int r);

  int getRank () const { return rank; }
  int getSuit () const { return suit; }
  int setRank (int r) { rank = r; }
  int setSuit (int s) { suit = s; }
};
\end{verbatim}
%
I replaced the word {\tt struct} with the word {\tt class} and
removed the {\tt private:} label.  This result of the two definitions
is exactly the same.

\index{public}
\index{private}

In fact, anything that can be written as a {\tt struct} can also
be written as a {\tt class}, just by adding or removing labels.
There is no real reason to choose one over the other, except that
as a stylistic choice, most C++ programmers use {\tt class}.

Also, it is common to refer to all user-defined types in C++ as
``classes,'' regardless of whether they are defined as a {\tt struct}
or a {\tt class}.

\section{Complex numbers}
\index{complex number}
\index{Complex}
\index{class!Complex}
\index{arithmetic!complex}

As a running example for the rest of this chapter we will consider a
class definition for complex numbers.  Complex numbers are useful for
many branches of mathematics and engineering, and many computations
are performed using complex arithmetic.  A complex number is the sum
of a real part and an imaginary part, and is usually written in the
form $x + yi$, where $x$ is the real part, $y$ is the imaginary part,
and $i$ represents the square root of -1.

The following is a class definition for a user-defined type called
{\tt Complex}:

\begin{verbatim}
class Complex
{
  double real, imag;

public:
  Complex () { }
  Complex (double r, double i) { real = r;  imag = i; }
};
\end{verbatim}
%
Because this is a {\tt class} definition, the instance variables
{\tt real} and {\tt imag} are private, and we have to include
the label {\tt public:} to allow client code to invoke the
constructors.

As usual, there are two constructors: one takes no parameters and does
nothing; the other takes two parameters and uses them to initialize
the instance variables.

\index{instance variable}
\index{variable!instance}
\index{constructor}

So far there is no real advantage to making the instance
variables private.  Let's make things a little more complicated;
then the point might be clearer.

\index{coordinate}
\index{coordinate!Cartesian}
\index{coordinate!polar}
\index{Cartesian coordinate}
\index{polar coordinate}

There is another common representation for complex numbers that is
sometimes called ``polar form'' because it is based on polar
coordinates.  Instead of specifying the real part and the imaginary
part of a point in the complex plane, polar coordinates specify the
direction (or angle) of the point relative to the origin, and
the distance (or magnitude) of the point.  

The following figure shows the two coordinate systems graphically.

\vspace {0.1in}
\centerline{\epsfig{figure=coordinates.eps}}
\vspace {0.1in}

Complex numbers in polar coordinates are written $r e^{i \theta}$,
where $r$ is the magnitude (radius), and $\theta$ is the angle in
radians.

Fortunately, it is easy to convert from one form to another.
To go from Cartesian to polar,

\begin{eqnarray*}
r       & = &  \sqrt{x^2 + y^2} \\
\theta  & = &  \arctan (y / x)
\end{eqnarray*}

To go from polar to Cartesian,

\begin{eqnarray*}
x       & = &  r \cos \theta \\
y       & = &  r \sin \theta
\end{eqnarray*}

So which representation should we use?  Well, the whole reason there
are multiple representations is that some operations are easier to
perform in Cartesian coordinates (like addition), and others are
easier in polar coordinates (like multiplication).  One option is that
we can write a class definition that uses {\em both} representations,
and that converts between them automatically, as needed.

\begin{verbatim}
class Complex
{
  double real, imag;
  double mag, theta;
  bool cartesian, polar;

public:
  Complex () { cartesian = false;  polar = false; }

  Complex (double r, double i)
  {
    real = r;  imag = i;
    cartesian = true;  polar = false;
  }
};
\end{verbatim}
%
There are now six instance variables, which means that
this representation will take up more space than either
of the others, but we will see that it is very versatile.

\index{instance variable}
\index{variable!instance}

Four of the instance variables are self-explanatory.  They
contain the real part, the imaginary part, the angle and
the magnitude of the complex number.  The other two
variables, {\tt cartesian} and {\tt polar} are flags that
indicate whether the corresponding values are currently
valid.

\index{flag}
\index{constructor}

For example, the do-nothing constructor sets both flags
to false to indicate that this object does not contain
a valid complex number (yet), in either representation.

The second constructor uses the parameters to initialize
the real and imaginary parts, but it does not calculate the
magnitude or angle.  Setting the {\tt polar} flag to false
warns other functions not to access {\tt mag} or {\tt theta}
until they have been set.

Now it should be clearer why we need to keep the instance
variables private.  If client programs were allowed unrestricted
access, it would be easy for them to make errors by reading
uninitialized values.  In the next few sections, we will
develop accessor functions that will make those kinds of mistakes
impossible.

\section{Accessor functions}
\index{accessor function}
\index{function!accessor}

By convention, accessor functions have names that
begin with {\tt get} and end with the name of the instance
variable they fetch.  The return type, naturally, is the type
of the corresponding instance variable.

\index{flag}

In this case, the accessor functions give us an opportunity
to make sure that the value of the variable is valid before
we return it.  Here's what {\tt getReal} looks like:

\begin{verbatim}
double Complex::getReal ()
{
  if (cartesian == false) calculateCartesian ();
  return real;
}
\end{verbatim}
%
If the {\tt cartesian} flag is true then {\tt real} contains
valid data, and we can just return it.  Otherwise, we have
to call {\tt calculateCartesian} to convert from polar coordinates
to Cartesian coordinates:

\begin{verbatim}
void Complex::calculateCartesian ()
{
  real = mag * cos (theta);
  imag = mag * sin (theta);
  cartesian = true;
}
\end{verbatim}
%
Assuming that the polar coordinates are valid, we
can calculate the Cartesian coordinates using the formulas
from the previous section.  Then we
set the {\tt cartesian} flag, indicating that {\tt real}
and {\tt imag} now contain valid data.

As an exercise, write a corresponding function called
{\tt calculatePolar} and then write {\tt getMag}
and {\tt getTheta}.  One unusual thing about these
accessor functions is that they are not {\tt const},
because invoking them might modify the instance variables.

\section{Output}
\index{output}

As usual when we define a new class, we want to be able to
output objects in a human-readable form.  For {\tt Complex}
objects, we could use two functions:

\begin{verbatim}
void Complex::printCartesian ()
{
  cout << getReal() << " + " << getImag() << "i" << endl;
}

void Complex::printPolar ()
{
  cout << getMag() << " e^ " << getTheta() << "i" << endl;
}
\end{verbatim}
%
The nice thing here is that we can output any {\tt Complex} object in
either format without having to worry about the representation.  Since
the output functions use the accessor functions, the program
will compute automatically any values that are needed.

The following code creates a {\tt Complex} object using the
second constructor.   Initially, it is in Cartesian format only.
When we invoke {\tt printCartesian} it accesses {\tt real} and
{\tt imag} without having to do any conversions.

\begin{verbatim}
  Complex c1 (2.0, 3.0);

  c1.printCartesian();
  c1.printPolar();
\end{verbatim}
%
When we invoke {\tt printPolar}, and {\tt printPolar} invokes
{\tt getMag}, the program is forced to convert to polar
coordinates and store the results in the instance variables.
The good news is that we only have to do the conversion
once.  When {\tt printPolar} invokes {\tt getTheta}, it will
see that the polar coordinates are valid and return {\tt theta}
immediately.

The output of this code is:

\begin{verbatim}
2 + 3i
3.60555 e^ 0.982794i
\end{verbatim}

\section{A function on {\tt Complex} numbers}
\index{pure function}

A natural operation we might want to perform on complex numbers is
addition.  If the numbers are in Cartesian coordinates, addition is
easy: you just add the real parts together and the imaginary parts
together.  If the numbers are in polar coordinates, it is easiest to
convert them to Cartesian coordinates and then add them.

Again, it is easy to deal with these cases if we use
the accessor functions:

\begin{verbatim}
Complex add (Complex& a, Complex& b)
{
  double real = a.getReal() + b.getReal();
  double imag = a.getImag() + b.getImag();
  Complex sum (real, imag);
  return sum;
}
\end{verbatim}
%
Notice that the arguments to {\tt add} are not {\tt const}
because they might be modified when we invoke the accessors.
To invoke this function, we would pass both operands as arguments:

\begin{verbatim}
  Complex c1 (2.0, 3.0);
  Complex c2 (3.0, 4.0);

  Complex sum = add (c1, c2);
  sum.printCartesian();
\end{verbatim}
%
The output of this program is

\begin{verbatim}
5 + 7i
\end{verbatim}
%


\section{Another function on {\tt Complex} numbers}
\index{pure function}

Another operation we might want is multiplication.  Unlike
addition, multiplication is easy if the numbers are in polar
coordinates and hard if they are in Cartesian coordinates
(well, a little harder, anyway).

In polar coordinates, we can just multiply the magnitudes and
add the angles.  As usual, we can use the accessor functions
without worrying about the representation of the objects.

\begin{verbatim}
Complex mult (Complex& a, Complex& b)
{
  double mag = a.getMag() * b.getMag()
  double theta = a.getTheta() + b.getTheta();
  Complex product;
  product.setPolar (mag, theta);
  return product;
}
\end{verbatim}
%
A small problem we encounter here is that we have no constructor
that accepts polar coordinates.  It would be nice to write one,
but remember that we can only overload a function (even a
constructor) if the different versions take different parameters.
In this case, we would like a second constructor that also takes
two {\tt double}s, and we can't have that.

An alternative it to provide an accessor function that {\em sets}
the instance variables.  In order to do that properly, though,
we have to make sure that when {\tt mag} and {\tt theta} are set,
we also set the {\tt polar} flag.  At the same time, we have to
make sure that the {\tt cartesian} flag is unset.  That's because
if we change the polar coordinates, the cartesian coordinates are
no longer valid.

\begin{verbatim}
void Complex::setPolar (double m, double t)
{
  mag = m;  theta = t;
  cartesian = false;  polar = true;
}
\end{verbatim}
%
As an exercise, write the corresponding function named
{\tt setCartesian}.

To test the {\tt mult} function, we can try something like:

\begin{verbatim}
  Complex c1 (2.0, 3.0);
  Complex c2 (3.0, 4.0);

  Complex product = mult (c1, c2);
  product.printCartesian();
\end{verbatim}
%
The output of this program is

\begin{verbatim}
-6 + 17i
\end{verbatim}
%
There is a lot of conversion going on in this program behind the
scenes.  When we call {\tt mult}, both arguments get converted to
polar coordinates.  The result is also in polar format, so when we
invoke {\tt printCartesian} it has to get converted back.  Really,
it's amazing that we get the right answer!


\section{Invariants}
\index{invariant}

There are several conditions we expect to be true for a proper
{\tt Complex} object.  For example, if the {\tt cartesian} flag
is set then we expect {\tt real} and {\tt imag} to contain valid
data.  Similarly, if {\tt polar} is set, we expect {\tt mag}
and {\tt theta} to be valid.  Finally, if both flags are set
then we expect the other four variables to be consistent;
that is, they should be specifying the same point in two different
formats.

These kinds of conditions are called {\tt invariants}, for the obvious
reason that they do not vary---they are always supposed to be true.
One of the ways to write good quality code that contains few bugs
is to figure out what invariants are appropriate for your classes,
and write code that makes it impossible to violate them.

\index{data encapsulation}
\index{encapsulation!data}

One of the primary things that data encapsulation is good for
is helping to enforce invariants.  The first step is to prevent
unrestricted access to the instance variables by making them
private.  Then the only way to modify the object is through
accessor functions and modifiers.  If we examine all the accessors
and modifiers, and we can show that every one of them maintains
the invariants, then we can prove that it is impossible for
an invariant to be violated.

Looking at the {\tt Complex} class, we can list the functions
that make assignments to one or more instance variables:

\begin{verbatim}
the second constructor
calculateCartesian
calculatePolar
setCartesian
setPolar
\end{verbatim}
%
In each case, it is straightforward to show that the function
maintains each of the invariants I listed.  We have to be a little
careful, though.  Notice that I said ``maintain'' the invariant.
What that means is ``If the invariant is true when the function
is called, it will still be true when the function is complete.''

That definition allows two loopholes.  First, there may be some
point in the middle of the function when the invariant is not
true.  That's ok, and in some cases unavoidable.  As long as the
invariant is restored by the end of the function, all is well.

The other loophole is that we only have to maintain the invariant
if it was true at the beginning of the function.  Otherwise, all
bets are off.  If the invariant was violated somewhere else in
the program, usually the best we can do is detect the error,
output an error message, and exit.

\section{Preconditions}
\index{precondition}
\index{postcondition}

Often when you write a function you make implicit assumptions
about the parameters you receive.  If those assumptions turn
out to be true, then everything is fine; if not, your program
might crash.

To make your programs more robust, it is a good idea to think
about your assumptions explicitly, document them as part of the
program, and maybe write code that checks them.

For example, let's take another look at {\tt calculateCartesian}.
Is there an assumption we make about the current object?  Yes,
we assume that the {\tt polar} flag is set and that {\tt mag}
and {\tt theta} contain valid data.  If that is not true, then
this function will produce meaningless results.

One option is to add a comment to the function that warns
programmers about the {\bf precondition}.

\begin{verbatim}
void Complex::calculateCartesian ()
// precondition: the current object contains valid polar coordinates
	and the polar flag is set
// postcondition: the current object will contain valid Cartesian
	coordinates and valid polar coordinates, and both the cartesian
	flag and the polar flag will be set
{
  real = mag * cos (theta);
  imag = mag * sin (theta);
  cartesian = true;
}
\end{verbatim}
%
At the same time, I also commented on the {\bf postconditions},
the things we know will be true when the function completes.

These comments are useful for people reading your programs, but
it is an even better idea to add code that {\em checks} the
preconditions, so that we can print an appropriate error message:

\begin{verbatim}
void Complex::calculateCartesian ()
{
  if (polar == false) {
    cout <<
    "calculateCartesian failed because the polar representation is invalid"
	 << endl;
    exit (1);
  }
  real = mag * cos (theta);
  imag = mag * sin (theta);
  cartesian = true;
}
\end{verbatim}
%
The {\tt exit} function causes the program to quit immediately.  The
return value is an error code that tells the system (or whoever
executed the program) that something went wrong.

\index{exit}
\index{assert}
\index{run-time error}

This kind of error-checking is so common that C++ provides
a built-in function to check preconditions and print error messages.
If you include the {\tt assert.h} header file, you get a function
called {\tt assert} that takes a boolean value (or a conditional
expression) as an argument.  As long as the argument is true,
{\tt assert} does nothing.  If the argument is false, assert
prints an error message and quits.  Here's how to use it:

\begin{verbatim}
void Complex::calculateCartesian ()
{
  assert (polar);
  real = mag * cos (theta);
  imag = mag * sin (theta);
  cartesian = true;
  assert (polar && cartesian);
}
\end{verbatim}
%
The first {\tt assert} statement checks the precondition
(actually just part of it); the second {\tt assert} statement
checks the postcondition.

In my development environment, I get the following message
when I violate an assertion:

\begin{verbatim}
Complex.cpp:63: void Complex::calculatePolar(): Assertion `cartesian' failed.
Abort
\end{verbatim}
%
There is a lot of information here to help me track down the error,
including the file name and line number of the assertion that
failed, the function name and the contents of the assert statement.


\section{Private functions}
\index{private!function}

In some cases, there are member functions that are used internally
by a class, but that should not be invoked by client programs.
For example, {\tt calculatePolar} and {\tt calculateCartesian}
are used by the accessor functions, but there is probably no
reason clients should call them directly (although it would not
do any harm).  If we wanted to protect these functions, we
could declare them {\tt private} the same way we do with instance
variables.  In that case the complete class definition for
{\tt Complex} would look like:

\begin{verbatim}
class Complex
{
private:
  double real, imag;
  double mag, theta;
  bool cartesian, polar;

  void calculateCartesian ();
  void calculatePolar ();

public:
  Complex () { cartesian = false;  polar = false; }

  Complex (double r, double i)
  {
    real = r;  imag = i;
    cartesian = true;  polar = false;
  }

  void printCartesian ();
  void printPolar ();

  double getReal ();
  double getImag ();
  double getMag ();
  double getTheta ();

  void setCartesian (double r, double i);
  void setPolar (double m, double t);
};
\end{verbatim}
%
The {\tt private} label at the beginning is not necessary,
but it is a useful reminder.

\section{Glossary}

\begin{description}

\item[class:]  In general use, a class is a user-defined type
with member functions.  In C++, a class is a structure with
private instance variables.

\item[accessor function:]  A function that provides access
(read or write) to a private instance variable.

\item[invariant:]  A condition, usually pertaining to an object, that
should be true at all times in client code, and that should be
maintained by all member functions.

\item[precondition:]  A condition that is assumed to be true at
the beginning of a function.  If the precondition is not true, the
function may not work.  It is often a good idea for functions to
check their preconditions, if possible.

\item[postcondition:]  A condition that is true at the end of a
function. 

\index{class}
\index{accessor function}
\index{invariant}
\index{precondition}
\index{postcondition}

\end{description}


% LaTeX source for textbook ``How to think like a computer scientist''
% Copyright (C) 1999  Allen B. Downey

% This LaTeX source is free software; you can redistribute it and/or
% modify it under the terms of the GNU General Public License as
% published by the Free Software Foundation (version 2).

% This LaTeX source is distributed in the hope that it will be useful,
% but WITHOUT ANY WARRANTY; without even the implied warranty of
% MERCHANTABILITY or FITNESS FOR A PARTICULAR PURPOSE.  See the GNU
% General Public License for more details.

% Compiling this LaTeX source has the effect of generating
% a device-independent representation of a textbook, which
% can be converted to other formats and printed.  All intermediate
% representations (including DVI and Postscript), and all printed
% copies of the textbook are also covered by the GNU General
% Public License.

% This distribution includes a file named COPYING that contains the text
% of the GNU General Public License.  If it is missing, you can obtain
% it from www.gnu.org or by writing to the Free Software Foundation,
% Inc., 59 Temple Place - Suite 330, Boston, MA 02111-1307, USA.


\chapter{File Input/Output and {\tt apmatrix}es}

In this chapter we will develop a program that reads and writes files,
parses input, and demonstrates the {\tt apmatrix} class.  We will also
implement a data structure called {\tt Set} that expands automatically
as you add elements.

Aside from demonstrating all these features, the real purpose of the
program is to generate a two-dimensional table of
the distances between cities in the United States.
The output is a table that looks like this:

\begin{verbatim}
Atlanta 0
Chicago 700     0
Boston  1100    1000    0
Dallas  800     900     1750    0
Denver  1450    1000    2000    800     0
Detroit 750     300     800     1150    1300    0
Orlando 400     1150    1300    1100    1900    1200    0
Phoenix 1850    1750    2650    1000    800     2000    2100    0
Seattle 2650    2000    3000    2150    1350    2300    3100    1450    0
        Atlanta Chicago Boston  Dallas  Denver  Detroit Orlando Phoenix Seattle
\end{verbatim}
%
The diagonal elements are all zero because that is the distance
from a city to itself.  Also, because
the distance from A to B is the same as the distance from B
to A, there is no need to print the top half of the matrix.

\section {Streams}
\index{stream}

To get input from a file or send output to a file, you have to
create an {\tt ifstream} object (for input files) or an
{\tt ofstream} object (for output files).  These objects
are defined in the header file {\tt fstream}, which you
have to include.

\index{header file}

A {\bf stream} is an abstract object that represents the flow
of data from a source like the keyboard or a file to a destination
like the screen or a file.

We have already worked with two streams: {\tt cin}, which has type
{\tt istream}, and {\tt cout}, which has type {\tt ostream}.
{\tt cin} represents the flow of data from the keyboard to
the program.  Each time the program uses the {\tt >>} operator
or the {\tt getline} function, it removes a piece of data
from the input stream.

\index{cin}
\index{cout}
\index{istream}
\index{ostream}

Similarly, when the program uses the {\tt <<} operator on
an {\tt ostream}, it adds a datum to the outgoing stream.

\section {File input}
\label{finput}
\index{file!input}
\index{ifstream}

To get data from a file, we have to create a stream that flows
from the file into the program.  
We can do that using the {\tt ifstream} constructor.

\begin{verbatim}
  ifstream infile ("file-name");
\end{verbatim}
%
The argument for this constructor is a string that
contains the name of the file you want to open.  The result
is an object named {\tt infile} that supports all the same
operations as {\tt cin}, including {\tt >>} and {\tt getline}.

\begin{verbatim}
  int x;
  apstring line;
    
  infile >> x;               // get a single integer and store in x
  getline (infile, line);    // get a whole line and store in line
\end{verbatim}
%
If we know ahead of time how much data is in a file, it is 
straightforward to write a loop that reads the entire file and
then stops.  More often, though, we want to read the entire
file, but don't know how big it is.

There are member functions for {\tt ifstreams} that check the status
of the input stream; they are called {\tt good}, {\tt eof}, {\tt fail}
and {\tt bad}.  We will use {\tt good} to make sure the file was
opened successfully and {\tt eof} to detect the ``end of file.''

\index{stream!status}
\index{good}
\index{eof}
\index{end of file}

Whenever you get data from an input stream, you don't
know whether the attempt succeeded until you check.  If the
return value from {\tt eof} is {\tt true} then we have reached
the end of the file and we know that the last attempt failed.
Here is a program that reads lines from a file and displays
them on the screen:

\begin{verbatim}
  apstring fileName = ...;
  ifstream infile (fileName.c_str());

  if (infile.good() == false) {
    cout << "Unable to open the file named " << fileName;
    exit (1);
  }

  while (true) {
    getline (infile, line);
    if (infile.eof()) break;
    cout << line << endl;
  }
\end{verbatim}
%
The function {\tt c\_str} converts an {\tt apstring} to a
native C string.  Because the {\tt ifstream} constructor
expects a C string as an argument, we have to convert
the {\tt apstring}.

\index{c\_str}
\index{C string}
\index{string!native C}

Immediately after opening the file, we invoke the {\tt good} function.
The return value is {\tt false} if the system could not open the file,
most likely because it does not exist, or you do not have permission
to read it.

\index{loop!infinite}
\index{infinite loop}

The statement {\tt while(true)} is an idiom for an infinite
loop.  Usually there will be a {\tt break} statement somewhere in
the loop so that the program does not really run forever (although
some programs do).  In this case, the {\tt break} statement allows
us to exit the loop as soon as we detect the end of file.

\index{break statement}
\index{statement!break}
\index{getline}

It is important to exit the loop between the input statement and
the output statement, so that when {\tt getline} fails at the
end of the file, we do not output the invalid data in {\tt line}.

\section{File output}
\index{file output}
\index{ofstream}

Sending output to a file is similar.  For example, we could
modify the previous program to copy lines from one file to
another.

\begin{verbatim}
  ifstream infile ("input-file");
  ofstream outfile ("output-file");

  if (infile.good() == false || outfile.good() == false) {
    cout << "Unable to open one of the files." << endl;
    exit (1);
  }

  while (true) {
    getline (infile, line);
    if (infile.eof()) break;
    outfile << line << endl;
  }
\end{verbatim}

\section{Parsing input}
\label{parsing}
\index{parsing}

In Section~\ref{formal} I defined ``parsing'' as the process of
analyzing the structure of a sentence in a natural language or a
statement in a formal language.  For example, the compiler has to
parse your program before it can translate it into machine language.

In addition, when you read input from a file or from the keyboard
you often have to parse it in order to extract the information
you want and detect errors.

For example, I have a file called {\tt distances} that contains
information about the distances between major cities in the
United States.  I got this information from a randomly-chosen
web page

\begin{verbatim}
http://www.jaring.my/usiskl/usa/distance.html
\end{verbatim}
%
so it may be wildly inaccurate, but that doesn't matter.  The
format of the file looks like this:

\begin{verbatim}
"Atlanta"       "Chicago"       700
"Atlanta"       "Boston"        1,100
"Atlanta"       "Chicago"       700
"Atlanta"       "Dallas"        800
"Atlanta"       "Denver"        1,450
"Atlanta"       "Detroit"       750
"Atlanta"       "Orlando"       400
\end{verbatim}
%
Each line of the file contains the names of two cities in quotation
marks and the distance between them in miles.  The quotation marks
are useful because they make it easy to deal with names that have
more than one word, like ``San Francisco.''

By searching for the quotation marks in a line of input, we
can find the beginning and end of each city name.
Searching for special characters like quotation marks can be a little
awkward, though, because the quotation mark is a special character
in C++, used to identify string values.

If we want to find the
first appearance of a quotation mark, we have to write something
like:

\begin{verbatim}
  int index = line.find ('\"');
\end{verbatim}
%
The argument here looks like a mess, but it represents a single
character, a double quotation mark.  The outermost single-quotes
indicate that this is a character value, as usual.  The backslash
(\verb+\+) indicates that we want to treat the next character
literally.  The sequence \verb+\"+ represents a quotation mark; the
sequence \verb+\'+ represents a single-quote.  Interestingly, the
sequence \verb+\\+ represents a single backslash.  The first backslash
indicates that we should take the second backslash seriously.

\index{special character}
\index{character!special sequence}
\index{backslash}

Parsing input lines consists of finding the beginning and
end of each city name and using
the {\tt substr} function to extract the cities and distance.
{\tt substr} is an {\tt apstring} member function;
it takes two arguments, the starting index of the substring
and the length.

\index{find}

\begin{verbatim}
void processLine (const apstring& line)
{
  // the character we are looking for is a quotation mark
  char quote = '\"';

  // store the indices of the quotation marks in a vector
  apvector<int> quoteIndex (4);

  // find the first quotation mark using the built-in find
  quoteIndex[0] = line.find (quote);

  // find the other quotation marks using the find from Chapter 7
  for (int i=1; i<4; i++) {
    quoteIndex[i] = find (line, quote, quoteIndex[i-1]+1);
  }

  // break the line up into substrings
  int len1 = quoteIndex[1] - quoteIndex[0] - 1;
  apstring city1 = line.substr (quoteIndex[0]+1, len1);
  int len2 = quoteIndex[3] - quoteIndex[2] - 1;
  apstring city2 = line.substr (quoteIndex[2]+1, len2);
  int len3 = line.length() - quoteIndex[2] - 1;
  apstring distString = line.substr (quoteIndex[3]+1, len3);

  // output the extracted information
  cout << city1 << "\t" << city2 << "\t" << distString << endl;
}
\end{verbatim}
%
Of course, just displaying the extracted information is not
exactly what we want, but it is a good starting place.

\section{Parsing numbers}
\index{parsing number}
\index{atoi}
\index{convert!to integer}

The next task is to convert the numbers in the file from strings to
integers.  When people write large numbers, they often use commas to
group the digits, as in 1,750.  Most of the time when computers write
large numbers, they don't include commas, and the built-in functions
for reading numbers usually can't handle them.  That makes the
conversion a little more difficult, but it also provides an
opportunity to write a comma-stripping function, so that's ok.  Once
we get rid of the commas, we can use the library function {\tt atoi}
to convert to integer.  {\tt atoi} is defined in the header file {\tt
cstdlib}.

\index{character!classification}
\index{isdigit}

To get rid of the commas, one option is to traverse the string and
check whether each character is a digit.  If so, we add it to the
result string.  At the end of the loop, the result string contains all
the digits from the original string, in order.

\begin{verbatim}
int convertToInt (const apstring& s)
{
  apstring digitString = "";

  for (int i=0; i<s.length(); i++) {
    if (isdigit (s[i])) {
      digitString += s[i];
    }
  }
  return atoi (digitString.c_str());
}
\end{verbatim}
%
The variable {\tt digitString} is an example of an {\bf accumulator}.  It is
similar to the counter we saw in Section~\ref{loopcount},
except that instead of getting incremented, it gets accumulates
one new character at a time, using string concatentation.

The expression

\begin{verbatim}
      digitString += s[i];
\end{verbatim}
%
is equivalent to

\begin{verbatim}
      digitString = digitString + s[i];
\end{verbatim}
%
Both statements add a single character onto the end of the existing
string.

\index{concatentation}
\index{string!concatentation}
\index{accumulator}
\index{pattern!accumulator}

Since {\tt atoi} takes a C string as a parameter, we have
to convert {\tt digitString} to a C string before passing it
as an argument.

\section{The {\tt Set} data structure}
\index{Set}
\index{data structure}

A data structure is a container for grouping a collection
of data into a single object.  We have seen some examples already,
including {\tt apstring}s, which are collections of characters,
and {\tt apvector}s which are collections on any type.

An ordered set is a collection of items with two defining
properties:

\begin{description}

\item[Ordering:] The elements of the set have indices associated
with them.  We can use these indices to identify elements of the set.

\item[Uniqueness:] No element appears in the set more than once.
If you try to add an element to a set, and it already exists, there
is no effect.

\end{description}

In addition, our implementation of an ordered set will have the
following property:

\begin{description}

\item[Arbitrary size:] As we add elements to the set, it expands
to make room for new elements.

\end{description}

Both {\tt apstring}s and {\tt apvector}s have an ordering; every
element has an index we can use to identify it.  Both none of
the data structures we have seen so far have the properties of
uniqueness or arbitrary size.

\index{ordering}

To achieve uniqueness, we have to write an {\tt add} function
that searches the set to see if it already exists.  To make the
set expand as elements are added, we can take advantage of the
{\tt resize} function on {\tt apvector}s.

Here is the beginning of a class definition for a {\tt Set}.

\begin{verbatim}
class Set {
private:
  apvector<apstring> elements;
  int numElements;

public:
  Set (int n);

  int getNumElements () const;
  apstring getElement (int i) const;
  int find (const apstring& s) const;
  int add (const apstring& s);
};

Set::Set (int n)
{
  apvector<apstring> temp (n);
  elements = temp;
  numElements = 0;
}
\end{verbatim}
%
The instance variables are an {\tt apvector} of strings and an
integer that keeps track of how many elements there are in the
set.  Keep in mind that the number of elements in the
set, {\tt numElements}, is not the same thing as the size
of the {\tt apvector}.  Usually it will be smaller.

\index{constructor}

The {\tt Set} constructor takes a single parameter, which is
the initial size of the {\tt apvector}.  The initial number
of elements is always zero.

{\tt getNumElements} and {\tt getElement} are accessor functions
for the instance variables, which are private.  {\tt numElements}
is a read-only variable, so we provide a {\tt get} function
but not a {\tt set} function.

\begin{verbatim}
int Set::getNumElements () const
{
  return numElements;
}
\end{verbatim}
%
Why do we have to prevent client programs from changing {\tt
getNumElements}?  What are the invariants for this type, and
how could a client program break an invariant.  As we look
at the rest of the {\tt Set} member function, see if you can
convince yourself that they all maintain the invariants.

\index{data encapsulation}
\index{encapsulation!data}

When we use the {\tt []} operator to access the {\tt apvector},
it checks to make sure the index is greater than or equal to zero
and less than the length of the {\tt apvector}.  To access the
elements of a set, though, we need to check a stronger condition.
The index has to be less than the number of elements, which 
might be smaller than the length of the {\tt apvector}.

\begin{verbatim}
apstring Set::getElement (int i) const
{
  if (i < numElements) {
    return elements[i];
  } else {
    cout << "Set index out of range." << endl;
    exit (1);
  }
}
\end{verbatim}
%
If {\tt getElement} gets an index that is out of range, it prints
an error message (not the most useful message, I admit), and
exits.

\index{run-time error}

The interesting functions are {\tt find} and {\tt add}.  By
now, the pattern for traversing and searching should be old
hat:

\begin{verbatim}
int Set::find (const apstring& s) const
{
  for (int i=0; i<numElements; i++) {
    if (elements[i] == s) return i;
  }
  return -1;
}
\end{verbatim}
%
So that leaves us with {\tt add}.  Often the return type for
something like {\tt add} would be void, but in this case it
might be useful to make it return the index of the element.

\begin{verbatim}
int Set::add (const apstring& s)
{
  // if the element is already in the set, return its index
  int index = find (s);
  if (index != -1) return index;

  // if the apvector is full, double its size
  if (numElements == elements.length()) {
    elements.resize (elements.length() * 2);
  }

  // add the new elements and return its index
  index = numElements;
  elements[index] = s;
  numElements++;
  return index;
}
\end{verbatim}
%
The tricky thing here is that {\tt numElements} is used in
two ways.  It is the number of elements in the set, of course,
but it is also the index of the next element to be added.

It takes a minute to convince yourself that that works, but
consider this: when the number of elements is zero, the index
of the next element is 0.  When the number of elements is
equal to the length of the {\tt apvector}, that means that the
vector is full, and we have to allocate more space (using
{\tt resize}) before we can add the new element.

\index{state diagram}

Here is a state diagram showing a {\tt Set} object that
initially contains space for 2 elements.

\vspace {0.1in}
\centerline{\epsfig{figure=set.eps,width=6in}}
\vspace {0.1in}

Now we can use the {\tt Set} class to keep track of the cities
we find in the file.  In {\tt main} we create the {\tt Set} with
an initial size of 2:

\begin{verbatim}
  Set cities (2);
\end{verbatim}
%
Then in {\tt processLine} we add both cities to the {\tt Set}
and store the index that gets returned.

\begin{verbatim}
  int index1 = cities.add (city1);
  int index2 = cities.add (city2);
\end{verbatim}
%
I modified {\tt processLine} to take the {\tt cities} object
as a second parameter.

\section {{\tt apmatrix}}
\index{matrix}
\index{apmatrix}

An {\tt apmatrix} is similar to an {\tt apvector} except it
is two-dimensional.  Instead of a length, it has two
dimensions, called {\tt numrows} and {\tt numcols}, for
``number of rows'' and ``number of columns.''

Each element in the matrix is indentified by two indices;
one specifies the row number, the other the column number.

\index{index}

To create a matrix, there are four constructors:

\begin{verbatim}
  apmatrix<char> m1;
  apmatrix<int> m2 (3, 4);
  apmatrix<double> m3 (rows, cols, 0.0);
  apmatrix<double> m4 (m3);
\end{verbatim}
%
The first is a do-nothing constructor that makes a matrix with both
dimensions 0.  The second takes two integers, which are the initial
number of rows and columns, in that order.  The third is the same as
the second, except that it takes an additional parameter that is used
to initialized the elements of the matrix.  The fourth is a copy
constructor that takes another {\tt apmatrix} as a parameter.

\index{constructor}

Just as with {\tt apvectors}, we can make {\tt apmatrix}es with any
type of elements (including {\tt apvector}s, and even {\tt
apmatrix}es).

To access the elements of a matrix, we use the {\tt []} operator
to specify the row and column:

\begin{verbatim}
  m2[0][0] = 1;
  m3[1][2] = 10.0 * m2[0][0];
\end{verbatim}
%
If we try to access an element that is out of range, the program
prints an error message and quits.

\index{run-time error}

The {\tt numrows} and {\tt numcols} functions get the number of
rows and columns.  Remember that the row indices run from 0 to
{\tt numrows() -1} and the column indices run from 0 to
{\tt numcols() -1}.

\index{loop!nested}

The usual way to traverse a matrix is with a nested loop.
This loop sets each element of the matrix to the sum of its
two indices:

\begin{verbatim}
  for (int row=0; row < m2.numrows(); row++) {
    for (int col=0; col < m2.numcols(); col++) {
      m2[row][col] = row + col;
    }
  }
\end{verbatim}
%
This loop prints each row of the matrix with tabs between the
elements and newlines between the rows:

\begin{verbatim}
  for (int row=0; row < m2.numrows(); row++) {
    for (int col=0; col < m2.numcols(); col++) {
      cout << m2[row][col] << "\t";
    }
    cout << endl;
  }
\end{verbatim}
%

\section{A distance matrix}

Finally, we are ready to put the data from the file into
a matrix.  Specifically, the matrix will have one row and
one column for each city.

We'll create the matrix in {\tt main}, with plenty of space
to spare:

\begin{verbatim}
  apmatrix<int> distances (50, 50, 0);
\end{verbatim}
%

Inside {\tt processLine}, we add new information to the
matrix by getting the indices of the two cities from the
{\tt Set} and using them as matrix indices:

\begin{verbatim}
  int dist = convertToInt (distString);
  int index1 = cities.add (city1);
  int index2 = cities.add (city2);

  distances[index1][index2] = distance;
  distances[index2][index1] = distance;
\end{verbatim}
%
Finally, in {\tt main} we can print the information in a
human-readable form:

\begin{verbatim}
  for (int i=0; i<cities.getNumElements(); i++) {
    cout << cities.getElement(i) << "\t";

    for (int j=0; j<=i; j++) {
      cout << distances[i][j] << "\t";
    }
    cout << endl;
  }

  cout << "\t";
  for (int i=0; i<cities.getNumElements(); i++) {
    cout << cities.getElement(i) << "\t";
  }
  cout << endl;
\end{verbatim}
%
This code produces the output shown at the beginning of the
chapter.  The original data is available from this book's web page.

\section{A proper distance matrix}

Although this code works, it is not as well organized as it
should be.  Now that we have written a prototype, we are in a
good position to evaluate the design and improve it.

What are some of the problems with the existing code?

\begin{enumerate}

\item We did not know ahead of time how big to make the distance
matrix, so we chose an arbitrary large number (50) and made it
a fixed size.  It would be better to allow the distance matrix
to expand in the same way a {\tt Set} does.  The {\tt apmatrix}
class has a function called {\tt resize} that makes this possible.

\index{resize}

\item The data in the distance matrix is not well-encapsulated.
We have to pass the set of city names and the matrix itself
as arguments to {\tt processLine}, which is awkward.  Also,
use of the distance matrix is error prone because we have not
provided accessor functions that perform error-checking.
It might be a good idea to take the {\tt Set} of city names
and the {\tt apmatrix} of distances, and combine them into a
single object called a {\tt DistMatrix}.

\end{enumerate}

Here is a draft of what the header for a {\tt DistMatrix}
might look like:

\begin{verbatim}
class DistMatrix {
private:
  Set cities;
  apmatrix<int> distances;

public:
  DistMatrix (int rows);

  void add (const apstring& city1, const apstring& city2, int dist);
  int distance (int i, int j) const;
  int distance (const apstring& city1, const apstring& city2) const;
  apstring cityName (int i) const;
  int numCities () const;
  void print ();
};
\end{verbatim}
%
Using this interface simplifies {\tt main}:

\begin{verbatim}
#include <iostream>
#include <fstream>
using namespace std;


int main ()
{
  apstring line;
  ifstream infile ("distances");
  DistMatrix distances (2);

  while (true) {
    getline (infile, line);
    if (infile.eof()) break;
    processLine (line, distances);
  }

  distances.print ();
  return 0;
}
\end{verbatim}
%
It also simplifies {\tt processLine}:

\begin{verbatim}
void processLine (const apstring& line, DistMatrix& distances)
{
  char quote = '\"';
  apvector<int> quoteIndex (4);
  quoteIndex[0] = line.find (quote);
  for (int i=1; i<4; i++) {
    quoteIndex[i] = find (line, quote, quoteIndex[i-1]+1);
  }

  // break the line up into substrings
  int len1 = quoteIndex[1] - quoteIndex[0] - 1;
  apstring city1 = line.substr (quoteIndex[0]+1, len1);
  int len2 = quoteIndex[3] - quoteIndex[2] - 1;
  apstring city2 = line.substr (quoteIndex[2]+1, len2);
  int len3 = line.length() - quoteIndex[2] - 1;
  apstring distString = line.substr (quoteIndex[3]+1, len3);
  int distance = convertToInt (distString);

  // add the new datum to the distances matrix
  distances.add (city1, city2, distance);
}
\end{verbatim}
%
I will leave it as an exercise to you to implement the
member functions of {\tt DistMatrix}.


\section{Glossary}

\begin{description}

\item[ordered set:]  A data structure in which every element appears
only once and every element has an index that identifies it.

\item[stream:]  A data structure that represents a ``flow'' or
sequence of data items from one place to another.  In C++ streams
are used for input and output.

\item[accumulator:]  A variable used inside a loop to accumulate
a result, often by getting something added or concatenated during each
iteration.

\index{ordered set}
\index{set!ordered}
\index{stream}
\index{accumulator}
\index{pattern!accumulator}

\end{description}




\appendix
% LaTeX source for textbook ``How to think like a computer scientist''
% Copyright (C) 1999  Allen B. Downey

% This LaTeX source is free software; you can redistribute it and/or
% modify it under the terms of the GNU General Public License as
% published by the Free Software Foundation (version 2).

% This LaTeX source is distributed in the hope that it will be useful,
% but WITHOUT ANY WARRANTY; without even the implied warranty of
% MERCHANTABILITY or FITNESS FOR A PARTICULAR PURPOSE.  See the GNU
% General Public License for more details.

% Compiling this LaTeX source has the effect of generating
% a device-independent representation of a textbook, which
% can be converted to other formats and printed.  All intermediate
% representations (including DVI and Postscript), and all printed
% copies of the textbook are also covered by the GNU General
% Public License.

% This distribution includes a file named COPYING that contains the text
% of the GNU General Public License.  If it is missing, you can obtain
% it from www.gnu.org or by writing to the Free Software Foundation,
% Inc., 59 Temple Place - Suite 330, Boston, MA 02111-1307, USA.


\chapter{Quick reference for AP classes}

These class definitions are copied from the College Board
web page,

\begin{verbatim}
http://www.collegeboard.org/ap/computer-science/html/quick_ref.htm
\end{verbatim}

with minor formatting changes.  This is probably a good time to
repeat the following text, also from the College Board web page.

\begin{quotation}
"Inclusion of the C++ classes defined for use in the Advanced
Placement Computer Science courses does not constitute endorsement of
the other material in this textbook by the College Board, Educational
Testing service, or the AP Computer Science Development Committee. The
versions of the C++ classes defined for use in the AP Computer Science
courses included in this textbook were accurate as of 20 July
1999.  Revisions to the classes may have been made since that time."
\end{quotation}

\section{{\tt apstring}}

\begin{verbatim}
extern const int npos;  // used to indicate not a position in the string

// public member functions

  // constructors/destructor
  apstring();                       // construct empty string ""
  apstring(const char * s);         // construct from string literal
  apstring(const apstring & str);   // copy constructor
  ~apstring();                      // destructor

  // assignment
  const apstring & operator= (const apstring & str); // assign str
  const apstring & operator= (const char * s);       // assign s
  const apstring & operator= (char ch);              // assign ch

  // accessors
  int length() const;                      // number of chars
  int find(const apstring & str) const;    // index of first occurrence of str
  int find(char ch) const;                 // index of first occurrence of ch
  apstring substr(int pos, int len) const; // substring of len chars, 
                                           // starting at pos
  const char * c_str() const;              // explicit conversion to char *

  // indexing
  char operator[ ](int k) const; // range-checked indexing
  char & operator[ ](int k);     // range-checked indexing

  // modifiers
  const apstring & operator+= (const apstring & str); // append str
  const apstring & operator+= (char ch);              // append char

  // The following free (non-member) functions operate on strings

  // I/O functions
  ostream & operator<< ( ostream & os, const apstring & str );
  istream & operator>> ( istream & is, apstring & str );
  istream & getline( istream & is, apstring & str );

  // comparison operators
  bool operator== ( const apstring & lhs, const apstring & rhs );
  bool operator!= ( const apstring & lhs, const apstring & rhs );
  bool operator<  ( const apstring & lhs, const apstring & rhs );
  bool operator<= ( const apstring & lhs, const apstring & rhs );
  bool operator>  ( const apstring & lhs, const apstring & rhs );
  bool operator>= ( const apstring & lhs, const apstring & rhs );

  // concatenation operator +
  apstring operator+ ( const apstring & lhs, const apstring & rhs );
  apstring operator+ ( char ch, const apstring & str );
  apstring operator+ ( const apstring & str, char ch );
\end{verbatim}

\section{{\tt apvector}}

\begin{verbatim}
template <class itemType>
class apvector

// public member functions

  // constructors/destructor
  apvector();                                 // default constructor (size==0)
  apvector(int size);                         // initial size of vector is size
  apvector(int size, const itemType & fillValue);  // all entries == fillValue
  apvector(const apvector & vec);             // copy constructor
  ~apvector();                                // destructor

  // assignment
  const apvector & operator= (const apvector & vec);

  // accessors
  int length() const;                            // capacity of vector

  // indexing
  // indexing with range checking
  itemType & operator[ ](int index);           
  const itemType & operator[ ](int index) const;

  // modifiers
  void resize(int newSize);               // change size dynamically
                                          //can result in losing values
\end{verbatim}

\section{{\tt apmatrix}}

\begin{verbatim}
template <class itemType>
class apmatrix

// public member functions

  // constructors/destructor
  apmatrix();                                   // default size is 0 x 0
  apmatrix(int rows, int cols);                 // size is rows x cols
  apmatrix(int rows, int cols, const itemType & fillValue);
                                                // all entries == fillValue
  apmatrix(const apmatrix & mat);               // copy constructor
  ~apmatrix( );                                 // destructor

  // assignment
  const apmatrix & operator = (const apmatrix & rhs);

  // accessors
  int numrows() const;                                  // number of rows
  int numcols() const;                                  // number of columns

  // indexing
  // range-checked indexing
  const apvector<itemType> & operator[ ](int k) const;  
  apvector<itemType> & operator[ ](int k);

  // modifiers
  void resize(int newRows, int newCols); // resizes matrix to newRows x newCols
                                         // (can result in losing values)
\end{verbatim}



\printindex

\end{document}



