% LaTeX source for textbook ``How to think like a computer scientist''
% Copyright (C) 1999  Allen B. Downey

% This LaTeX source is free software; you can redistribute it and/or
% modify it under the terms of the GNU General Public License as
% published by the Free Software Foundation (version 2).

% This LaTeX source is distributed in the hope that it will be useful,
% but WITHOUT ANY WARRANTY; without even the implied warranty of
% MERCHANTABILITY or FITNESS FOR A PARTICULAR PURPOSE.  See the GNU
% General Public License for more details.

% Compiling this LaTeX source has the effect of generating
% a device-independent representation of a textbook, which
% can be converted to other formats and printed.  All intermediate
% representations (including DVI and Postscript), and all printed
% copies of the textbook are also covered by the GNU General
% Public License.

% This distribution includes a file named COPYING that contains the text
% of the GNU General Public License.  If it is missing, you can obtain
% it from www.gnu.org or by writing to the Free Software Foundation,
% Inc., 59 Temple Place - Suite 330, Boston, MA 02111-1307, USA.

% This is an Icelandic translation/adaptation of the orginal book by Allen B. Downey


\chapter{Forritun}

Megin markmiðið með þessari bók er að kenna þér, lesandi góður, að hugsa eins og tölvunarfræðingur.
Tölvunarfræðileg hugsun sameinar marga af bestu eiginleikum úr stærðfræði, verkfræði og náttúruvísindum.
Tölvunarfræðingar nota, eins og stærðfræðingar, formleg mál til að lýsa hugmyndum (sérstaklega útreikningum/vinnslu (e. computation)).
Eins og verkfræðingar, þá hanna þeir hluti, setja þá saman til að mynda kerfi og meta kosti og galla mismunandi valkosta.
Eins og vísindamenn, þá fylgjast þeir með hegðun flókinna kerfa, setja fram tilgátur og prófa þær. 

Einn mikilvægasti eiginleiki tölvunarfræðinga er hæfni til að {\bf leysa vandamál/verkefni}.
Hér er átt við að geta skilgreint verkefni, hugsað á skapandi hátt um lausnir og tjáð lausnaraðferðir á skýran og nákvæman máta.
Það vill einmitt svo vel til að það að læra að forrita er mjög góð leið til að þróa með sér hæfni til að leysa vandamál.
%That's why this chapter is called ``The way of the program.''

Þessi bók er hugsuð sem kennslubók í fyrsta forritunarnámskeiðinu (Forritun) við tölvunarfræðideild Háskólans í Reykjavík (HR).
Annað markmið bókarinnar er því að kenna þér grundvallaratriði í forritun -- grunn sem er nauðsynlegur fyrir önnur námskeið í tölvunarfræðináminu, eins og Gagnaskipan, Reiknirit og Forritunarmál.
%Of course, the other goal of this book is to prepare you for the Computer Science AP Exam.
%We may not take the most direct approach to that goal, though.
%For example, there are not many exercSo the
following is illegal: ises in this book that are similar to the AP questions.
%On the other hand, if you understand the concepts in this book, along with the details of programming in C++, you will have all the tools you need to do well on the exam.

\section{Hvað er forritunarmál?}
\index{forritunarmál}
\index{forritun}
\index{mál!forritun}

Í þessari bók er forritun kennd með því að nota forritunarmálið C++ sem er mjög útbreitt forritunarmál.
C++ er {\bf æðra forritunarmál} (e. high-level language) en önnur sambærileg forritunarmál sem þú hefur kannski heyrt um eru t.d. Java, C{\#} og Python.
Einkenni æðri forritunarmála er að smáatriði varðandi innri virkni tölvunnar hafa verið dregin út (e. abstracted away) sem merkir að forritarinn getur hugsað um verkefnið á æðra plani'' í stað þess að þurfa að hugsa um eiginleika tiltekins vélbúnaðar.

Andstæðan við æðri forritunarmál eru {\bf lágtæknimál} (e. low-level languages), sem stundum eru einnig kölluð vélarmál (e. machine languages) eða smalamál (e. assembly languages).
Í stuttu máli sagt þá geta tölvur aðeins keyrt forrit sem skrifuð eru í vélarmáli.
Þess vegna þarf að þýða (e. compile) forrit sem skrifað er í æðra forritunarmáli yfir á vélarmál tiltekinnar tölvu áður en forritið er keyrt.
Ef hægt er að tala um einhvern ókost við æðri forritunarmál þá er hann sá að umrædd þýðing getur tekið nokkurn tíma.

\index{færanleiki}
\index{æðra forritunarmál}
\index{lágtæknimál}
\index{mál!æðra}
\index{mál!lágtækni}

Á hinn bóginn eru kostirnir miklir.
Í fyrsta lagi er miklu auðveldara að forrita í æðra forritunarmáli heldur en í lágtæknimáli.
Með ``auðveldara'' er hér átt við að forritunin tekur minni tíma, forritið er styttra og læsilegra og mun líklegra til að vera rétt.
Í öðru lagi eru æðri forritunarmál oftast {\bf færanleg} (e. portable), í þeim skilningi að hægt er að keyra þau á mismunandi vélum með engum eða litlum breytingum.
Forrit sem skrifað eru í lágtæknimáli er aftur á móti eingöngu hægt að keyra á einni tiltekinni vél og þarfnast endurskriftar til keyrslu á annarri vélartegund.

Vegna þessara kosta eru langflest forrit í dag skrifuð í æðri forritunarmálum.
Lágtæknimál eru notuð við sérstakar aðstæður, t.d. þegar forritið er á móti tilteknum örgjörvum.

\index{þýða}
\index{túlka}

Það eru tvær leiðir til að umbreyta forriti, sem skrifað er í æðra forritunarmáli, þannig að hægt sé að keyra það á tiltekinni vél: {\bf túlkun} (e. interpretation) eða {\bf
þýðing} (e. compilation).
{\em Túlkur} (e. interpreter) er forrit sem les annað forrit $S$, sem skrifað er í æðra forritunarmáli, og líkir eftir sérhverri skipun í $S$.
Túlkur les forrit línu fyrir línu, varpar sérhverri skipun í $S$ yfir í skipun/skipanir í því máli sem túlkurinn er skrifaður í, og framkvæmir síðan þær skipanir.
Úttakið úr túlkinum er úttakið úr $S$.

\vspace{0.1in}
\centerline{\epsfig{figure=interpret.eps}}
\vspace{0.1in}

Þýðandi er forrit sem les annað forrit (sem oSo the
following is illegal: ftast er skrifað í æðra forritunarmáli) og þýðir það yfir á annað mál (oftast lágtæknimál).
Hið þýdda forrit er þá tilbúið til keyrslu síðar meir.
Inntakið í þýðandann er kallað {\bf frumkóði} (e. source code) og úttakið er {\bf markkóði} (e. target code).

Gefum okkur t.d. að þú skrifir forrit í C++.
Þú gætir notað textaritil (e. text editor) eða sérstakt C++ þróunarumhverfi (e. development environment) til að skrifa forritið og vistað það í skrá með nafninu {\tt program.cpp}.
Hér er {\tt program} eitthvað nafn sem þú gefur forritinu og viðskeytið {\tt .cpp} gefur til kynna að umrædd skrá innihaldi C++ frumkóða.

Þegar forritið hefur verið skrifað þá gætir þú keyrt þýðanda á það.
Þýðandinn myndi lesa frumkóðann, þýða hann og búa til nýja skrá með nafninu {\tt program.o} fyrir markkóðann,
og/eða {\tt program.exe} fyrir hina endanlegu keyrsluskrá (e. executable file).

\vspace{0.1in}
\centerline{\epsfig{figure=compile.eps}}
\vspace{0.1in}

Þegar keyrsluskráin er síðan keyrð þá sér undirliggjandi stýrikerfi um að hlaða forritinu upp í minni (afrita það af diski yfir í minni) og lætur síðan tölvuna byrja að keyra forritið.

Þetta ferli virðist vera flókið en sem betur fer þá eru einstök skref í ferlinu framkvæmd á sjálfvirkan máta í flestum þróunarumhverfum.
Yfirleitt þarft þú aðeins að skrifa forritið og ýta síðan á hnapp (eða skrifa skipun) sem setur þýðingu og keyrslu af stað.
Á hinn bóginn er mikilvægt fyrir þig að vita hvaða einstöku hlutar eiga sér stað í ferlinu þannig að þú getir lagfært þá ef eitthvað bjátar á. 

\section{Hvað er forrit?}

Forrit er röð skipana sem segja til um hvernig framkvæma á tiltekna vinnslu.
Vinnslan gæti t.d. verið stærðfræðileg, eins og að leysa jöfnuhneppi eða að finna rætur á margliðu.
Hún gæti jafnframt líka verið táknræn (e. symbolic), t.d. að leita að og skipta út texta í skrá eða að þýða forrit!

\index{setning}

Skipanir eða setningar líta mismunandi út í mismunandi forritunarmálum en nokkrar grunnaðgerðir eru sameiginlegar flestum málum:

\begin{description}

\item[inntak (e. input):] Sækja gögn af lyklaborði, úr skrá eða af einhverju öðru tæki.

\item[úttak (e. output):] Sýna gögn á skjánum, skrifa gögn í skrá eða í annað tæki. 

\item[stærðfræðilegar aðgerðir:] Framkvæma stærðfræðilegar grunnaðgerðir eins og samlagningu og margföldun.

\item[prófun (e. test):] Athuga hvort tiltekið skilyrði er satt/ósatt og framkvæma síðan viðeigandi röð setninga.

\item[endurtekning (e. repitition):] Endurtaka tilteknar aðgerðir, yfirleitt með einhverjum frávikum á milli endurtekninga.

\end{description}

Það kann að hljóma ótrúlega en þetta er í raun allt og sumt!
Sérhvert forrit, sem þú hefur notað/keyrt, samanstendur af aðgerðum sem þessum.
Forritun er því í raun það ferli að brjóta stórt, flókið verkefni upp í smærri og smærri einingar þangað til að sérhver eining er orðin það einföld að hægt er að leysa hana með einhverjum af þessum aðgerðum.

\section{Hvað er kembing?}
\index{kembing}
\index{aflúsun}
\index{böggur}
\index{villa}

Forritun er flókið ferli sem framkvæmt er mönnum (en ekki vélum) og þess vegna eiga forritunarvillur oft sér stað.
Af sögulegum ástæðum eru forritunarvillur oft kallaðar {\bf böggar} (e. bugs) og ferlið við að finna og leiðrétta þær er kallað {\bf kembing} eða {\bf aflúsun} (e. debugging).

Mismunandi forritunarvillur geta komið upp í forriti og það er mikilvægt að gera greinarmun á þeim til að geta fundið þær og leiðrétt hraðar.

\subsection{Þýðingarvillur}
%\index{compile-time error}
\index{þýðingarvilla}
\index{villa!þýðing}

Þýðandinn getur aðeins þýtt (búið til markkóða) fyrir forrit sem er setningafræðilega rétt.
Að öðrum kosti mistekst þýðingin og þá er ekki hægt að keyra forritið.
{\bf Málskipan} (e. syntax) vísar til hvernig forrit er samansett og hvaða reglur gilda um samsetninguna.

\index{málskipan}

Í íslensku þarf sérhver setning t.d. að byrja á stórum staf og enda á punkti.
þessi setning inniheldur málskipunarvillu. Einnig þessi hér 

Flestir lesendur eiga ekki í erfiðleikum með að skilja texta sem inniheldur nokkrar málskipunarvillur.
Við getum t.d. lesið ljóð eftir e e cummings án þess að spýja út villuskilaboðum!

Þýðendur líta aftur á móti málskipunarvillur alvarlegum augum.
Ef þýðandi finnur t.d. aðeins eina villu í forritinu þínu þá mun hann skrifa út villuskilaboð og hætta án þess að búa til keyrsluskrá.
Í því tilviki munt þú ekki geta keyrt forritið þitt.

Til að gera illt verra þá eru margar málskipunarreglur í C++ og villuskilaboðin sem koma frá þýðandanum eru stundum ekki mjög hjálpleg.
Á fyrstu vikum forritunarferils þíns munt þú líklega eyða verulegum tíma í að finna {\bf þýðingarvillur} (e. compile-time errors).
Með reynslunni munt þú hins vegar gera færri villur og finna þær mun hraðar.

\subsection{Keyrsluvillur}
\label{run-time}
\index{keyrsluvilla}
\index{villa!keyrsla}
%\index{safe language}
%\index{language!safe}

Önnur tegund af villum eru svokallaðar {\bf keyrsluvillur} (e. run-time errors), þ.e. villur sem koma ekki upp fyrr en við keyrslu forrits.

Í þeim einföldu forritum sem við munum skrifa á næstu vikum verða keyrsluvillur sjaldgæfar þannig að líklega mun líða dálítill tími þangað til þú sérð þess konar villur.


\subsection{Rökvillur og merking}
\index{merking}
\index{rökvilla}
\index{villa!rök}

{\bf Rökvillur} (e. logical errors) eða {\bf merkingarvillur} (e. semantic errors) eru þriðja tegundin af villum.
Ef forritið þitt inniheldur rökvillur þá mun það samt sem áður þýðast og keyra án þess að villuskilboð prentist út.
Forritið mun hins vegar ekki gera það sem til stóð en þó það sem þú sagðir því að gera!

Vandamálið er sem sagt það að forritið sem þú skrifaðir er ekki forritið sem þú ætlaðir þér að skrifa.
Merking forritsins er því röng.
Það getur verið erfitt að finna rökvillur því það krefst þess að þú vinnir ``aftur á bak'' með þvi að skoða úttakið og reyna að finna út hvað geti verið að.

\subsection{Tilraunakembing}

Ein mikilvægasta færnin sem þú ættir að öðlast við að lesa og vinna með þessa bók er kembing.
Þó kembing geti stundum verið pirrandi, þá getur hún jafnframt verið sá hluti forritunar sem er hvað mest krefjandi og áhugaverður.

Að sumu leyti er kembing eins og starf rannsóknarlögreglumanns.
Vísbendingar eru gefnar og finna þarf út hvaða ferli og atburðir leiddu til niðurstöðunnar. 

Kembing er einnig eins og tilraunavísindi.
Þegar þig grunar hvað fór úrskeiðis þá breytir þú forritinu og keyrir það á ný.
Ef tilgáta þín reynist rétt þá skilar forritið þitt réttri útkomu.
Ef tilgátan er aftur á móti röng þá þarftu að finna nýja.
Eða eins og Sherlock Holmes benti á: ``When you have eliminated the
impossible, whatever remains, however improbable, must be the truth.''
(úr {\em The Sign of Four} eftir A. Conan Doyle).

\index{Holmes, Sherlock}
\index{Doyle, Arthur Conan}

Sumir líta svo á að forritun og kembing sé einn og sami hluturinn, þ.e.a.s. að forritun sé það ferli að kemba forrit þangað til að það gerir það sem til stóð.
Hugmyndin er sú að þú ættir alltaf að byrja með forrit sem virkar, þ.e. forrit sem gerir {\em eitthvað}, framkvæma síðan litlar breytingar á því, með kembingu ef þess er þörf, þannig að þú sért ávallt með forrit sem virkar.

Sem dæmi um þetta má nefna stýrikerfið Linux sem inniheldur þúsundir forritslína en byrjaði sem lítið forrit sem Linus Torvalds notaði til að kanna Intel 80386 örgjörvann.
``One of Linus's earlier projects was a program that would switch between printing AAAA and BBBB.
This later evolved to Linux'' (úr {\em The Linux Users' Guide} Beta Version 1).

\index{Linux}

Í síðari köflum þessarar bókar mun ég koma með fleiri tillögur um kembingu og aðrar forritunaraðferðir.

\section{Formleg og náttúruleg mál}
\label{formal}
\index{formlegt mál}
\index{náttúrulegt mál}
\index{mál!formlegt}
\index{mál!náttúrulegt}

{\bf Náttúruleg mál} (e. natural languages) eru mál sem fólk talar, eins og enska, spænska, franska og íslenska.
Þessi mál hafa ekki verið hönnuð af mönnum (þó svo að menn reyni oft að búa þeim einhverjar reglur) heldur hafa þau þróast á náttúrulegan hátt.

{\bf Formleg mál} (e. formal languages) eru mál sem eru hönnuð af mönnum fyrir tiltekna notkun.
Stærðfræðingar nota t.d. sérstakt formlegt mál sem hentar vel til að lýsa sambandi á milli talna og tákna.
Efnafræðingar nota formlegt mál sem lýsir efnafræðilegri byggingu sameinda.
Og síðast en ekki síst:

\begin{quote}
{\bf Forritunarmál eru formleg mál sem hafa verið hönnuð til að tjá vinnslu.} 
%{\bf Programming languages are formal languages that have been designed to express computations.}
\end{quote}

Flest formleg mál hafa stífar málfræðireglur.
$3+3=6$ er t.d. setningafræðilega rétt stærðfræðileg setning en ekki $3=+6\$$.
Sem annað dæmi má nefna að $H_2O$ er rétt nafn á frumefni en ekki $_2Zz$.

Málfræðireglurnar eru af tvennum toga.
Í fyrsta lagi eru reglur um hvað eru leyfilegir tókar (e. tokens) í viðkomandi máli. 
Tókar eru grunneiningar í öllum málum, t.d. orð, tölur og nöfn á frumefnum.

Vandamálið við {\tt 3=+6\$} er að {\tt \$} er ekki leyfilegur tóki í stærðfræði.
Á sama hátt er $_2Zz$ ekki leyfilegt því það er ekkert frumefni með skammstöfunina $Zz$.

Hin tegundin af málfræðireglum, málskipunarreglur, hefur að gera með hvernig tókar eru settir saman til að mynda setningar.
T.d. er setningin {\tt 3=+6\$} er ekki rétt sett saman því á eftir gildisveitingarvirkja (e. assignment operator) getur ekki komið virkinn plús.
%Similarly, molecular formulas have to have subscripts after the element name, not before.

Þegar þú lest setningu í þínu móðurmáli eða setningu í formlegu máli þá þarftu að finna út hvernig setningin er samansett af einstökum tókum (í náttúrulegu máli þá gerum við þetta reyndar ómeðvitað). 
Þetta ferli er kallað {\bf þáttun} (e. parsing).

\index{þáttun}

Þegar þú heyrir t.d. setninguna ``The other shoe fell'' þá áttar þú þig á því (með þáttun) að ``the other shoe'' er frumlagið (e. subject) og ``fell'' er sögnin (e. verb).
Þegar þú hefur þáttað setninguna þá getur þú fundið út hver merking hennar er.
Að því gefnu að þú vitir hvað ``shoe'' er og þú vitir hver merkingin ``to fall'' er þá skilur þú heildmerkingu setningarinnar.

Þrátt fyrir að formleg mál og náttúruleg mál eigi sér margar hliðstæður, t.d. í tókum, málskipan og merkingu, þá er jafnframt margt ólíkt með þessum tegundum mála:

\index{margræðni}
\index{ofauki}
\index{bókstafsmerking}

\begin{description}

\item[margræðni (e. ambiguity):] Margræðni er mjög algeng í náttúrlegum málum en við leysum margræðni með því að nýta okkur samhengið og aðrar upplýsingar.
Formleg mál eru aftur á móti hönnuð til að vera laus við margræðni, þ.e. sérhver setning hefur nákvæmlega eina merkingu, án tillits til samhengis.

\item[ofauki (e. redundancy):] Til að koma í veg fyrir misskilning vegna margræðni þá er orðum oft ofaukið í setningum í náttúrulegum málum.
Í formlegum málum eru hins vegar litlar sem engar málalengingar og setningar í þeim eru því gagnyrtar.

\item[bókstafsmerking (e. literalness):] Orðatiltæki og myndlíkingar eru algengar í náttúrulegum málum.
Ef ég segi t.d. ``Sjaldan er ein báran stök'', þá er ég í rauninni ekki að tala um báru.  
Merking setningar í formlegu máli er hins vegar bókstafsleg, þ.e. nákvæmlega sú sem lesa má úr setningunni. 

\end{description}

Þeir sem alast upp við náttúruleg mál (þ.e. við öll) eiga oft erfitt með að venjast formlegum málum.
Að sumu leyti er munurinn á milli náttúrulegs og formlegs máls eins og munurinn á milli bundins og óbundins máls, en þó meiri:

\index{bundið mál}
\index{óbundið mál}

\begin{description}

\item[Bundið mál:] Orð eru notuð vegna þeirra hljóða sem þau mynda og einnig vegna merkingar þeirra.
Í heild sinni hefur ljóðið einhver áhrif og veldur tilfinningalegum viðbrögðum.
Margræðni er ekki bara algeng heldur oft meðvituð.

\item[Óbundið mál:] Raunveruleg merking orða er mikilvæg og uppbygging setningar hefur mikið með merkingu að gera.
Óbundið mál er yfirleitt auðveldara að greina en bundið mál, en er samt sem áður oft margrætt. 

\item[Forrit:] Merking forrits er ekki margræð og hún er bókstafleg.
Hægt er að skilja merkinguna með því að greina tóka og málskipan. 

\end{description}

Hér fylgja nokkrar ráðleggingar við lestur forrits (eða annarra formlegra mála).
Í fyrsta lagi mundu að formleg mál eru mun ``samþjappaðri'' en náttúrleg mál og því tekur lengri tíma að lesa þau.
Í öðru lagi er uppbygging forrita mjög mikilvæg og því er yfirleitt ekki hentugt að lesa þau frá byrjun til enda, vinstri til hægri.
Í staðinn skaltu læra að þátta forritið í huganum, bera kennsl á tókana og greina uppbygginguna.
Mundu að lokum að smáatriðin skipta máli.  
Smáatriði, eins og stafsetningarvillur eða greinamerkjavillur sem hægt er að komast upp með í náttúrulegum málum, geta skipt miklu máli í formlegum málum.


\section{Fyrsta forritið}
\label{hello}
\index{hello world}

Sú venja hefur skapast að fyrsta forritið sem fólk skrifar í nýju forritunarmáli er kallað ``Hello, World'' vegna þess að eina sem það gerir er að prenta út orðin ``Hello, World''.
Svona lítur forritið út í forritunarmálinu C++: 

\begin{verbatim}
#include <iostream>
using namespace std;

// main: generate some simple output

int main ()
{
  cout << "Hello, world." << endl;
  return 0;
}
\end{verbatim}
%
Sumir dæma gæði forritunarmáls með því að skoða hversu einfalt er að skrifa ``Hello, World'' forrit í málinu.
Ef við notum þennan mælikvarða þá kemur C++ ágætlega út.
Samt sem áður inniheldur þetta einfalda forrit ýmsa eiginleika sem er erfitt að skýra út fyrir byrjendum í forritun.
Til að byrja með munum við ekki skýra suma þeirra, eins og fyrstu tvær línurnar í þessu forriti.

\index{athugasemd}
\index{setning!athugasemd}

Þriðja línan byrjar á  {\tt //} sem gefur til kynna að línan er {\bf athugasemd} (e. comment).
Athugasemd er texti (oftast skrifaður í náttúrulegu máli) sem hægt er að setja inn hvar sem er í forriti í þeim tilgangi að skýra út hvað forritið, eða hluti þess, gerir.
Þegar þýðandinn rekst á {\tt //} þá hunsar hann allan textann frá þeim stað og að enda línunnar.

Í fjórðu línunni getur þú, sem stendur, hunsað orðið (tókann) {\tt int} en taktu eftir orðinu (tókanum) {\tt main}.
sem er sérstakt nafn sem gefur til kynna staðinn þar sem keyrsla forritsins hefst.
Þegar forritið hefur keyrslu þá mun það byrja með því að keyra fyrstu setninguna í {\tt main}, heldur síðan áfram í réttri röð og hættir eftir framkvæmd síðustu setningarinnar.

\index{úttak}
\index{setning!úttak}

Það eru engin takmörk á því hversu margar setningar {\tt main} getur innihaldið en í dæminu að ofan er aðeins um eina setningu að ræða.
Sú setning er úttakssetning, þ.e. setning sem skrifar tiltekin skilaboð út á skjáinn. 

{\tt cout} er sérstakur hlutur (e. object), innbyggður í C++, sem gerir þér kleift að senda tiltekið úttak á skjáinn (cout er úttaksstraumur).
Táknið {\tt <<} er {\bf virki} (e. operator) sem beitt er á {\tt cout} og streng, og veldur því að strengurinn er skrifaður út. 

\index{virki}

{\tt endl} er sérstakt tákn sem stendur fyrir enda á línu.
Þegar þú sendir {\tt endl} á {\tt cout} þá færist bendillinn (e. cursor) í næstu línu á skjánum.
Eftir það mun texti sem skrifaður er út birtast í þessari næstu línu. 

Eins og gildir um allar setningar þá endar úttakssetningin með semikommu ({\tt ;}).

Það er nokkur önnur atriði sem vert er að hafa í huga varðandi málskipan þessa fyrsta forrits.
Í fyrsta lagi að C++ notar slaufusviga (e. curly-braces) (\{ og
\}) til að gefa til kynna að hlutir eigi saman.
Í þessu tilviki er úttakssetningin inni í slaufusvigum sem þýðir að 
hún er hluti af skilgreiningunni á {\tt main}. 
Einnig er vert að benda á að setninginn er inndregin (e. indented) sem sýnir á skýran hátt hvaða setningar eru hluti af skilgreiningunni.

Á þessum tímapunkti mæli ég með því að þú setjist fyrir framan tölvuna þína og þýðir og keyrir þetta forrit.
Hvernig það er gert er háð því forritunarumhverfi\footnote{Í fyrsta forritunarnámskeiðinu í tölvunarfræði í HR er forritunarumhverfið Code::Blocks \url{http://www.codeblocks.org/} notað.} sem þú notar en ég mun héðan í frá í þessari bók gera ráð fyrir að þú kunnir að gera það.

Eins og ég nefndi áður þá tekur C++ þýðandinn hart á málskipunarvillum.
Ef þú gerir einhverjar innsláttarvillur í forritinu þá mun forritið líklega ekki þýðast.
T.d. ef þú slærð inn {\tt oistream} í stað {\tt iostream} þá færðu villuskilaboð eins og þessi:

\begin{verbatim}
hello.cpp:1: oistream.h: No such file or directory
\end{verbatim}
%
Það eru verulegar upplýsingar fólgnar í þessum villuskilaboðum en þær eru á samþjöppuðu formi og það er ekki einfalt að túlka skilaboðin.
Notandavænni þýðandi myndi sjálfsagt segja eitthvað á þessa leið: 

\begin{quote}
``Í línu 1 í frumkóðanum með nafninu hello.cpp reyndir þú að taka inn `header' skrá með nafninu oistream.
Ég fann enga skrá með því nafni en fann aftur á móti skrá með nafninu iostream.
Getur verið að þú hafir ætlað þér að taka þá skrá inn?''
\end{quote}

Því miður eru fæstir þýðendur svona vingjarnlegir!
Þýðandi er í raun ekki mjög ``greindur'' og í mörgum tilvikum eru villuskilaboðin sem þú færð í raun bara ábending um hvað geti verið að.
Það mun taka tíma fyrir þig að læra að túlka villuskilaboðin á réttan hátt.

Þrátt fyrir þetta eru þýðendur mikilvægt tól til að læra málskipunarreglur tiltekins forritunarmáls.
Gott er að byrja með forrit sem virkar (eins og hello.cpp), breyta því á ýmsan hátt og sjá hvað gerist.
Ef þú færð villuskilaboð reyndu þá að muna hver þau eru og hvað olli þeim þannig að ef þú sérð þau aftur síðar meir þá veistu hvað þarf að gera til að lagfæra villuna.

\section{Orðalisti}

\begin{description}

\item[lausn verkefnis/vandamáls (e. problem-solving):]
Ferli sem felur í sér að skilgreina vandamál, finna lausn á því og setja lausnaraðferðina fram á skýran og nákvæman máta.

\item[æðra forritunarmál (e. high-level language):]
Forritunarmál eins og C++ sem er hannað til að fólk geti auðveldlega lesið það og skrifað.

\item[lágtæknimál (e. low-level language):]
Forritunarmál sem er hannað til að tölvur geti á auðveldan hátt keyrt forrit í málinu.
Einnig kallað vélarmál (e. machine language) eða smalamál (e. assembly language).

\item[færanleiki (e. portability):]  Einkenni forrits sem gerir það að verkum að hægt er að keyra það á fleiri en einni tegund véla.

\item[formlegt mál (e. formal language):]  Öll mál sem fólk hefur búið til í sérstökum tilgangi, eins og t.d. mál sem standa fyrir stærðfræðilegar hugmyndir eða tölvuforrit, eru formleg mál.
Öll forritunarmál eru formleg mál.

\item[náttúrlegt mál (e. natural language):]
Öll mál sem töluð eru af mönnum og hafa þróast á náttúrulegan hátt.

\item[túlka (e. interpret):]  Það að keyra forrit, sem skrifað er í tilteknu máli, með því að túlka það línu fyrir línu.

\item[þýða (e. compile):]  Það að þýða forrit yfir í lágtæknimál í þeim tilgangi að geta keyrt það síðar, aftur og aftur.

\item[frumkóði (e. source code):]  Forrit, sem er (yfirleitt) skrifað í æðra forritunarmáli, áður en það er þýtt.

\item[markkóði (e. object code):]  Úttak þýðandans, þ.e. hið þýdda forrit. 

\item[keyrsluskrá (e. executable):] Markkóði sem hægt er að keyra beint á tölvu. 

%\item[algorithm:]  A general process for solving a category of problems. 

\item[forritunarvilla (e. bug):]  Villa í forriti.

\item[málskipan (e. syntax):]  Uppbygging forrits.

\item[merking (e. semantics):]  Merking forrits.

\item[þátta (e. parse):]  Að greina forrit í einstakar setningafræðilegar einingar.

\item[málskipunarvilla (e. syntax error):]  Villa í forriti sem veldur því að ekki er hægt að þátta forritið (og þar með ekki hægt að þýða það).

\item[keyrsluvilla (e. run-time error):]  Villa í forrit sem kemur upp við keyrslu þess. 

\item[rökvilla (e. logical error):]  Villa í forriti sem veldur því að það gerir eitthvað annað en það sem forritarinn ætlaðist til.

\item[kembing (e. debugging):]  Ferlið við að finna og fjarlægja allar tegundir af forritunarvillum.

\index{lausn vandamála}
\index{æðra forritunarmál}
\index{lágtæknimál}
\index{formlegt mál}
\index{náttúrulegt mál}
\index{túlka}
\index{þýða}
\index{málskipan}
\index{merking}
\index{þátta}
\index{villa}
\index{forritun!villa}
\index{kembing}
\index{aflúsun}

\end{description}