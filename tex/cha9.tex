% LaTeX source for textbook ``How to think like a computer scientist''
% Copyright (C) 1999  Allen B. Downey

% This LaTeX source is free software; you can redistribute it and/or
% modify it under the terms of the GNU General Public License as
% published by the Free Software Foundation (version 2).

% This LaTeX source is distributed in the hope that it will be useful,
% but WITHOUT ANY WARRANTY; without even the implied warranty of
% MERCHANTABILITY or FITNESS FOR A PARTICULAR PURPOSE.  See the GNU
% General Public License for more details.

% Compiling this LaTeX source has the effect of generating
% a device-independent representation of a textbook, which
% can be converted to other formats and printed.  All intermediate
% representations (including DVI and Postscript), and all printed
% copies of the textbook are also covered by the GNU General
% Public License.

% This distribution includes a file named COPYING that contains the text
% of the GNU General Public License.  If it is missing, you can obtain
% it from www.gnu.org or by writing to the Free Software Foundation,
% Inc., 59 Temple Place - Suite 330, Boston, MA 02111-1307, USA.

% This is an Icelandic translation/adaptation of the orginal book by Allen B. Downey

\chapter{Meira um strúktúra}
\label{time}
\index{struct}

\section{Time}
\index{struct!Time}
\index{Time}

Í þeim tilgangi að sýna annað dæmi um strúktur munum við nú skilgreina tag með nafnið {\tt Time} sem notað er til að halda utan um tíma innan dags.
Tími samanstendur af klukkustund (hour), mínútu (minute) og sekúndu (second) þannig að þetta munu verða tilvikabreytur strúktúrsins.

Fyrsta skrefið er þá að ákveða hvert tag sérhverrar tilvikabreytu á að vera.
Það virðist ljóst að {\tt hour} og {\tt minute} ættu að vera heiltölur.
Til að hafa þetta áhugaverðara þá skulum við láta {\tt second} vera {\tt double} þannig að við getum haldið utan um brot af sekúndu.

Svona lítur þá strúktúrinn okkar út: 

\begin{verbatim}
struct Time {
  int hour, minute;
  double second;
};
\end{verbatim}
%
Við getum þá búið til {\tt Time} hlut á venjulegan hátt: 

\begin{verbatim}
  Time time = { 11, 59, 3.14159 };
\end{verbatim}
%
Stöðuritið fyrir þennan hlut lítur svona út:

\vspace{0.1in}
\centerline{\epsfig{figure=time.eps}}
\vspace{0.1in}

Orðið ``tilvik'' (e. instance) er stundum notað þegar við tölum um hluti (e. objects) vegna þess að sérhver hlutur er tilvik af einhverju tagi.
Ástæðan fyrir því að tilvikabreytur bera það nafn er að sérhvert tilvik af einhverju tagi eiga sér afrit af breytum þess tags.

\section{{\tt printTime}}
\label{printobject}
\index{úttak}
\index{setning!úttak}
\index{hlutur!úttak}

Þegar við skilgreinum nýtt tag þá er góð regla að skrifa fall sem skrifar út tilvikabreyturnar á læsilegan hátt.
Dæmi:

\begin{verbatim}
void printTime (Time& t) {
  cout << t.hour << ":" << t.minute << ":" << t.second << endl;
}
\end{verbatim}
%
Ef við sendum inn {\tt time} sem viðfang í þetta fall þá er úttakið {\tt 11:59:3.14159}.

\begin{verbatim}
#include <iostream>

using namespace std;

struct Time {
  int hour, minute;
  double second;
};


void printTime (Time& t) {
  cout << t.hour << ":" << t.minute << ":" << t.second << endl;
  cout << "Time is " << t.hour << " hour " << t.minute << " minutes " << t.second << "  seconds  "<<endl;
}


int main ()
{
 Time time = { 11, 59, 3.14159 };
 printTime(time);
 
 return 0;
}
\end{verbatim}
%

\section{Föll fyrir hluti}
\label{objectops}
\index{hlutur}
\index{föll!fyrir hluti}

Í næstu köflum mun ég sýna dæmi um nokkur möguleg skil (e. interfaces) fyrir föll sem vinna með hluti.
Þú munt hafa val um nokkur möguleg skil fyrir tilteknar aðgerðir þannig að þú ættir að meta kosti og galla sérhvers möguleika:

\begin{description}

\item[hreint fall (e. pure function):]  Tekur hlut og/eða grunntag sem viðföng en breytir ekki hlutnum.
Skilagildið er annaðhvort grunntag eða nýr hlutur sem búinn er til í fallinu. 

\item[breytir (e. modifier):]  Tekur hluti sem viðföng og breytir sumum eða öllum þeirra.  Skilar oft void. \index{void}

\item[fyllir (e. fill-in function):]  Eitt af viðföngunum er ``tómur'' hlutur sem fallið fyllir inn í.
Tæknilega séð er þetta þá í raun tegund af breyti. 

\end{description}

\section{Hrein föll}
\index{hreint fall}
\index{fall}
\index{fall!hreint}

Fall er sagt vera hreint ef skilagildi þess er eingöngu háð viðföngunum og að það hafi ekki neinar aukaverkanir (e. side effects)
eins og að breyta viðfangi eða skrifa eitthvað út.
Það eina sem gerist þegar kallað er á hreint fall er að skilagildi þess kemur til baka.

Eitt dæmi um hreint fall er {\tt after} sem ber saman tvo {\tt Time} hluti og skilar {\tt bool} sem gefur til kynna hvort fyrra viðfangið komi á eftir því seinna (tímalega séð):

\begin{verbatim}
bool after (Time& time1, Time& time2) {
  if (time1.hour > time2.hour) return true;
  if (time1.hour < time2.hour) return false;

  if (time1.minute > time2.minute) return true;
  if (time1.minute < time2.minute) return false;

  if (time1.second > time2.second) return true;
  return false;
}
\end{verbatim}
%
Hvert er skilagildi þessa falls ef tímarnir tveir eru jafnir?
Finnst þér það vera viðeigandi skilagildi fyrir þetta fall?
Ef þú værir að skjala þetta fall, myndir þú nefna þetta tilfelli sérstaklega?

Annað dæmi er {\tt addTime} sem reiknar summuna af tveimur tímasetningum.
Ef tíminn er t.d. {\tt 9:14:30} og brauðgerðin þín tekur 3 klukkustundir og 35 mínútur þá gætir þú notað {\tt addTime} til að finna út hvenær brauðið verður tilbúið.

Hér eru drög, sem eru reyndar ekki alveg rétt, af þessu falli:

\begin{verbatim}
Time addTime (Time& t1, Time& t2) {
  Time sum;
  sum.hour = t1.hour + t2.hour;
  sum.minute = t1.minute + t2.minute;
  sum.second = t1.second + t2.second;
  return sum;
}
\end{verbatim}
%
Hér er dæmi um hvernig hægt er að nota þetta fall.
Ef {\tt currentTime} inniheldur núverandi tíma og {\tt breadTime} inniheldur tímann sem tekur að baka brauðið þá getur þú notað {\tt addTime} til að reikna út hvenær brauðið verður tilbúið.

\begin{verbatim}
  Time currentTime = { 9, 14, 30.0 };
  Time breadTime = { 3, 35, 0.0 };
  Time doneTime = addTime (currentTime, breadTime);
  printTime (doneTime);
\end{verbatim}
%
Úttak þessa forrits er {\tt 12:49:30} sem er rétt.
Á hinn bóginn eru tilvik þar sem niðurstaðan er ekki rétt.
Getur þú fundið dæmi um þess konar tilvik?

Vandamálið er að þetta fall ræður ekki við þau tilvik þar sem summa sekúndna eða mínútna er stærri en 60.
Þegar það gerirst þá þurfum við að ``færa'' auka sekúndur yfir í mínútudálkinn eða auka mínútur yfir í klukkustundadálkinn.

Hér er önnur, nú rétt, útgáfa af fallinu:

\begin{verbatim}
Time addTime (Time& t1, Time& t2) {
  Time sum;
  sum.hour = t1.hour + t2.hour;
  sum.minute = t1.minute + t2.minute;
  sum.second = t1.second + t2.second;

  if (sum.second >= 60.0) {
    sum.second -= 60.0;
    sum.minute += 1;
  }
  if (sum.minute >= 60) {
    sum.minute -= 60;
    sum.hour += 1;
  }
  return sum;
}
\end{verbatim}
%
Þó svo að þessi útgáfa sé rétt þá er hún orðin dálítið löng.
Ég mun síðar leggja til aðra lausnaraðferð sem er mun styttri.

\index{increment}
\index{decrement}
\index{auki}
\index{frádrag}
\index{virki!auki}
\index{virki!frádrag}

Kóðinn að ofan sýnir dæmi um tvo virkja sem við höfum ekki séð áður, {\tt +=} og {\tt -=}.
Þessir virkjar eru notaðir sem samþjöppuð leið til að hækka eða lækka breytur.
Setningin {\tt sum.second -= 60.0;} er t.d. jafngild setningunni {\tt sum.second = sum.second - 60;}

\section{{\tt const} leppar}

Þú hefur væntanlega tekið eftir því að viðföngin í föllin {\tt after}
og {\tt addTime} eru send með tilvísun (e. by reference).
Þar sem um er að ræða hrein föll þá breyta þau ekki viðföngunum og því hefði ég alveg eins getað sent þau sem gildi (e. by value).

Kosturinn við að senda viðföng sem gildi er að fallið sem kallað er á og sá sem kallar eru hjúpuð á viðeigandi hátt -- breyting í öðrum þeirra getur ekki leitt til breytingar í hinum, nema í tengslum við skilagildið.

Á hinn bóginn er yfirleitt skilvirkara að senda viðföng með tilvísun vegna þess að þá þarf ekki að afrita nein gildi úr viðföngunum yfir í leppana.
Svo vill líka til að C++ inniheldur eiginleika sem kallaður er {\tt const} og gerir það að verkum að það er jafn öruggt að senda tilvísunarviðföng eins og gildisviðföng.

Ef þú skrifar fall og ætlar ekki að breyta viðfangi þá getur þú skilgreint leppinn sem {\bf constant reference parameter} (fast tilvísunarviðfang).
Málskipanin lítur svona út:

\begin{verbatim}
void printTime (const Time& time) ...
Time addTime (const Time& t1, const Time& t2) ...
\end{verbatim}
%
Hér sýni ég aðeins fyrstu línuna í föllunum.
Ef þú lætur þýðandann vita að þú ætlir ekki að breyta viðfangi í falli þá getur hann minnt þig á það!
Ef þú reynir að breyta viðfangi þá mun þýðandinn kvarta.

\index{keyrsluvilla}
\index{villa!við keyrslu}

\section{Breytiföll}
\index{modifier}
\index{fall!breytir}

Auðvitað vill svo til stundum að þú vilt einmitt breyta viðfangi.  Fall sem gerir það eru kallað breytir (e. modifier).

Skoðum fallið {\tt increment} sem dæmi um breyti en það bætir tilteknum fjölda sekúndna við {\tt Time} hlut.
Gróf drög fyrir þetta fall gæti litið svona út:

\begin{verbatim}
void increment (Time& time, double secs) {
  time.second += secs;

  if (time.second >= 60.0) {
    time.second -= 60.0;
    time.minute += 1;
  }
  if (time.minute >= 60) {
    time.minute -= 60;
    time.hour += 1;
  }
}
\end{verbatim}
%
Fyrsta línan framkvæmir grunnaðgerð en það sem á eftir kemur sér um sérstök tilfelli, eins og við sáum áður.

Er þetta fall rétt? Hvað gerist ef leppurinn {\tt secs} er mikið stærri en 60?
Í því tilfelli er ekki nóg að draga 60 frá einu sinni -- við þurfum að gera það þangað til {\tt second} er lægra en 60.

Við getum gert það með því að skipta {\tt if} setningum út fyrir {\tt while} setningar:

\begin{verbatim}
void increment (Time& time, double secs) {
  time.second += secs;

  while (time.second >= 60.0) {
    time.second -= 60.0;
    time.minute += 1;
  }
  while (time.minute >= 60) {
    time.minute -= 60;
    time.hour += 1;
  }
}
\end{verbatim}
%
Þessi lausn er rétt en ekki mjög skilvirk.
Getur þú séð fyrir þér lausn sem þarfnast ekki ítrunar?

\section{Fylliföll}
\index{fyllir}
\index{fall!fyllir}

Af og til sérð þú fall eins og {\tt addTime} skrifað með því að nota önnur skil (e. interface), þ.e. önnur viðföng og annað skilagildi.
Í stað þess að búa til nýjan hlut í sérhvert sinn sem kallað er á {\tt addTime} þá gætum við krafist þess að sá sem kallar útvegi ``tóman'' hlut sem {\tt addTime} getur sett upplýsingar inn í.
Berðu saman eftirfarandi útgáfu og þá fyrri:

\begin{verbatim}
void addTimeFill (const Time& t1, const Time& t2, Time& sum) {
  sum.hour = t1.hour + t2.hour;
  sum.minute = t1.minute + t2.minute;
  sum.second = t1.second + t2.second;

  if (sum.second >= 60.0) {
    sum.second -= 60.0;
    sum.minute += 1;
  }
  if (sum.minute >= 60) {
    sum.minute -= 60;
    sum.hour += 1;
  }
}
\end{verbatim}
%
Kosturinn við þessa útfærslu er að sá sem kallar hefur möguleika á að endurnýta sama hlutinn aftur og aftur til að framkvæma röð af samlagningum á {\tt Time} hlutum.
Þetta getur verið aðeins skilvirkara þó að þetta geti verið ruglandi og valdið lúmskum villum.
Fyrir flest forritunarverkekfni getur verið gott að eyða meiri keyrslutíma til að koma í veg fyrir langan kembitíma síðar meir.

Athugaðu að hægt er að skilgreina fyrstu tvo leppana sem {\tt const} en ekki þann þriðja.

\section{Hvað er best?}
\index{forritunarstíll}

Það sem hægt er að gera með breyti eða fylli er líka hægt að gera með hreinu falli.
Það vill reyndar svo til að ákveðin tegund forritunarmála, {\bf fallaforritunarmál}, leyfa eingöngu hrein föll.
Sumir forritarar trúa því að fljótlegra sé að skrifa forrit sem nota eingöngu hrein föll og þau séu líka áreiðanlegri (innihaldi færri villur) en þau sem nota breytiföll.
Aftur á móti geta breytiföll stundum verið hentug og tilvik koma upp þar sem fallaforrit eru ekki eins skilvirk.

Almennt séð mæli ég með því að þú skrifir hrein föll þar sem það liggur beint við og notir aðeins breytiföll ef það er augljós kostur.
Þetta viðhorf mætti kalla fallaforritunarstíl (e. functional programming style).

\section{Stigvaxandi þróun vs. áætlunargerð}
\index{stigvaxandi þróun}
\index{frumgerð}
\index{forritunarþróun!stigvaxandi}
\index{forritunarþróun!áætlunargerð}

Ég hef í þessum kafla sýnt dæmi um aðferð við forritunarþróun sem leiðir af sér {\bf skjóta frumgerð með stigvaxandi endurbótum} (e. rapid prototyping with iterative improvement).
Í sérhverju tilviki skrifaði ég drög (frumgerð) sem framkvæmdi grunnútreikninga, síðan prófaði ég þá á nokkrum tilvikum og leiðrétti villur þegar þær komu í ljós.

Þó svo að þessi aðferð geti verið markvirk (e. effective) þá getur kóðinn orðið óþarflega flókinn, vegna þess að hann meðhöndlar mörg mismunandi tilvik,
og óáreiðanlegur, vegna þess að það getur verið erfitt að fullvissa sig um að maður hafi fundið allar villurnar.

Önnur aðferð er hönnun eða áætlunargerð sem felur í sér að smá innsýn í vandamálið getur gert forritunina mikið einfaldari.
Í þessu tilviki er innsýnin sú að {\tt Time} er í raun tala með þremur tölustöfum með grunn 60!
Sekúndan ({\tt second}) er ``ones column'', mínútan ({\tt minute}) er ``60's column'', og klukkustundin ({\tt hour}) er ``3600's column''.

Þegar við skrifuðum {\tt addTime} og {\tt increment} þá gerðum við í raun samlagningu með grunn 60, sem er einmitt ástæðan fyrir því að við þurftum að ``færa'' frá einum dálki yfir í annan.

\index{útreikningur!grunnur 60}
\index{útreikningur!kommutala}

Önnur leið til að leysa vandamálið er því að breyta {\tt Time} hlutum yfir í {\tt double} breytur og 
nýta sér það að tölvan veit þegar hvernig gera á útreikninga með {\tt double}.
Hér er fall sem breytir {\tt Time} yfir í {\tt double}:

\begin{verbatim}
double convertToSeconds (const Time& t) {
  int minutes = t.hour * 60 + t.minute;
  double seconds = minutes * 60 + t.second;
  return seconds;
}
\end{verbatim}
%
Allt sem við þurfum þá til viðbótar er leið til að breyta {\tt double} yfir í {\tt Time} hlut:

\begin{verbatim}
Time makeTime (double secs) {
  Time time;
  time.hour = int (secs / 3600.0);
  secs -= time.hour * 3600.0;
  time.minute = int (secs / 60.0);
  secs -= time.minute * 60;
  time.second = secs;
  return time;
}
\end{verbatim}
%
Þú þarft líklega að hugsa dálítið um þetta til að fullvissa þig um að aðferðin sem ég nota hér til að breyta úr einum grunn yfir í annan sé rétt.
Að því gefnu að þú sért sannfærð(ur) þá getum við notað þessi föll til að endurskrifa {\tt addTime}:

\begin{verbatim}
Time addTime (const Time& t1, const Time& t2) {
  double seconds = convertToSeconds (t1) + convertToSeconds (t2);
  return makeTime (seconds);
}
\end{verbatim}
%
Þessi útgáfa er miklu styttri en upphaflega útgáfan og það er líka mun einfaldara að sýna að hún sé rétt.
Þú ættir núna að æfa þig með því að endurskrifa {\tt increment} á sama hátt.

\section{Alhæfing}
\index{alhæfing}

Á vissan hátt er erfiðara að breyta úr grunni 60 yfir í grunn 10, og til baka, en að meðhöndla tíma.
Breyting á grunni er meira abstrakt, tilfinning okkar fyrir tíma er meiri.

Á hinn bóginn má segja að ef við meðhöndlum tíma sem tölur með grunn 60 og skrifum umbreytingaföllin
({\tt convertToSeconds} og {\tt makeTime}) þá endum við með forrit sem er styttra, læsilegra og einfaldara að kemba og jafnframt áreiðanlegra.

Það er jafnframt auðveldara að bæta eiginleikum við forritið síðar.
Gefum okkur t.d. að við þurfum að geta dregið einn tíma frá öðrum til að finna tímann á milli þeirra.
Bein leið væri að útfæra frádráttinn með ``láni'' (e. borrowing).
Það væri hins vegar auðveldara að nota umbreytingaföllin og jafnframt líklegra til að vera rétt.

Það er kaldhæðnislegt að stundum verður vandamál auðveldara (færri sérstök tilvik og færri möguleikar á villum) ef það er gert erfiðara (almennara)!

\section{Reiknirit}
\label{algorithm}
\index{reiknirit}

Þegar þú skrifar almenna lausn fyrir safn af vandamálum, í stað þess að skrifa sérstaka lausn fyrir eitt tiltekið vandamál, þá hefur þú útfært 
{\bf reiknirit} (e. algorithm).
Ég nefndi þetta orð í kafla 1 en skilgreindi það ekki nákvæmlega.
Það er reyndar ekki auðvelt að skilgreina en hér mun ég reyna það með tveimur aðferðum.

Í fyrsta lagi skaltu hugsa um eitthvað sem er ekki reiknirit.
Þegar þú lærðir t.d. að margfalda saman tölur með einum tölustaf þá lærðir þú væntanlega margföldunartöfluna utan að.
Í rauninni settir þú 100 mismunandi lausnir á minnið!
Þess konar þekking er í raun ekki reikniritanleg.

En ef þú varst ``löt/latur'' þá svindlaðir þú líklega og lærðir nokkur trikk.
Til að finna t.d. margfeldið á $n$ and 9 getur þú skrifað $n-1$ sem fyrri tölustafinn og $10-n$ seinni stafinn.
Þetta trikk er almenn lausn til að margfalda tölu (með einum tölustaf) með 9.
Þetta er reiknirit!

Á sama hátt eru aðferðirnar sem þú lærðir til að leggja saman, með því færa yfir, og að draga frá, með því að taka að láni, líka reiknirit.
Eitt sem einkennir reiknirit er að þau krefjast ekki neinnar sérstakar greindar til að framkvæma þau.
Þau eru sjálfvirk ferli þar sem sérhvert skref er framkvæmd á eftir undanfarandi skrefi í samræmi við einfalt mengi af reglum.

Það er mín skoðun að það sé í raun vandræðalegt hversu löngum tíma við eyðum í skóla að keyra reiknirit sem þarfnast í raun engrar greindar.

Á hinn bóginn er ferlið við að hanna reiknirit mjög áhugavert og krefjandi og í raun hryggjarstykkið í því sem við köllum forritun.

Sumt af því sem manneskjan gerir á náttúrulegan hátt, án vandræða eða meðvitaðrar hugsunar, er hvað erfiðast að tjá með reikniriti.
Það að greina og skilja náttúrulegt tungumál er gott dæmi.
Við gerum það öll en hingað til hefur engum tekist að skýra út {\em hvernig} við gerum það, a.m.k. ekki í formi reiknirits.

Seinna í þessari bók færðu tækifæri til að skrifa einföld reiknirit fyrir ýmis vandamál.
Ef þú ert í tölvunarfræðinámi þá muntu síðar taka námskeiðið Reiknirit (e. Data Structures) og kynnast
mörgum áhugaverðum, snjöllum og gagnlegum reikniritum sem tölvunarfræðingar hafa þróað.

\section{Orðalisti}

\begin{description}

\item[tilvik (e.instance):]  Eitt dæmi úr tilteknum flokki.  Kötturinn minn er tilvik úr flokknum ``læður''. 
Sérhver hlutur (e. object) er tilvik af einhverju tagi.

\item[tilvikabreyta (e. instance variable):]  Ein af þeim breytum sem mynda strúktúr. Sérhver tilvik af strúktúr á sér eigin afrit af tilvikabreytunum fyrir viðkomandi tag.

\item[fast tilvísunarviðfang (e. constant reference parameter):]  Viðfang sem er sent sem tilvísun í fall en er samt sem áður ekki hægt að breyta í fallinu.

\item[hreint fall (e. pure function):]  Fall, hvers skilagildi er eingöngu háð viðföngum þess, og sem hefur engar aukaverkanir en þær að skila gildi.

\item[fallaforritunarstíll (e. functional programming style):]  Forritunarstíll sem leggur áherslu á að meginhluti falla séu hrein.

\item[breytir (e. modifier):]  Fall sem breytir einu eða fleiri viðföngum og skilar yfirleitt {\tt void}.

\item[fyllir (e. fill-in function):]  Fall sem tekur ``tóman'' hlut sem viðfang og fyllir inn í tilvikabreytur hlutarins í stað þess að skila gildi.

\item[reiknirit (e. algorithm):]  Eins konar uppskrift til að leysa tiltekna tegund af vandamálum með sjálfvirku ferli.

\index{tilvik}
\index{tilvikabreyta}
\index{hreint fall}
\index{fallaforritun}
\index{breytir}
\index{fyllir}
\index{reiknirit}

\end{description}

