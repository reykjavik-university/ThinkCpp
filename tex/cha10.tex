% LaTeX source for textbook ``How to think like a computer scientist''
% Copyright (C) 1999  Allen B. Downey

% This LaTeX source is free software; you can redistribute it and/or
% modify it under the terms of the GNU General Public License as
% published by the Free Software Foundation (version 2).

% This LaTeX source is distributed in the hope that it will be useful,
% but WITHOUT ANY WARRANTY; without even the implied warranty of
% MERCHANTABILITY or FITNESS FOR A PARTICULAR PURPOSE.  See the GNU
% General Public License for more details.

% Compiling this LaTeX source has the effect of generating
% a device-independent representation of a textbook, which
% can be converted to other formats and printed.  All intermediate
% representations (including DVI and Postscript), and all printed
% copies of the textbook are also covered by the GNU General
% Public License.

% This distribution includes a file named COPYING that contains the text
% of the GNU General Public License.  If it is missing, you can obtain
% it from www.gnu.org or by writing to the Free Software Foundation,
% Inc., 59 Temple Place - Suite 330, Boston, MA 02111-1307, USA.

% This is an Icelandic translation/adaptation of the orginal book by Allen B. Downey

\chapter{Vektorar}
\label{vectors}
\index{vektor}
\index{tag!vektor}

{\bf Vektor} (e. vector) er mengi gilda þar sem hvert þeirra er auðkennt með tölu sem kölluð er vísir (e. index).
Strengur ({\tt string}) er svipaður og vektor því hann samanstendur af mengi af stöfum sem hægt er að vísa í með tölu.
Vektorar í C++ eru hentugir í notkun því þeir geta geymt ýmiss konar gögn, þ.m.t. grunntög eins og {\tt int} og {\tt double}, 
eða tög sem skilgreind eru af notanda, eins og {\tt Point} og {\tt Time}.

Tagið {\tt vector} er skilgreint í C++ ``Standard Template Library'' (STL).
Til að nota það þarf að taka inn (e. include) hausaskrána {\tt vector} -- hvernig það er gert er háð þínu forritunarumhverfi.

Þú getur búið til vektor á sama hátt og þú býrð til breytu af hvaða öðru tagi:

\begin{verbatim}
  vector<int> count;
  vector<double> doubleVector;
\end{verbatim}
%
Tagið sem vektorinn mun geyma kemur fram í hornsvigunum (e. angle brackets) ({\tt <} og {\tt >}).
Fyrri línan býr til vektor af heiltölum með nafnið {\tt count} en seinni línan býr til vektor af {\tt double}.
Þrátt fyrir að þessar setningar séu löglegar þá eru þær ekki mjög gagnlegar vegna þess að þær búa til vektora sem hafa engin stök (stærð þeirra er núll).
Það er því mun algengara að skilgreina stærð vektorsins innan sviga: 

\begin{verbatim}
  vector<int> count (4);
\end{verbatim}
%
Málskipanin er hér dálítið skrýtin því hún lítur út eins og samsetning breytuyfirlýsingar og fallakalls.
Það er reyndar nákvæmlega það sem hún er!
Fallið sem við erum að kalla á er svokallaður smiður í {\tt vector} klasanum.
{\bf Smiður} (e. constructor) er sérstakt fall sem býr til nýtt tilvik og upphafsstillir tilvikabreytur þess.
Í þessu tilviki tekur smiðurinn eitt viðfang sem er stærð vektorsins.

\index{smiður}

Eftirfarandi mynd sýnir hvernig vektorar eru táknaðir á stöðuritum:

\vspace{0.1in}
\centerline{\epsfig{figure=vector.eps}}
\vspace{0.1in}

Stóru tölurnar innan í boxunum eru {\bf stök} (e. elements) vektorsins. 
Litlu tölurnar fyrir utan boxin eru vísarnir (e. indices) sem notaðir eru til að einkenna sérhvert box.
Stök vektors eru ekki upphafsstillt þegar nýr vektor er búinn til og því gætu þau gætu í raun innihaldið hvaða gildi sem er.

Það er til annar smiður fyrir {\tt vector} sem tekur tvö viðföng; seinna viðfangið er ``upphafsstillingargildi'', þ.e. gildið sem sérhvert stak vektorsins fær í upphafi.

\begin{verbatim}
  vector<int> count (4, 0);
\end{verbatim}
%
Þessi setning býr til vektor með fjórum stökum og upphafsstillir öll stökin með núlli.

\section{Aðgangur að stökum}
\index{stak}
\index{vektor!stak}

Virkjann {\tt []} er hægt að nota til að lesa og skrifa stök vektors á sambærilegan hátt og gert er með strengi.
Vísarnir byrja í núlli, þannig að {\tt count[0]} vísar til ``núllta'' staks vektorsins {\tt count} og
{\tt count[1]} vísar til ``fyrsta'' staksins.
Þú getur notað virkjann {\tt []} hvar sem er í segð:

\begin{verbatim}
  count[0] = 7;
  count[1] = count[0] * 2;
  count[2]++;
  count[3] -= 60;
\end{verbatim}
%
Allar þessar setningar eru löglegar og áhrif þessa kóða á vektorinn er: 

\vspace{0.1in}
\centerline{\epsfig{figure=vector2.eps}}
\vspace{0.1in}

Ekkert stak er með vísinn 4 þar sem stök þessa vektors eru númeruð frá 0 to 3.
Það er algeng villa að reyna að vísa ``út fyrir'' vektor og það veldur keyrsluvillu.
Í því tilfelli skrifar forritið út villuskilaboð eins og ``Illegal vector index'', og hættir síðan keyrslu.

\index{keyrsluvilla}
\index{vísir}
\index{segð}

Þú getur notað hvaða segð sem er sem vísi svo framarlega sem hún hefur tagið {\tt int}.
Ein algengasta leiðin til að vísa í stök vektors er að nota lykkjubreytu (e. loop variable).
Dæmi:

\begin{verbatim}
  int i = 0;
  while (i < 4) {
    cout << count[i] << endl;
    i++;
  }
\end{verbatim}
%
Þessi {\tt while} lykkja ``hleypur'' frá 0 í 4.
Þegar lykkjubreytan {\tt i} fær gildið 4 þá verður skilyrðið {\tt false} og lykkjan hættir.
Meginmál lykkjunnar er því eingöngu keyrt þegar {\tt i} er 0, 1, 2 og 3.

\index{lykkja}
\index{lykkjubreyta}
\index{breyta!lykkja}

Í sérhverri ítrun lykkjunnar notum við {\tt i} sem vísi inn í vektorinn og skrifum út {\tt i}-ta stakið.
Þessi tegund af ``vektorrölti'' er mjög algeng enda vinna vektorar og lykkjur vel saman.

\section{Afritun vektora}
\index{vektor!afritun}

Það er til einn smiður í viðbót fyrir {\tt vector} sem kallaður ``afritatökusmiður'' (e. copy constructor)
vegna þess að hann tekur einn {\tt vector} sem viðfang og býr til nýjan vektor af sömu stærð og með sömu stök.

\begin{verbatim}
  vector<int> copy (count);
\end{verbatim}
%
Þrátt fyrir að þessi málskipan sé lögleg þá er hún sjaldan notuð fyrir vektorar því til er betri leið:

\begin{verbatim}
  vector<int> copy = count;
\end{verbatim}
%
Gildisveitingarvirkinn {\tt =} virkar fyrir vektorar á þann hátt sem ætla má.

\section{{\tt for} lykkjur}

Lykkjurnar sem við höfum skrifað hingað til eiga ýmislegt sameiginlegt.
Allar byrja þær á því að upphafsstilla breytur.
Þær innihalda jafnframt skilyrði sem er háð breytunni og meginmál lykkjunnar gerir eitthvað við breytuna, eins og að hækka hana. 

\index{lykkja!for}
\index{for}
\index{setning!for}

Þessi tegund af lykkju er svo algeng að það er til annars konar lykkjusetning, kölluð {\tt for}-setning, sem tjáir þetta á samþjappaðri hátt.
Málskipanin lítur svona út:

\begin{verbatim}
  for (INITIALIZER; CONDITION; INCREMENTOR) {
    BODY
  }
\end{verbatim}
%
Ofangreind setning er nákvæmlega jafngild þessu:

\begin{verbatim}
  INITIALIZER;
  while (CONDITION) {
    BODY
    INCREMENTOR
  }
\end{verbatim}
%
{\tt for}-setningin er aftur á móti gagnorðaðri og læsilegri því í henni eru allar lykkjusetningarnar á einum stað.
Dæmi:

\begin{verbatim}
  int i;
  for (i = 0; i < 4; i++) {
    cout << count[i] << endl;
  }
\end{verbatim}
%
er jafngilt:

\begin{verbatim}
  int i = 0;
  while (i < 4) {
    cout << count[i] << endl;
    i++;
  }
\end{verbatim}

\section{Stærð vektors}
\index{stærð!vektor}
\index{vektor!stærð}

Það eru margvísleg föll sem hægt er að keyra fyrir {\tt vector}.
Eitt af þeim er þó sérstaklega gagnlegt: {\tt size()}.
Þetta fall skilar stærð vektors, þ.e. fjölda staka.

Það er góð regla að nota þetta gildi sem efri mörk (e. upper bound) lykkju frekar heldur en einhvern fasta.
Þannig þarf ekki að breyta lykkjunni ef stærð vektorsins breytist.

\begin{verbatim}
  int i;
  for (i = 0; i < count.size(); i++) {
    cout << count[i] << endl;
  }
\end{verbatim}
%
Í síðasta skiptið sem meginmál lykkjunnar er keyrt er gildið á {\tt i} jafnt {\tt count.size() - 1}, sem er vísirinn á síðasta stakinu.
Þegar {\tt i} er jafnt {\tt count.size()} mun skilyrðið verða {\tt false} og meginmál lykkjunnar verður því ekki keyrt lengur.
Það er gott því annars myndi keyrsluvilla koma upp!

Taktu eftir því að kallað er á {\tt size()} fallið í hverri ítrun lykkjunnar.
Það að kalla á fall aftur og aftur hefur áhrif á keyrslutímann þannig að í raun væri betra að geyma stærð vektorsins í einhverri breytu með því að kalla á 
{\tt size()} áður en lykkjan hefst og nota síðan breytuna til að tékka á síðasta gildinu.
Þú ættir að æfa þig með því að prófa að gera þessa breytingu.

\section{Vektorföll}
\index{föll!vektor}
\index{vektor!föll}

Einn besti eiginleiki vektors er geta hans til að stækka eða minnka að þörf.
Eftir að vektor hefur verið búinn til þá er hægt að stækka hann eða minnka hvar sem er í forritinu.
Gerum t.d. ráð fyrir að við lesum inn tölur frá notanda inn í vektor þangað til notandi slær inn {\tt -1} en þá skrifum við tölurnar út.
Í svona tilviki vitum við ekki stærðina á vektornum fyrirfram.
Við þurfum því að geta bætt nýjum stökum við enda vektorsins um leið og notandinn slær inn ný gildi.
Við getum notað fallið {\tt push\_back()} í þessum tilgangi:

\begin{verbatim}
  #include<iostream>
  #include<vector>
  using namespace std;
  int main()
  {
    vector<int> values;
    int c,i,len;
    cin>>c;
    
    while(c != -1) {
      values.push_back(c);
      cin >> c;
    }
    len=values.size();
    for(i = 0; i < len; i++) {
      cout << values[i] << endl;
    }
  }

\end{verbatim}

\section{Slembitölur}
\label{random}
\label{pseudorandom}
\index{slembitala}
%\index{deterministic}
\index{löggengur}
%\index{nondeterministic}
\index{brigðgengur}

Flestar tölvur gera það sama í hvert sinn sem þær eru keyrðar með sama forritinu og eru kallaðar {\bf löggengar} (e. deterministic). 
Löggengni er yfirleitt kostur því við viljum jú að sömu útreikningar skili alltaf af sér sömu niðurstöðu.
Í sumum tilvikum gætum við aftur á móti viljað að tölvur væru óútreiknanlegar.  Tölvuleikir eru gott dæmi um þetta.

Það að gera forrit algerlega {\bf brigðgengt} (e. nondeterministic) er ekki auðvelt en það eru til leiðir til að láta það líta svo út.
Ein leið er að búa til hermislembitölur (e. pseudo random numbers) og nota þær til að stýra útkomunni úr forriti.
Hermislembitölur eru ekki algerlega handahófskenndar í stærðfræðilegum skilningi en þær duga fyrir það sem við ætlum að gera.

Í hausaskránni {\tt cstdlib} (sem inniheldur ýmis konar ``standard library'' föll) er fallið {\tt random} skilgreint en það býr til hermislembitölur.

Skilagildið úr {\tt random} er heiltala á milli 0 og {\tt RAND\_MAX}
en {\tt RAND\_MAX} er stór tala (um það bil 2 milljarðar á minni tölvu) sem einnig er skilgreind í hausaskránni.
Í sérhvert sinn sem þú kallar á {\tt random} færðu nýja slembitölu.
Til að sjá dæmi um þetta skaltu keyra eftirfarandi lykkju:

\begin{verbatim}
#include <iostream>
#include <cstdlib>
using namespace std;

int main ()
{
  for (int i = 0; i < 4; i++) {
    int x = random ();
    cout << x << endl;
  }
  return 0;
}

  
\end{verbatim}
%
Ég fæ eftirfarandi úttak á minni vél: 

\begin{verbatim}
1804289383
846930886
1681692777
1714636915
\end{verbatim}
%
Þú færð væntanlega eitthvað annað, en þó svipað, á þinni tölvu.

Auðvitað viljum við ekki alltaf vinna með svona stórar heiltölur.
Það er algengara að búa til heiltölur á milli 0 og einhvers efri marks.
Einföld leið til að gera það er að nota ``modulus'' virkjann:
Dæmi:

\begin{verbatim}
  int x = random ();
  int y = x % upperBound;
\end{verbatim}
%
Þar sem {\tt y} er afgangurinn sem fæst með þvi að deila {\tt x} með {\tt upperBound} þá liggja gildin fyrir {\tt y} á bilinu 0 og {\tt upperBound - 1} (að báðum meðtöldum).
Hafðu í huga að {\tt y} er aldrei jafnt og {\tt upperBound}.

Það er einnig oft gagnlegt að búa til slembikommutölur
Algeng leið til að gera það er að deila með {\tt RAND\_MAX}.
Dæmi:

\begin{verbatim}
  int x = random ();
  double y = double(x) / RAND_MAX;
\end{verbatim}
%
Þessi kóði gefur {\tt y} gildi slembikommutölu á milli 0,0 and 1,0 (að báðum meðtöldum).
Þú ættir núna að íhuga hvernig hægt er að búa til slembikommutölu á ákveðnu bili, t.d. á milli 100,0 og 200,0.

\section{Tölfræði}
\index{statistics}
\index{dreifing}
\index{meðaltal}

Tölurnar sem {\tt random} fallið býr til eiga að dreifast jafn (e. distributed uniformly).
Það þýðir að sérhvert gildi á hinu valda bili ætti að vera jafn líklegt.
Fjöldinn af sérhverju gildi ætti að vera nokkurn veginn sá sami að því gefnu að við látum {\tt random} búa til nógu mörg gildi fyrir okkur.

Hér á eftir munum við skrifa forrit sem mynda röð af slembitölum og athugar hvort þessi staðhæfing sé rétt.

\section{Vektor af slembitölum}

Fyrsta skrefið er að búa til mikinn fjölda af slembitölum og geyma þær í vektor.
Með ``mikinn fjölda'' á ég auðvitað við 20!  
Það er góð regla að byrja með lítinn fjölda (sem auðveldar kembun) og fjölga tölum síðar.

Eftirfarandi fall tekur eitt viðfang, stærð vektors.
Það úthlutar minni fyrir nýjan vektor fyrir {\tt int} og fyllir hann með slembitölum á bilinu 0 og {\tt upperBound-1}.

\begin{verbatim}
vector<int> randomVector (int n, int upperBound) {
  vector<int> vec (n);
  for (int i = 0; i<vec.size(); i++) {
    vec[i] = random () % upperBound;
  }
  return vec;
}
\end{verbatim}
%
Skilagildið er {\tt vector<int>}, þ.e. vektor af heiltölum.
Til að prófa þetta fall er þægilegt að hafa fall sem skrifar út innihalds vektors:

\begin{verbatim}
void printVector (const vector<int>& vec) {
  for (int i = 0; i<vec.size(); i++) {
    cout << vec[i] << " ";
  }
}
\end{verbatim}
%
Taktu eftir því að það er löglegt að senda {\tt vector} með tilvísun.
Það er einmitt mjög algengt því þá þarf ekki að afrita öll stök vektorsins (eins og gera þyrfti ef vektorinn væri sendur sem gildi).
Við skilgreinum leppinn sem {\tt const} þar sem {\tt printVector} breytir ekki viðfangi sínu. 

Eftirfarandi kóði býr til vektor og skrifar innihald hans út: 

\begin{verbatim}
  int numValues = 20;
  int upperBound = 10;
  vector<int> vector = randomVector (numValues, upperBound);
  printVector (vector);
\end{verbatim}
%
Á minni vél er úttakið 

\begin{verbatim}
3 6 7 5 3 5 6 2 9 1 2 7 0 9 3 6 0 6 2 6 
\end{verbatim}
%
sem virðist vera nokkuð handahófskennt.  Þitt úttak er væntanlega öðruvísi.

Ef þessar tölur eru raunverulegar slembitölur þá megum við gera ráð fyrir því að sérhver tala komi jafn oft fyrir, þ.e. tvisvar sinnum.
Það vill reyndar svo til að talan 6 kemur fyrir fimm sinnum og tölurnar 4 og 8 koma ekki fyrir.

En þýðir þetta þá að gildin fylgja ekki jafnri dreifingu?
Það er erfitt að segja til um, því þegar gildin eru svona fá þá eru afskaplega litlar líkur á því að fá nákvæmlega það sem við eigum von á.
Þegar gildum fjölgar þá ætti útkoman, aftur á móti, að vera fyrirsjáanlegri.

Til að prófa þessa tilgátu munum við skrifa forrit sem telur hversu oft sérhvert gildi kemur fyrir og síðan athuga hvað gerist
þegar við hækkum {\tt numValues}.

\section{Talning}
\label{counting}
\index{ferðast um!talning}
\index{lykkja!talning}
\index{teljari}

Góð leið til að leysa vandamál eins og þetta er að skrifa einföld föll sem er auðvelt að skrifa og gera má ráð fyrir að verði gagnleg.
Þá er hægt að skrifa heildarlausnina með því að nýta sér þessi einföldu föll.
Þessi forritunaraðferð er stundum kölluð {\bf neðansækin hönnun} (e. bottom-up deisgn).
Auðvitað er það svo að það er ekki auðvelt að vita fyrirfram hvaða föll koma til með að verða gagnleg
en eftir því sem reynslan eykst því auðveldara verður fyrir þig að átta þig á því.

\index{neðansækin hönnun}
\index{forritunarþróun!neðansækin}

Að auki er ekki alltaf augljóst hvers konar föll er auðvelt að skrifa en 
það er góð leið að leita að hlutvandamálum sem passa við eitthvað mynstur sem þú hefur séð áður.

\index{mynstur!teljari}

Í kafla~\ref{loopcount} skoðuðum við lykkju sem ``ferðaðist um'' í streng og taldi hversu oft tiltekinn stafur kom fyrir í strengnum.
Þú getur litið á það forrit sem dæmi um mynstur sem kallast ``ferðast um og telja''.
Einstakir hlutar þessa mynsturs eru:

\begin{itemize}

\item Mengi eða gámur (e. container) sem hægt er að ferðast um í, t.d. strengur eða vektor. 

\item Próf (e. test) sem hægt er að beita á sérhvert stak í gámnum.

\item Teljari sem heldur utan um hversu mörg stök standast prófið. 

\end{itemize}

Í þessu tilviki hef ég fall í huga sem ég kalla {\tt howMany} sem telur fjölda staka í vektor sem eru jöfn tilteknu gildi. 
Viðföngin í fallið eru vektor og heiltalan sem leitað er að.
Skilagildið er gildi sem segir til um hversu oft heiltalan fannst í vektornum.

\begin{verbatim}
int howMany (const vector<int>& vec, int value) {
  int count = 0;
  for (int i=0; i< vec.size(); i++) {
    if (vec[i] == value) count++;
  }
  return count;
}
\end{verbatim}


\section{Athugun á öðrum gildum}

{\tt howMany} telur aðeins fjölda tilvika á af tilteknu gildi í vektornum en við höfum áhuga á hversu oft sérhvert gildi kemur fyrir.
Það getum við leyst með því að nota lykkju:

\begin{verbatim}
  int numValues = 20;
  int upperBound = 10;
  vector<int> vector = randomVector (numValues, upperBound);

  cout << "value\thowMany" << endl;

  for (int i = 0; i<upperBound; i++) {
    cout << i << '\t' << howMany (vector, i) << endl;
  }
\end{verbatim}
%
Taktu eftir því að það er löglegt að skilgreina breytu innan í {\tt for}-setningu.
Þessi málskipan er stundum hentug en athugaðu að breyta sem skilgreind er innan í lykkju er eingöngu lifandi í lykkjunni.
Þú færð villu frá þýðandanum ef þú reynir að vísa í {\tt i} utan lykkjunnar.

Þessi kóði sendir lykkjubreytuna sem viðfang í {\tt howMany} í þeim tilgangi að tékka á sérhverju gildi á milli 0 og 9 í réttri röð.
Niðurstaðan er:

\begin{verbatim}
value   howMany
0       2
1       1
2       3
3       3
4       0
5       2
6       5
7       2
8       0
9       2
\end{verbatim}
%
Það er erfitt að segja hvort tölurnar birtast í raun jafn oft.
Ef við hækkum hins vegar {\tt numValues} í 100.000 þá fæst:

\begin{verbatim}
value   howMany
0       10130
1       10072
2       9990
3       9842
4       10174
5       9930
6       10059
7       9954
8       9891
9       9958
\end{verbatim}
%
Í sérhverju tilviki er fjöldi tilvika innan við 1\% af væntu gildi (10.000) þannig að við getum ályktað sem svo að tölurnar dreifist líklega jafnt.

\section {Súlurit}
\index{histogram}

Það er oft gagnlegt að geyma gögn, eins og úr töflunni hér að ofan, og sækja þau síðar í stað þess að prenta þau bara út.
Til þess þurfum við einhverja leið til að geyma 10 heiltölur.
Við gætum búið til 10 heiltölubreytur með nöfn eins og {\tt howManyOnes}, {\tt howManyTwos}, o.s.frv.
Þá þyrftum við hins vegar að slá mikið inn og það yrði erfitt fyrir okkur að breyta forritinu ef bilið (nú 0-9) breytist.

Miklu betri lausn er að nota vektor af stærðinni 10.
Þannig getum við búið til minnið fyrir allar 10 heiltölurnar í einu og getum nálgast þær með því að nota vísa (e. indices) í stað 10 mismunandi breytunafna.
Hér er dæmi um þetta:

\begin{verbatim}
  int numValues = 100000;
  int upperBound = 10;
  vector<int> vector = randomVector (numValues, upperBound);
  vector<int> histogram (upperBound);

  for (int i = 0; i<upperBound; i++) {
    int count = howMany (vector, i);
    histogram[i] = count;
  }
\end{verbatim}
%
Ég kallaði vektorinn {\bf histogram} (súlurit) því það er tölfræðilegt hugtak fyrir vektor af tölum sem heldur utan um fjölda tilvika af gildum á ákveðnu bili.

\index{súlurit}

Taktu eftir að hér nota ég lykkjubreytuna á tvo mismunandi vegu.
Í fyrsta lagi sem viðfang í {\tt howMany} fallið til að tilgreina hvaða gildi ég hef áhuga á í hverri ítrun.
Í öðru lagi er breytan notuð sem vísir inn í súluritið, þ.e. til að tilgreina hvar (í súluritinu) geyma eigi fjöldann.

\section{Skilvirkari lausn}

Þrátt fyrir að þessi kóði virki þá er hann ekki eins skilvirkur eins og hann gæti verið.
Ferðast er um í öllum vektornum í hvert sinn sem kallað er á {\tt howMany}.
Í þessu dæmi er því ferðast um í vektornum 10 sinnum!

Það væri miklu betra að renna aðeins einu sinni (e. make a single pass) í gegnum vektorinn.
Fyrir sérhvert gildi í vektornum gætum við fundið samsvarandi teljari og hækkað hann.
Við getum, m.ö.o., notað gildið í vektornum sem vísi inn í súluritið.
Svona myndi þetta líta út:

\begin{verbatim}
  vector<int> histogram (upperBound, 0);

  for (int i = 0; i<numValues; i++) {
    int index = vector[i];
    histogram[index]++;
  }
\end{verbatim}
%
Fyrsta línan upphafsstillir stök súluritsins með núlli.
Það þýðir að þegar við notum virkjann {\tt ++} innan í lykkjunni þá vitum við að byrjað er í 0.
Það er einmitt algeng villa að gleyma að upphafsstilla með núlli.

Þú ættir núna að hjúpa þennan kóða í fall sem kallað er {\tt histogram}.
Fallið tekur vektor og bil (í þessu tilviki 0-9) sem viðföng og skilar súluriti yfir gildin í vektornum.

\section{Slembifræ}
\index{fræ}
\index{slembitölur}

Ef þú hefur keyrt kóðann í þessum kafla nokkrum sinnum þá hefur þú kannski tekið eftir því að þú hefur fengið sömu ``slembitölurnar'' aftur og aftur.
Það er ekki mjög handahófskennt!

Einn af eiginleikum hermislembitalna er að ef byrjað er á sama stað þá myndast alltaf sama talnaröðin.
Byrjunarpunkturinn er kallað ``fræ'' (e. seed) og C++ notar allaf sama fræið í hvert sinn sem þú keyrir forritið.

Það getur oft verið gagnlegt að sjá sömu niðurstöðu (talnaröðina) aftur og aftur á meðan þú ert að kemba forrit.
Þannig getur þú auðveldlega séð hvort breyting á forritinu þínu hefur í raun áhrif á úttak þess.

Ef þú vilt hins vegar nota mismunandi fræ fyrir myndun slembitalna þá getur þú notað {\tt srand} fallið.
Það tekur eitt viðfang sem er heiltala á milli 0 og {\tt RAND\_MAX}.

Í mörgum forritum, t.d. leikjum, er þörf á því að fá mismunandi slembitölurunur í hvert sinn sem forritið er keyrt.
Algeng leið til að gera það er að nota fall eins og {\tt gettimeofday} sem býr til eitthvað sem er tiltölulega ófyrrisjáanlegt og ekki auðvelt að endurtaka,
eins og fjöldi millisekúndna síðan kallað var síðast, og nota það gildi sem fræ.

\section{Orðalisti}

\begin{description}

\item[vektor (e. vector):]  Safn af gildum sem öll hafa sama tag. Sérhvert gildi er einkennt með vísi (e. index).

\item[stak (e. element):]  Eitt af gildunum í vektor. Virkinn {\tt []} velur stök í vektor.

\item[vísir (e. index):]  Heiltölubreyta eða gildi sem notað er til að einkenna stak í vektor. 

\item[smiður (e. constructor):]  Sérstakt fall sem býr til nýtt tilvik og upphafsstillir tilvikabreytur þess.

\item[löggengt forrit (e. deterministic program):]  Forrit sem gerir það sama í hvert skipti sem það er keyrt.

\item[hermislembitöluruna (e. pseudo random sequence):]  Runa af tölum sem virðist vera handahófskennd en er í raun niðurstaðan af löggengum útreikningum.

\item[fræ (e. seed):]  Gildi sem notað er til að upphafsstilla slembitölurunu.
Ef sama fræ er notað þá ætti að fást sama slembitöluruna.

\item[neðansækin hönnun (e. bottom-up design):]  Forritunaraðferð sem byggir á því að skrifa fyrst lítil, gagnleg föll, sem síðan eru nýtt í stærri heildarlausn.

\item[súlurit (e. histogram):]  Vektor af heiltölugildum þar sem hver tala stendur fyrir fjölda gilda á ákveðnu bili.

\index{vektor}
\index{stak}
\index{vísir}
\index{smiður}
\index{löggeng}
\index{slembitöluruna}
\index{fræ}
\index{súlurit}

\end{description}
