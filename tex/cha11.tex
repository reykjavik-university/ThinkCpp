% LaTeX source for textbook ``How to think like a computer scientist''
% Copyright (C) 1999  Allen B. Downey

% This LaTeX source is free software; you can redistribute it and/or
% modify it under the terms of the GNU General Public License as
% published by the Free Software Foundation (version 2).

% This LaTeX source is distributed in the hope that it will be useful,
% but WITHOUT ANY WARRANTY; without even the implied warranty of
% MERCHANTABILITY or FITNESS FOR A PARTICULAR PURPOSE.  See the GNU
% General Public License for more details.

% Compiling this LaTeX source has the effect of generating
% a device-independent representation of a textbook, which
% can be converted to other formats and printed.  All intermediate
% representations (including DVI and Postscript), and all printed
% copies of the textbook are also covered by the GNU General
% Public License.

% This distribution includes a file named COPYING that contains the text
% of the GNU General Public License.  If it is missing, you can obtain
% it from www.gnu.org or by writing to the Free Software Foundation,
% Inc., 59 Temple Place - Suite 330, Boston, MA 02111-1307, USA.

% This is an Icelandic translation/adaptation of the orginal book by Allen B. Downey

\chapter{Meðlimaföll}

\section{Hlutir og föll}
\index{meðlimafall}
\index{fall!meðlimur}

C++ er hlutbundið forritunarmál sem merkir að það hefur eiginleika sem styðja við hlutbundna forritun.

Það er ekki auðvelt að skilgreina hlutbundna forritun en við höfum þó þegar séð suma eiginleika hennar:

\begin{enumerate}

\item Forrit samanstanda af safni af strúktúrskilgreiningum og fallaskilgreiningum 
þar sem flest föllin vinna með tiltekna tegundur af strúktúrum (eða hlutum).

\item Sérhver strúktúrskilgreining stendur fyrir einhvern hlut eða hugtak í raunveruleikanum og föllin sem framkvæma aðgerðir á viðkomandi strúktúr
líkja eftir því hvernig raunverulegir hlutir eiga samskipti.

\end{enumerate}

{\tt Time} strúktúrinn, t.d., sem við skilgreindum í kafla~\ref{time}
samsvarar augljóslega hvernig fólk skráir tíma innan dagsins og 
aðgerðirnar sem við skilgreindum samsvara því hvað fólk gerir við hugtakið tími.

Á sambærilegan hátt má segja að {\tt Point} og
{\tt Rectangle} hlutirnir samsvari stærðfræðilegu hugtökunum hnit og rétthyrningur.

Hingað til höfum við ekki nýtt okkur þá eiginleika sem C++ býður upp á til að styðja við hlutbundna forritun.
Strangt til tekið eru þessir eiginleikar reyndar ekki nauðsynlegir.
Að mestu leyti gera þeir forriturum kleift að nota aðra málskipan til að framkvæma eitthvað sem við höfum þegar gert áður
en í mörgum tilvikum er þessi nýja málskipan samanþjappaðri (e. more concise) og sýnir á nákvæmari hátt högun forritsins. 
 
Í {\tt Time} forritinu er t.d. ekkert augljóst samband á milli strúktúrskilgreiningarinnar og fallanna sem á eftir koma.
Við nánari athugun sést hins vegar er ljóst að sérhvert fallanna tekur a.m.k. einn {\tt Time} hlut sem viðfang og það leiðir okkur að hugtakinu  {\bf meðlimafall} (e. member function).
Meðlimaföll eru ólík öðrum föllum sem við höfum skrifað á tvennan hátt:

\begin{enumerate}

\item Þegar við köllum á meðlimafall þá tengjum við kallið við tiltekinn hlut (e. object).
Stundum er sagt að kall á meðlimafall hafi í för með sér aðgerð á hlut (e. operation on an object) 
eða verið sé að senda boð til hlutar (e. sending a message to an object).

\item {\em Yfirlýsing} (e. declaration) meðlimafallsins á sér stað innan {\tt struct}
skilgreiningar í þeim tilgangi að gera sambandið á milli strúktúrsins og fallsins skýrt (e. explicit). 

\end{enumerate}

Í því sem á eftir kemur munum við taka föllin úr kafla~\ref{time} og umbreyta þeim í meðlimaföll.
Athugaðu að þessi umbreyting er algerlega vélræn, m.ö.o. hægt er að framkvæma hana með því að fylgja ákveðinni forskrift.

%\index{nonmember function}
\index{fall!nonmember}

Eins og ég nefndi að ofan þá getur allt það sem hægt er að gera með meðlimafalli líka verið gert með falli sem stendur eitt og sér (e. nonmember function / free-standing function).
Oft eru hins vegar kostir/ókostir sem fylgja notkun annars í stað hins.
Ef þú getur á auðveldan hátt umbreytt öðru forminu yfir í hitt þá munt þú geta valið betri leiðina fyrir það sem þú vinnur að sérhverju sinni.

\section{{\tt print}}

Í kafla~\ref{time} skilgreindum við strúktúrinn {\tt Time}
og skrifuðum fallið {\tt printTime}

\begin{verbatim}
struct Time {
  int hour, minute;
  double second;
};

void printTime (const Time& time) {
  cout << time.hour << ":" << time.minute << ":" << time.second << endl;
}
\end{verbatim}
%
Til að kalla á þetta fall þurftum við að senda {\tt Time} hlut sem viðfang. 

\begin{verbatim}
  Time currentTime = { 9, 14, 30.0 };
  printTime (currentTime);
\end{verbatim}
%
Fyrsta skrefið í áttina að gera {\tt printTime} að meðlimafalli er að breyta nafninu á fallinu úr {\tt printTime} í {\tt Time::print}.
Virkinn {\tt ::} kemur á milli nafns strúktúrsins og nafns fallsins.
Til saman gefa þessi nöfn til kynna að um sé að ræða fallið {\tt print} sem tilheyrir {\tt Time} strúktúr.

Næsta skref felst í því að fjarlægja leppinn.
Í stað þess að senda hlut inn sem viðfang þá vekjum við fallið með tilteknum hlut (e. invoke the function on an object).

Niðurstaðan er sú að inni í fallinu er ekki lengur leppur með nafnið {\tt time}.
Í staðinn er um að ræða {\bf núverandi hlut} (e. current object) sem er hluturinn sem fallið er keyrt á.
Hægt er að vísa í núverandi hlut með því að nota C++ lykilorðið {\tt this}.

\index{núverandi hlutur}
\index{hlutur!núverandi}
\index{bendir}
\index{this}

Það flækir aðeins málið að {\tt this} er í raun {\bf bendir} á strúktúr frekar en strúktúrinn sjálfur.
Bendir er svipaður og tilvísun en ég vil þó ekki fjalla nákvæmlega um bendanotkun enn sem komið er.
Eina bendaaðgerðin sem við þurfum á að halda núna er {\tt *} virkinn (e. dereference operator) sem breytir bendi á strúktúr yfir í strúktur.
Í eftirfarandi falli notum við þennan virkja til að gefa staðværu (e. local) breytunnni {\tt time} gildið á {\tt this}:

\begin{verbatim}
void Time::print () {
  Time time = *this;
  cout << time.hour << ":" << time.minute << ":" << time.second << endl;
}
\end{verbatim}
%
Fyrstu tvær línurnar í þessu falli breyttust þó nokkuð þegar við umbreyttum fallinu yfir í meðlimafall 
en taktu eftir því að úttakssetningin sjálf breyttist ekki neitt.

Þessa nýju útgáfu af {\tt print} keyrum við síðan á {\tt Time} hlut (taktu eftir punktatáknuninni):

\begin{verbatim}
  Time currentTime = { 9, 14, 30.0 };
  currentTime.print ();
\end{verbatim}
%
Síðasta skrefið í umbreytingarferlinu er að við þurfum að lýsa yfir fallinu inni í strúktúrskilgreiningunni:

\begin{verbatim}
struct Time {
  int hour, minute;
  double second;

  void print ();
};
\end{verbatim}
%
{\bf Yfirlýsing} falls lítur eins út og fyrsta línan í skilgreiningu fallsins nema að því leyti að yfirlýsingin er með semíkommu í endann og ekki þarf að tiltaka nafn strúktúrsins sem fallið tilheyrir (í þessu tilfelli Time).
Yfirlýsingin lýsir {\bf skilum} (e. interface) fallsins, þ.e. fjölda og tagi leppa og tagi skilagildisins.

Þegar þú lýsir yfir falli þá ``lofar'' þú þýðandanum að þú munir, á einhverjum öðrum stað í forritinu, setja skilgreiningu fallsins fram.
Skilgreiningin sjálf er {\bf útfærsla) (e. implementation) fallsins því hún inniheldur nákvæmar upplýsingar um hvernig fallið virkar.
Þýðandinn mun kvarta ef þú sleppir skilgreiningunni eða setur fram skilgreiningu sem inniheldur önnur skil en þú lofaðir.

\section {Dulinn aðgangur að breytum (e. Implicit variable access)}

Nýja útgáfan okkar af {\tt Time::print} er reyndar flóknari en hún þarf að vera.
Við þurfum í raun ekki að búa til staðværa breytu í þeim tilgangi að vísa í tilvikabreytur núverandi hlutar.

Ef fallið vísar í {\tt hour}, {\tt minute} eða {\tt second}, án þess að nota punktatáknun, þá veit C++ þýðandinn að um er að ræða tilvísun í núverandi hlut.
Við gætum því skrifað fallið á þennan hátt: 

\begin{verbatim}
void Time::print ()
{
  cout << hour << ":" << minute << ":" << second << endl;
}
\end{verbatim}
%
Þessi tegund af aðgangi að breytum er kallaður ``dulinn'' (e. implicit) vegna þess að nafn hlutarins er ekki ljóst (e. explicit). 
Þetta er ein ástæða þess að meðlimaföll eru oft samþjappaðri en föll sem eru ekki meðlimaföll.

\section {Annað dæmi}

Nú skulum við umbreyta {\tt increment} í meðlimafall.
Við munum breyta einu viðfanginu í dulda viðfangið {\tt this}.
Síðan getum við gert allan aðgang að einstökum breytum dulinn. 

\begin{verbatim}
void Time::increment (double secs) {
  second += secs;

  while (second >= 60.0) {
    second -= 60.0;
    minute += 1;
  }
  while (minute >= 60) {
    minute -= 60.0;
    hour += 1;
  }
}
\end{verbatim}
%
Athugaðu að þessi útfærsla fallsins er ekki sú skilvirkasta.
Þú ættir núna að skrifa skilvirkari útfærslu ef þú gerðir það ekki þegar þú last kafla~\ref{time}

Til að lýsa fallinu yfir þurfum við aðeins að afrita fyrstu línuna úr skilgreiningu þess (og sleppa tilvísuninni í Time strúktúrinn):

\begin{verbatim}
struct Time {
  int hour, minute;
  double second;

  void print ();
  void increment (double secs);
};
\end{verbatim}
%
Nú getum við kallað á fallið með því að vekja það með tilteknum {\tt Time} hlut:

\begin{verbatim}
  Time currentTime = { 9, 14, 30.0 };
  currentTime.increment (500.0);
  currentTime.print ();
\end{verbatim}
%
Útttakið úr þessu forriti er {\tt 9:22:50}.

\section{Enn eitt dæmi}

Upphaflega útgáfan af {\tt convertToSeconds} leit svona út:

\begin{verbatim}
double convertToSeconds (const Time& time) {
  int minutes = time.hour * 60 + time.minute;
  double seconds = minutes * 60 + time.second;
  return seconds;
}
\end{verbatim}
%
Það er einfalt að umbreyta þessu í meðlimafall: 

\begin{verbatim}
double Time::convertToSeconds () const {
  int minutes = hour * 60 + minute;
  double seconds = minutes * 60 + second;
  return seconds;
}
\end{verbatim}
%
Hér er athyglisvert að dulda viðfanginu er lýst yfir sem 
{\tt const} vegna þess að við breytum því ekki inni í fallinu. 
Það er reyndar ekki augljóst hvar setja á upplýsingar um viðfang sem er ekki til!
Eins og sést í þessu dæmi felst lausnin í því að setja þetta á eftir viðfangalistanum (sem er reyndar tómur í þessu tilviki).

Fallið {\tt print} í fyrri kafla hefði líka á að lýsa dulda viðfanginu yfir sem {\tt const}.

\section {Flóknara dæmi}

Þrátt fyrri að ferlið við að umbreyta föllum í meðlimaföll sé vélrænt þá geta aðstæður verið sérkennilegar.
Fallið {\tt after}, t.d., vinnur á tveimur {\tt Time} hlutum (ekki bara einum) og við getum ekki gert þá báða að duldum viðföngum.
Í staðinn köllum við á fallið í gegnum annan þeirra en sendum hinn sem viðfang.

Inni í fallinu getum við notað dulinn aðgang að breytum hlutarins sem notaður var til að vekja fallið en fáum síðan aðgang að breytum hins hlutarins með því að nota punktatáknun.

\begin{verbatim}
bool Time::after (const Time& time2) const {
  if (hour > time2.hour) return true;
  if (hour < time2.hour) return false;

  if (minute > time2.minute) return true;
  if (minute < time2.minute) return false;

  if (second > time2.second) return true;
  return false;
}
\end{verbatim}
%
Svona köllum við þá á þetta fall:

\begin{verbatim}
  if (doneTime.after (currentTime)) {
    cout << "The bread will be done after it starts." << endl;
  }
\end{verbatim}
%
Það er nánast hægt að lesa þennan kóða eins og ensku:
``If the done-time is after the current-time, then...''

\section{Smiðir}

Í kafla~\ref{time} skrifuðum við fallið {\tt makeTime}:

\begin{verbatim}
Time makeTime (double secs) {
  Time time;
  time.hour = int (secs / 3600.0);
  secs -= time.hour * 3600.0;
  time.minute = int (secs / 60.0);
  secs -= time.minute * 60.0;
  time.second = secs;
  return time;
}
\end{verbatim}
%
Fyrir sérhvert nýtt tag (e. type) þurfum við að geta búið til nýja hluti af því tagi.
Það vill einmitt reyndar svo til að föll eins og {\tt makeTime} eru svo algeng að sérstök málskipan er notuð fyrir þau.
Þessi föll eru kölluð {\bf smiðir} (e. constructors} og málskipanin er þessi: 

\begin{verbatim}
Time::Time (double secs) {
  hour = int (secs / 3600.0);
  secs -= hour * 3600.0;
  minute = int (secs / 60.0);
  secs -= minute * 60.0;
  second = secs;
}
\end{verbatim}
%
Hér eru tvö atriði sem vert er að minnast á.
Í fyrsta lagi ber smiðurinn sama nafn og strúktúrinn sjálfur og hann hefur ekkert skilagildi.
Viðföngin hafa hins vegar ekkert breyst.

Í öðru lagi þurfum við ekki að búa til nýjan Time hlut og við þurfum ekki að skila neinu.
Bæði þessi skref eru gerð fyrir okkur af þýðandanum.
Við getum vísað í nýja hlutinn -- þann sem við erum að smíða -- með því að nota lykilorðið
{\tt this}, eða á dulinn hátt eins og sést að hér. 
Þegar við notum nöfnin {\tt hour}, {\tt minute}
og {\tt second} þá veit þýððandinn að við erum að vísa í meðlimabreytur (tilvikabreytur) nýja hlutarins. 

Til að vekja (kalla á) smiðinn þá notum við sérstaka málskipan sem er einhvers konar millivegur á milli breytuyfirlýsingar og fallakalls:

\begin{verbatim}
  Time time (seconds);
\end{verbatim}
%
Þessi setning lýsir yfir að breytan {\tt time} hafi tagið {\tt Time} og vekur smiðinn, sem við vorum að skrifa, 
með því að senda {\tt seconds} sem viðfang.
Kerfið úthlutar minni fyrir nýja hlutinn og smiðurinn upphafsstillir meðlimabreyturnar.
Niðurstaðan er gildi sem breytan {\tt time} fær.


\section {Að upphafsstilla eða smíða?}

Í fyrri kafla lýstum við yfir og upphafsstilltum {\tt Time} strúktúr með því að nota slaufusviga (e. squiggly-braces):

\begin{verbatim}
  Time currentTime = { 9, 14, 30.0 };
  Time breadTime = { 3, 35, 0.0 };
\end{verbatim}
%
Með því að nota smiði getum við nú notað aðra aðferð við yfirlýsingu og upphafsstillingu:

\begin{verbatim}
  Time time (seconds);
\end{verbatim}
%
Þessar tvær aðferðir standa fyrir mismunandi forritunarstíl og reyndar mismunandi tíma í sögu C++.
Það er kannski ástæðan fyrir því að C++ þýðandinn krefst þess að þú notar aðra aðferðina, en ekki báðar, í sama forritinu.

Ef þú skilgreinir smið fyrir strúktúr þá verður þú að nota smiðinn til að upphafsstilla öll ný tilvik af viðkomandi tagi.
Hin málskipanin, þ.e. að nota slaufusviga við upphafsstillingu, er þá ekki leyfð.

Sem betur fer er leyfilegt að fjölbinda (e. overload) smiði og sama hátt og hægt er að fjölbinda föll.
M.ö.o, það geta verið fleiri en einn smiður með sama nafn svo framarlega sem smiðirnir taki mismunandi viðföng.
Þegar við upphafsstillum nýtjan hlut mun þýðandinn reyna að finna þann smið sem tekur viðeigandi viðföng.
Fortunately, it is legal to overload constructors in the same

Það er t.d. algengt að hafa einn smið sem tekur eitt viðfang fyrir sérhverja meðlimabreytu og gefur meðlimabreytunum gildi viðfanganna:

\begin{verbatim}
Time::Time (int h, int m, double s)
{
  hour = h;  minute = m;  second = s;
}
\end{verbatim}
%
Við notum sömu skrýtnu málskipanina og áður til að vekja þennan smið nema að nú verða viðföngin að vera tvær heiltölur og ein {\tt double}:

\begin{verbatim}
  Time currentTime (9, 14, 30.0);
\end{verbatim}

\section {Eitt dæmi að lokum}

Síðasta dæmið sem við skoðum er {\tt addTime}:

\begin{verbatim}
Time addTime2 (const Time& t1, const Time& t2) {
  double seconds = convertToSeconds (t1) + convertToSeconds (t2);
  return makeTime (seconds);
}
\end{verbatim}
%
Við þurfum að gera nokkrar breytingar á þessu falli, þ.m.t.:

\begin{enumerate}

\item Breyta nafninu á fallinu úr {\tt addTime} í {\tt Time::add}.

\item Fjarlægja fyrsta viðfangið og gera það að duldu {\tt const} viðfangi.

\item Skipta út kalli í {\tt makeTime} fyrir vakningu á smið. 

\end{enumerate}
%
Hér er niðurstaðan:

\begin{verbatim}
Time Time::add (const Time& t2) const {
  double seconds = convertToSeconds () + t2.convertToSeconds ();
  Time time (seconds);
  return time;
}
\end{verbatim}
%
Þegar við köllum á {\tt convertToSeconds} í fyrsta sinn þá virðist vanta hlutinn sem vekja á fallið með!
Inni í meðlimafalli (eins og Time::add) gerir þýðandinn ráð fyrir þvi að við viljum vekja upp föll með núverandi hlut (þ.e. {\tt this}). 
Þess vegna á fyrsta kallið í {\tt convertToSeconds} við {\tt this} en í seinna kallinu er {\tt convertToSeconds} vakið upp með {\tt t2}.

Í næstu línu fallsins er smiður vakinn upp sem tekur eitt {\tt double} sem viðfang.
Í síðustu línunni er síðan nýja hlutnum (time) skilað til baka úr fallinu.

\section {Header files}

It might seem like a nuisance to declare functions inside
the structure definition and then define the functions later.
Any time you change the interface to a function, you have
to change it in two places, even if it is a small change
like declaring one of the parameters {\tt const}.

There is a reason for the hassle, though, which is that it
is now possible to separate the structure definition and the
functions into two files: the {\bf header file},
which contains the structure definition, and the implementation
file, which contains the functions.

Header files usually have the same name as the implementation
file, but with the suffix {\tt .h} instead of {\tt .cpp}.  For
the example we have been looking at, the header file is called
{\tt Time.h}, and it contains the following:

\begin{verbatim}
struct Time {
  // instance variables
  int hour, minute;
  double second;

  // constructors
  Time (int hour, int min, double secs);
  Time (double secs);

  // modifiers
  void increment (double secs);

  // functions
  void print () const;
  bool after (const Time& time2) const;
  Time add (const Time& t2) const;
  double convertToSeconds () const;
};
\end{verbatim}
%
Notice that in the structure definition I don't really have
to include the prefix {\tt Time::} at the beginning of every
function name.  The compiler knows that we are declaring functions
that are members of the {\tt Time} structure.

{\tt Time.cpp} contains the definitions of the member functions
(I have elided the function bodies to save space):

\begin{verbatim}
#include <iostream>
using namespace std;
#include "Time.h"

Time::Time (int h, int m, double s)  ...

Time::Time (double secs) ...

void Time::increment (double secs) ...

void Time::print () const ...

bool Time::after (const Time& time2) const ...

Time Time::add (const Time& t2) const ...

double Time::convertToSeconds () const ...
\end{verbatim}
%
In this case the definitions in {\tt Time.cpp} appear in the
same order as the declarations in {\tt Time.h}, although it
is not necessary.

On the other hand, it is necessary to include the header
file using an {\tt include} statement.  That way, while the
compiler is reading the function definitions, it knows enough
about the structure to check the code and catch errors.

Finally, {\tt main.cpp} contains the function {\tt main} along
with any functions we want that are not members of the {\tt Time}
structure (in this case there are none):

\begin{verbatim}
#include <iostream>
using namespace std;
#include "Time.h"

int main ()
{
  Time currentTime (9, 14, 30.0);
  currentTime.increment (500.0);
  currentTime.print ();

  Time breadTime (3, 35, 0.0);
  Time doneTime = currentTime.add (breadTime);
  doneTime.print ();

  if (doneTime.after (currentTime)) {
    cout << "The bread will be done after it starts." << endl;
  }
  return 0;
}

\end{verbatim}
%
Again, {\tt main.cpp} has to include the header file.

It may not be obvious why it is useful to break such a small
program into three pieces.  In fact, most of the advantages come
when we are working with larger programs:

\begin{description}

\item[Reuse:]  Once you have written a structure like {\tt Time},
you might find it useful in more than one program.  By separating
the definition of {\tt Time} from {\tt main.cpp}, you make is easy
to include the {\tt Time} structure in another program.

\item[Managing interactions:]  As systems become large, the number
of interactions between components grows and quickly becomes
unmanageable.  It is often useful to minimize these interactions
by separating modules like {\tt Time.cpp} from the programs that
use them.

\item[Separate compilation:]  Separate files can be compiled
separately and then linked into a single program later.  The details
of how to do this depend on your programming environment.  As
the program gets large, separate compilation can save a lot of time,
since you usually need to compile only a few files at a time.

\end{description}

For small programs like the ones in this book, there is
no great advantage to splitting up programs.  But it is good
for you to know about this feature, especially since it explains
one of the statements that appeared in the first program we
wrote:

\begin{verbatim}
#include <iostream>
using namespace std;
\end{verbatim}
%
{\tt iostream} is the header file that contains declarations
for {\tt cin} and {\tt cout} and the functions that operate on
them.  When you compile your program, you need the information
in that header file.

The implementations of those functions are stored in a library,
sometimes called the ``Standard Library'' that gets linked to
your program automatically.  The nice thing is that you don't
have to recompile the library every time you compile a program.
For the most part the library doesn't change, so there is no
reason to recompile it.

\section{Glossary}

\begin{description}

\item[member function:]  A function that operates on an object
that is passed as an implicit parameter named {\tt this}.

\item[nonmember function:]  A function that is not a member
of any structure definition.  Also called a ``free-standing''
function.

\item[invoke:] To call a function ``on'' an object, in order to
pass the object as an implicit parameter.

\item[current object:]  The object on which a member function
is invoked.  Inside the member function, we can refer to the
current object implicitly, or by using the keyword {\tt this}.

\item[this:]  A keyword that refers to the current object.
{\tt this} is a pointer, which makes it difficult to use, since
we do not cover pointers in this book.

\item[interface:] A description of how a function is used, including
the number and types of the parameters and the type of the return
value.

\item[function declaration:] A statement that declares the interface
to a function without providing the body.  Declarations of
member functions appear inside structure definitions even if the
definitions appear outside.

\item[implementation:] The body of a function, or the details of how
a function works.

\item[constructor:] A special function that initializes the instance
variables of a newly-created object.

\index{member function}
\index{nonmember function}
\index{function!member}
\index{function!nonmember}
\index{interface}
\index{implementation}
\index{invoke}
\index{constructor}

\end{description}

